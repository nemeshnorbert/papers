% Chapter Template

\chapter{Preliminaries} % Main chapter title

\label{ChapterPreliminaries} % Change X to a consecutive number; for referencing this chapter elsewhere, use \ref{ChapterX}

\lhead{Chapter 1. \emph{Preliminaries}} % Change X to a consecutive number; this is for the header on each page - perhaps a shortened title

In what follows, we present some parts in parallel fashion by listing the respective options in order, enclosed and separate like this: $\langle$~.../...~$\rangle$. For example: a real number $x$ is $\langle$~positive / non negative~$\rangle$ if $\langle$~$x>0$ / $x\geq 0$~$\rangle$. Sometimes one of the parts might be empty. We use symbol $:=$ for equality by definition.

We use the following standard notation for some commonly used sets of numbers: $\mathbb{C}$ denotes the complex numbers, $\mathbb{R}$ denotes the real numbers, $\mathbb{Z}$ denotes the integers, $\mathbb{N}$ denotes the natural numbers, $\mathbb{N}_n$ denotes the set of first $n$ natural numbers, $\mathbb{R}_+$ denotes the set of non negative real numbers, $\mathbb{T}$ denotes the set of complex numbers of modulus $1$, finally, $\mathbb{D}$ denotes the set of complex numbers with modulus less than $1$. For $z\in\mathbb{C}$ the symbol $\overline{z}$ stands for the complex conjugate number.

For a given map $f:M\to M'$ and subset $\langle$~$N\subset M$ / $N'\subset M'$ such that $\operatorname{Im}(f)\subset N'$~$\rangle$ by $\langle$~$f|_N$ / $f|^{N'}$~$\rangle$ we denote the  $\langle$~restriction of $f$ onto $N$ / corestriction of $f$ onto $N'$~$\rangle$, that is $\langle$~$f|_N:N\to M':x\mapsto f(x)$ / $f|^{N'}:M\to N':x\mapsto f(x)$~$\rangle$. The indicator function of a subset $N$ of the set $M$ is denoted by $\chi_{N}$, so that $\chi_N(x)=1$ for $x\in N$ and $\chi_N(x)=0$ for $x\in M\setminus N$. We also use the shortcut $\delta_x=\chi_{\{x\}}$ where $x\in M$. By $\mathcal{P}(M)$ we denote the set of all subsets of $M$, and $\mathcal{P}_0(M)$ stands for the set of all finite subsets of $M$. The symbol $M^N$ stands for the set of all functions from $N$ to $M$. By $\operatorname{Card}(M)$ we denote the cardinality of $M$. By $\aleph_0$ we denote the cardinality of $\mathbb{N}$.

%----------------------------------------------------------------------------------------
%	Broad foundations
%----------------------------------------------------------------------------------------

\section{Broad foundations}

\label{SectionBroadFoundations} 

%----------------------------------------------------------------------------------------
%	Categorical language
%----------------------------------------------------------------------------------------

\subsection{Categorical language}
\label{SubSectionCategoricalLanguage}

Here we recall some basic facts and definitions from category theory and fix notation we shall use. We assume that our reader is familiar with such basics of category theory as category, functor, morphism. Otherwise see [\cite{HelLectAndExOnFuncAn}, chapter 0] for a quick introduction or [\cite{KashivShapCatsAndSheavs}, chapter 1] for more details.

For a given category $\mathbf{C}$ by $\operatorname{Ob}(\mathbf{C})$ we denote the class of its objects. The symbol $\mathbf{C}^o$ stands for the opposite category. For a given objects $X$ and $Y$ by $\operatorname{Hom}_{\mathbf{C}}(X, Y)$ we denote the set of morphisms from $X$ to $Y$. Often we shall write $\phi:X\to Y$ instead of $\phi\in\operatorname{Hom}_{\mathbf{C}}(X,Y)$. A morphism $\phi:X\to Y$ is called $\langle$~retraction / coretraction~$\rangle$ if it has a $\langle$~right / left~$\rangle$ inverse morphism. Morphism $\phi$ is called an isomorphism if it is retraction and coretraction. Usually we shall express existence of isomorphism between $X$ and $Y$ as $X\isom{\mathbf{C}} Y$. We say that two morphisms $\phi:X_1\to Y_1$ and $\psi:X_2\to Y_2$ are equivalent in $\mathbf{C}$ if there exist isomorphisms $\alpha:X_1\to X_2$ and $\beta:Y_1\to Y_2$ such that $\beta\phi=\psi\alpha$.

The first obvious example of the category that comes to mind is the category of all sets and all maps between them. We denote this category by $\mathbf{Set}$. Other examples will be given later. Two main examples of functors that any category has are functors of morphisms. For a fixed $X\in\operatorname{Ob}(\mathbf{C})$ we define covariant and a contravariant functors
$$
\operatorname{Hom}_{\mathbf{C}}(X,-):\mathbf{C}\to\mathbf{Set}:Y\mapsto \operatorname{Hom}_{\mathbf{C}}(X,Y), \phi\mapsto(\psi\mapsto \phi\circ\psi),
$$
$$
\operatorname{Hom}_{\mathbf{C}}(-,X):\mathbf{C}\to\mathbf{Set}:Y\mapsto \operatorname{Hom}_{\mathbf{C}}(Y,X), \phi\mapsto(\psi\mapsto \psi\circ\phi).
$$
This construction has its reminiscent analogs in many categories of mathematics with slight modification of categories between which these functors act.

We say that two covariant functors $F:\mathbf{C}\to\mathbf{D}$, $G:\mathbf{C}\to\mathbf{D}$ are isomorphic if there exists a class of isomorphisms $\{\eta_X:X\in\operatorname{Ob}(\mathbf{C})\}$ in $\mathbf{D}$ (called natural isomorphisms), such that $G(f)\eta_X=\eta_Y F(f)$ for all $f:X\to Y$. In this case we simply write $F\cong G$. A $\langle$~covariant / contravariant~$\rangle$ functor $F:\mathbf{C}\to\mathbf{D}$ is called representable by object $X$ if $\langle$~$F\cong\operatorname{Hom}_{\mathbf{C}}(X,-)$ / $F\cong\operatorname{Hom}_{\mathbf{C}}(-,X)$~$\rangle$. If functor is representable, then its representing object is unique up to isomorphism in $\mathbf{C}$.

Constructions of categorical product and coproduct shall play an important role in this thesis. We say that $X$ is the the $\langle$~product / coproduct~$\rangle$ of the family of objects $\{X_\lambda:\lambda\in\Lambda\}$ if the functor $\langle$~$\prod_{\lambda\in\Lambda}\operatorname{Hom}_{\mathbf{C}}(-,X_{\lambda}):\mathbf{C}\to\mathbf{Set}$ / $\prod_{\lambda\in\Lambda}\operatorname{Hom}_{\mathbf{C}}(X_{\lambda},-):\mathbf{C}\to\mathbf{Set}$~$\rangle$ is representable by object $X$. As the consequence we get that a $\langle$~product / coproduct~$\rangle$, if it exists, is unique up to an isomorphism. Later we shall give examples of the $\langle$~products / coproducts~$\rangle$ in different categories of functional analysis. 


%----------------------------------------------------------------------------------------
%	Topology
%----------------------------------------------------------------------------------------

\subsection{Topology}
\label{SubSectionTopology}

Let $(S,\tau)$ be a topological space. Elements of $\tau$ are called open sets, and their complements are called closed sets. Let $E$ be an arbitrary subset of $S$. By $\operatorname{cl}_S(E)$ we denote the closure of $E$ in $S$, that is the smallest closed set that contains $E$. Similarly, by $\operatorname{int}_S(E)$ we denote the interior of $E$, that is the largest open set that contained in $E$. We say that $E$ is a neighborhood of point $s\in S$ if $s\in \operatorname{int}_S(E)$. We say that a set $E$ is dense in a set $F$ if $F\subset\operatorname{cl}_S(E)$. Note that $E$ may be regarded as topological space, if we endow it with subspace topology which equals $\{U\cap E:U\in\tau\}$. Topological space is called Hausdorff if any two distinct points have disjoint open neighborhoods. In this thesis we shall work with Hausdorff spaces only.

A map $f:X\to Y$ between topological spaces is called continuous if preimage under $f$ of any open set is open. By $\mathbf{Top}$ we denote the category of topological spaces with continuous maps in the role of morphisms. Isomorphisms in $\mathbf{Top}$ are called homeomorphisms. The category $\mathbf{Top}$ admits  products. For a given family of topological spaces $\{S_\lambda:\lambda\in\Lambda\}$ their product is the Tychonoff product $\prod_{\lambda\in\Lambda}S_\lambda$, that is the Cartesian product of the family $\{S_\lambda:\lambda\in\Lambda\}$ with the coarsest topology making all natural projections $p_\lambda:\prod_{\lambda\in\Lambda}S_\lambda\to S_\lambda$ continuous.

In this thesis we shall encounter four types of topological spaces: compact spaces, paracompact spaces, locally compact spaces and extremely disconnected spaces. Before giving their definitions we need to recall the notion of cover. Let $\mathcal{E}$ be a family of subsets of topological space $S$. We say that $\mathcal{E}$ is a cover if its union equals $S$. We say that cover is open if all its elements are open sets. A cover is called locally finite if any point of $S$ has a neighborhood that intersects only finitely many elements of the cover. We say that cover $\mathcal{E}_1$ is inscribed into cover $\mathcal{E}_2$ if any element of $\mathcal{E}_1$ is a subset of some element of $\mathcal{E}_2$. A cover $\mathcal{E}_1$ is called a subcover of $\mathcal{E}_2$ if $\mathcal{E}_1\subset\mathcal{E}_2$. Finally, a topological space is called 

$i)$ compact if any its open cover admits a finite open subcover; 

$ii)$ paracompact if any its open cover is inscribed into some locally finite open cover; 

$iii)$ locally compact if any its point has a compact neighborhood;

$iv)$ extremely disconnected spaces if the closure of any its open set is open;

$v)$ Stonean if it is an extremely disconnected Hausdorff compact space.

The property of being $\langle$~compact / paracompact / locally compact~$\rangle$ space is preserved by $\langle$~closed / closed / open and closed~$\rangle$ subspaces. Any non compact locally compact Hausdorff space $S$ can be regarded as dense subspace of some compact Hausdorff space. There is the smallest and the largest such compactification. The smallest one is called the Alexandroff compactification $\alpha S$.  By definition $\alpha S:=S\cup \{S\}$. A subset of $\alpha S$ is called open if it is an open subset of $S$ or has the form $\{S\}\cup S\setminus K$ for some compact set $K\subset S$. The largest compactification $\beta S$ is called the Stone-Cech compactification. It may be represented as the image of the embedding $j:S\to\prod_{f\in C}[0,1]:s\mapsto \prod_{f\in C}f(s)$, where $C$ is a set of all continuous maps from $S$ to $[0,1]$. Stone-Cech compactification is highly non constructive. Even $\beta\mathbb{N}$ has no explicit description, though it is known that $\beta\mathbb{N}$ is an extremely disconnected Hausdorff compact.

Occasionally we shall apply the Urysohn's lemma to locally compact Hausdorff spaces. It states that for any compact subset $K$ of open set $V$ in a locally compact Hausdorff space $S$ there exists a continuous function $f:S\to [0,1]$ such that $f|_K=1$ and $f|_{S\setminus V}=0$. 

For more details on topological spaces see comprehensive treatise \cite{EngelGenTop}. 

%----------------------------------------------------------------------------------------
%	Filters, nets and limits
%----------------------------------------------------------------------------------------

\subsection{Filters, nets and limits}

\label{SubSectionFiltersNetsAndLimits} 

We will use two generalizations of the notion of the sequence and the limit of the sequence.

A family $\mathfrak{F}$ of subsets of the set $M$ is called a filter on a set $M$ if $\mathfrak{F}$ doesn't contain the empty set, $\mathfrak{F}$ is closed under finite intersections and $\mathfrak{F}$ contains all supersets of its elements. In general filters are too large to be described explicitly. To overcome this difficulty we shall use filterbases. A non empty family $\mathfrak{B}$ of subset of a set $M$ is called a filterbase on a set $M$ if $\mathfrak{B}$ doesn't contain empty set and the intersection of any two elements of $\mathfrak{B}$ contains some element of $\mathfrak{B}$. Given a filterbase $\mathfrak{B}$ we can construct a filter by adding to $\mathfrak{B}$ all supersets of elements of $\mathfrak{B}$.

We say that filter $\mathfrak{F}_1$ dominates filter $\mathfrak{F}_2$ if $\mathfrak{F}_2\subset\mathfrak{F}_1$. Therefore the set of all filters on a given set is partially ordered set. Filters that are maximal with respect to this order are called ultrafilters. An easy application of Zorn's lemma gives that any filter is dominated by some ultrafilter.

Let $\mathfrak{F}$ be a filter on  a set $M$, and $\phi:M\to S$ be a map from $M$ to the Hausdorff topological space $S$. We say that $x$ is a limit of $\phi$ along $\mathfrak{F}$ and write $x=\lim_{\mathfrak{F}} \phi(m)$ if for every open neighborhood $U$ of $x$ holds $\phi^{-1}(U)\in\mathfrak{F}$. Directly from the definition it follows that if $\phi$ has a limit along $\mathfrak{F}$ then it has the same limit along any filter that dominates $\mathfrak{F}$. 

Limit along filter preserve order structure of $\mathbb{R}$. More precisely: if two functions $\phi:M\to\mathbb{R}$ and $\psi:M\to\mathbb{R}$ have limits along filter $\mathfrak{F}$ and $\phi\leq\psi$, then 
$$
\lim_{\mathfrak{F}}\phi(m)\leq\lim_{\mathfrak{F}}\psi(m).
$$

Limit along filter respects continuous functions. Rigorously this formulates as follows. Assume for each $\lambda\in\Lambda$ a function $\phi_\lambda:M\to S_\lambda$ has a limit along filter $\mathfrak{F}$, then for any continuous function $g:\prod_{\lambda\in\Lambda}S_\lambda\to Y$ holds
$$
\lim_{\mathfrak{F}}g\left(\prod_{\lambda\in\Lambda}\phi_\lambda(m)\right)=g\left(\prod_{\lambda\in\Lambda}\lim_{\mathfrak{F}}\phi_\lambda(m)\right).
$$
In particular limit along filter is linear and multiplicative. Just like ordinary sequences.

The most important feature of filters and the reason of our interest is the following: if $\mathfrak{U}$ is an ultrafilter on the set $M$ and $\phi:M\to K$ is a function with values in the compact Hausdorff space $K$, then $\lim_{\mathfrak{U}}\phi(m)$ exists. In particular, we always can speak of limits along ultrafilters of bounded scalar valued functions.

Another approach to the generalization of the notion of the limit is a limit of the net. A directed set is a partially ordered set $(N,\leq)$ in which any two elements have upper bound. Every directed set gives rise to the so called section filter, whose filterbase consist of so called sections $\{\nu':\nu\leq\nu'\}$ for some $\nu\in N$.
Any function $x:N\to X$ from the directed set $(N,\leq)$ into the topological space $X$ is called a net. Usually it is denoted as $(x_\nu)_{\nu\in N}$ to allude to sequences. A limit of the net $x:N\to X$ is a limit of the function $x$ along section filter of the directed set $N$. It is denoted $\lim_\nu x_\nu$. We shall exploit both notions of the limit.

More on this matters can be found in [\cite{BourbElemMathGenTopLivIII}, section 7].


%----------------------------------------------------------------------------------------
%	Measure theory basics
%----------------------------------------------------------------------------------------

\subsection{Measure theory}
\label{SubSectionMeasureTheory}

A family $\Sigma$ of subsets of the set $\Omega$ is called a $\sigma$-algebra if it contains an empty set, contains complements of all its elements and closed under countable unions. If $\Sigma$ is a $\sigma$-algebra of subsets of $\Omega$ we call $(\Omega,\Sigma)$ a measurable subspace. Elements of $\Sigma$ are called measurable sets. 

A function $\mu:\Sigma\to[0,+\infty]$ such that:  

$i)$ $\mu(\varnothing)=0$; 

$ii)$ $\mu\left(\bigcup\limits_{n\in\mathbb{N}} E_n\right)=\sum\limits_{n\in\mathbb{N}}\mu(E_n)$ for any family of disjoint sets $(E_n)_{n\in\mathbb{N}}$ in $\Sigma$; 

is called a measure. The triple $(\Omega,\Sigma,\mu)$ is called a measure space. If $\mu$ attains only finite values we may drop the first condition. The second condition is essential and called the $\sigma$-additivity. The simplest example of measure space is an  arbitrary set $\Lambda$ with $\sigma$-algebra of all subsets and so called counting measure $\mu_c:\mathcal{P}(\Lambda)\to[0,+\infty]$. By definition $\mu_c(E)$ equals $\operatorname{Card}(E)$ if $E$ is finite and $+\infty$ otherwise. If $E$ is a measurable set, by $\Sigma|_E$ we denote the $\sigma$-algebra $\{F\cap E:F\in\Sigma\}$ and by $\mu|_E$ we denote the restriction of $\mu$ to $\Sigma|_E$. A set $E$ in $\Omega$ is called negligible if there exists a measurable set $F$ of measure $0$ that contains $E$. Similarly, a set $E$ in $\Omega$ is conegligible if $\Omega\setminus E$ is negligible. Let $P$ be some property that depends on points of $\Omega$. We say that $P$ holds almost everywhere if the set where $P$ is violated is negligible. A measure space is called $\sigma$-finite if there exists a countable family of measurable sets of finite measure whose union is the whole space. The class of $\sigma$-finite spaces is enough for most applications but we shall encounter a more generic measure spaces.

A measurable space $(\Omega,\Sigma,\mu)$ is called strictly localizable if there exists a  family of disjoint measurable subsets $\{E_\lambda:\lambda\in\Lambda\}$ of finite measure such that: 

$i)$ $\bigcup_{\lambda\in\Lambda}E_\lambda=\Omega$;

$ii)$ $E$ is measurable iff $E\cap E_\lambda$ is measurable for all $\lambda\in\Lambda$;

$iii)$ for any measurable $E$ holds $\mu(E)=\sum_{\lambda\in\Lambda}\mu(E\cap E_\lambda)$. 
  
  The class of strictly localizable measure spaces is huge. It includes all $\sigma$-finite measure spaces, their arbitrary unions, Haar measures of locally compact groups, counting measures and much more. In what follows we shall consider only strictly localizable measure spaces.

We shall exploit a more detailed classification of measure spaces. We say that a measurable set $E$ is an atom if $\mu(E)>0$ and for any measurable subset $F$ of $E$ either $F$ or $E\setminus F$ is negligible. Directly from the definition it follows that that all atoms of strictly localizable measure spaces are of finite measure. In general an atom may not be a mere singleton.

We say that a measure space is non atomic if there is no atoms for its measure. A measure space is called purely atomic if every measurable set of positive measure contains an atom. A straightforward application of Zorn's lemma gives that a purely atomic measure space can be represented as disjoint union of some family of atoms. This family is countable if measure space is $\sigma$-finite. These facts allow us to say that the structure of purely atomic measure space is well understood. 

The structure of strictly localizable non atomic measure spaces is given by Maharam's theorem [\cite{FremMeasTh}, 332B]. We shall exploit only the following property of non atomic measures [\cite{FremMeasTh}, proposition 215D]: if $E$ is a measurable set of positive measure in a non atomic measure space, then for all $0\leq c\leq \mu(E)$ there exists a measurable subset $F$ of $E$ such that $\mu(F)=c$.

For completeness we shall say a few words on constructions with measures. The product measure of two measure spaces $(\Omega_1,\Sigma_1,\mu_1)$ and $(\Omega_2,\Sigma_2,\mu_2)$ we denote by $\mu_1\times \mu_2$. The definition of product measure for localizable measure spaces is rather involved [\cite{FremMeasTh}, definition 251F] and we don't give it here.  For our purposes it is enough to know that the product of two strictly localizable measure spaces is again strictly localizable [\cite{FremMeasTh}, proposition 251N]. By direct sum of measure spaces $\{(\Omega_\lambda, \Sigma_\lambda, \mu_\lambda):\lambda\in\Lambda\}$ we denote disjoint union of set $\{\Omega_\lambda:\lambda\in\Lambda\}$ with $\sigma$-algebra defined as $\Sigma=\{E\subset \Omega: E\cap E_\lambda\in\Sigma_\lambda\mbox{ for all }\lambda\in\Lambda\}$ and measure given by the formula $\mu(E)=\sum_{\lambda\in\Lambda}\mu_\lambda(E\cap E_\lambda)$. It is clear now that strictly localizable measure space are exactly direct sums of finite measure spaces.

Assume $(\Omega,\Sigma,\mu)$ is a $\sigma$-finite measure space, then there exists a purely atomic measure $\mu_1:\Sigma\to[0,+\infty]$ and a non atomic measure $\mu_2:\Sigma\to[0,+\infty]$ such that $\mu=\mu_1+\mu_2$. Even more there exist measurable sets $\Omega_a^{\mu}$ and $\Omega_{na}^{\mu}=\Omega\setminus \Omega_a^{\mu}$ such that $\mu_1(\Omega_{na}^{\mu})=\mu_2(\Omega_a^{\mu})=0$. The sets $\Omega_a^{\mu}$ and $\Omega_{na}^{\mu}$ are called respectively the atomic and the non atomic parts of measure space $(\Omega,\Sigma,\mu)$.

By measurable function we always mean a complex or real valued function on measurable space, with the property that preimage of every open set is measurable. We say that two measurable functions are equivalent if the set where they are different is negligible. If $f:\Omega\to\mathbb{R}$ is an  integrable function on $(\Omega,\Sigma,\mu)$, then we may define a new measure 
$$
f\mu:\Sigma\to[0,+\infty]:E\mapsto\int_{E}f(\omega)d\mu(\omega).
$$

The notion of measure can be extended by changing the range of values that the measure can attain. Any $\sigma$-additive function $\mu:\Sigma\to\mathbb{C}$ on a measurable space $(\Omega,\Sigma)$ is called a complex measure. Any complex measure $\mu$ can be represented as $\mu=\mu_1-\mu_2+i(\mu_3-\mu_4)$, where $\mu_1,\mu_2,\mu_3,\mu_4$ --- are finite measures. As the consequence every complex measure is finite and therefore we have a well defined total variation measure:
$$
|\mu|:\Sigma\to[0,+\infty):E\mapsto\sup\left\{\sum_{n\in\mathbb{N}}|\mu(E_n)|:\{E_n:n\in\mathbb{N}\}\subset\Sigma -\mbox{partition of }E\right\}
$$

Let $\mu$ and $\nu$ be two measures on a measurable space $(\Omega,\Sigma)$. We say that $\mu$ and $\nu$ are mutually singular and write $\mu\perp\nu$ if there exists a measurable set $E$ such that $\mu(E)=\nu(\Omega\setminus E)=0$. The opposite property is absolute continuity. We say that $\nu$ is absolutely continuous with respect to $\mu$ and write $\nu\ll\mu$ if $\nu(E)=0$ for every measurable set $E$ with $\mu(E)=0$. In general, two measures may neither be absolutely continuous nor singular with respect to each other. We have a Lebesgue decomposition theorem for this case. For a given two $\sigma$-finite measures $\mu$ and $\nu$ on a measurable space $(\Omega,\Sigma)$ there exists a measurable function $\rho_{\nu,\mu}:\Omega\to\mathbb{C}$, a $\sigma$-finite measure $\nu_s:\Sigma\to[0,+\infty]$ and two measurable sets $\Omega_s^{\nu,\mu}$, $\Omega_c^{\nu,\mu}=\Omega\setminus\Omega_s^{\nu,\mu}$ such that
$\nu=\rho_{\nu,\mu}\mu+\nu_s$ and $\mu(\Omega_s^{\nu,\mu})=\nu_s(\Omega_c^{\nu,\mu})=0$, i.e. $\mu\perp\nu_s$.

Finally we shall say a few words on measures defined on topological spaces. Given a topological space $S$ we may consider the minimal $\sigma$-algebra containing all open subsets of $S$. It is called the Borel $\sigma$-algebra of $S$ and denoted by $Bor(S)$. Measures and complex measures defined on Borel $\sigma$-algebras are supported with adjective Borel. We shall not consider measures on general topological spaces and stick with locally compact Hausdorff spaces. This significantly simplifies further considerations. We say that a complex Borel measure $\mu:Bor(S)\to\mathbb{C}$ defined on a locally compact Hausdorff space $S$ is regular if for any Borel set $E$ and any $\epsilon>0$ there exists a compact set $K\subset E$ such that $|\mu|(E\setminus K)<\epsilon$. The support of complex Borel measure $\mu$ is a set of all points $s\in S$ for which every open neighborhood of $s$ has positive measure. We denote the set of such points by $\operatorname{supp}(\mu)$. The support is always closed. 

Most of the results and definitions in this section can be found in the first, second and the fourth volumes of \cite{FremMeasTh}.

%----------------------------------------------------------------------------------------
%	Banach spaces, algebras and modules
%----------------------------------------------------------------------------------------

\section{Banach structures}

\label{SectionBanachStructures}

%----------------------------------------------------------------------------------------
%	Banach spaces
%----------------------------------------------------------------------------------------

\subsection{Banach spaces}
\label{SubSectionBanachSpaces}

We assume that our reader is familiar with fundamentals of functional analysis and its constructions, otherwise consult \cite{HelLectAndExOnFuncAn} or \cite{ConwACoursInFuncAn}. In this thesis we will highly rely on results about geometry of Banach spaces. See \cite{CarothShortCourseBanSp}, \cite{KalAlbTopicsBanSpTh} or \cite{FabHabBanSpTh} for a quick introduction. All Banach spaces are considered over complex field, unless otherwise stated. 

By $\langle$~$B_E$ / $B_E^\circ$~$\rangle$ we denote the $\langle$~closed / open~$\rangle$ unit ball of Banach space $E$ with center at zero. If $F$ is a closed subspace of $E$, then $E/F$ stands for the quotient Banach space. By $E^{cc}$ we denote a Banach space with the same set of vectors as in $E$, the same addition but with new multiplication by conjugate scalars:  $\alpha \overline{x}:=\overline{\overline{\alpha}x}$ for $\alpha\in\mathbb{C}$ and $x\in E$. Note: elements of $E^{cc}$ we denote by $\overline{x}$. Clearly, $(E^{cc})^{cc}=E$.

Now fix two Banach spaces $E$ and $F$. A map $T:E\to F$ is called conjugate linear if the respective map $T:E^{cc}\to F$ is linear. A linear operator $T:E\to F$ is called:

$i)$ bounded if its norm $\Vert T\Vert:=\sup\{\Vert T(x)\Vert:x\in B_E\}$ is finite.

$ii)$ contractive if its norm is at most $1$;

$iii)$ compact if $T(B_E)$ is relatively compact in $F$;

$iv)$ nuclear if it can be represented as an absolutely convergent series of rank one operators.

Is well known that any nuclear operator is compact. any compact operator is bounded, any bounded operator is continuous. By $\langle$~$\mathcal{B}(E,F)$ / $\mathcal{K}(E,F)$ / $\mathcal{N}(E,F)$~$\rangle$ we denote the Banach space of $\langle$~bounded / compact / nuclear~$\rangle$ linear operators from $E$ to $F$. If $F=E$ we use the shortcut $\langle$~$\mathcal{B}(E)$ / $\mathcal{K}(E)$ / $\mathcal{N}(E)$~$\rangle$ for this space. The norms in $\mathcal{B}(E,F)$ and $\mathcal{K}(E,F)$ are just the usual operator norm. The norm of a nuclear operator $T$ is defined by equality
$$
\Vert T\Vert:=\inf\left\{\sum_{n=1}^\infty\Vert S_n\Vert: T=\sum_{n=1}^\infty S_n,\quad (S_n)_{n\in\mathbb{N}} - \mbox{ rank one operators}\right\}.
$$

By $\mathbf{Ban}$ we shall denote the category of Banach spaces with bounded linear operators in the role of morphisms, while $\mathbf{Ban}_1$ stands for the category of Banach spaces with contractive operators in the role of morphisms. As the consequence, $\operatorname{Hom}_{\mathbf{Ban}}(E,F)$ is just another name for $\mathcal{B}(E,F)$.

Let $E$, $F$ and $G$ be three Banach spaces, then a bilinear operator $\Phi:E\times F\to G$ is called bounded if its norm $\Vert \Phi\Vert:=\sup\{\Vert \Phi(x,y)\Vert:x\in B_E, y\in B_F\}$ is finite. The Banach space of all bounded bilinear operators on $E\times F$ with values in $G$ is denoted by $\mathcal{B}(E\times F,G)$.

A few words on classification of bounded linear operators. A bounded linear operator $T:E\to F$ is called:

$i)$ topologically injective if it performs homeomorphism on its image;

$ii)$ topologically surjective if it is an open map;

$iii)$ coisometric if it maps open unit ball onto open unit ball;

$iv)$ strictly coisometric if it maps closed unit ball onto closed unit ball. 

But we shall refine these definitions. We say a bounded linear operator $T:E\to F$ is:

$v)$ $c$-topologically injective, if $\Vert x\Vert\leq c\Vert  T(x)\Vert$ for all $x\in E$;

$vi)$ $c$-topologically surjective, if $cT(B_E^\circ)\supset B_F^\circ$;

$vii)$ strictly $c$-topologically surjective, if $cT(B_E)\supset B_F$; 

Note that $T$ is topologically $\langle$~injective / surjective~$\rangle$ iff it is $c$-topologically $\langle$~injective / surjective~$\rangle$ for some $c>0$. Obviously $\langle$~coisometric / strictly coisometric~$\rangle$ operators are exactly contractive $\langle$~$1$-topologically surjective / strictly $1$-topologically surjective~$\rangle$ operators.

Two Banach spaces $E$ and $F$ are $\langle$~isometrically isomorphic / topologically isomorphic~$\rangle$ as Banach spaces if there exists a bounded linear operator $T:E\to F$ which is both $\langle$~isometric and surjective / topologically injective and topologically surjective~$\rangle$. The fact that $E$ and $F$ are $\langle$~isometrically isomorphic / topologically isomorphic~$\rangle$ Banach spaces means that $\langle$~$E\isom{\mathbf{Ban}_1}F$ / $E\isom{\mathbf{Ban}}F$~$\rangle$. The Banach-Mazur distance between $E$ and $F$ is defined by the formula 
$$
d_{BM}(E,F):=\inf\{\Vert T\Vert\Vert T^{-1}\Vert: T \in \mathcal{B}(E,F) \mbox{ --- a topological isomorphism}\}.
$$ 
If $E$ and $F$ are not topologically isomorphic the Banach-Mazur distance between them is infinite.

One more important class of operators is the class of bounded projections. A bounded linear operator $P:E\to E$ is called a projection if $P^2=P$. If $F=P(E)$, then we say that $P$ is a projection from $E$ onto $F$ and $F$ is complemented in $E$. If $\Vert P\Vert\leq c$ we say that $F$ is $c$-complemented in $E$. Finally, we say that $F$ is contractively complemented in $E$ if it is $1$-complemented in $E$. Another equivalent characterization says that $F$ is a complemented subspace of Banach space $E$ if there exists a closed subspace $G$ in $E$ such that $E\isom{\mathbf{Ban}}F\bigoplus G$. All finite dimensional subspaces are complemented, but not necessarily contractively complemented. An example of contractively complemented subspace is the following: consider arbitrary Banach space $E$, then $E^*$ is contractively complemented in $E^{***}$ via Dixmier projection $P=\iota_{E^*}(\iota_E)^*$, where $\iota_E$ is the natural embedding of $E$ into its second dual $E^{**}$. A canonical example of uncomplemented subspace is $c_0(\mathbb{N})$ in $\mathbb{\ell_\infty}(\mathbb{N})$  [\cite{KalAlbTopicsBanSpTh}, theorem 2.5.5].

Now consider the algebraic tensor product $E\otimes F$ of Banach spaces $E$ and $F$. This linear space can be endowed with different norms, but the most important is the projective norm. For $u\in E\otimes F$ we define its projective norm as
$$
\Vert u\Vert:=\inf\left\{\sum_{i=1}^n \Vert x_i\Vert\Vert y_i\Vert: u=\sum_{i=1}^n x_i\otimes y_i, (x_i)_{i\in\mathbb{N}_n}\subset E, (y_i)_{i\in\mathbb{N}_n}\subset F\right\}
$$
It is indeed a norm, but not complete in general. The symbol $E\projtens F$ stands for the completion of $E\otimes F$ under projective norm. We call the resulting completion the projective tensor product of Banach spaces $E$ and $F$. Let $T:E_1\to E_2$ and $S:F_1\to F_2$ be two bounded linear operators between Banach spaces, then there exists a unique bounded linear operator $T\projtens S:E_1\projtens F_1\to E_2\projtens F_2$ such that $(T\projtens S)(x\projtens y)=T(x)\projtens S(y)$ for all $x\in E_1$ and $y\in F_1$. Even more $\Vert T\projtens S\Vert=\Vert T\Vert\Vert S\Vert$. The main feature of projective tensor product which makes it so important is the following universal property: for any Banach spaces $E$, $F$ and $G$ there is a natural isometric isomorphism:
$$
\mathcal{B}(E\projtens F,G)\isom{\mathbf{Ban}_1}\mathcal{B}(E\times F,G)
$$
In other words, projective tensor product linearizes bounded bilinear operators. Also we have the following two (natural in $E$, $F$ and $G$) isometric isomorphisms:
$$
\mathcal{B}(E\projtens F,G)
\isom{\mathbf{Ban}_1}\mathcal{B}(E,\mathcal{B}(F,G))
\isom{\mathbf{Ban}_1}\mathcal{B}(F,\mathcal{B}(E,G))
$$
The last isomorphism is called the law of adjoint associativity. There are many other tensor norms on the algebraic tensor product of Banach spaces. Their thorough treatment can be found in \cite{DiestMetTheoryOfTensProd}.

Now we are able to craft four very important functors:
$$
\mathcal{B}(-,E):\mathbf{Ban}\to\mathbf{Ban}
\qquad\qquad
\mathcal{B}(E,-):\mathbf{Ban}\to\mathbf{Ban}
$$
$$
-\projtens E:\mathbf{Ban}\to\mathbf{Ban}
\qquad\qquad
E\projtens -:\mathbf{Ban}\to\mathbf{Ban}.
$$
We shall often encounter them. For example, the well known adjoint functor ${}^*$ is nothing more than $\mathcal{B}(-,\mathbb{C})$. All these functors have their obvious analogs on $\mathbf{Ban}_1$.

Now we  proceed to classical examples of Banach spaces. 

An important source of examples of Banach spaces are $L_p$-spaces, also known as Lebesgue spaces. A detailed discussion of basic properties of $L_p$-spaces can be found in \cite{CarothShortCourseBanSp}.  Let $(\Omega,\Sigma,\mu)$ be a measure space.  For $1\leq p<\infty$, as usually, the symbol $L_p(\Omega,\mu)$ stands for the Banach space of equivalence classes of functions $f:\Omega\to\mathbb{C}$ such that $|f|^p$ is Lebesgue integrable with respect to measure $\mu$. The norm of such function is defined as
$$
\Vert f\Vert:=\left(\int\limits_{\Omega}|f(\omega)|^pd\mu(\omega)\right)^{1/p}.
$$ 
By $L_\infty(\Omega,\mu)$ we denote the Banach space of equivalence classes of bounded measurable functions with norm defined as 
$$
\Vert f\Vert:=\inf\left\{\sup_{\omega\in\Omega\setminus N}|f(\omega)|:N\subset\Omega - \mbox{is negligible}\right\}.
$$
For simplicity we shall speak of functions in $L_p(\Omega,\mu)$ instead of their equivalence classes. All equalities and inequalities about functions of $L_p$-spaces are understood up to negligible sets. It is well know that $L_p(\Omega,\mu)^*\isom{\mathbf{Ban}_1}L_{p^*}(\Omega,\mu)$ for $1\leq p<+\infty$ [\cite{FremMeasTh}, theorems 243G, 244K]. One more well known fact is that, $L_p$-spaces are reflexive for $1<p<+\infty$. Here we exploited the standard notation $p^*=+\infty$ if $p=1$ and $p^*=p/(p-1)$ if $1<p<+\infty$. Clearly, $p^{**}=p$ for $1<p<+\infty$.

The most well known classes of Banach spaces are related to continuous functions. Let $S$ be a locally compact Hausdorff space. We say that a function $f:S\to\mathbb{C}$ vanishes at infinity  if for any $\epsilon>0$ there exists a compact $K\subset S$ such that $|f(s)|\leq\epsilon$ for all $s\in S\setminus K$. The linear space of continuous functions on $S$ vanishing at infinity is denoted by $C_0(S)$. When endowed with $\sup$-norm $C_0(S)$ becomes a Banach space. Any set $\Lambda$ with discrete topology may be regarded as a locally compact space and following the traditional notation we shall write $c_0(\Lambda)$ instead of $C_0(\Lambda)$. If $K$ is a compact Hausdorff space then all functions on $K$ vanish at infinity. We use the notation $C(K)$ for $C_0(K)$ to indicate that $K$ is compact. Some Banach spaces in fact are $C(K)$-spaces in disguise. For example, if we are given a measure space $(\Omega,\Sigma,\mu)$, then $B(\Omega,\Sigma)$ --- the space of bounded measurable functions with $\sup$-norm or $L_\infty(\Omega,\mu)$ are $C(K)$ spaces for some compact Hausdorff space $K$ [\cite{KalAlbTopicsBanSpTh}, remark 4.2.8]. If $S$ is a locally compact Hausdorff space, then $M(S)$ stands for the Banach space of complex finite Borel regular measures on $S$. The norm of measure $\mu\in M(S)$ is defined by equality $\Vert\mu\Vert=|\mu|(S)$, where $|\mu|$ is a total variation measure of measure $\mu$. By Riesz-Markov-Kakutani theorem  [\cite{ConwACoursInFuncAn}, section C.18] we have $C_0(S)^*\isom{\mathbf{Ban}_1}M(S)$. In fact $M(S)$ is an $L_1$-space, see discussion after [\cite{DalLauSecondDualOfMeasAlg}, proposition 2.14]. 

We shall also mention one important specific case of $L_p$-spaces. For a given index set $\Lambda$ and a counting measure $\mu_c:\mathcal{P}(\Lambda)\to[0,+\infty]$ the respective $L_p$-space is denoted by $\ell_p(\Lambda)$. For this type of measure spaces we have one more important isomorphism $c_0(\Lambda)^*\isom{\mathbf{Ban}_1}\ell_1(\Lambda)$. For convenience we define $c_0(\varnothing)=\ell_p(\varnothing)=\{0\}$ for $1\leq p\leq+\infty$. This example motivates the following construction.

Let $\{E_\lambda:\lambda\in\Lambda\}$ be an arbitrary family of Banach spaces. For each $x\in \prod_{\lambda\in\Lambda} E_\lambda$ we define
$\Vert x\Vert_p=\Vert(\Vert x_\lambda\Vert)_{\lambda\in\Lambda}\Vert_{\ell_p(\Lambda)}$ for $1\leq p\leq +\infty$ and $\Vert x\Vert_0=\Vert(\Vert x_\lambda\Vert)_{\lambda\in\Lambda}\Vert_{c_0(\Lambda)}$. Then the Banach space $\left\{x\in \prod_{\lambda\in\Lambda} E_\lambda: \Vert x\Vert_p<+\infty\right\}$ with the norm $\Vert\cdot\Vert_p$ is denoted by $\bigoplus_p\{E_\lambda:\lambda\in\Lambda\}$. We call these objects $\bigoplus_p$-sums of Banach spaces $\{E_\lambda:\lambda\in\Lambda\}$. It is almost tautological that the Banach space $\ell_p(\Lambda)$ is the $\bigoplus_p$-sum of the family $\{\mathbb{C}:\lambda\in\Lambda\}$. A nice property of $\bigoplus_p$-sums is their interrelation with duality:
$$
\left(\bigoplus\nolimits_p\{E_\lambda:\lambda\in\Lambda\}\right)^*\isom{\mathbf{Ban}_1}
\bigoplus\nolimits_{p^*}\{E_\lambda^*:\lambda\in\Lambda\}
$$
for all $1\leq p<+\infty$ and 
$$
\left(\bigoplus\nolimits_0\{E_\lambda:\lambda\in\Lambda\}\right)^*\isom{\mathbf{Ban}_1}
\bigoplus\nolimits_1\{E_\lambda^*:\lambda\in\Lambda\}
$$
If $\{T_\lambda\in\mathcal{B}(E_\lambda, F_\lambda):\lambda\in\Lambda\}$ is a family of bounded linear operators, then for all $1\leq p\leq+\infty$ and $p=0$ we have a well defined linear operator
$$
T:\bigoplus\nolimits_p\{E_\lambda:\lambda\in\Lambda\}\to \bigoplus\nolimits_p\{ F_\lambda:\lambda\in\Lambda\}:x\mapsto \bigoplus\nolimits_p\{ T_\lambda(x_\lambda):\lambda\in\Lambda\}
$$
which we shall denote by $\bigoplus_p\{T_\lambda:\lambda\in\Lambda\}$. Its norm equals $\sup_{\lambda\in\Lambda}\Vert T_\lambda\Vert$.

Among different $\bigoplus_p$-sums the $\langle$~$\bigoplus_1$-sums / $\bigoplus_\infty$-sums~$\rangle$ play a special role in Banach space theory. The reason is that any family of Banach spaces admit $\langle$~product / coproduct~$\rangle$ in $\mathbf{Ban}_1$ which in fact is their $\langle$~$\bigoplus_1$-sum / $\bigoplus_\infty$-sum~$\rangle$. The same statement holds for $\mathbf{Ban}$ if we restrict ourselves to finite families of objects [\cite{HelLectAndExOnFuncAn}, chapter 2, section 5].

We proceed to advanced topics of Banach space theory. Below we shall discuss several geometric properties of Banach spaces such as the property of being an $\mathscr{L}_p^g$-space, weak sequential completeness, the Dunford-Pettis property, the l.u.st. property and the approximation property. In what follows, imitating Banach space geometers, we shall say that a Banach space $E$ contains $\langle$~an isometric copy / a copy~$\rangle$ of Banach space $F$ if $F$ is $\langle$~isometrically isomorphic / topologically isomorphic~$\rangle$ to some closed subspace of $E$.

Let $1\leq p\leq +\infty$. We say that $E$ is an $\mathscr{L}_{p,C}^g$-space if for any $\epsilon>0$ and any finite dimensional subspace $F$ of $E$ there exists a finite dimensional $\ell_p$-space $G$ and two bounded linear operators $S:F\to G$, $T:G\to E$ such that $TS|^F=1_F$ and $\Vert T\Vert\Vert S\Vert\leq C+\epsilon$. If $E$ is an $\mathscr{L}_{p,C}^g$-space for some $C\geq 1$ we simply say, that $E$ is an $\mathscr{L}_p^g$-space. This definition [\cite{DefFloTensNorOpId}, definition 23.1] is an improvement of the definition of $\mathscr{L}_p$-spaces given by Lindenstrauss and Pelczynski in their pioneering work \cite{LinPelAbsSumOpInLpSpAndApp}. Clearly, any finite dimensional Banach space is an $\mathscr{L}_p^g$-space for all $1\leq p\leq +\infty$. Any $L_p$-space is an $\mathscr{L}_{p,1}^g$-space  [\cite{DefFloTensNorOpId}, exercise 4.7], but the converse is not true. Any $c$-complemented subspace of $\mathscr{L}_{p,C}^g$-space is an $\mathscr{L}_{p,cC}^g$-space [\cite{DefFloTensNorOpId}, corollary 23.2.1(2)]. A Banach space is an $\mathscr{L}_{p,C}^g$-space iff its dual is an $\mathscr{L}_{p^*,C}^g$-space [\cite{DefFloTensNorOpId}, corollary 23.2.1(1)]. All $C(K)$-spaces are $\mathscr{L}_{\infty, 1}^g$-spaces [\cite{DefFloTensNorOpId}, lemma 4.4]. Note that, for a given locally compact Hausdorff space $S$ the Banach space $C_0(S)$ is complemented in $C(\alpha S)$. Therefore $C_0(S)$-spaces are $\mathscr{L}_\infty^g$-spaces too. We will mainly concern in $\mathscr{L}_1^g$- and $\mathscr{L}_\infty^g$-spaces.

We say that a Banach space $E$ is weakly sequentially complete if for any sequence $(x_n)_{n\in\mathbb{N}}\subset E$ such that $(f(x_n))_{n\in\mathbb{N}}\subset\mathbb{C}$ is a Cauchy sequence for any $f\in E^*$ there exists a vector $x\in E$ such that $\lim_n f(x_n)=f(x)$ for all $f\in E^*$. That is any weakly Cauchy sequence converges in the weak topology. A typical example of weakly sequentially complete Banach space is any $L_1$-space [\cite{WojBanSpForAnalysts}, corollary III.C.14]. This property is preserved by closed subspaces. A typical example of Banach space that is not weakly sequentially complete is $c_0(\mathbb{N})$, just consider the sequence $(\sum_{k=1}^n \delta_k)_{n\in\mathbb{N}}$.

Now we proceed to the discussion of the Dunford-Pettis property. A bounded linear operator $T:E\to F$ is called weakly compact if it maps the unit ball of $E$ into a relatively weakly compact subset of $F$. A bounded linear operator is called completely continuous if the image of any weakly compact subset of $E$ is norm compact in $F$. A Banach space $E$ is said to have the Dunford-Pettis property if any weakly compact operator from $E$ to any Banach space $F$ is completely continuous. There is a simple internal characterization [\cite{KalAlbTopicsBanSpTh}, theorem 5.4.4]: a Banach space $E$ has the Dunford-Pettis property if $\lim_n f_n(x_n)=0$ for all sequences $(x_n)_{n\in\mathbb{N}}\subset E$ and $(f_n)_{n\in\mathbb{N}}\subset E^*$, that both weakly converge to $0$. Now it is easy to deduce, that if a Banach space $E^*$ has the Dunford-Pettis property, then so does $E$. In his seminal work \cite{GrothApllFaiblCompSpCK} Grothendieck showed that all $L_1$-spaces and $C(K)$-spaces have this property. The Dunford-Pettis property passes to complemented subspaces [\cite{FabHabBanSpTh}, proposition 13.44]. This property behaves badly with reflexive spaces: since the unit ball of a reflexive space is weakly compact [\cite{MeggIntroBanSpTh}, theorem 2.8.2], then reflexive Banach space with the Dunford-Pettis property has norm compact unit ball and therefore this space is finite dimensional. 

To introduce the next Banach geometric property we need definitions of Banach lattice and unconditional Schauder basis. 

A real Riesz space $E$ is a vector space over $\mathbb{R}$ with the structure of partially ordered set such that $x\leq y$ implies $x+z\leq y+z$ for every $x,y,z\in E$ and $ax\geq 0$ for every $x\geq 0$, $a\in\mathbb{R}_+$. A partially ordered set is a lattice if any two elements ${x,y}$ have the least upper bound $x\vee y$ and the greatest lower bound $x\wedge y$. A real vector lattice is real Riesz space which is lattice as partially ordered set. If $E$ is a real vector lattice, then for every $x\in E$ we define its absolute value by equality $|x|:=x\vee(-x)$. A complex vector lattice $E$ is a vector space over $\mathbb{C}$ such that there exists a real vector subspace $\operatorname{Re}(E)$ which is real vector lattice and

$i)$ for any $x\in E$ there are unique $\operatorname{Re}(x),\operatorname{Im}(x)\in \operatorname{Re}(E)$ such that $x=\operatorname{Re}(x)+i\operatorname{Im}(x)$;

$ii)$ for any $x\in E$ there exist an absolute value $|x|:=\sup\{\operatorname{Re}(e^{i\theta}x):\theta\in\mathbb{R}\}$.

A Banach lattice is a Banach space with the structure of the complex vector lattice such that $\Vert x\Vert\leq \Vert y\Vert$ whenever $|x|\leq |y|$. A classical example of Banach lattice $E$ is an $L_p$-space or a $C(K)$-space. In both cases $\operatorname{Re}(E)$ consist of real valued functions in $E$. If $\{E_\lambda:\lambda\in\Lambda\}$ is a family of Banach lattices then for any $1\leq p\leq +\infty$ or $p=0$ their $\bigoplus_p$-sum is a Banach lattice with lattice operation defined as $x\leq y$ if $x_\lambda\leq y_\lambda$ for all $\lambda\in\Lambda$, where $x,y\in\bigoplus_p\{ E_\lambda:\lambda\in\Lambda\}$. The dual space $E^*$ of a Banach lattice $E$ is again a Banach lattice with lattice operation defined by $f\leq g$ if $f(x)\leq g(x)$ for all $x\geq 0$, where $f,g\in  E^*$. A very nice account of Banach lattices can be found in [\cite{LaceyIsomThOfClassicBanSp}, section 1].

The property of being a Banach lattice puts some restrictions on the geometry of the space \cite{SherOrderInOpAlg}, \cite{KadOrderPropOfBoundSAOps}. To explain the Banach geometric reason of this phenomena we need the definition of an unconditional Schauder basis. Let $E$ be a Banach space. A collection of functionals $(f_\lambda)_{\lambda\in\Lambda}$ in $E^*$ is called a biorthogonal system for vectors $(x_\lambda)_{\lambda\in\Lambda}$  from $E$ if $f_\lambda(x_{\lambda'})=1$ whenever $\lambda=\lambda'$ and $0$ otherwise. A collection $(x_\lambda)_{\lambda\in\Lambda}$ in $E$ is called an unconditional Schauder basis if there exists a biorthogonal system $(f_\lambda)_{\lambda\in\Lambda}$ in $E^*$ for it such that
the series $\sum_{\lambda\in\Lambda} f_\lambda(x)x_\lambda$ unconditionally converges to $x$ for any $x\in E$. All $\ell_p$-spaces with $1\leq p<+\infty$ have an unconditional Schauder basis, for example, it is $(\delta_\lambda)_{\lambda\in\Lambda}$. A typical example of space without unconditional basis is $C([0,1])$. Even more this Banach space can not even be a subspace of the space with unconditional basis [\cite{KalAlbTopicsBanSpTh}, proposition 3.5.4].  Any unconditional Schauder basis $(x_\lambda)_{\lambda\in\Lambda}$ in $E$ satisfy the following property  [\cite{DiestAbsSumOps}, proposition 1.6]: there exists a constant $\kappa\geq 1$ such that
$$
\left\Vert \sum_{\lambda\in\Lambda}t_\lambda f_\lambda(x)x_\lambda\right\Vert
\leq
\kappa\left\Vert \sum_{\lambda\in\Lambda}f_\lambda(x)x_\lambda\right\Vert
$$
for all $x\in E$ and $t\in\ell_\infty(\Lambda)$. The least such constant $\kappa$ among all unconditional Schauder bases of $E$ is denoted by $\kappa(E)$. Similar constant could be defined for Banach spaces without unconditional Schauder bases. The local unconditional constant $\kappa_u(E)$ of Banach space $E$ is defined to be the infimum of all scalars $c$ with the following property: given any finite dimensional subspace $F$ of $E$ there exists a Banach space $G$ with unconditional Schauder basis and two bounded linear operators $S:F\to G$, $T:G\to E$ such that $TS|^{F}=1_F$ and $\Vert T\Vert\Vert S\Vert\kappa(G)\leq c$. We say that a Banach space $E$ has the local unconditional structure property (the l.u.st. property for short) if $\kappa_u(E)$ is finite. Clearly any Banach space $E$ with unconditional Schauder basis has the l.u.st. property with $\kappa_u(E)=\kappa(E)$. In particular, all finite dimensional Banach spaces have the l.u.st. property. Though a general Banach lattice $E$ may not have an unconditional Schauder basis it is still has the l.u.st. property with $\kappa_u(E)=1$  [\cite{DiestAbsSumOps}, theorem 17.1]. Directly from the definition it follows that the l.u.st. property is preserved by complemented subspaces. More precisely: if $F$ is a $c$-complemented subspace of $E$, then $\kappa_u(F)\leq c\kappa_u(E)$. Therefore all complemented subspaces of Banach lattices have the l.u.st. property. This sufficient condition is not far from criterion [\cite{DiestAbsSumOps}, theorem 17.5]: a Banach space $E$ has the l.u.st. property iff $E^{**}$ is topologically isomorphic to a complemented subspace of some Banach lattice. As the corollary of this criterion we get that $E$ has the l.u.st. property iff so does $E^*$ [\cite{DiestAbsSumOps}, corollary 17.6].

The last property we shall discuss is a well known approximation property introduced by Grothendieck in \cite{GrothProdTenTopNucl}. We say that a Banach space $E$ has the approximation property if for any compact set $K\subset E$ and any $\epsilon>0$ there exists a finite rank operator $T:E\to E$ such that $\Vert T(x)-x\Vert<\epsilon$ for all $x\in K$. If $T$ can be chosen with $\Vert T\Vert\leq c$, then $E$ is said to have the $c$-bounded approximation property. The metric approximation property is another name for $1$-bounded approximation property. We say that $E$ has the bounded approximation property if $E$ has the $c$-bounded approximation property for some $c\geq 1$. None of these properties are preserved by subspaces or quotient spaces, but the approximation property and the bounded approximation properties are inherited by complemented subspaces [\cite{DefFloTensNorOpId}, exercise 5.5]. All $L_p$-spaces and $C(K)$-spaces have the metric approximation property [\cite{DefFloTensNorOpId}, section 5.2(3)], but their subspaces may fail the approximation property [\cite{DefFloTensNorOpId}, section 5.2(1)]. Any Banach space with unconditional Schauder basis has the approximation property [\cite{RyanIntroTensNormsBanSp}, example 4.4]. If $E^*$ has the approximation property, then so does $E$ [\cite{DefFloTensNorOpId}, corollary 5.7.2]. The reason why approximation property is so important is rather simple --- it has a lot of equivalent reformulations that involve many nice properties of Banach spaces. For example, the following properties of Banach space $E$ are equivalent [\cite{DefFloTensNorOpId}, sections 5.3, 5.6]:

$i)$ $E$ has the approximation property;

$ii)$ the natural mapping $Gr:E^*\projtens E\to\mathcal{N}(E)$ is an isometric isomorphism;

$iii)$ for any Banach space $F$ every compact operator $T:F\to E$ can be approximated in the operator norm by finite rank operators.

There is much more to list, but we confine ourselves with these three properties.

%----------------------------------------------------------------------------------------
%	Banach algebras and their modules
%----------------------------------------------------------------------------------------

\subsection{Banach algebras and their modules}
\label{SubSectionBanachAlgebrasAndTheirModules}

A thorough treatment of Banach algebras and Banach modules can be found in \cite{HelBanLocConvAlg} or \cite{HelHomolBanTopAlg} or \cite{DalBanAlgAutCont}. We shall describe only the bare minimum required for us.

A Banach algebra $A$ is an associative algebra over $\mathbb{C}$ which is a Banach space and the multiplication bilinear operator $\cdot:A\times A\to A:(a,b)\mapsto ab$ is of the norm at most $1$. A typical example of commutative Banach algebra is the algebra of continuous functions on a compact Hausdorff space with pointwise multiplication. A typical non commutative example is the algebra of bounded linear operators on the Hilbert space with composition in the role of multiplication. Both examples belong to a very important class of $C^*$-algebras to be discussed below. By $\langle$~left / right / two-sided~$\rangle$ ideal $I$ of a Banach algebra $A$ we always mean a closed subalgebra of $A$ such that $\langle$~$ax$ / $xa$ / $ax$ and $xa$~$\rangle$ belong to $I$ for all $a\in A$ and $x\in I$.

We say that an element $p$ of a Banach algebra $A$ is a $\langle$~left / right~$\rangle$ identity of $A$ if $\langle$~$pa=a$ / $ap=a$~$\rangle$ for all $a\in A$. The element which is both left and right identity is called the identity and denoted by $e_A$. In general we do not assume that Banach algebras are unital, i.e. has an identity. Even if a Banach algebra $A$ is unital we do not require its identity to be of norm $1$. We use notation $A_+=A\bigoplus_1\mathbb{C}$ for the standard unitization of Banach algebras. The multiplication in $A_+$ is defined as $(a\oplus_1 z)(b\oplus_1 w)=(ab+wa+zb)\oplus_1 zw$, for $a,b\in A$ and $z,w\in\mathbb{C}$. Clearly $(0,1)$ is the identity of $A_+$. By $A_\times$ we denote the conditional unitization of $A$, i.e. $A_\times=A$ if $A$ has identity of norm one and $A_\times=A_+$ otherwise. Even in the absence of identity in case of Banach algebras there are good substitutes for it which are called approximate identities. We say that a net $(e_\nu)_{\nu\in N}$ in $A$ is a $\langle$~left / right / two-sided~$\rangle$ approximate identity if $\langle$~$\lim_\nu e_\nu a=a$ / $\lim_\nu ae_\nu=a$ / $\lim_\nu e_\nu a=\lim_\nu ae_\nu=a$~$\rangle$ for all $a\in A$. In all these three definitions convergence of nets is understood in the norm topology. If we will consider weak topology, we shall get definitions of left, right and two-sided weak approximate identities. We say that an approximate identity $(e_\nu)_{\nu\in N}$ is $c$-bounded if $\sup_\nu\Vert e_\nu\Vert\leq c$. An approximate identity $(e_\nu)_{\nu\in N}$ is called $\langle$~bounded / contractive~$\rangle$ if it is $\langle$~$1$-bounded / $c$-bounded for some $c\geq 1$~$\rangle$. Occasionally we will use the following simple fact: if $A$ is a Banach algebra with $\langle$~left / right~$\rangle$ identity $p$ and $\langle$~right / left~$\rangle$ approximate identity $(e_\nu)_{\nu\in N}$, then $A$ is unital with identity $p$ of norm $\lim_\nu\Vert e_\nu\Vert$. 

If $A$ is a unital Banach algebra we define the spectrum $\operatorname{sp}_A(a)$ of element $a$ in $A$ as the set of all complex numbers $z$ such that $a-ze_A$ is not invertible in $A$. For Banach algebras the spectrum of any element is a non empty compact subset of $\mathbb{C}$ [\cite{HelBanLocConvAlg}, corollary 2.1.16].

A character on a Banach algebra $A$ is a non zero linear homomorphism $\varkappa:A\to\mathbb{C}$. All characters are continuous and are contained in the unit ball of $A^*$  [\cite{HelBanLocConvAlg}, theorem 1.2.6]. Therefore we may consider the set of characters with the induced weak$^*$ topology. The resulting topological space is Hausdorff and locally compact. It is called the spectrum of Banach algebra $A$ and denoted by $\operatorname{Spec}(A)$. If $A$ is unital then its spectrum is compact [\cite{HelBanLocConvAlg}, theorem 1.2.50]. Now for a given Banach algebra $A$ with non empty spectrum we can construct a contractive homomorphism $\Gamma_A:A\to C_0(\operatorname{Spec}(A)):a\mapsto(\varkappa\mapsto \varkappa(a))$ called the Gelfand transform of $A$ [\cite{HelBanLocConvAlg}, theorem 4.2.11]. The kernel of this homomorphism is called the Jacobson's radical and denoted by $\operatorname{Rad}(A)$. For Banach algebra $A$ with empty spectrum we define $\operatorname{Rad}(A)=A$. If $\operatorname{Rad}(A)=\{0\}$, then $A$ is called semisimple. By Shilov's idempotent theorem [\cite{KaniBanAlg}, section 3.5] any semisimple Banach algebra with compact spectrum is unital.

Most of standard constructions for Banach spaces have their counterparts for Banach algebras. For example $\bigoplus_p$-sum of Banach algebras endowed with componentwise multiplication is a Banach algebra, the quotient of a given Banach algebra by its two sided ideal is a Banach algebra. Even the projective tensor product of two Banach algebras is a Banach algebra with multiplication defined on elementary tensors the same way as in pure algebra.

We shall proceed to the discussion of the most important class of Banach algebras. Let $A$ be an associative algebra over $\mathbb{C}$, then a conjugate linear operator ${}^*:A\to A$ is called an involution if $(ab)^*=b^*a^*$ and $a^{**}=a$ for all $a,b\in A$. Algebras with involution are called ${}^*$-algebras. Homomorphisms between ${}^*$-algebras that preserve involution are called ${}^*$-homomorphisms. A Banach algebra with isometric involution is called a ${}^*$-Banach algebra. An example of such algebra is the Banach algebra of bounded linear operators on Hilbert space with operation of taking the Hilbert adjoint operator in the role of involution. In fact there is much more to this algebra than one could expect. We say that a ${}^*$-Banach algebra $A$ is a $C^*$-algebra if it satisfies $\Vert a^*a\Vert=\Vert a\Vert^2$ for all $a\in A$. One of the biggest advantages of $C^*$-algebras is their celebrated representation theorems by Gelfand and Naimark. The first representation theorem [\cite{HelBanLocConvAlg}, theorem 4.7.13] states that any commutative $C^*$-algebra $A$ is isometrically isomorphic as ${}^*$-algebra to $C_0(\operatorname{Spec}(A))$. The second theorem [\cite{HelBanLocConvAlg}, theorem 4.7.57] gives a description of generic $C^*$-algebras as closed ${}^*$-Banach subalgebras of $\mathcal{B}(H)$ for some Hilbert space $H$. Such representation is not unique, but a norm (if it exists) that turn a ${}^*$-algebra into a $C^*$-algebra is always unique. If a ${}^*$-subalgebra of $\mathcal{B}(H)$ is weak${}^*$ closed it is called a von Neumann algebra. If a $C^*$-algebra is isomorphic as ${}^*$-algebra to a von Neumann algebra it is called a $W^*$-algebra. By well known Sakai's theorem [\cite{BlackadarOpAlg}, theorem III.2.4.2] each $C^*$-algebra which is dual as Banach space is a $W^*$-algebra, but beware a $W^*$-algebra may be represented as non weak${}^*$ closed ${}^*$-subalgebra in $\mathcal{B}(H)$ for some Hilbert space $H$. 

A lot of standard constructions pass to $C^*$-algebras from Banach algebras, but not all. For example a $\bigoplus_\infty$-sum of $C^*$-algebras is again a $C^*$-algebra. A quotient of $C^*$-algebra by closed two-sided ideal is a $C^*$-algebra too. Meanwhile the projective tensor product of $C^*$-algebras is rarely a $C^*$-algebra, though there a lot of norms that may turn their algebraic tensor product into a $C^*$-algebra. In this thesis we shall exploit one specific and highly
important for $C^*$-algebras construction of matrix algebras. For a given $C^*$-algebra $A$ by $M_n(A)$ we denote the linear space of $n\times n$ matrices with entries in $A$. In fact $M_n(A)$ is ${}^*$-algebra with involution and multiplication defined by equalities 
$$
(ab)_{i,j}=\sum_{k=1}^n a_{i,k}b_{k,j}
\qquad\qquad
(a^*)_{i,j}=(a_{j,i}^*)
$$ 
for all $a,b\in M_n(A)$ and $i,j\in\mathbb{N}_n$. There is a unique norm on $M_n(A)$ that makes it a $C^*$-algebra [\cite{MurphyCStarAlgsAndOpTh}, theorem 3.4.2]. Obviously, $M_n(\mathbb{C})$ is isometrically isomorphic as ${}^*$-algebra to $\mathcal{B}(\ell_2(\mathbb{N}_n))$. From [\cite{MurphyCStarAlgsAndOpTh}, remark 3.4.1] it follows that the natural embeddings $i_{k,l}:A\to M_n(A):a\mapsto(a\delta_{i,k}\delta_{j,l})_{i,j\in\mathbb{N}_n}$ and projections $\pi_{k,l}:M_n(A)\to A:a\mapsto a_{k,l}$ are continuous. Therefore for a given bounded linear operator $\phi:A\to B$ between $C^*$-algebras $A$ and $B$ the linear operator 
$$
M_n(\phi):M_n(A)\to M_n(B):a\mapsto (\phi(a_{i,j}))_{i,j\in\mathbb{N}_n}
$$ 
is also bounded. Even more if $\phi$ is an $A$-morphism or ${}^*$-homomorphism, then so does $M_n(\phi)$. Finally we shall mention two isometric isomorphisms that will be of use:
$$
M_n\left(\bigoplus\nolimits_\infty\{A_\lambda:\lambda\in\Lambda\}\right)
\isom{\mathbf{Ban}_1}
\bigoplus\nolimits_\infty\{M_n\left(A_\lambda\right):\lambda\in\Lambda\},
$$
$$
M_n(C(K))\isom{\mathbf{Ban}_1}C(K,M_n(\mathbb{C}))
$$

Now a few facts on approximate identities and identities of $C^*$-algebras and their ideals. Any two-sided closed ideal of $C^*$-algebra has a two-sided contractive positive approximate identity [\cite{HelBanLocConvAlg}, theorem 4.7.79], and any left ideal has a right contractive positive approximate identity. In some cases even an approximate identity is not enough, so for this situation there is a procedure to endow a $C^*$-algebra with identity and preserve $C^*$-algebraic structure [\cite{HelBanLocConvAlg}, proposition 4.7.6]. This type of unitization we shall denote as $A_\#$. Till the end of this paragraph we assume that $A$ is a unital $C^*$-algebra. An element $a\in A$ is called a projection (or an orthogonal projection) if $a=a^*=a^2$, self-adjoint if $a=a^*$, positive if $a=b^*b$ for some $b\in A$, unitary if $a^*a=aa^*=e_A$. The set $A_{pos}$ of all positive elements of $A$ is a closed cone in $A$. If an element $a\in A$ is $\langle$~self-adjoint / positive~$\rangle$, then $\langle$~$\operatorname{sp}_A(a)\subset[-\Vert a\Vert, \Vert a\Vert]$ / $\operatorname{sp}_A(a)\subset[0,\Vert a\Vert]$~$\rangle$. For a given self-adjoint element $a\in A$, there always exists the isometric ${}^*$-homomorphism $\operatorname{Cont}_a:C(\operatorname{sp}_A(a))\to A$ such that $\operatorname{Cont}_a(f)=a$, where $f:\operatorname{sp}_A(a)\to\mathbb{C}:t\mapsto t$. It is called the continuous functional calculus [\cite{HelBanLocConvAlg}, theorem 4.7.24]. Loosely speaking it allows to take continuous functions of self-adjoint elements of $C^*$-algebras, so following standard convention we shall write $f(a)$ instead of $\operatorname{Cont}_a(f)$. Another related result called the spectral mapping theorem allows to compute the spectrum of elements given by continuous functional calculus: $\operatorname{sp}_A(f(a))=f(\operatorname{sp}_A(a))$.

We proceed to the discussion of more general objects --- Banach modules. Let $A$ be a Banach algebra, we say that $X$ is a $\langle$~left / right~$\rangle$ Banach $A$-module if $X$ is a Banach space endowed with bilinear operator $\langle$~$\cdot:A\times X\to X$ / $\cdot: X\times A\to X$~$\rangle$ of norm at most $1$ (called a module action), such that $\langle$~$a\cdot(b\cdot x)=ab\cdot x$ / $(x\cdot a)\cdot b=x\cdot ab$~$\rangle$ for all $a,b\in A$ and $x\in X$. Any Banach space $E$ can be turned into a $\langle$~left / right~$\rangle$ Banach $A$-module be defining $\langle$~$a\cdot x=0$ / $x\cdot a=0$~$\rangle$ for all $a\in A$ and $x\in E$. Any Banach algebra $A$ can be regarded as a left and right Banach $A$-module --- the module action coincides with algebra multiplication. Of course, there are more meaningful examples too.  Usually we shall discuss only left Banach modules since for their right sided counterparts all definitions and results are similar. We call a left Banach module $X$ over unital Banach algebra $A$ unital if $e_A\cdot x=x$ for all $x\in X$. For a given left Banach $A$-module $X$ and $S\subset A$, $M\subset X$ we define their products $S\cdot M=\{a\cdot x:a\in S, x\in M\}$, $SM=\operatorname{span} (S\cdot M)$ and annihilators $S^{\perp M}=\{a\in S:a\cdot M=\{0\}\}$, ${}^{S\perp}M=\{x\in M: S\cdot x=\{0\}\}$. The essential and annihilator parts of $X$ are defined as $X_{ess}=\operatorname{cl}_X(A X)$, $X_{ann}={}^{A\perp}X$. The module $X$ is called $\langle$~faithful / annihilator / essential~$\rangle$ if $\langle$~${}^{A\perp}X=\{0\}$ / $X=X_{ann}$ / $X=X_{ess}$~$\rangle$. An obvious application of Hahn-Banach theorem shows that $X$ is an essential $A$-module iff $X^*$ is a faithful $A$-module.

Let $X$ and $Y$ be $\langle$~left / right~$\rangle$ Banach $A$-modules. We say that a linear operator $\phi:X\to Y$ is an $A$-module map of $\langle$~left / right~$\rangle$ modules if $\langle$~$\phi(a\cdot x)=a\cdot \phi(x)$ / $\phi(x\cdot a)=\phi(x)\cdot a$~$\rangle$ for all $a\in A$ and $x\in X$. A bounded $A$-module map is called an $A$-morphism. The set of $A$-morphisms between $\langle$~left / right~$\rangle$ $A$-modules $X$ and $Y$ we denote as $\langle$~${}_A\mathcal{B}(X,Y)$ / $\mathcal{B}_A(X,Y)$~$\rangle$. Note that if $X$ and $Y$ are $\langle$~left / right~$\rangle$ annihilator $A$-modules, then $\langle$~${}_A\mathcal{B}(X,Y)=\mathcal{B}(X,Y)$ / $\mathcal{B}_A(X,Y)=\mathcal{B}(X,Y)$~$\rangle$.

By $\langle A-\mathbf{mod}$ / $\mathbf{mod}-A\rangle$ we shall denote the category of $\langle$~left / right~$\rangle$ $A$-modules with continuous $A$-module maps in the role of morphisms. By $\langle$~$A-\mathbf{mod}_1$ / $\mathbf{mod}_1-A$~$\rangle$ we denote its subcategory of $\langle$~$A-\mathbf{mod}$ / $\mathbf{mod}-A$~$\rangle$ with the same objects and contractive morphisms only. Therefore $\langle$~ $\operatorname{Hom}_{A-\mathbf{mod}}(X,Y)={}_A\mathcal{B}(X,Y)$ /  $\operatorname{Hom}_{\mathbf{mod}-A}(X,Y)=\mathcal{B}_A(X,Y)$~$\rangle$. 

As in any category we can speak of retraction and	 coretractions in the category of Banach modules. But for this particular case we have several refinements for the standard definitions. An $A$-morphism $\xi:X\to Y$ is called a $\langle$~$c$-retraction / $c$-coretraction~$\rangle$ if there exist an $A$-morphism $\eta:Y\to X$ such that $\langle$~$\xi\eta=1_Y$ / $\eta\xi=1_X$~$\rangle$ and $\Vert\xi\Vert\Vert\eta\Vert\leq c$. From the definition it follows that composition of $\langle$~$c_1$- and $c_2$-retraction / $c_1$- and $c_2$-coretraction~$\rangle$ gives a $\langle$~$c_1c_2$-retraction / $c_1c_2$-coretraction~$\rangle$. Clearly, the adjoint of $\langle$~$c$-retraction / $c$-coretraction~$\rangle$ is a $\langle$~$c$-coretraction / $c$-retraction~$\rangle$. Finally, an $A$-morphism $\xi:X\to Y$ is called a $c$-isomorphism if there exists an $A$-morphism $\eta:Y\to X$ such that $\xi\eta=1_Y$, $\eta\xi=1_X$ and $\Vert\xi\Vert\Vert\eta\Vert\leq c$. In this case we say that $A$-modules $X$ and $Y$ are $c$-isomorphic.

Now we mention several constructions over Banach modules that we will encounter in this thesis.  Any left Banach $A$-module can be regarded as unital Banach module over $A_+$, and we put by definition $(a\oplus_1 z)\cdot x=a\cdot x+zx$ for all $a\in A$, $x\in X$ and $z\in\mathbb{C}$. Most constructions used for Banach spaces transfer to Banach modules.  We say that a linear subspace $ Y$ of a left Banach $A$-module $X$ is a left $A$-submodule of $X$ if $A\cdot Y\subset Y$. For example, any left ideal $I$ of a Banach algebra $A$ is a left $A$-submodule of $A$. If $Y$ is a closed left $A$-submodule of the left Banach $A$-module $X$, then the Banach space $X/Y$ can be endowed with the structure of the left Banach $A$-module, just put by definition $a\cdot(x+Y)=a\cdot x+Y$ for all $a\in A$ and $x+Y\in X/Y$. This object is called the quotient $A$-module.  Quotient modules of the form $A/I$, where $I$ is a left ideal of $A$, are called cyclic modules. For motivation for this term see [\cite{HelBanLocConvAlg}, proposition 6.2.2]. Clearly, $X/X_{ess}$ is an annihilator $A$-module. If $X$ is a left Banach $A$-module and $E$ is a Banach space, then $\langle$~$\mathcal{B}(X,E)$ / $\mathcal{B}(E,X)$~$\rangle$ is a $\langle$~right / left~$\rangle$ Banach $A$-module with module action defined by $\langle$~$(T\cdot a)(x)=T(a\cdot x)$ for all $a\in A$, $x\in X$ and $T\in\mathcal{B}(X, E)$ / $(a\cdot T)(x)=a\cdot T(x)$ for all $a\in A$, $x\in E$ and $T\in\mathcal{B}(E, X)$~$\rangle$. In particular, $X^*$ is a right Banach $A$-module. If $\{X_\lambda:\lambda\in\Lambda\}$ is a family of left Banach $A$-modules and $1\leq p\leq +\infty$ or $p=0$, then their $\bigoplus_p$-sum is a left Banach $A$-module with module action defined by $a\cdot x=\bigoplus_p\{ a\cdot x_\lambda:\lambda\in\Lambda\}$, where $a\in A$, $x\in\bigoplus_p\{ X_\lambda:\lambda\in\Lambda\}$. Again, as in Banach space theory, any family of $A$-modules admits the $\langle$~product / coproduct~$\rangle$ in $A-\mathbf{mod}_1$ which in fact is their $\langle$~$\bigoplus_1$-sum / $\bigoplus_\infty$-sum~$\rangle$. The category $A-\mathbf{mod}$ admits $\langle$~products / coproducts~$\rangle$ only for finite families of objects. Similar statements are valid for $\mathbf{mod}-A$ and $\mathbf{mod}_1-A$.

Projective tensor product of Banach spaces also has its module version, it is called the projective module tensor product. Assume $X$ is a right and $Y$ is a left Banach $A$-module. Their projective module tensor product $X\projmodtens{A}Y$ is defined as quotient space $X\projtens Y / N$ where $N=\operatorname{cl}_{X\projtens Y}(\operatorname{span}\{x\cdot a\projtens y-x\projtens a\cdot y:x\in X,y\in Y,a\in A\})$. Let $\phi\in\mathcal{B}_A(X_1,X_2)$ and $\psi\in{}_A\mathcal{B}(Y_1,Y_2)$ for right Banach $A$-modules $X_1$, $X_2$ and left Banach $A$-modules $Y_1$, $Y_2$, then there exists a unique bounded linear operator $\phi\projmodtens{A} \psi:X_1\projmodtens{A} Y_1\to X_2\projmodtens{A} Y_2$ such that $(\phi\projmodtens{A} \psi)(x\projmodtens{A} y)=\phi(x)\projmodtens{A} \psi(y)$ for all $x\in X_1$ and $y\in Y_1$. Even more $\Vert \phi\projmodtens{A} \psi\Vert\leq\Vert \phi\Vert\Vert \psi\Vert$. The projective module tensor product has its own universal property: for any right Banach $A$-module $X$, any left Banach $A$-module $Y$ and any Banach space $E$ there exists an isometric isomorphism:
$$
\mathcal{B}(X\projmodtens{A}Y,E)\isom{\mathbf{Ban}_1}\mathcal{B}_{bal}(X\times Y, E)
$$
where $\mathcal{B}_{bal}(X\times Y, E)$ stands for the Banach space of bilinear operators $\Phi:X\times Y\to E$ satisfying $\Phi(x\cdot a,y)=\Phi(x,a\cdot y)$ for all $x\in X$, $y\in Y$ and $a\in A$. Such bilinear operators are called balanced.
Furthermore we have two (natural in $X$, $Y$ and $E$) isometric isomorphisms:
$$
\mathcal{B}(X\projmodtens{A}Y,E)
\isom{\mathbf{Ban}_1}
{}_A\mathcal{B}(Y,\mathcal{B}(X,E))
\isom{\mathbf{Ban}_1}
\mathcal{B}_A(X,\mathcal{B}(Y,E))
$$
Analogously to Banach space theory we may define the following functors:
$$
\mathcal{B}(-,E):A-\mathbf{mod}\to \mathbf{mod}-A
\qquad\qquad
\mathcal{B}(E,-):\mathbf{mod}-A\to \mathbf{mod}-A
$$
$$
-\projmodtens{A} Y:\mathbf{mod}-A\to\mathbf{Ban}
\qquad\qquad
X\projmodtens{A} -:A-\mathbf{mod}\to\mathbf{Ban}
$$
where $E$ is a Banach space, $X$ is a right $A$-module and $Y$ is a left $A$-module. All these functors have their counterparts for categories $A-\mathbf{mod}_1$, $\mathbf{mod}_1-A$. 

In some cases it is possible to explicitly compute the projective module tensor product. For example [\cite{HelBanLocConvAlg}, proposition 6.3.24] if $I$ is a left closed ideal of $A_+$ with left $\langle$~contractive / bounded~$\rangle$ approximate identity, and $X$ is a left Banach module then the linear operator 
$$
i_{I,X}:I\projmodtens{A}X \to \operatorname{cl}_X(IX):a\projmodtens{A} x\mapsto a\cdot x
$$
is $\langle$~a topological isomorphism / an isometric isomorphism~$\rangle$ of Banach spaces. If $I$ is a two-sided ideal, then $i_{I,X}$ is a morphism of left $A$-modules. We call reduced all left Banach modules of the form $A\projmodtens{A}X$. 

Most of what have been said here can be generalized to Banach bimodules, but in this thesis we shall not exploit them much. In those rare case when we shall encounter bimodules, the respective definitions and results are easily recoverable from their one sided counterparts.

%----------------------------------------------------------------------------------------
%	Banach homology
%----------------------------------------------------------------------------------------

\section{Banach homology}
\label{SectionBanachHomology}

%----------------------------------------------------------------------------------------
%	Relative homology
%----------------------------------------------------------------------------------------

\subsection{Relative homology}
\label{SubSectionRelativeHomology}

Further we briefly discuss ABCs of relative homology introduced and intensively studied by Helemskii. Fix an arbitrary Banach algebra $A$. We say that a morphism $\xi:X\to Y$ of left $A$-modules $X$ and $Y$ is a relatively admissible epimorphism if it admits a right inverse bounded linear operator. A left $A$-module $P$ is called relatively projective if for any relatively admissible  epimorphism $\xi:X\to Y$ and for any $A$-morphism $\phi:P\to Y$ there exists an $A$-morphism $\psi:P\to X$ such that the diagram
$$
\xymatrix{
& {X} \ar[d]^{\xi}\\
{P} \ar@{-->}[ur]^{\psi} \ar[r]^{\phi} &{Y}}
$$
is commutative. Such $A$-morphism $\psi$ is called a lifting of $\phi$ and it is not unique in general. Similarly,  we say that a morphism $\xi:Y\to X$ of right $A$-modules $X$ and $Y$ is a relatively admissible monomorphism if it admits a left inverse bounded linear operator. A right $A$-module $J$ is called relatively injective if for any relatively admissible  monomorphism $\xi:Y\to X$ and for any $A$-morphism $\phi:Y\to J$ there exists an $A$-morphism $\psi:X\to J$ such that the diagram
$$
\xymatrix{
& {X} \ar@{-->}[dl]_{\psi} \\
{J} &{Y} \ar[l]_{\phi} \ar[u]_{\xi}}
$$
is commutative. Such $A$-morphism $\psi$ is called an extension of $\phi$ and it is not unique in general.

The reason for considering relatively admissible morphisms in these definitions is the intention of separation Banach geometric and algebraic motives that may prevent an $A$-module to be relatively projective or injective. A straightforward check shows that any retract of relatively $\langle$~projective / injective~$\rangle$ $A$-module is again relatively $\langle$~projective / injective~$\rangle$. Obviously, any relatively admissible $\langle$~epimorphism / monomorphism~$\rangle$ $\langle$~onto / from~$\rangle$ a relatively $\langle$~projective / injective~$\rangle$ $A$-module is a $\langle$~retraction / coretraction~$\rangle$.

A special class of relatively $\langle$~projective / injective~$\rangle$ $A$-modules is the so-called relatively $\langle$~free / cofree~$\rangle$ modules. These are modules of the form $\langle$~$A_+\projtens E$ / $\mathcal{B}(A_+,E)$~$\rangle$ for some Banach space $E$. Their main feature is the following: for any $A$-module $X$ there exists a relatively $\langle$~free / cofree~$\rangle$ $A$-module $F$, which in fact is $\langle$~$A_+\projtens X$ / $\mathcal{B}(A_+,X)$~$\rangle$ and a relatively admissible $\langle$~epimorphism $\xi:F\to X$ / monomorphism $\xi:X\to F$~$\rangle$. If $X$ is relatively $\langle$~projective / injective~$\rangle$ we immediately get that $\xi$ is a $\langle$~retraction / coretraction~$\rangle$. Therefore an $A$-module is relatively $\langle$~projective / injective~$\rangle$ iff it is a retract of relatively $\langle$~free / cofree~$\rangle$ $A$-module. 

It is worth to emphasize one more time that major nuance of relative Banach homology is deliberate balance between algebra and topology in choice of admissible morphisms. This choice allowed one to build homological theory with some interesting phenomena with no analogs in pure algebra. We demonstrate one example related to Banach algebras. Consider morphism of $A$-bimodules  $\Pi_A:A\projtens A\to A:a\projtens b\mapsto ab$. We say that a Banach algebra $A$ is 

$i)$ relatively $c$-biprojective if $\Pi_A$ is a $c$-retraction of $A$-bimodules;

$ii)$ relatively $c$-biflat if $\Pi_A^*$ is a $c$-coretraction of $A$-bimodules;

$iii)$ relatively $c$-contractible if $\Pi_{A_+}$ is a $c$-retraction of $A$-bimodules;

$iv)$ relatively $c$-amenable if $\Pi_{A_+}^*$ is a $c$-coretraction of $A$-bimodules.

We say that $A$ is relatively $\langle$~biprojective / biflat / contractive / amenable~$\rangle$ if it is relatively $\langle$~$c$-biprojective / $c$-biflat / $c$-contractive / $c$-amenable~$\rangle$ for some $c\geq 1$. The infimum of the constants $c$ is called the $\langle$~biprojectivity / biflatness / contractivity / amenability~$\rangle$ constant.  With slight modifications of [\cite{HelBanLocConvAlg}, proposition 7.1.72] one can show that $A$ is relatively $\langle$~$c$-contractible / $c$-amenable~$\rangle$ iff there exists $\langle$~an element $d\in A\projtens A$ / a net $(d_\nu)_{\nu\in N}\subset A\projtens A$~$\rangle$ with norm not greater than $c$ such that for all $a\in A$ holds $\langle$~$a\cdot d-d\cdot a=0$ and $a\Pi_A(d)=a$ / $\lim_\nu(a\cdot d_\nu-d_\nu\cdot a)=0$ and $\lim_\nu a\Pi_A(d_\nu)=a$~$\rangle$. Note that $\langle$~such element $d$ / such net $(d_\nu)_{\nu\in N}$~$\rangle$ is called $\langle$~a diagonal / an approximate diagonal~$\rangle$. From homological point of view, the main advantage of relatively $\langle$~biprojective / biflat / contractible / amenable~$\rangle$ Banach algebras is that $\langle$~any reduced / any reduced / any / any~$\rangle$ left and right Banach $A$-module is relatively $\langle$~projective / flat / projective / flat~$\rangle$ [\cite{HelBanLocConvAlg}, theorem 7.1.60]. As for flatness such phenomena is typical for relative Banach homology, but not for the purely algebraic one.

%----------------------------------------------------------------------------------------
%	Rigged categories
%----------------------------------------------------------------------------------------

\subsection{Rigged categories}
\label{SubSectionRiggedCategories}

Claims on projectivity and injectivity from previous section have their analogs for a lot of other types of projectivity and injectivity in other categories of mathematics \cite{SemadeniProjInjDual}. Even more, one may easily see that injectivity and projectivity are somewhat dual to each other. All these observations suggest that there is a general categorical approach to study basic properties of homologically trivial objects. Such approach has been promoted by Helemskii in \cite{HelMetrFrQMod}. As we shall see it covers relative theory while results given above are obvious consequences of more general facts. 

Let $\mathbf{C}$ and $\mathbf{D}$ be two fixed categories. An ordered pair ($\mathbf{C}, \square:\mathbf{C}\to\mathbf{D}$), where $\square$ is a faithful covariant functor, is called a rigged category. We say that a morphism $\xi$ in $\mathbf{C}$ is $\square$-admissible epimorphism if $\square (\xi)$ is a retraction in $\mathbf{D}$. An object $P$ in $\mathbf{C}$ is called $\square$-projective, if for every $\square$-admissible epimorphism $\xi$ in $\mathbf{C}$ the map $\operatorname{Hom}_{\mathbf{C}}(P,\xi)$ is surjective. An object $F$ in $\mathbf{C}$ is called $\square$-free with base $M$ in  $\mathbf{D}$, if there exists an isomorphism of functors $\operatorname{Hom}_{\mathbf{D}}(M,\square(-))\cong\operatorname{Hom}_{\mathbf{C}}(F,-)$. A rigged category $(\mathbf{C},\square)$ is called  freedom-loving [\cite{HelMetrFrQMod}, definition 2.10], if every object in $\mathbf{D}$ is a base of some $\square$-free object in $\mathbf{C}$. We may summarize results of propositions 2.3, 2.11  and 2.12 in \cite{HelMetrFrQMod} as follows:

$i)$ any retract of $\square$-projective object is $\square$-projective;

$ii)$ any $\square$-admissible epimorphism into $\square$-projective object is a retraction;

$iii)$ any $\square$-free object is $\square$-projective;

$iv)$ if $(\mathbf{C},\square)$ is freedom-loving rigged category, then any object is $\square$-projective iff it is a retract of $\square$-free object;

$v)$ coproduct of the family of $\square$-projective objects is $\square$-projective.

The opposite rigged category of $(\mathbf{C}, \square)$ 
is a rigged category $(\mathbf{C}^{o},\square^{o}:\mathbf{C}^{o}\to\mathbf{D}^{o})$. 
Thus by passing to the opposite rigged category we may define admissible monomorphisms, injectivity and cofreedom. A morphism $\xi$ in called $\square$-admissible monomorphism if it is $\square^o$-admissible epimorphism. An object $J$ in $\mathbf{C}$ is called $\square$-injective if it is $\square^o$-projective. Finally, an object $F$ in $\mathbf{C}$ is called $\square$-cofree if it is $\square^o$-free. This gives us analogs of results as above for injectivity and cofreedom.

Now consider faithful functor $\square_{rel}:A-\mathbf{mod}\to\mathbf{Ban}$ that just `forgets'' the module structure. One can easily see that $(A-\mathbf{mod},\square_{rel})$ is a rigged category whose $\square_{rel}$-admissible $\langle$~epimorphisms / monomorphisms~$\rangle$ are exactly relatively admissible $\langle$~epimorphisms / monomorphisms~$\rangle$ and $\langle$~$\square_{rel}$-projective / $\square_{rel}$-injective~$\rangle$ objects are exactly relatively $\langle$~projective / injective~$\rangle$ $A$-modules. Even more all $\langle$~$\square_{rel}$-free / $\square_{rel}$-cofree~$\rangle$ objects are isomorphic in $A-\mathbf{mod}$ to $\langle$~$A_+\projtens E$ / $\mathcal{B}(A_+,E)$ ~$\rangle$ for some Banach space $E$. This example shows, that relative theory perfectly fits into the realm of rigged categories.

We shall apply this scheme for metric and topological theory in the next chapter. These two theories put much weaker restrictions on their admissible morphisms. The proverb ``all covet, all lose'' perfectly explains what will happen next.
