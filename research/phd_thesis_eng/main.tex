 %\title{PhD thesis}
%%%%%%%%%%%%%%%%%%%%%%%%%%%%%%%%%%%%%%%%%
% Thesis 
% LaTeX Template
% Version 1.3 (21/12/12)
%
% This template has been downloaded from:
% http://www.latextemplates.com
%
% Original authors:
% Steven Gunn 
% http://users.ecs.soton.ac.uk/srg/softwaretools/document/templates/
% and
% Sunil Patel
% http://www.sunilpatel.co.uk/thesis-template/
%
% License:
% CC BY-NC-SA 3.0 (http://creativecommons.org/licenses/by-nc-sa/3.0/)
%
% Note:
% Make sure to edit document variables in the Thesis.cls file
%
%%%%%%%%%%%%%%%%%%%%%%%%%%%%%%%%%%%%%%%%%

% TODO(norberrt)
% Get rid of the section on the "small category"
% Get rid of excessively detailed introduction on approximation property
% 

%----------------------------------------------------------------------------------------
%	PACKAGES AND OTHER DOCUMENT CONFIGURATIONS
%----------------------------------------------------------------------------------------

\documentclass[11pt, a4paper, oneside]{Thesis} % Paper size, default font size and one-sided paper 

\graphicspath{{./Pictures/}} % Specifies the directory where pictures are stored

\usepackage[square, numbers, comma, sort&compress]{natbib} % Use the natbib reference package - read up on this to edit the reference style; if you want text (e.g. Smith et al., 2012) for the in-text references (instead of numbers), remove 'numbers' 
\hypersetup{urlcolor=blue, colorlinks=true} % Colors hyperlinks in blue - change to black if annoying
\title{\ttitle} % Defines the thesis title - don't touch this

\usepackage{amssymb,amsmath}
\usepackage[utf8]{inputenc} 
\usepackage{mathrsfs}
\usepackage[matrix,arrow,curve]{xy}


%----------------------------------------------------------------------------------------
%	MY COMMANDS AND DEFINITIONS
%----------------------------------------------------------------------------------------

\newcommand{\projtens}{\mathbin{\widehat{\otimes}}}
\newcommand{\convol}{\ast}
\newcommand{\projmodtens}[1]{\mathbin{\widehat{\otimes}}_{#1}}
\newcommand{\isom}[1]{\mathop{\mathbin{\cong}}\limits_{#1}}

\begin{document}

\frontmatter % Use roman page numbering style (i, ii, iii, iv...) for the pre-content pages

\setstretch{1.3} % Line spacing of 1.3

% Define the page headers using the FancyHdr package and set up for one-sided printing
\fancyhead{} % Clears all page headers and footers
\rhead{\thepage} % Sets the right side header to show the page number
\lhead{} % Clears the left side page header

\pagestyle{fancy} % Finally, use the "fancy" page style to implement the FancyHdr headers

\newcommand{\HRule}{\rule{\linewidth}{0.5mm}} % New command to make the lines in the title page

% PDF meta-data
\hypersetup{pdftitle={\ttitle}}
\hypersetup{pdfsubject=\subjectname}
\hypersetup{pdfauthor=\authornames}
\hypersetup{pdfkeywords=\keywordnames}

%----------------------------------------------------------------------------------------
%	TITLE PAGE
%----------------------------------------------------------------------------------------

\begin{titlepage}

\begin{center}

\textsc{\LARGE \univname}\\[1.5cm] % University name
\textsc{\Large Doctoral Thesis}\\[0.5cm] % Thesis type

\HRule \\[0.4cm] % Horizontal line
{\huge \bfseries \ttitle}\\[0.4cm] % Thesis title
\HRule \\[1.5cm] % Horizontal line
 
\begin{minipage}{0.4\textwidth}
\begin{flushleft} \large
\emph{Author:}\\

{\authornames} % Author name. Full example \href{http://www.johnsmith.com}{\authornames}. Remove the \href bracket to remove the link 
\end{flushleft}
\end{minipage}
\begin{minipage}{0.4\textwidth}
\begin{flushright} \large
\emph{Supervisor:} \\
{\supname} % Supervisor name. Full example \href{http://www.jamessmith.com}{\supname}. Remove the \href bracket to remove the link
\end{flushright}
\end{minipage}\\[3cm]
 
\large \textit{A thesis submitted in fulfilment of the requirements\\ for the degree of \degreename}\\[0.3cm] % University requirement text
\textit{in the}\\[0.4cm]
\groupname\\\deptname\\[2cm] % Research group name and department name
 
{\large \today}\\[4cm] % Date
%\includegraphics{Logo} % University/department logo - uncomment to place it
 
\vfill
\end{center}

\end{titlepage}

%----------------------------------------------------------------------------------------
%	DECLARATION PAGE
%	Your institution may give you a different text to place here
%----------------------------------------------------------------------------------------

\Declaration{

\addtocontents{toc}{\vspace{1em}} % Add a gap in the Contents, for aesthetics

I, \authornames, declare that this thesis titled, '\ttitle' and the work presented in it are my own. I confirm that:

\begin{itemize} 
\item[\tiny{$\blacksquare$}] This work was done wholly or mainly while in candidature for a research degree at this University.
\item[\tiny{$\blacksquare$}] Where any part of this thesis has previously been submitted for a degree or any other qualification at this University or any other institution, this has been clearly stated.
\item[\tiny{$\blacksquare$}] Where I have consulted the published work of others, this is always clearly attributed.
\item[\tiny{$\blacksquare$}] Where I have quoted from the work of others, the source is always given. With the exception of such quotations, this thesis is entirely my own work.
\item[\tiny{$\blacksquare$}] I have acknowledged all main sources of help.
\item[\tiny{$\blacksquare$}] Where the thesis is based on work done by myself jointly with others, I have made clear exactly what was done by others and what I have contributed myself.\\
\end{itemize}
 
Signed:\\
\rule[1em]{25em}{0.5pt} % This prints a line for the signature
 
Date:\\
\rule[1em]{25em}{0.5pt} % This prints a line to write the date
}

\clearpage % Start a new page

%----------------------------------------------------------------------------------------
%	QUOTATION PAGE
%----------------------------------------------------------------------------------------

\pagestyle{empty} % No headers or footers for the following pages

\null\vfill % Add some space to move the quote down the page a bit

\textit{Never trust any result proved after 11 PM. }

\begin{flushright}
Professional secret
\end{flushright}

\vfill\vfill\vfill\vfill\vfill\vfill\null % Add some space at the bottom to position the quote just right

\clearpage % Start a new page

%----------------------------------------------------------------------------------------
%	ABSTRACT PAGE
%----------------------------------------------------------------------------------------

\addtotoc{Abstract} % Add the "Abstract" page entry to the Contents

\abstract{\addtocontents{toc}{\vspace{1em}}} % Add a gap in the Contents, for aesthetics

In this thesis we study metric and topological versions of projectivity injectivity and flatness of Banach modules over Banach algebras. These two non standard versions of Banach homology theories are studied in parallel under unified approach.

Chapter 1 gives background required for our studies. In paragraph 1.1 we collect all necessary facts on category theory, topology and measure theory. Paragraph 1.2 contains a brief introduction into Banach structures. Here we discuss Banach spaces, Banach algebras and Banach modules. In paragraph 1.3 we give a short introduction into relative Banach homology. Then we list main results from the theory of rigged categories. This theory is a common ground for many known versions of projectivity and injectivity, in particular of metric and topological ones.

In chapter 2 we establish general properties of projective injective and flat modules. In some cases we give complete characterizations of such modules. Results of this chapter are extensively used in chapter 3 when dealing with specific modules of analysis. Let us discuss contents of this chapter in more detail. In paragraph 2.1 we derive basic properties of projective injective and flat modules from the theory of rigged categories. We also study different constructions that preserve homological triviality of Banach modules. These results are used to characterize projectivity and flatness of cyclic modules. We also give necessary conditions for projectivity of left ideals of Banach algebras. As the consequence we describe projective ideals of commutative Banach algebras that admit bounded approximate identities. Paragraph 2.2 is devoted to Banach geometric properties of homologically trivial modules. We characterize projective injective and flat annihilator modules and establish their strong relation to projective injective and flat Banach spaces. Then we give several examples confirming that homologically trivial module and its Banach algebra have similar Banach geometric properties. Examples include the property of being an $\mathscr{L}_1^g$-space, the Dunford-Pettis property and the l.u.st. property. Paragraph 2.3 is quite short. Here we list conditions under which projectivity injectivity and flatness are preserved under transition between modules over algebra to modules over ideal. Finally, we give a necessary and sufficient conditions of topological flatness of a Banach module and necessary condition of injectivity of two-sided ideals.

In chapter 3 we apply general results to specific modules of analysis. In paragraph 3.1 we investigate projectivity injectivity and flatness of ideals of $C^*$-algebras. We describe projective left ideals of $C^*$-algebras and give a criterion for injectivity of $AW^*$-algebras. These characterizations are indispensable in description of homologically trivial modules over algebras of bounded and compact operators on a Hilbert space. We perform similar research for commutative case of algebras of bounded and vanishing functions on discrete sets and locally compact Hausdorff spaces. In paragraph 3.2 we proceed to study of standard modules of harmonic analysis. Due to specific Banach geometric structure of convolution algebra and measure algebra we easily show that most of standard modules of harmonic analysis are homologically non trivial. The most intriguing result of the paragraph is non-projectivity of convolution algebra of a non discrete group. In paragraph 3.3 we construct an example of category which behaves essentially different than the standard categories of analysis. This category consist of $L_p$-spaces with the structure of Banach module over algebra of bounded measurable functions. All modules of this category turn out to be homologically trivial. To show this we characterize topologically injective, topologically surjective, isometric and coisometric multiplication operators between $L_p$-spaces.

\clearpage % Start a new page

%----------------------------------------------------------------------------------------
%	ACKNOWLEDGEMENTS
%----------------------------------------------------------------------------------------

\setstretch{1.3} % Reset the line-spacing to 1.3 for body text (if it has changed)

\acknowledgements{\addtocontents{toc}{\vspace{1em}} % Add a gap in the Contents, for aesthetics

Firstly, I would like to thank my supervisor Professor Alexander Helemskii for formulation of interesting problems and continuous encouragement to finish this work.  I'm grateful to Narutaka Ozawa, Leonel Robert and Alexei Pirkovskii for answering my silly questions. Also I would like to thank Tomasz Kania for bringing my attention to the l.u.st. property. Lastly, but definitely not least, I want to thank Viktoria for continued support and love, without her this project would not have been possible.
}
\clearpage % Start a new page

%----------------------------------------------------------------------------------------
%	LIST OF CONTENTS/FIGURES/TABLES PAGES
%----------------------------------------------------------------------------------------

\pagestyle{fancy} % The page style headers have been "empty" all this time, now use the "fancy" headers as defined before to bring them back

\lhead{\emph{Contents}} % Set the left side page header to "Contents"
\tableofcontents % Write out the Table of Contents

%\lhead{\emph{List of Figures}} % Set the left side page header to "List of Figures"
%\listoffigures % Write out the List of Figures

%\lhead{\emph{List of Tables}} % Set the left side page header to "List of Tables"
%\listoftables % Write out the List of Tables

%----------------------------------------------------------------------------------------
%	ABBREVIATIONS
%----------------------------------------------------------------------------------------

%\clearpage % Start a new page

%\setstretch{1.5} % Set the line spacing to 1.5, this makes the following tables easier to read

%\lhead{\emph{Abbreviations}} % Set the left side page header to "Abbreviations"
%\listofsymbols{ll} % Include a list of Abbreviations (a table of two columns)
%{
%\textbf{LAH} & \textbf{L}ist \textbf{A}bbreviations \textbf{H}ere \\
%\textbf{Acronym} & \textbf{W}hat (it) \textbf{S}tands \textbf{F}or \\
%}

%----------------------------------------------------------------------------------------
%	PHYSICAL CONSTANTS/OTHER DEFINITIONS
%----------------------------------------------------------------------------------------

%\clearpage % Start a new page

%\lhead{\emph{Physical Constants}} % Set the left side page header to "Physical Constants"

%\listofconstants{lrcl} % Include a list of Physical Constants (a four column table)
%{
%Speed of Light & $c$ & $=$ & $2.997\ 924\ 58\times10^{8}\ \mbox{ms}^{-\mbox{s}}$ (exact)\\
% Constant Name & Symbol & = & Constant Value (with units) \\
%}

%----------------------------------------------------------------------------------------
%	DEDICATION
%----------------------------------------------------------------------------------------

%\setstretch{1.3} % Return the line spacing back to 1.3

%\pagestyle{empty} % Page style needs to be empty for this page

%\dedicatory{For/Dedicated to/To my\ldots} % Dedication text

%\addtocontents{toc}{\vspace{2em}} % Add a gap in the Contents, for aesthetics

%----------------------------------------------------------------------------------------
%	THESIS CONTENT - CHAPTERS
%----------------------------------------------------------------------------------------

\mainmatter % Begin numeric (1,2,3...) page numbering

\pagestyle{fancy} % Return the page headers back to the "fancy" style

% Include the chapters of the thesis as separate files from the Chapters folder
% Uncomment the lines as you write the chapters

% Chapter Template

\chapter{Preliminaries} % Main chapter title

\label{ChapterPreliminaries} % Change X to a consecutive number; for referencing this chapter elsewhere, use \ref{ChapterX}

\lhead{Chapter 1. \emph{Preliminaries}} % Change X to a consecutive number; this is for the header on each page - perhaps a shortened title

In what follows, we present some parts in parallel fashion by listing the respective options in order, enclosed and separate like this: $\langle$~.../...~$\rangle$. For example: a real number $x$ is $\langle$~positive / non negative~$\rangle$ if $\langle$~$x>0$ / $x\geq 0$~$\rangle$. Sometimes one of the parts might be empty. We use symbol $:=$ for equality by definition.

We use the following standard notation for some commonly used sets of numbers: $\mathbb{C}$ denotes the complex numbers, $\mathbb{R}$ denotes the real numbers, $\mathbb{Z}$ denotes the integers, $\mathbb{N}$ denotes the natural numbers, $\mathbb{N}_n$ denotes the set of first $n$ natural numbers, $\mathbb{R}_+$ denotes the set of non negative real numbers, $\mathbb{T}$ denotes the set of complex numbers of modulus $1$, finally, $\mathbb{D}$ denotes the set of complex numbers with modulus less than $1$. For $z\in\mathbb{C}$ the symbol $\overline{z}$ stands for the complex conjugate number.

For a given map $f:M\to M'$ and subset $\langle$~$N\subset M$ / $N'\subset M'$ such that $\operatorname{Im}(f)\subset N'$~$\rangle$ by $\langle$~$f|_N$ / $f|^{N'}$~$\rangle$ we denote the  $\langle$~restriction of $f$ onto $N$ / corestriction of $f$ onto $N'$~$\rangle$, that is $\langle$~$f|_N:N\to M':x\mapsto f(x)$ / $f|^{N'}:M\to N':x\mapsto f(x)$~$\rangle$. The indicator function of a subset $N$ of the set $M$ is denoted by $\chi_{N}$, so that $\chi_N(x)=1$ for $x\in N$ and $\chi_N(x)=0$ for $x\in M\setminus N$. We also use the shortcut $\delta_x=\chi_{\{x\}}$ where $x\in M$. By $\mathcal{P}(M)$ we denote the set of all subsets of $M$, and $\mathcal{P}_0(M)$ stands for the set of all finite subsets of $M$. The symbol $M^N$ stands for the set of all functions from $N$ to $M$. By $\operatorname{Card}(M)$ we denote the cardinality of $M$. By $\aleph_0$ we denote the cardinality of $\mathbb{N}$.

%----------------------------------------------------------------------------------------
%	Broad foundations
%----------------------------------------------------------------------------------------

\section{Broad foundations}

\label{SectionBroadFoundations} 

%----------------------------------------------------------------------------------------
%	Categorical language
%----------------------------------------------------------------------------------------

\subsection{Categorical language}
\label{SubSectionCategoricalLanguage}

Here we recall some basic facts and definitions from category theory and fix notation we shall use. We assume that our reader is familiar with such basics of category theory as category, functor, morphism. Otherwise see [\cite{HelLectAndExOnFuncAn}, chapter 0] for a quick introduction or [\cite{KashivShapCatsAndSheavs}, chapter 1] for more details.

For a given category $\mathbf{C}$ by $\operatorname{Ob}(\mathbf{C})$ we denote the class of its objects. The symbol $\mathbf{C}^o$ stands for the opposite category. For a given objects $X$ and $Y$ by $\operatorname{Hom}_{\mathbf{C}}(X, Y)$ we denote the set of morphisms from $X$ to $Y$. Often we shall write $\phi:X\to Y$ instead of $\phi\in\operatorname{Hom}_{\mathbf{C}}(X,Y)$. A morphism $\phi:X\to Y$ is called $\langle$~retraction / coretraction~$\rangle$ if it has a $\langle$~right / left~$\rangle$ inverse morphism. Morphism $\phi$ is called an isomorphism if it is retraction and coretraction. Usually we shall express existence of isomorphism between $X$ and $Y$ as $X\isom{\mathbf{C}} Y$. We say that two morphisms $\phi:X_1\to Y_1$ and $\psi:X_2\to Y_2$ are equivalent in $\mathbf{C}$ if there exist isomorphisms $\alpha:X_1\to X_2$ and $\beta:Y_1\to Y_2$ such that $\beta\phi=\psi\alpha$.

The first obvious example of the category that comes to mind is the category of all sets and all maps between them. We denote this category by $\mathbf{Set}$. Other examples will be given later. Two main examples of functors that any category has are functors of morphisms. For a fixed $X\in\operatorname{Ob}(\mathbf{C})$ we define covariant and a contravariant functors
$$
\operatorname{Hom}_{\mathbf{C}}(X,-):\mathbf{C}\to\mathbf{Set}:Y\mapsto \operatorname{Hom}_{\mathbf{C}}(X,Y), \phi\mapsto(\psi\mapsto \phi\circ\psi),
$$
$$
\operatorname{Hom}_{\mathbf{C}}(-,X):\mathbf{C}\to\mathbf{Set}:Y\mapsto \operatorname{Hom}_{\mathbf{C}}(Y,X), \phi\mapsto(\psi\mapsto \psi\circ\phi).
$$
This construction has its reminiscent analogs in many categories of mathematics with slight modification of categories between which these functors act.

We say that two covariant functors $F:\mathbf{C}\to\mathbf{D}$, $G:\mathbf{C}\to\mathbf{D}$ are isomorphic if there exists a class of isomorphisms $\{\eta_X:X\in\operatorname{Ob}(\mathbf{C})\}$ in $\mathbf{D}$ (called natural isomorphisms), such that $G(f)\eta_X=\eta_Y F(f)$ for all $f:X\to Y$. In this case we simply write $F\cong G$. A $\langle$~covariant / contravariant~$\rangle$ functor $F:\mathbf{C}\to\mathbf{D}$ is called representable by object $X$ if $\langle$~$F\cong\operatorname{Hom}_{\mathbf{C}}(X,-)$ / $F\cong\operatorname{Hom}_{\mathbf{C}}(-,X)$~$\rangle$. If functor is representable, then its representing object is unique up to isomorphism in $\mathbf{C}$.

Constructions of categorical product and coproduct shall play an important role in this thesis. We say that $X$ is the the $\langle$~product / coproduct~$\rangle$ of the family of objects $\{X_\lambda:\lambda\in\Lambda\}$ if the functor $\langle$~$\prod_{\lambda\in\Lambda}\operatorname{Hom}_{\mathbf{C}}(-,X_{\lambda}):\mathbf{C}\to\mathbf{Set}$ / $\prod_{\lambda\in\Lambda}\operatorname{Hom}_{\mathbf{C}}(X_{\lambda},-):\mathbf{C}\to\mathbf{Set}$~$\rangle$ is representable by object $X$. As the consequence we get that a $\langle$~product / coproduct~$\rangle$, if it exists, is unique up to an isomorphism. Later we shall give examples of the $\langle$~products / coproducts~$\rangle$ in different categories of functional analysis. 


%----------------------------------------------------------------------------------------
%	Topology
%----------------------------------------------------------------------------------------

\subsection{Topology}
\label{SubSectionTopology}

Let $(S,\tau)$ be a topological space. Elements of $\tau$ are called open sets, and their complements are called closed sets. Let $E$ be an arbitrary subset of $S$. By $\operatorname{cl}_S(E)$ we denote the closure of $E$ in $S$, that is the smallest closed set that contains $E$. Similarly, by $\operatorname{int}_S(E)$ we denote the interior of $E$, that is the largest open set that contained in $E$. We say that $E$ is a neighborhood of point $s\in S$ if $s\in \operatorname{int}_S(E)$. We say that a set $E$ is dense in a set $F$ if $F\subset\operatorname{cl}_S(E)$. Note that $E$ may be regarded as topological space, if we endow it with subspace topology which equals $\{U\cap E:U\in\tau\}$. Topological space is called Hausdorff if any two distinct points have disjoint open neighborhoods. In this thesis we shall work with Hausdorff spaces only.

A map $f:X\to Y$ between topological spaces is called continuous if preimage under $f$ of any open set is open. By $\mathbf{Top}$ we denote the category of topological spaces with continuous maps in the role of morphisms. Isomorphisms in $\mathbf{Top}$ are called homeomorphisms. The category $\mathbf{Top}$ admits  products. For a given family of topological spaces $\{S_\lambda:\lambda\in\Lambda\}$ their product is the Tychonoff product $\prod_{\lambda\in\Lambda}S_\lambda$, that is the Cartesian product of the family $\{S_\lambda:\lambda\in\Lambda\}$ with the coarsest topology making all natural projections $p_\lambda:\prod_{\lambda\in\Lambda}S_\lambda\to S_\lambda$ continuous.

In this thesis we shall encounter four types of topological spaces: compact spaces, paracompact spaces, locally compact spaces and extremely disconnected spaces. Before giving their definitions we need to recall the notion of cover. Let $\mathcal{E}$ be a family of subsets of topological space $S$. We say that $\mathcal{E}$ is a cover if its union equals $S$. We say that cover is open if all its elements are open sets. A cover is called locally finite if any point of $S$ has a neighborhood that intersects only finitely many elements of the cover. We say that cover $\mathcal{E}_1$ is inscribed into cover $\mathcal{E}_2$ if any element of $\mathcal{E}_1$ is a subset of some element of $\mathcal{E}_2$. A cover $\mathcal{E}_1$ is called a subcover of $\mathcal{E}_2$ if $\mathcal{E}_1\subset\mathcal{E}_2$. Finally, a topological space is called 

$i)$ compact if any its open cover admits a finite open subcover; 

$ii)$ paracompact if any its open cover is inscribed into some locally finite open cover; 

$iii)$ locally compact if any its point has a compact neighborhood;

$iv)$ extremely disconnected spaces if the closure of any its open set is open;

$v)$ Stonean if it is an extremely disconnected Hausdorff compact space.

The property of being $\langle$~compact / paracompact / locally compact~$\rangle$ space is preserved by $\langle$~closed / closed / open and closed~$\rangle$ subspaces. Any non compact locally compact Hausdorff space $S$ can be regarded as dense subspace of some compact Hausdorff space. There is the smallest and the largest such compactification. The smallest one is called the Alexandroff compactification $\alpha S$.  By definition $\alpha S:=S\cup \{S\}$. A subset of $\alpha S$ is called open if it is an open subset of $S$ or has the form $\{S\}\cup S\setminus K$ for some compact set $K\subset S$. The largest compactification $\beta S$ is called the Stone-Cech compactification. It may be represented as the image of the embedding $j:S\to\prod_{f\in C}[0,1]:s\mapsto \prod_{f\in C}f(s)$, where $C$ is a set of all continuous maps from $S$ to $[0,1]$. Stone-Cech compactification is highly non constructive. Even $\beta\mathbb{N}$ has no explicit description, though it is known that $\beta\mathbb{N}$ is an extremely disconnected Hausdorff compact.

Occasionally we shall apply the Urysohn's lemma to locally compact Hausdorff spaces. It states that for any compact subset $K$ of open set $V$ in a locally compact Hausdorff space $S$ there exists a continuous function $f:S\to [0,1]$ such that $f|_K=1$ and $f|_{S\setminus V}=0$. 

For more details on topological spaces see comprehensive treatise \cite{EngelGenTop}. 

%----------------------------------------------------------------------------------------
%	Filters, nets and limits
%----------------------------------------------------------------------------------------

\subsection{Filters, nets and limits}

\label{SubSectionFiltersNetsAndLimits} 

We will use two generalizations of the notion of the sequence and the limit of the sequence.

A family $\mathfrak{F}$ of subsets of the set $M$ is called a filter on a set $M$ if $\mathfrak{F}$ doesn't contain the empty set, $\mathfrak{F}$ is closed under finite intersections and $\mathfrak{F}$ contains all supersets of its elements. In general filters are too large to be described explicitly. To overcome this difficulty we shall use filterbases. A non empty family $\mathfrak{B}$ of subset of a set $M$ is called a filterbase on a set $M$ if $\mathfrak{B}$ doesn't contain empty set and the intersection of any two elements of $\mathfrak{B}$ contains some element of $\mathfrak{B}$. Given a filterbase $\mathfrak{B}$ we can construct a filter by adding to $\mathfrak{B}$ all supersets of elements of $\mathfrak{B}$.

We say that filter $\mathfrak{F}_1$ dominates filter $\mathfrak{F}_2$ if $\mathfrak{F}_2\subset\mathfrak{F}_1$. Therefore the set of all filters on a given set is partially ordered set. Filters that are maximal with respect to this order are called ultrafilters. An easy application of Zorn's lemma gives that any filter is dominated by some ultrafilter.

Let $\mathfrak{F}$ be a filter on  a set $M$, and $\phi:M\to S$ be a map from $M$ to the Hausdorff topological space $S$. We say that $x$ is a limit of $\phi$ along $\mathfrak{F}$ and write $x=\lim_{\mathfrak{F}} \phi(m)$ if for every open neighborhood $U$ of $x$ holds $\phi^{-1}(U)\in\mathfrak{F}$. Directly from the definition it follows that if $\phi$ has a limit along $\mathfrak{F}$ then it has the same limit along any filter that dominates $\mathfrak{F}$. 

Limit along filter preserve order structure of $\mathbb{R}$. More precisely: if two functions $\phi:M\to\mathbb{R}$ and $\psi:M\to\mathbb{R}$ have limits along filter $\mathfrak{F}$ and $\phi\leq\psi$, then 
$$
\lim_{\mathfrak{F}}\phi(m)\leq\lim_{\mathfrak{F}}\psi(m).
$$

Limit along filter respects continuous functions. Rigorously this formulates as follows. Assume for each $\lambda\in\Lambda$ a function $\phi_\lambda:M\to S_\lambda$ has a limit along filter $\mathfrak{F}$, then for any continuous function $g:\prod_{\lambda\in\Lambda}S_\lambda\to Y$ holds
$$
\lim_{\mathfrak{F}}g\left(\prod_{\lambda\in\Lambda}\phi_\lambda(m)\right)=g\left(\prod_{\lambda\in\Lambda}\lim_{\mathfrak{F}}\phi_\lambda(m)\right).
$$
In particular limit along filter is linear and multiplicative. Just like ordinary sequences.

The most important feature of filters and the reason of our interest is the following: if $\mathfrak{U}$ is an ultrafilter on the set $M$ and $\phi:M\to K$ is a function with values in the compact Hausdorff space $K$, then $\lim_{\mathfrak{U}}\phi(m)$ exists. In particular, we always can speak of limits along ultrafilters of bounded scalar valued functions.

Another approach to the generalization of the notion of the limit is a limit of the net. A directed set is a partially ordered set $(N,\leq)$ in which any two elements have upper bound. Every directed set gives rise to the so called section filter, whose filterbase consist of so called sections $\{\nu':\nu\leq\nu'\}$ for some $\nu\in N$.
Any function $x:N\to X$ from the directed set $(N,\leq)$ into the topological space $X$ is called a net. Usually it is denoted as $(x_\nu)_{\nu\in N}$ to allude to sequences. A limit of the net $x:N\to X$ is a limit of the function $x$ along section filter of the directed set $N$. It is denoted $\lim_\nu x_\nu$. We shall exploit both notions of the limit.

More on this matters can be found in [\cite{BourbElemMathGenTopLivIII}, section 7].


%----------------------------------------------------------------------------------------
%	Measure theory basics
%----------------------------------------------------------------------------------------

\subsection{Measure theory}
\label{SubSectionMeasureTheory}

A family $\Sigma$ of subsets of the set $\Omega$ is called a $\sigma$-algebra if it contains an empty set, contains complements of all its elements and closed under countable unions. If $\Sigma$ is a $\sigma$-algebra of subsets of $\Omega$ we call $(\Omega,\Sigma)$ a measurable subspace. Elements of $\Sigma$ are called measurable sets. 

A function $\mu:\Sigma\to[0,+\infty]$ such that:  

$i)$ $\mu(\varnothing)=0$; 

$ii)$ $\mu\left(\bigcup\limits_{n\in\mathbb{N}} E_n\right)=\sum\limits_{n\in\mathbb{N}}\mu(E_n)$ for any family of disjoint sets $(E_n)_{n\in\mathbb{N}}$ in $\Sigma$; 

is called a measure. The triple $(\Omega,\Sigma,\mu)$ is called a measure space. If $\mu$ attains only finite values we may drop the first condition. The second condition is essential and called the $\sigma$-additivity. The simplest example of measure space is an  arbitrary set $\Lambda$ with $\sigma$-algebra of all subsets and so called counting measure $\mu_c:\mathcal{P}(\Lambda)\to[0,+\infty]$. By definition $\mu_c(E)$ equals $\operatorname{Card}(E)$ if $E$ is finite and $+\infty$ otherwise. If $E$ is a measurable set, by $\Sigma|_E$ we denote the $\sigma$-algebra $\{F\cap E:F\in\Sigma\}$ and by $\mu|_E$ we denote the restriction of $\mu$ to $\Sigma|_E$. A set $E$ in $\Omega$ is called negligible if there exists a measurable set $F$ of measure $0$ that contains $E$. Similarly, a set $E$ in $\Omega$ is conegligible if $\Omega\setminus E$ is negligible. Let $P$ be some property that depends on points of $\Omega$. We say that $P$ holds almost everywhere if the set where $P$ is violated is negligible. A measure space is called $\sigma$-finite if there exists a countable family of measurable sets of finite measure whose union is the whole space. The class of $\sigma$-finite spaces is enough for most applications but we shall encounter a more generic measure spaces.

A measurable space $(\Omega,\Sigma,\mu)$ is called strictly localizable if there exists a  family of disjoint measurable subsets $\{E_\lambda:\lambda\in\Lambda\}$ of finite measure such that: 

$i)$ $\bigcup_{\lambda\in\Lambda}E_\lambda=\Omega$;

$ii)$ $E$ is measurable iff $E\cap E_\lambda$ is measurable for all $\lambda\in\Lambda$;

$iii)$ for any measurable $E$ holds $\mu(E)=\sum_{\lambda\in\Lambda}\mu(E\cap E_\lambda)$. 
  
  The class of strictly localizable measure spaces is huge. It includes all $\sigma$-finite measure spaces, their arbitrary unions, Haar measures of locally compact groups, counting measures and much more. In what follows we shall consider only strictly localizable measure spaces.

We shall exploit a more detailed classification of measure spaces. We say that a measurable set $E$ is an atom if $\mu(E)>0$ and for any measurable subset $F$ of $E$ either $F$ or $E\setminus F$ is negligible. Directly from the definition it follows that that all atoms of strictly localizable measure spaces are of finite measure. In general an atom may not be a mere singleton.

We say that a measure space is non atomic if there is no atoms for its measure. A measure space is called purely atomic if every measurable set of positive measure contains an atom. A straightforward application of Zorn's lemma gives that a purely atomic measure space can be represented as disjoint union of some family of atoms. This family is countable if measure space is $\sigma$-finite. These facts allow us to say that the structure of purely atomic measure space is well understood. 

The structure of strictly localizable non atomic measure spaces is given by Maharam's theorem [\cite{FremMeasTh}, 332B]. We shall exploit only the following property of non atomic measures [\cite{FremMeasTh}, proposition 215D]: if $E$ is a measurable set of positive measure in a non atomic measure space, then for all $0\leq c\leq \mu(E)$ there exists a measurable subset $F$ of $E$ such that $\mu(F)=c$.

For completeness we shall say a few words on constructions with measures. The product measure of two measure spaces $(\Omega_1,\Sigma_1,\mu_1)$ and $(\Omega_2,\Sigma_2,\mu_2)$ we denote by $\mu_1\times \mu_2$. The definition of product measure for localizable measure spaces is rather involved [\cite{FremMeasTh}, definition 251F] and we don't give it here.  For our purposes it is enough to know that the product of two strictly localizable measure spaces is again strictly localizable [\cite{FremMeasTh}, proposition 251N]. By direct sum of measure spaces $\{(\Omega_\lambda, \Sigma_\lambda, \mu_\lambda):\lambda\in\Lambda\}$ we denote disjoint union of set $\{\Omega_\lambda:\lambda\in\Lambda\}$ with $\sigma$-algebra defined as $\Sigma=\{E\subset \Omega: E\cap E_\lambda\in\Sigma_\lambda\mbox{ for all }\lambda\in\Lambda\}$ and measure given by the formula $\mu(E)=\sum_{\lambda\in\Lambda}\mu_\lambda(E\cap E_\lambda)$. It is clear now that strictly localizable measure space are exactly direct sums of finite measure spaces.

Assume $(\Omega,\Sigma,\mu)$ is a $\sigma$-finite measure space, then there exists a purely atomic measure $\mu_1:\Sigma\to[0,+\infty]$ and a non atomic measure $\mu_2:\Sigma\to[0,+\infty]$ such that $\mu=\mu_1+\mu_2$. Even more there exist measurable sets $\Omega_a^{\mu}$ and $\Omega_{na}^{\mu}=\Omega\setminus \Omega_a^{\mu}$ such that $\mu_1(\Omega_{na}^{\mu})=\mu_2(\Omega_a^{\mu})=0$. The sets $\Omega_a^{\mu}$ and $\Omega_{na}^{\mu}$ are called respectively the atomic and the non atomic parts of measure space $(\Omega,\Sigma,\mu)$.

By measurable function we always mean a complex or real valued function on measurable space, with the property that preimage of every open set is measurable. We say that two measurable functions are equivalent if the set where they are different is negligible. If $f:\Omega\to\mathbb{R}$ is an  integrable function on $(\Omega,\Sigma,\mu)$, then we may define a new measure 
$$
f\mu:\Sigma\to[0,+\infty]:E\mapsto\int_{E}f(\omega)d\mu(\omega).
$$

The notion of measure can be extended by changing the range of values that the measure can attain. Any $\sigma$-additive function $\mu:\Sigma\to\mathbb{C}$ on a measurable space $(\Omega,\Sigma)$ is called a complex measure. Any complex measure $\mu$ can be represented as $\mu=\mu_1-\mu_2+i(\mu_3-\mu_4)$, where $\mu_1,\mu_2,\mu_3,\mu_4$ --- are finite measures. As the consequence every complex measure is finite and therefore we have a well defined total variation measure:
$$
|\mu|:\Sigma\to[0,+\infty):E\mapsto\sup\left\{\sum_{n\in\mathbb{N}}|\mu(E_n)|:\{E_n:n\in\mathbb{N}\}\subset\Sigma -\mbox{partition of }E\right\}
$$

Let $\mu$ and $\nu$ be two measures on a measurable space $(\Omega,\Sigma)$. We say that $\mu$ and $\nu$ are mutually singular and write $\mu\perp\nu$ if there exists a measurable set $E$ such that $\mu(E)=\nu(\Omega\setminus E)=0$. The opposite property is absolute continuity. We say that $\nu$ is absolutely continuous with respect to $\mu$ and write $\nu\ll\mu$ if $\nu(E)=0$ for every measurable set $E$ with $\mu(E)=0$. In general, two measures may neither be absolutely continuous nor singular with respect to each other. We have a Lebesgue decomposition theorem for this case. For a given two $\sigma$-finite measures $\mu$ and $\nu$ on a measurable space $(\Omega,\Sigma)$ there exists a measurable function $\rho_{\nu,\mu}:\Omega\to\mathbb{C}$, a $\sigma$-finite measure $\nu_s:\Sigma\to[0,+\infty]$ and two measurable sets $\Omega_s^{\nu,\mu}$, $\Omega_c^{\nu,\mu}=\Omega\setminus\Omega_s^{\nu,\mu}$ such that
$\nu=\rho_{\nu,\mu}\mu+\nu_s$ and $\mu(\Omega_s^{\nu,\mu})=\nu_s(\Omega_c^{\nu,\mu})=0$, i.e. $\mu\perp\nu_s$.

Finally we shall say a few words on measures defined on topological spaces. Given a topological space $S$ we may consider the minimal $\sigma$-algebra containing all open subsets of $S$. It is called the Borel $\sigma$-algebra of $S$ and denoted by $Bor(S)$. Measures and complex measures defined on Borel $\sigma$-algebras are supported with adjective Borel. We shall not consider measures on general topological spaces and stick with locally compact Hausdorff spaces. This significantly simplifies further considerations. We say that a complex Borel measure $\mu:Bor(S)\to\mathbb{C}$ defined on a locally compact Hausdorff space $S$ is regular if for any Borel set $E$ and any $\epsilon>0$ there exists a compact set $K\subset E$ such that $|\mu|(E\setminus K)<\epsilon$. The support of complex Borel measure $\mu$ is a set of all points $s\in S$ for which every open neighborhood of $s$ has positive measure. We denote the set of such points by $\operatorname{supp}(\mu)$. The support is always closed. 

Most of the results and definitions in this section can be found in the first, second and the fourth volumes of \cite{FremMeasTh}.

%----------------------------------------------------------------------------------------
%	Banach spaces, algebras and modules
%----------------------------------------------------------------------------------------

\section{Banach structures}

\label{SectionBanachStructures}

%----------------------------------------------------------------------------------------
%	Banach spaces
%----------------------------------------------------------------------------------------

\subsection{Banach spaces}
\label{SubSectionBanachSpaces}

We assume that our reader is familiar with fundamentals of functional analysis and its constructions, otherwise consult \cite{HelLectAndExOnFuncAn} or \cite{ConwACoursInFuncAn}. In this thesis we will highly rely on results about geometry of Banach spaces. See \cite{CarothShortCourseBanSp}, \cite{KalAlbTopicsBanSpTh} or \cite{FabHabBanSpTh} for a quick introduction. All Banach spaces are considered over complex field, unless otherwise stated. 

By $\langle$~$B_E$ / $B_E^\circ$~$\rangle$ we denote the $\langle$~closed / open~$\rangle$ unit ball of Banach space $E$ with center at zero. If $F$ is a closed subspace of $E$, then $E/F$ stands for the quotient Banach space. By $E^{cc}$ we denote a Banach space with the same set of vectors as in $E$, the same addition but with new multiplication by conjugate scalars:  $\alpha \overline{x}:=\overline{\overline{\alpha}x}$ for $\alpha\in\mathbb{C}$ and $x\in E$. Note: elements of $E^{cc}$ we denote by $\overline{x}$. Clearly, $(E^{cc})^{cc}=E$.

Now fix two Banach spaces $E$ and $F$. A map $T:E\to F$ is called conjugate linear if the respective map $T:E^{cc}\to F$ is linear. A linear operator $T:E\to F$ is called:

$i)$ bounded if its norm $\Vert T\Vert:=\sup\{\Vert T(x)\Vert:x\in B_E\}$ is finite.

$ii)$ contractive if its norm is at most $1$;

$iii)$ compact if $T(B_E)$ is relatively compact in $F$;

$iv)$ nuclear if it can be represented as an absolutely convergent series of rank one operators.

Is well known that any nuclear operator is compact. any compact operator is bounded, any bounded operator is continuous. By $\langle$~$\mathcal{B}(E,F)$ / $\mathcal{K}(E,F)$ / $\mathcal{N}(E,F)$~$\rangle$ we denote the Banach space of $\langle$~bounded / compact / nuclear~$\rangle$ linear operators from $E$ to $F$. If $F=E$ we use the shortcut $\langle$~$\mathcal{B}(E)$ / $\mathcal{K}(E)$ / $\mathcal{N}(E)$~$\rangle$ for this space. The norms in $\mathcal{B}(E,F)$ and $\mathcal{K}(E,F)$ are just the usual operator norm. The norm of a nuclear operator $T$ is defined by equality
$$
\Vert T\Vert:=\inf\left\{\sum_{n=1}^\infty\Vert S_n\Vert: T=\sum_{n=1}^\infty S_n,\quad (S_n)_{n\in\mathbb{N}} - \mbox{ rank one operators}\right\}.
$$

By $\mathbf{Ban}$ we shall denote the category of Banach spaces with bounded linear operators in the role of morphisms, while $\mathbf{Ban}_1$ stands for the category of Banach spaces with contractive operators in the role of morphisms. As the consequence, $\operatorname{Hom}_{\mathbf{Ban}}(E,F)$ is just another name for $\mathcal{B}(E,F)$.

Let $E$, $F$ and $G$ be three Banach spaces, then a bilinear operator $\Phi:E\times F\to G$ is called bounded if its norm $\Vert \Phi\Vert:=\sup\{\Vert \Phi(x,y)\Vert:x\in B_E, y\in B_F\}$ is finite. The Banach space of all bounded bilinear operators on $E\times F$ with values in $G$ is denoted by $\mathcal{B}(E\times F,G)$.

A few words on classification of bounded linear operators. A bounded linear operator $T:E\to F$ is called:

$i)$ topologically injective if it performs homeomorphism on its image;

$ii)$ topologically surjective if it is an open map;

$iii)$ coisometric if it maps open unit ball onto open unit ball;

$iv)$ strictly coisometric if it maps closed unit ball onto closed unit ball. 

But we shall refine these definitions. We say a bounded linear operator $T:E\to F$ is:

$v)$ $c$-topologically injective, if $\Vert x\Vert\leq c\Vert  T(x)\Vert$ for all $x\in E$;

$vi)$ $c$-topologically surjective, if $cT(B_E^\circ)\supset B_F^\circ$;

$vii)$ strictly $c$-topologically surjective, if $cT(B_E)\supset B_F$; 

Note that $T$ is topologically $\langle$~injective / surjective~$\rangle$ iff it is $c$-topologically $\langle$~injective / surjective~$\rangle$ for some $c>0$. Obviously $\langle$~coisometric / strictly coisometric~$\rangle$ operators are exactly contractive $\langle$~$1$-topologically surjective / strictly $1$-topologically surjective~$\rangle$ operators.

Two Banach spaces $E$ and $F$ are $\langle$~isometrically isomorphic / topologically isomorphic~$\rangle$ as Banach spaces if there exists a bounded linear operator $T:E\to F$ which is both $\langle$~isometric and surjective / topologically injective and topologically surjective~$\rangle$. The fact that $E$ and $F$ are $\langle$~isometrically isomorphic / topologically isomorphic~$\rangle$ Banach spaces means that $\langle$~$E\isom{\mathbf{Ban}_1}F$ / $E\isom{\mathbf{Ban}}F$~$\rangle$. The Banach-Mazur distance between $E$ and $F$ is defined by the formula 
$$
d_{BM}(E,F):=\inf\{\Vert T\Vert\Vert T^{-1}\Vert: T \in \mathcal{B}(E,F) \mbox{ --- a topological isomorphism}\}.
$$ 
If $E$ and $F$ are not topologically isomorphic the Banach-Mazur distance between them is infinite.

One more important class of operators is the class of bounded projections. A bounded linear operator $P:E\to E$ is called a projection if $P^2=P$. If $F=P(E)$, then we say that $P$ is a projection from $E$ onto $F$ and $F$ is complemented in $E$. If $\Vert P\Vert\leq c$ we say that $F$ is $c$-complemented in $E$. Finally, we say that $F$ is contractively complemented in $E$ if it is $1$-complemented in $E$. Another equivalent characterization says that $F$ is a complemented subspace of Banach space $E$ if there exists a closed subspace $G$ in $E$ such that $E\isom{\mathbf{Ban}}F\bigoplus G$. All finite dimensional subspaces are complemented, but not necessarily contractively complemented. An example of contractively complemented subspace is the following: consider arbitrary Banach space $E$, then $E^*$ is contractively complemented in $E^{***}$ via Dixmier projection $P=\iota_{E^*}(\iota_E)^*$, where $\iota_E$ is the natural embedding of $E$ into its second dual $E^{**}$. A canonical example of uncomplemented subspace is $c_0(\mathbb{N})$ in $\mathbb{\ell_\infty}(\mathbb{N})$  [\cite{KalAlbTopicsBanSpTh}, theorem 2.5.5].

Now consider the algebraic tensor product $E\otimes F$ of Banach spaces $E$ and $F$. This linear space can be endowed with different norms, but the most important is the projective norm. For $u\in E\otimes F$ we define its projective norm as
$$
\Vert u\Vert:=\inf\left\{\sum_{i=1}^n \Vert x_i\Vert\Vert y_i\Vert: u=\sum_{i=1}^n x_i\otimes y_i, (x_i)_{i\in\mathbb{N}_n}\subset E, (y_i)_{i\in\mathbb{N}_n}\subset F\right\}
$$
It is indeed a norm, but not complete in general. The symbol $E\projtens F$ stands for the completion of $E\otimes F$ under projective norm. We call the resulting completion the projective tensor product of Banach spaces $E$ and $F$. Let $T:E_1\to E_2$ and $S:F_1\to F_2$ be two bounded linear operators between Banach spaces, then there exists a unique bounded linear operator $T\projtens S:E_1\projtens F_1\to E_2\projtens F_2$ such that $(T\projtens S)(x\projtens y)=T(x)\projtens S(y)$ for all $x\in E_1$ and $y\in F_1$. Even more $\Vert T\projtens S\Vert=\Vert T\Vert\Vert S\Vert$. The main feature of projective tensor product which makes it so important is the following universal property: for any Banach spaces $E$, $F$ and $G$ there is a natural isometric isomorphism:
$$
\mathcal{B}(E\projtens F,G)\isom{\mathbf{Ban}_1}\mathcal{B}(E\times F,G)
$$
In other words, projective tensor product linearizes bounded bilinear operators. Also we have the following two (natural in $E$, $F$ and $G$) isometric isomorphisms:
$$
\mathcal{B}(E\projtens F,G)
\isom{\mathbf{Ban}_1}\mathcal{B}(E,\mathcal{B}(F,G))
\isom{\mathbf{Ban}_1}\mathcal{B}(F,\mathcal{B}(E,G))
$$
The last isomorphism is called the law of adjoint associativity. There are many other tensor norms on the algebraic tensor product of Banach spaces. Their thorough treatment can be found in \cite{DiestMetTheoryOfTensProd}.

Now we are able to craft four very important functors:
$$
\mathcal{B}(-,E):\mathbf{Ban}\to\mathbf{Ban}
\qquad\qquad
\mathcal{B}(E,-):\mathbf{Ban}\to\mathbf{Ban}
$$
$$
-\projtens E:\mathbf{Ban}\to\mathbf{Ban}
\qquad\qquad
E\projtens -:\mathbf{Ban}\to\mathbf{Ban}.
$$
We shall often encounter them. For example, the well known adjoint functor ${}^*$ is nothing more than $\mathcal{B}(-,\mathbb{C})$. All these functors have their obvious analogs on $\mathbf{Ban}_1$.

Now we  proceed to classical examples of Banach spaces. 

An important source of examples of Banach spaces are $L_p$-spaces, also known as Lebesgue spaces. A detailed discussion of basic properties of $L_p$-spaces can be found in \cite{CarothShortCourseBanSp}.  Let $(\Omega,\Sigma,\mu)$ be a measure space.  For $1\leq p<\infty$, as usually, the symbol $L_p(\Omega,\mu)$ stands for the Banach space of equivalence classes of functions $f:\Omega\to\mathbb{C}$ such that $|f|^p$ is Lebesgue integrable with respect to measure $\mu$. The norm of such function is defined as
$$
\Vert f\Vert:=\left(\int\limits_{\Omega}|f(\omega)|^pd\mu(\omega)\right)^{1/p}.
$$ 
By $L_\infty(\Omega,\mu)$ we denote the Banach space of equivalence classes of bounded measurable functions with norm defined as 
$$
\Vert f\Vert:=\inf\left\{\sup_{\omega\in\Omega\setminus N}|f(\omega)|:N\subset\Omega - \mbox{is negligible}\right\}.
$$
For simplicity we shall speak of functions in $L_p(\Omega,\mu)$ instead of their equivalence classes. All equalities and inequalities about functions of $L_p$-spaces are understood up to negligible sets. It is well know that $L_p(\Omega,\mu)^*\isom{\mathbf{Ban}_1}L_{p^*}(\Omega,\mu)$ for $1\leq p<+\infty$ [\cite{FremMeasTh}, theorems 243G, 244K]. One more well known fact is that, $L_p$-spaces are reflexive for $1<p<+\infty$. Here we exploited the standard notation $p^*=+\infty$ if $p=1$ and $p^*=p/(p-1)$ if $1<p<+\infty$. Clearly, $p^{**}=p$ for $1<p<+\infty$.

The most well known classes of Banach spaces are related to continuous functions. Let $S$ be a locally compact Hausdorff space. We say that a function $f:S\to\mathbb{C}$ vanishes at infinity  if for any $\epsilon>0$ there exists a compact $K\subset S$ such that $|f(s)|\leq\epsilon$ for all $s\in S\setminus K$. The linear space of continuous functions on $S$ vanishing at infinity is denoted by $C_0(S)$. When endowed with $\sup$-norm $C_0(S)$ becomes a Banach space. Any set $\Lambda$ with discrete topology may be regarded as a locally compact space and following the traditional notation we shall write $c_0(\Lambda)$ instead of $C_0(\Lambda)$. If $K$ is a compact Hausdorff space then all functions on $K$ vanish at infinity. We use the notation $C(K)$ for $C_0(K)$ to indicate that $K$ is compact. Some Banach spaces in fact are $C(K)$-spaces in disguise. For example, if we are given a measure space $(\Omega,\Sigma,\mu)$, then $B(\Omega,\Sigma)$ --- the space of bounded measurable functions with $\sup$-norm or $L_\infty(\Omega,\mu)$ are $C(K)$ spaces for some compact Hausdorff space $K$ [\cite{KalAlbTopicsBanSpTh}, remark 4.2.8]. If $S$ is a locally compact Hausdorff space, then $M(S)$ stands for the Banach space of complex finite Borel regular measures on $S$. The norm of measure $\mu\in M(S)$ is defined by equality $\Vert\mu\Vert=|\mu|(S)$, where $|\mu|$ is a total variation measure of measure $\mu$. By Riesz-Markov-Kakutani theorem  [\cite{ConwACoursInFuncAn}, section C.18] we have $C_0(S)^*\isom{\mathbf{Ban}_1}M(S)$. In fact $M(S)$ is an $L_1$-space, see discussion after [\cite{DalLauSecondDualOfMeasAlg}, proposition 2.14]. 

We shall also mention one important specific case of $L_p$-spaces. For a given index set $\Lambda$ and a counting measure $\mu_c:\mathcal{P}(\Lambda)\to[0,+\infty]$ the respective $L_p$-space is denoted by $\ell_p(\Lambda)$. For this type of measure spaces we have one more important isomorphism $c_0(\Lambda)^*\isom{\mathbf{Ban}_1}\ell_1(\Lambda)$. For convenience we define $c_0(\varnothing)=\ell_p(\varnothing)=\{0\}$ for $1\leq p\leq+\infty$. This example motivates the following construction.

Let $\{E_\lambda:\lambda\in\Lambda\}$ be an arbitrary family of Banach spaces. For each $x\in \prod_{\lambda\in\Lambda} E_\lambda$ we define
$\Vert x\Vert_p=\Vert(\Vert x_\lambda\Vert)_{\lambda\in\Lambda}\Vert_{\ell_p(\Lambda)}$ for $1\leq p\leq +\infty$ and $\Vert x\Vert_0=\Vert(\Vert x_\lambda\Vert)_{\lambda\in\Lambda}\Vert_{c_0(\Lambda)}$. Then the Banach space $\left\{x\in \prod_{\lambda\in\Lambda} E_\lambda: \Vert x\Vert_p<+\infty\right\}$ with the norm $\Vert\cdot\Vert_p$ is denoted by $\bigoplus_p\{E_\lambda:\lambda\in\Lambda\}$. We call these objects $\bigoplus_p$-sums of Banach spaces $\{E_\lambda:\lambda\in\Lambda\}$. It is almost tautological that the Banach space $\ell_p(\Lambda)$ is the $\bigoplus_p$-sum of the family $\{\mathbb{C}:\lambda\in\Lambda\}$. A nice property of $\bigoplus_p$-sums is their interrelation with duality:
$$
\left(\bigoplus\nolimits_p\{E_\lambda:\lambda\in\Lambda\}\right)^*\isom{\mathbf{Ban}_1}
\bigoplus\nolimits_{p^*}\{E_\lambda^*:\lambda\in\Lambda\}
$$
for all $1\leq p<+\infty$ and 
$$
\left(\bigoplus\nolimits_0\{E_\lambda:\lambda\in\Lambda\}\right)^*\isom{\mathbf{Ban}_1}
\bigoplus\nolimits_1\{E_\lambda^*:\lambda\in\Lambda\}
$$
If $\{T_\lambda\in\mathcal{B}(E_\lambda, F_\lambda):\lambda\in\Lambda\}$ is a family of bounded linear operators, then for all $1\leq p\leq+\infty$ and $p=0$ we have a well defined linear operator
$$
T:\bigoplus\nolimits_p\{E_\lambda:\lambda\in\Lambda\}\to \bigoplus\nolimits_p\{ F_\lambda:\lambda\in\Lambda\}:x\mapsto \bigoplus\nolimits_p\{ T_\lambda(x_\lambda):\lambda\in\Lambda\}
$$
which we shall denote by $\bigoplus_p\{T_\lambda:\lambda\in\Lambda\}$. Its norm equals $\sup_{\lambda\in\Lambda}\Vert T_\lambda\Vert$.

Among different $\bigoplus_p$-sums the $\langle$~$\bigoplus_1$-sums / $\bigoplus_\infty$-sums~$\rangle$ play a special role in Banach space theory. The reason is that any family of Banach spaces admit $\langle$~product / coproduct~$\rangle$ in $\mathbf{Ban}_1$ which in fact is their $\langle$~$\bigoplus_1$-sum / $\bigoplus_\infty$-sum~$\rangle$. The same statement holds for $\mathbf{Ban}$ if we restrict ourselves to finite families of objects [\cite{HelLectAndExOnFuncAn}, chapter 2, section 5].

We proceed to advanced topics of Banach space theory. Below we shall discuss several geometric properties of Banach spaces such as the property of being an $\mathscr{L}_p^g$-space, weak sequential completeness, the Dunford-Pettis property, the l.u.st. property and the approximation property. In what follows, imitating Banach space geometers, we shall say that a Banach space $E$ contains $\langle$~an isometric copy / a copy~$\rangle$ of Banach space $F$ if $F$ is $\langle$~isometrically isomorphic / topologically isomorphic~$\rangle$ to some closed subspace of $E$.

Let $1\leq p\leq +\infty$. We say that $E$ is an $\mathscr{L}_{p,C}^g$-space if for any $\epsilon>0$ and any finite dimensional subspace $F$ of $E$ there exists a finite dimensional $\ell_p$-space $G$ and two bounded linear operators $S:F\to G$, $T:G\to E$ such that $TS|^F=1_F$ and $\Vert T\Vert\Vert S\Vert\leq C+\epsilon$. If $E$ is an $\mathscr{L}_{p,C}^g$-space for some $C\geq 1$ we simply say, that $E$ is an $\mathscr{L}_p^g$-space. This definition [\cite{DefFloTensNorOpId}, definition 23.1] is an improvement of the definition of $\mathscr{L}_p$-spaces given by Lindenstrauss and Pelczynski in their pioneering work \cite{LinPelAbsSumOpInLpSpAndApp}. Clearly, any finite dimensional Banach space is an $\mathscr{L}_p^g$-space for all $1\leq p\leq +\infty$. Any $L_p$-space is an $\mathscr{L}_{p,1}^g$-space  [\cite{DefFloTensNorOpId}, exercise 4.7], but the converse is not true. Any $c$-complemented subspace of $\mathscr{L}_{p,C}^g$-space is an $\mathscr{L}_{p,cC}^g$-space [\cite{DefFloTensNorOpId}, corollary 23.2.1(2)]. A Banach space is an $\mathscr{L}_{p,C}^g$-space iff its dual is an $\mathscr{L}_{p^*,C}^g$-space [\cite{DefFloTensNorOpId}, corollary 23.2.1(1)]. All $C(K)$-spaces are $\mathscr{L}_{\infty, 1}^g$-spaces [\cite{DefFloTensNorOpId}, lemma 4.4]. Note that, for a given locally compact Hausdorff space $S$ the Banach space $C_0(S)$ is complemented in $C(\alpha S)$. Therefore $C_0(S)$-spaces are $\mathscr{L}_\infty^g$-spaces too. We will mainly concern in $\mathscr{L}_1^g$- and $\mathscr{L}_\infty^g$-spaces.

We say that a Banach space $E$ is weakly sequentially complete if for any sequence $(x_n)_{n\in\mathbb{N}}\subset E$ such that $(f(x_n))_{n\in\mathbb{N}}\subset\mathbb{C}$ is a Cauchy sequence for any $f\in E^*$ there exists a vector $x\in E$ such that $\lim_n f(x_n)=f(x)$ for all $f\in E^*$. That is any weakly Cauchy sequence converges in the weak topology. A typical example of weakly sequentially complete Banach space is any $L_1$-space [\cite{WojBanSpForAnalysts}, corollary III.C.14]. This property is preserved by closed subspaces. A typical example of Banach space that is not weakly sequentially complete is $c_0(\mathbb{N})$, just consider the sequence $(\sum_{k=1}^n \delta_k)_{n\in\mathbb{N}}$.

Now we proceed to the discussion of the Dunford-Pettis property. A bounded linear operator $T:E\to F$ is called weakly compact if it maps the unit ball of $E$ into a relatively weakly compact subset of $F$. A bounded linear operator is called completely continuous if the image of any weakly compact subset of $E$ is norm compact in $F$. A Banach space $E$ is said to have the Dunford-Pettis property if any weakly compact operator from $E$ to any Banach space $F$ is completely continuous. There is a simple internal characterization [\cite{KalAlbTopicsBanSpTh}, theorem 5.4.4]: a Banach space $E$ has the Dunford-Pettis property if $\lim_n f_n(x_n)=0$ for all sequences $(x_n)_{n\in\mathbb{N}}\subset E$ and $(f_n)_{n\in\mathbb{N}}\subset E^*$, that both weakly converge to $0$. Now it is easy to deduce, that if a Banach space $E^*$ has the Dunford-Pettis property, then so does $E$. In his seminal work \cite{GrothApllFaiblCompSpCK} Grothendieck showed that all $L_1$-spaces and $C(K)$-spaces have this property. The Dunford-Pettis property passes to complemented subspaces [\cite{FabHabBanSpTh}, proposition 13.44]. This property behaves badly with reflexive spaces: since the unit ball of a reflexive space is weakly compact [\cite{MeggIntroBanSpTh}, theorem 2.8.2], then reflexive Banach space with the Dunford-Pettis property has norm compact unit ball and therefore this space is finite dimensional. 

To introduce the next Banach geometric property we need definitions of Banach lattice and unconditional Schauder basis. 

A real Riesz space $E$ is a vector space over $\mathbb{R}$ with the structure of partially ordered set such that $x\leq y$ implies $x+z\leq y+z$ for every $x,y,z\in E$ and $ax\geq 0$ for every $x\geq 0$, $a\in\mathbb{R}_+$. A partially ordered set is a lattice if any two elements ${x,y}$ have the least upper bound $x\vee y$ and the greatest lower bound $x\wedge y$. A real vector lattice is real Riesz space which is lattice as partially ordered set. If $E$ is a real vector lattice, then for every $x\in E$ we define its absolute value by equality $|x|:=x\vee(-x)$. A complex vector lattice $E$ is a vector space over $\mathbb{C}$ such that there exists a real vector subspace $\operatorname{Re}(E)$ which is real vector lattice and

$i)$ for any $x\in E$ there are unique $\operatorname{Re}(x),\operatorname{Im}(x)\in \operatorname{Re}(E)$ such that $x=\operatorname{Re}(x)+i\operatorname{Im}(x)$;

$ii)$ for any $x\in E$ there exist an absolute value $|x|:=\sup\{\operatorname{Re}(e^{i\theta}x):\theta\in\mathbb{R}\}$.

A Banach lattice is a Banach space with the structure of the complex vector lattice such that $\Vert x\Vert\leq \Vert y\Vert$ whenever $|x|\leq |y|$. A classical example of Banach lattice $E$ is an $L_p$-space or a $C(K)$-space. In both cases $\operatorname{Re}(E)$ consist of real valued functions in $E$. If $\{E_\lambda:\lambda\in\Lambda\}$ is a family of Banach lattices then for any $1\leq p\leq +\infty$ or $p=0$ their $\bigoplus_p$-sum is a Banach lattice with lattice operation defined as $x\leq y$ if $x_\lambda\leq y_\lambda$ for all $\lambda\in\Lambda$, where $x,y\in\bigoplus_p\{ E_\lambda:\lambda\in\Lambda\}$. The dual space $E^*$ of a Banach lattice $E$ is again a Banach lattice with lattice operation defined by $f\leq g$ if $f(x)\leq g(x)$ for all $x\geq 0$, where $f,g\in  E^*$. A very nice account of Banach lattices can be found in [\cite{LaceyIsomThOfClassicBanSp}, section 1].

The property of being a Banach lattice puts some restrictions on the geometry of the space \cite{SherOrderInOpAlg}, \cite{KadOrderPropOfBoundSAOps}. To explain the Banach geometric reason of this phenomena we need the definition of an unconditional Schauder basis. Let $E$ be a Banach space. A collection of functionals $(f_\lambda)_{\lambda\in\Lambda}$ in $E^*$ is called a biorthogonal system for vectors $(x_\lambda)_{\lambda\in\Lambda}$  from $E$ if $f_\lambda(x_{\lambda'})=1$ whenever $\lambda=\lambda'$ and $0$ otherwise. A collection $(x_\lambda)_{\lambda\in\Lambda}$ in $E$ is called an unconditional Schauder basis if there exists a biorthogonal system $(f_\lambda)_{\lambda\in\Lambda}$ in $E^*$ for it such that
the series $\sum_{\lambda\in\Lambda} f_\lambda(x)x_\lambda$ unconditionally converges to $x$ for any $x\in E$. All $\ell_p$-spaces with $1\leq p<+\infty$ have an unconditional Schauder basis, for example, it is $(\delta_\lambda)_{\lambda\in\Lambda}$. A typical example of space without unconditional basis is $C([0,1])$. Even more this Banach space can not even be a subspace of the space with unconditional basis [\cite{KalAlbTopicsBanSpTh}, proposition 3.5.4].  Any unconditional Schauder basis $(x_\lambda)_{\lambda\in\Lambda}$ in $E$ satisfy the following property  [\cite{DiestAbsSumOps}, proposition 1.6]: there exists a constant $\kappa\geq 1$ such that
$$
\left\Vert \sum_{\lambda\in\Lambda}t_\lambda f_\lambda(x)x_\lambda\right\Vert
\leq
\kappa\left\Vert \sum_{\lambda\in\Lambda}f_\lambda(x)x_\lambda\right\Vert
$$
for all $x\in E$ and $t\in\ell_\infty(\Lambda)$. The least such constant $\kappa$ among all unconditional Schauder bases of $E$ is denoted by $\kappa(E)$. Similar constant could be defined for Banach spaces without unconditional Schauder bases. The local unconditional constant $\kappa_u(E)$ of Banach space $E$ is defined to be the infimum of all scalars $c$ with the following property: given any finite dimensional subspace $F$ of $E$ there exists a Banach space $G$ with unconditional Schauder basis and two bounded linear operators $S:F\to G$, $T:G\to E$ such that $TS|^{F}=1_F$ and $\Vert T\Vert\Vert S\Vert\kappa(G)\leq c$. We say that a Banach space $E$ has the local unconditional structure property (the l.u.st. property for short) if $\kappa_u(E)$ is finite. Clearly any Banach space $E$ with unconditional Schauder basis has the l.u.st. property with $\kappa_u(E)=\kappa(E)$. In particular, all finite dimensional Banach spaces have the l.u.st. property. Though a general Banach lattice $E$ may not have an unconditional Schauder basis it is still has the l.u.st. property with $\kappa_u(E)=1$  [\cite{DiestAbsSumOps}, theorem 17.1]. Directly from the definition it follows that the l.u.st. property is preserved by complemented subspaces. More precisely: if $F$ is a $c$-complemented subspace of $E$, then $\kappa_u(F)\leq c\kappa_u(E)$. Therefore all complemented subspaces of Banach lattices have the l.u.st. property. This sufficient condition is not far from criterion [\cite{DiestAbsSumOps}, theorem 17.5]: a Banach space $E$ has the l.u.st. property iff $E^{**}$ is topologically isomorphic to a complemented subspace of some Banach lattice. As the corollary of this criterion we get that $E$ has the l.u.st. property iff so does $E^*$ [\cite{DiestAbsSumOps}, corollary 17.6].

The last property we shall discuss is a well known approximation property introduced by Grothendieck in \cite{GrothProdTenTopNucl}. We say that a Banach space $E$ has the approximation property if for any compact set $K\subset E$ and any $\epsilon>0$ there exists a finite rank operator $T:E\to E$ such that $\Vert T(x)-x\Vert<\epsilon$ for all $x\in K$. If $T$ can be chosen with $\Vert T\Vert\leq c$, then $E$ is said to have the $c$-bounded approximation property. The metric approximation property is another name for $1$-bounded approximation property. We say that $E$ has the bounded approximation property if $E$ has the $c$-bounded approximation property for some $c\geq 1$. None of these properties are preserved by subspaces or quotient spaces, but the approximation property and the bounded approximation properties are inherited by complemented subspaces [\cite{DefFloTensNorOpId}, exercise 5.5]. All $L_p$-spaces and $C(K)$-spaces have the metric approximation property [\cite{DefFloTensNorOpId}, section 5.2(3)], but their subspaces may fail the approximation property [\cite{DefFloTensNorOpId}, section 5.2(1)]. Any Banach space with unconditional Schauder basis has the approximation property [\cite{RyanIntroTensNormsBanSp}, example 4.4]. If $E^*$ has the approximation property, then so does $E$ [\cite{DefFloTensNorOpId}, corollary 5.7.2]. The reason why approximation property is so important is rather simple --- it has a lot of equivalent reformulations that involve many nice properties of Banach spaces. For example, the following properties of Banach space $E$ are equivalent [\cite{DefFloTensNorOpId}, sections 5.3, 5.6]:

$i)$ $E$ has the approximation property;

$ii)$ the natural mapping $Gr:E^*\projtens E\to\mathcal{N}(E)$ is an isometric isomorphism;

$iii)$ for any Banach space $F$ every compact operator $T:F\to E$ can be approximated in the operator norm by finite rank operators.

There is much more to list, but we confine ourselves with these three properties.

%----------------------------------------------------------------------------------------
%	Banach algebras and their modules
%----------------------------------------------------------------------------------------

\subsection{Banach algebras and their modules}
\label{SubSectionBanachAlgebrasAndTheirModules}

A thorough treatment of Banach algebras and Banach modules can be found in \cite{HelBanLocConvAlg} or \cite{HelHomolBanTopAlg} or \cite{DalBanAlgAutCont}. We shall describe only the bare minimum required for us.

A Banach algebra $A$ is an associative algebra over $\mathbb{C}$ which is a Banach space and the multiplication bilinear operator $\cdot:A\times A\to A:(a,b)\mapsto ab$ is of the norm at most $1$. A typical example of commutative Banach algebra is the algebra of continuous functions on a compact Hausdorff space with pointwise multiplication. A typical non commutative example is the algebra of bounded linear operators on the Hilbert space with composition in the role of multiplication. Both examples belong to a very important class of $C^*$-algebras to be discussed below. By $\langle$~left / right / two-sided~$\rangle$ ideal $I$ of a Banach algebra $A$ we always mean a closed subalgebra of $A$ such that $\langle$~$ax$ / $xa$ / $ax$ and $xa$~$\rangle$ belong to $I$ for all $a\in A$ and $x\in I$.

We say that an element $p$ of a Banach algebra $A$ is a $\langle$~left / right~$\rangle$ identity of $A$ if $\langle$~$pa=a$ / $ap=a$~$\rangle$ for all $a\in A$. The element which is both left and right identity is called the identity and denoted by $e_A$. In general we do not assume that Banach algebras are unital, i.e. has an identity. Even if a Banach algebra $A$ is unital we do not require its identity to be of norm $1$. We use notation $A_+=A\bigoplus_1\mathbb{C}$ for the standard unitization of Banach algebras. The multiplication in $A_+$ is defined as $(a\oplus_1 z)(b\oplus_1 w)=(ab+wa+zb)\oplus_1 zw$, for $a,b\in A$ and $z,w\in\mathbb{C}$. Clearly $(0,1)$ is the identity of $A_+$. By $A_\times$ we denote the conditional unitization of $A$, i.e. $A_\times=A$ if $A$ has identity of norm one and $A_\times=A_+$ otherwise. Even in the absence of identity in case of Banach algebras there are good substitutes for it which are called approximate identities. We say that a net $(e_\nu)_{\nu\in N}$ in $A$ is a $\langle$~left / right / two-sided~$\rangle$ approximate identity if $\langle$~$\lim_\nu e_\nu a=a$ / $\lim_\nu ae_\nu=a$ / $\lim_\nu e_\nu a=\lim_\nu ae_\nu=a$~$\rangle$ for all $a\in A$. In all these three definitions convergence of nets is understood in the norm topology. If we will consider weak topology, we shall get definitions of left, right and two-sided weak approximate identities. We say that an approximate identity $(e_\nu)_{\nu\in N}$ is $c$-bounded if $\sup_\nu\Vert e_\nu\Vert\leq c$. An approximate identity $(e_\nu)_{\nu\in N}$ is called $\langle$~bounded / contractive~$\rangle$ if it is $\langle$~$1$-bounded / $c$-bounded for some $c\geq 1$~$\rangle$. Occasionally we will use the following simple fact: if $A$ is a Banach algebra with $\langle$~left / right~$\rangle$ identity $p$ and $\langle$~right / left~$\rangle$ approximate identity $(e_\nu)_{\nu\in N}$, then $A$ is unital with identity $p$ of norm $\lim_\nu\Vert e_\nu\Vert$. 

If $A$ is a unital Banach algebra we define the spectrum $\operatorname{sp}_A(a)$ of element $a$ in $A$ as the set of all complex numbers $z$ such that $a-ze_A$ is not invertible in $A$. For Banach algebras the spectrum of any element is a non empty compact subset of $\mathbb{C}$ [\cite{HelBanLocConvAlg}, corollary 2.1.16].

A character on a Banach algebra $A$ is a non zero linear homomorphism $\varkappa:A\to\mathbb{C}$. All characters are continuous and are contained in the unit ball of $A^*$  [\cite{HelBanLocConvAlg}, theorem 1.2.6]. Therefore we may consider the set of characters with the induced weak$^*$ topology. The resulting topological space is Hausdorff and locally compact. It is called the spectrum of Banach algebra $A$ and denoted by $\operatorname{Spec}(A)$. If $A$ is unital then its spectrum is compact [\cite{HelBanLocConvAlg}, theorem 1.2.50]. Now for a given Banach algebra $A$ with non empty spectrum we can construct a contractive homomorphism $\Gamma_A:A\to C_0(\operatorname{Spec}(A)):a\mapsto(\varkappa\mapsto \varkappa(a))$ called the Gelfand transform of $A$ [\cite{HelBanLocConvAlg}, theorem 4.2.11]. The kernel of this homomorphism is called the Jacobson's radical and denoted by $\operatorname{Rad}(A)$. For Banach algebra $A$ with empty spectrum we define $\operatorname{Rad}(A)=A$. If $\operatorname{Rad}(A)=\{0\}$, then $A$ is called semisimple. By Shilov's idempotent theorem [\cite{KaniBanAlg}, section 3.5] any semisimple Banach algebra with compact spectrum is unital.

Most of standard constructions for Banach spaces have their counterparts for Banach algebras. For example $\bigoplus_p$-sum of Banach algebras endowed with componentwise multiplication is a Banach algebra, the quotient of a given Banach algebra by its two sided ideal is a Banach algebra. Even the projective tensor product of two Banach algebras is a Banach algebra with multiplication defined on elementary tensors the same way as in pure algebra.

We shall proceed to the discussion of the most important class of Banach algebras. Let $A$ be an associative algebra over $\mathbb{C}$, then a conjugate linear operator ${}^*:A\to A$ is called an involution if $(ab)^*=b^*a^*$ and $a^{**}=a$ for all $a,b\in A$. Algebras with involution are called ${}^*$-algebras. Homomorphisms between ${}^*$-algebras that preserve involution are called ${}^*$-homomorphisms. A Banach algebra with isometric involution is called a ${}^*$-Banach algebra. An example of such algebra is the Banach algebra of bounded linear operators on Hilbert space with operation of taking the Hilbert adjoint operator in the role of involution. In fact there is much more to this algebra than one could expect. We say that a ${}^*$-Banach algebra $A$ is a $C^*$-algebra if it satisfies $\Vert a^*a\Vert=\Vert a\Vert^2$ for all $a\in A$. One of the biggest advantages of $C^*$-algebras is their celebrated representation theorems by Gelfand and Naimark. The first representation theorem [\cite{HelBanLocConvAlg}, theorem 4.7.13] states that any commutative $C^*$-algebra $A$ is isometrically isomorphic as ${}^*$-algebra to $C_0(\operatorname{Spec}(A))$. The second theorem [\cite{HelBanLocConvAlg}, theorem 4.7.57] gives a description of generic $C^*$-algebras as closed ${}^*$-Banach subalgebras of $\mathcal{B}(H)$ for some Hilbert space $H$. Such representation is not unique, but a norm (if it exists) that turn a ${}^*$-algebra into a $C^*$-algebra is always unique. If a ${}^*$-subalgebra of $\mathcal{B}(H)$ is weak${}^*$ closed it is called a von Neumann algebra. If a $C^*$-algebra is isomorphic as ${}^*$-algebra to a von Neumann algebra it is called a $W^*$-algebra. By well known Sakai's theorem [\cite{BlackadarOpAlg}, theorem III.2.4.2] each $C^*$-algebra which is dual as Banach space is a $W^*$-algebra, but beware a $W^*$-algebra may be represented as non weak${}^*$ closed ${}^*$-subalgebra in $\mathcal{B}(H)$ for some Hilbert space $H$. 

A lot of standard constructions pass to $C^*$-algebras from Banach algebras, but not all. For example a $\bigoplus_\infty$-sum of $C^*$-algebras is again a $C^*$-algebra. A quotient of $C^*$-algebra by closed two-sided ideal is a $C^*$-algebra too. Meanwhile the projective tensor product of $C^*$-algebras is rarely a $C^*$-algebra, though there a lot of norms that may turn their algebraic tensor product into a $C^*$-algebra. In this thesis we shall exploit one specific and highly
important for $C^*$-algebras construction of matrix algebras. For a given $C^*$-algebra $A$ by $M_n(A)$ we denote the linear space of $n\times n$ matrices with entries in $A$. In fact $M_n(A)$ is ${}^*$-algebra with involution and multiplication defined by equalities 
$$
(ab)_{i,j}=\sum_{k=1}^n a_{i,k}b_{k,j}
\qquad\qquad
(a^*)_{i,j}=(a_{j,i}^*)
$$ 
for all $a,b\in M_n(A)$ and $i,j\in\mathbb{N}_n$. There is a unique norm on $M_n(A)$ that makes it a $C^*$-algebra [\cite{MurphyCStarAlgsAndOpTh}, theorem 3.4.2]. Obviously, $M_n(\mathbb{C})$ is isometrically isomorphic as ${}^*$-algebra to $\mathcal{B}(\ell_2(\mathbb{N}_n))$. From [\cite{MurphyCStarAlgsAndOpTh}, remark 3.4.1] it follows that the natural embeddings $i_{k,l}:A\to M_n(A):a\mapsto(a\delta_{i,k}\delta_{j,l})_{i,j\in\mathbb{N}_n}$ and projections $\pi_{k,l}:M_n(A)\to A:a\mapsto a_{k,l}$ are continuous. Therefore for a given bounded linear operator $\phi:A\to B$ between $C^*$-algebras $A$ and $B$ the linear operator 
$$
M_n(\phi):M_n(A)\to M_n(B):a\mapsto (\phi(a_{i,j}))_{i,j\in\mathbb{N}_n}
$$ 
is also bounded. Even more if $\phi$ is an $A$-morphism or ${}^*$-homomorphism, then so does $M_n(\phi)$. Finally we shall mention two isometric isomorphisms that will be of use:
$$
M_n\left(\bigoplus\nolimits_\infty\{A_\lambda:\lambda\in\Lambda\}\right)
\isom{\mathbf{Ban}_1}
\bigoplus\nolimits_\infty\{M_n\left(A_\lambda\right):\lambda\in\Lambda\},
$$
$$
M_n(C(K))\isom{\mathbf{Ban}_1}C(K,M_n(\mathbb{C}))
$$

Now a few facts on approximate identities and identities of $C^*$-algebras and their ideals. Any two-sided closed ideal of $C^*$-algebra has a two-sided contractive positive approximate identity [\cite{HelBanLocConvAlg}, theorem 4.7.79], and any left ideal has a right contractive positive approximate identity. In some cases even an approximate identity is not enough, so for this situation there is a procedure to endow a $C^*$-algebra with identity and preserve $C^*$-algebraic structure [\cite{HelBanLocConvAlg}, proposition 4.7.6]. This type of unitization we shall denote as $A_\#$. Till the end of this paragraph we assume that $A$ is a unital $C^*$-algebra. An element $a\in A$ is called a projection (or an orthogonal projection) if $a=a^*=a^2$, self-adjoint if $a=a^*$, positive if $a=b^*b$ for some $b\in A$, unitary if $a^*a=aa^*=e_A$. The set $A_{pos}$ of all positive elements of $A$ is a closed cone in $A$. If an element $a\in A$ is $\langle$~self-adjoint / positive~$\rangle$, then $\langle$~$\operatorname{sp}_A(a)\subset[-\Vert a\Vert, \Vert a\Vert]$ / $\operatorname{sp}_A(a)\subset[0,\Vert a\Vert]$~$\rangle$. For a given self-adjoint element $a\in A$, there always exists the isometric ${}^*$-homomorphism $\operatorname{Cont}_a:C(\operatorname{sp}_A(a))\to A$ such that $\operatorname{Cont}_a(f)=a$, where $f:\operatorname{sp}_A(a)\to\mathbb{C}:t\mapsto t$. It is called the continuous functional calculus [\cite{HelBanLocConvAlg}, theorem 4.7.24]. Loosely speaking it allows to take continuous functions of self-adjoint elements of $C^*$-algebras, so following standard convention we shall write $f(a)$ instead of $\operatorname{Cont}_a(f)$. Another related result called the spectral mapping theorem allows to compute the spectrum of elements given by continuous functional calculus: $\operatorname{sp}_A(f(a))=f(\operatorname{sp}_A(a))$.

We proceed to the discussion of more general objects --- Banach modules. Let $A$ be a Banach algebra, we say that $X$ is a $\langle$~left / right~$\rangle$ Banach $A$-module if $X$ is a Banach space endowed with bilinear operator $\langle$~$\cdot:A\times X\to X$ / $\cdot: X\times A\to X$~$\rangle$ of norm at most $1$ (called a module action), such that $\langle$~$a\cdot(b\cdot x)=ab\cdot x$ / $(x\cdot a)\cdot b=x\cdot ab$~$\rangle$ for all $a,b\in A$ and $x\in X$. Any Banach space $E$ can be turned into a $\langle$~left / right~$\rangle$ Banach $A$-module be defining $\langle$~$a\cdot x=0$ / $x\cdot a=0$~$\rangle$ for all $a\in A$ and $x\in E$. Any Banach algebra $A$ can be regarded as a left and right Banach $A$-module --- the module action coincides with algebra multiplication. Of course, there are more meaningful examples too.  Usually we shall discuss only left Banach modules since for their right sided counterparts all definitions and results are similar. We call a left Banach module $X$ over unital Banach algebra $A$ unital if $e_A\cdot x=x$ for all $x\in X$. For a given left Banach $A$-module $X$ and $S\subset A$, $M\subset X$ we define their products $S\cdot M=\{a\cdot x:a\in S, x\in M\}$, $SM=\operatorname{span} (S\cdot M)$ and annihilators $S^{\perp M}=\{a\in S:a\cdot M=\{0\}\}$, ${}^{S\perp}M=\{x\in M: S\cdot x=\{0\}\}$. The essential and annihilator parts of $X$ are defined as $X_{ess}=\operatorname{cl}_X(A X)$, $X_{ann}={}^{A\perp}X$. The module $X$ is called $\langle$~faithful / annihilator / essential~$\rangle$ if $\langle$~${}^{A\perp}X=\{0\}$ / $X=X_{ann}$ / $X=X_{ess}$~$\rangle$. An obvious application of Hahn-Banach theorem shows that $X$ is an essential $A$-module iff $X^*$ is a faithful $A$-module.

Let $X$ and $Y$ be $\langle$~left / right~$\rangle$ Banach $A$-modules. We say that a linear operator $\phi:X\to Y$ is an $A$-module map of $\langle$~left / right~$\rangle$ modules if $\langle$~$\phi(a\cdot x)=a\cdot \phi(x)$ / $\phi(x\cdot a)=\phi(x)\cdot a$~$\rangle$ for all $a\in A$ and $x\in X$. A bounded $A$-module map is called an $A$-morphism. The set of $A$-morphisms between $\langle$~left / right~$\rangle$ $A$-modules $X$ and $Y$ we denote as $\langle$~${}_A\mathcal{B}(X,Y)$ / $\mathcal{B}_A(X,Y)$~$\rangle$. Note that if $X$ and $Y$ are $\langle$~left / right~$\rangle$ annihilator $A$-modules, then $\langle$~${}_A\mathcal{B}(X,Y)=\mathcal{B}(X,Y)$ / $\mathcal{B}_A(X,Y)=\mathcal{B}(X,Y)$~$\rangle$.

By $\langle A-\mathbf{mod}$ / $\mathbf{mod}-A\rangle$ we shall denote the category of $\langle$~left / right~$\rangle$ $A$-modules with continuous $A$-module maps in the role of morphisms. By $\langle$~$A-\mathbf{mod}_1$ / $\mathbf{mod}_1-A$~$\rangle$ we denote its subcategory of $\langle$~$A-\mathbf{mod}$ / $\mathbf{mod}-A$~$\rangle$ with the same objects and contractive morphisms only. Therefore $\langle$~ $\operatorname{Hom}_{A-\mathbf{mod}}(X,Y)={}_A\mathcal{B}(X,Y)$ /  $\operatorname{Hom}_{\mathbf{mod}-A}(X,Y)=\mathcal{B}_A(X,Y)$~$\rangle$. 

As in any category we can speak of retraction and	 coretractions in the category of Banach modules. But for this particular case we have several refinements for the standard definitions. An $A$-morphism $\xi:X\to Y$ is called a $\langle$~$c$-retraction / $c$-coretraction~$\rangle$ if there exist an $A$-morphism $\eta:Y\to X$ such that $\langle$~$\xi\eta=1_Y$ / $\eta\xi=1_X$~$\rangle$ and $\Vert\xi\Vert\Vert\eta\Vert\leq c$. From the definition it follows that composition of $\langle$~$c_1$- and $c_2$-retraction / $c_1$- and $c_2$-coretraction~$\rangle$ gives a $\langle$~$c_1c_2$-retraction / $c_1c_2$-coretraction~$\rangle$. Clearly, the adjoint of $\langle$~$c$-retraction / $c$-coretraction~$\rangle$ is a $\langle$~$c$-coretraction / $c$-retraction~$\rangle$. Finally, an $A$-morphism $\xi:X\to Y$ is called a $c$-isomorphism if there exists an $A$-morphism $\eta:Y\to X$ such that $\xi\eta=1_Y$, $\eta\xi=1_X$ and $\Vert\xi\Vert\Vert\eta\Vert\leq c$. In this case we say that $A$-modules $X$ and $Y$ are $c$-isomorphic.

Now we mention several constructions over Banach modules that we will encounter in this thesis.  Any left Banach $A$-module can be regarded as unital Banach module over $A_+$, and we put by definition $(a\oplus_1 z)\cdot x=a\cdot x+zx$ for all $a\in A$, $x\in X$ and $z\in\mathbb{C}$. Most constructions used for Banach spaces transfer to Banach modules.  We say that a linear subspace $ Y$ of a left Banach $A$-module $X$ is a left $A$-submodule of $X$ if $A\cdot Y\subset Y$. For example, any left ideal $I$ of a Banach algebra $A$ is a left $A$-submodule of $A$. If $Y$ is a closed left $A$-submodule of the left Banach $A$-module $X$, then the Banach space $X/Y$ can be endowed with the structure of the left Banach $A$-module, just put by definition $a\cdot(x+Y)=a\cdot x+Y$ for all $a\in A$ and $x+Y\in X/Y$. This object is called the quotient $A$-module.  Quotient modules of the form $A/I$, where $I$ is a left ideal of $A$, are called cyclic modules. For motivation for this term see [\cite{HelBanLocConvAlg}, proposition 6.2.2]. Clearly, $X/X_{ess}$ is an annihilator $A$-module. If $X$ is a left Banach $A$-module and $E$ is a Banach space, then $\langle$~$\mathcal{B}(X,E)$ / $\mathcal{B}(E,X)$~$\rangle$ is a $\langle$~right / left~$\rangle$ Banach $A$-module with module action defined by $\langle$~$(T\cdot a)(x)=T(a\cdot x)$ for all $a\in A$, $x\in X$ and $T\in\mathcal{B}(X, E)$ / $(a\cdot T)(x)=a\cdot T(x)$ for all $a\in A$, $x\in E$ and $T\in\mathcal{B}(E, X)$~$\rangle$. In particular, $X^*$ is a right Banach $A$-module. If $\{X_\lambda:\lambda\in\Lambda\}$ is a family of left Banach $A$-modules and $1\leq p\leq +\infty$ or $p=0$, then their $\bigoplus_p$-sum is a left Banach $A$-module with module action defined by $a\cdot x=\bigoplus_p\{ a\cdot x_\lambda:\lambda\in\Lambda\}$, where $a\in A$, $x\in\bigoplus_p\{ X_\lambda:\lambda\in\Lambda\}$. Again, as in Banach space theory, any family of $A$-modules admits the $\langle$~product / coproduct~$\rangle$ in $A-\mathbf{mod}_1$ which in fact is their $\langle$~$\bigoplus_1$-sum / $\bigoplus_\infty$-sum~$\rangle$. The category $A-\mathbf{mod}$ admits $\langle$~products / coproducts~$\rangle$ only for finite families of objects. Similar statements are valid for $\mathbf{mod}-A$ and $\mathbf{mod}_1-A$.

Projective tensor product of Banach spaces also has its module version, it is called the projective module tensor product. Assume $X$ is a right and $Y$ is a left Banach $A$-module. Their projective module tensor product $X\projmodtens{A}Y$ is defined as quotient space $X\projtens Y / N$ where $N=\operatorname{cl}_{X\projtens Y}(\operatorname{span}\{x\cdot a\projtens y-x\projtens a\cdot y:x\in X,y\in Y,a\in A\})$. Let $\phi\in\mathcal{B}_A(X_1,X_2)$ and $\psi\in{}_A\mathcal{B}(Y_1,Y_2)$ for right Banach $A$-modules $X_1$, $X_2$ and left Banach $A$-modules $Y_1$, $Y_2$, then there exists a unique bounded linear operator $\phi\projmodtens{A} \psi:X_1\projmodtens{A} Y_1\to X_2\projmodtens{A} Y_2$ such that $(\phi\projmodtens{A} \psi)(x\projmodtens{A} y)=\phi(x)\projmodtens{A} \psi(y)$ for all $x\in X_1$ and $y\in Y_1$. Even more $\Vert \phi\projmodtens{A} \psi\Vert\leq\Vert \phi\Vert\Vert \psi\Vert$. The projective module tensor product has its own universal property: for any right Banach $A$-module $X$, any left Banach $A$-module $Y$ and any Banach space $E$ there exists an isometric isomorphism:
$$
\mathcal{B}(X\projmodtens{A}Y,E)\isom{\mathbf{Ban}_1}\mathcal{B}_{bal}(X\times Y, E)
$$
where $\mathcal{B}_{bal}(X\times Y, E)$ stands for the Banach space of bilinear operators $\Phi:X\times Y\to E$ satisfying $\Phi(x\cdot a,y)=\Phi(x,a\cdot y)$ for all $x\in X$, $y\in Y$ and $a\in A$. Such bilinear operators are called balanced.
Furthermore we have two (natural in $X$, $Y$ and $E$) isometric isomorphisms:
$$
\mathcal{B}(X\projmodtens{A}Y,E)
\isom{\mathbf{Ban}_1}
{}_A\mathcal{B}(Y,\mathcal{B}(X,E))
\isom{\mathbf{Ban}_1}
\mathcal{B}_A(X,\mathcal{B}(Y,E))
$$
Analogously to Banach space theory we may define the following functors:
$$
\mathcal{B}(-,E):A-\mathbf{mod}\to \mathbf{mod}-A
\qquad\qquad
\mathcal{B}(E,-):\mathbf{mod}-A\to \mathbf{mod}-A
$$
$$
-\projmodtens{A} Y:\mathbf{mod}-A\to\mathbf{Ban}
\qquad\qquad
X\projmodtens{A} -:A-\mathbf{mod}\to\mathbf{Ban}
$$
where $E$ is a Banach space, $X$ is a right $A$-module and $Y$ is a left $A$-module. All these functors have their counterparts for categories $A-\mathbf{mod}_1$, $\mathbf{mod}_1-A$. 

In some cases it is possible to explicitly compute the projective module tensor product. For example [\cite{HelBanLocConvAlg}, proposition 6.3.24] if $I$ is a left closed ideal of $A_+$ with left $\langle$~contractive / bounded~$\rangle$ approximate identity, and $X$ is a left Banach module then the linear operator 
$$
i_{I,X}:I\projmodtens{A}X \to \operatorname{cl}_X(IX):a\projmodtens{A} x\mapsto a\cdot x
$$
is $\langle$~a topological isomorphism / an isometric isomorphism~$\rangle$ of Banach spaces. If $I$ is a two-sided ideal, then $i_{I,X}$ is a morphism of left $A$-modules. We call reduced all left Banach modules of the form $A\projmodtens{A}X$. 

Most of what have been said here can be generalized to Banach bimodules, but in this thesis we shall not exploit them much. In those rare case when we shall encounter bimodules, the respective definitions and results are easily recoverable from their one sided counterparts.

%----------------------------------------------------------------------------------------
%	Banach homology
%----------------------------------------------------------------------------------------

\section{Banach homology}
\label{SectionBanachHomology}

%----------------------------------------------------------------------------------------
%	Relative homology
%----------------------------------------------------------------------------------------

\subsection{Relative homology}
\label{SubSectionRelativeHomology}

Further we briefly discuss ABCs of relative homology introduced and intensively studied by Helemskii. Fix an arbitrary Banach algebra $A$. We say that a morphism $\xi:X\to Y$ of left $A$-modules $X$ and $Y$ is a relatively admissible epimorphism if it admits a right inverse bounded linear operator. A left $A$-module $P$ is called relatively projective if for any relatively admissible  epimorphism $\xi:X\to Y$ and for any $A$-morphism $\phi:P\to Y$ there exists an $A$-morphism $\psi:P\to X$ such that the diagram
$$
\xymatrix{
& {X} \ar[d]^{\xi}\\
{P} \ar@{-->}[ur]^{\psi} \ar[r]^{\phi} &{Y}}
$$
is commutative. Such $A$-morphism $\psi$ is called a lifting of $\phi$ and it is not unique in general. Similarly,  we say that a morphism $\xi:Y\to X$ of right $A$-modules $X$ and $Y$ is a relatively admissible monomorphism if it admits a left inverse bounded linear operator. A right $A$-module $J$ is called relatively injective if for any relatively admissible  monomorphism $\xi:Y\to X$ and for any $A$-morphism $\phi:Y\to J$ there exists an $A$-morphism $\psi:X\to J$ such that the diagram
$$
\xymatrix{
& {X} \ar@{-->}[dl]_{\psi} \\
{J} &{Y} \ar[l]_{\phi} \ar[u]_{\xi}}
$$
is commutative. Such $A$-morphism $\psi$ is called an extension of $\phi$ and it is not unique in general.

The reason for considering relatively admissible morphisms in these definitions is the intention of separation Banach geometric and algebraic motives that may prevent an $A$-module to be relatively projective or injective. A straightforward check shows that any retract of relatively $\langle$~projective / injective~$\rangle$ $A$-module is again relatively $\langle$~projective / injective~$\rangle$. Obviously, any relatively admissible $\langle$~epimorphism / monomorphism~$\rangle$ $\langle$~onto / from~$\rangle$ a relatively $\langle$~projective / injective~$\rangle$ $A$-module is a $\langle$~retraction / coretraction~$\rangle$.

A special class of relatively $\langle$~projective / injective~$\rangle$ $A$-modules is the so-called relatively $\langle$~free / cofree~$\rangle$ modules. These are modules of the form $\langle$~$A_+\projtens E$ / $\mathcal{B}(A_+,E)$~$\rangle$ for some Banach space $E$. Their main feature is the following: for any $A$-module $X$ there exists a relatively $\langle$~free / cofree~$\rangle$ $A$-module $F$, which in fact is $\langle$~$A_+\projtens X$ / $\mathcal{B}(A_+,X)$~$\rangle$ and a relatively admissible $\langle$~epimorphism $\xi:F\to X$ / monomorphism $\xi:X\to F$~$\rangle$. If $X$ is relatively $\langle$~projective / injective~$\rangle$ we immediately get that $\xi$ is a $\langle$~retraction / coretraction~$\rangle$. Therefore an $A$-module is relatively $\langle$~projective / injective~$\rangle$ iff it is a retract of relatively $\langle$~free / cofree~$\rangle$ $A$-module. 

It is worth to emphasize one more time that major nuance of relative Banach homology is deliberate balance between algebra and topology in choice of admissible morphisms. This choice allowed one to build homological theory with some interesting phenomena with no analogs in pure algebra. We demonstrate one example related to Banach algebras. Consider morphism of $A$-bimodules  $\Pi_A:A\projtens A\to A:a\projtens b\mapsto ab$. We say that a Banach algebra $A$ is 

$i)$ relatively $c$-biprojective if $\Pi_A$ is a $c$-retraction of $A$-bimodules;

$ii)$ relatively $c$-biflat if $\Pi_A^*$ is a $c$-coretraction of $A$-bimodules;

$iii)$ relatively $c$-contractible if $\Pi_{A_+}$ is a $c$-retraction of $A$-bimodules;

$iv)$ relatively $c$-amenable if $\Pi_{A_+}^*$ is a $c$-coretraction of $A$-bimodules.

We say that $A$ is relatively $\langle$~biprojective / biflat / contractive / amenable~$\rangle$ if it is relatively $\langle$~$c$-biprojective / $c$-biflat / $c$-contractive / $c$-amenable~$\rangle$ for some $c\geq 1$. The infimum of the constants $c$ is called the $\langle$~biprojectivity / biflatness / contractivity / amenability~$\rangle$ constant.  With slight modifications of [\cite{HelBanLocConvAlg}, proposition 7.1.72] one can show that $A$ is relatively $\langle$~$c$-contractible / $c$-amenable~$\rangle$ iff there exists $\langle$~an element $d\in A\projtens A$ / a net $(d_\nu)_{\nu\in N}\subset A\projtens A$~$\rangle$ with norm not greater than $c$ such that for all $a\in A$ holds $\langle$~$a\cdot d-d\cdot a=0$ and $a\Pi_A(d)=a$ / $\lim_\nu(a\cdot d_\nu-d_\nu\cdot a)=0$ and $\lim_\nu a\Pi_A(d_\nu)=a$~$\rangle$. Note that $\langle$~such element $d$ / such net $(d_\nu)_{\nu\in N}$~$\rangle$ is called $\langle$~a diagonal / an approximate diagonal~$\rangle$. From homological point of view, the main advantage of relatively $\langle$~biprojective / biflat / contractible / amenable~$\rangle$ Banach algebras is that $\langle$~any reduced / any reduced / any / any~$\rangle$ left and right Banach $A$-module is relatively $\langle$~projective / flat / projective / flat~$\rangle$ [\cite{HelBanLocConvAlg}, theorem 7.1.60]. As for flatness such phenomena is typical for relative Banach homology, but not for the purely algebraic one.

%----------------------------------------------------------------------------------------
%	Rigged categories
%----------------------------------------------------------------------------------------

\subsection{Rigged categories}
\label{SubSectionRiggedCategories}

Claims on projectivity and injectivity from previous section have their analogs for a lot of other types of projectivity and injectivity in other categories of mathematics \cite{SemadeniProjInjDual}. Even more, one may easily see that injectivity and projectivity are somewhat dual to each other. All these observations suggest that there is a general categorical approach to study basic properties of homologically trivial objects. Such approach has been promoted by Helemskii in \cite{HelMetrFrQMod}. As we shall see it covers relative theory while results given above are obvious consequences of more general facts. 

Let $\mathbf{C}$ and $\mathbf{D}$ be two fixed categories. An ordered pair ($\mathbf{C}, \square:\mathbf{C}\to\mathbf{D}$), where $\square$ is a faithful covariant functor, is called a rigged category. We say that a morphism $\xi$ in $\mathbf{C}$ is $\square$-admissible epimorphism if $\square (\xi)$ is a retraction in $\mathbf{D}$. An object $P$ in $\mathbf{C}$ is called $\square$-projective, if for every $\square$-admissible epimorphism $\xi$ in $\mathbf{C}$ the map $\operatorname{Hom}_{\mathbf{C}}(P,\xi)$ is surjective. An object $F$ in $\mathbf{C}$ is called $\square$-free with base $M$ in  $\mathbf{D}$, if there exists an isomorphism of functors $\operatorname{Hom}_{\mathbf{D}}(M,\square(-))\cong\operatorname{Hom}_{\mathbf{C}}(F,-)$. A rigged category $(\mathbf{C},\square)$ is called  freedom-loving [\cite{HelMetrFrQMod}, definition 2.10], if every object in $\mathbf{D}$ is a base of some $\square$-free object in $\mathbf{C}$. We may summarize results of propositions 2.3, 2.11  and 2.12 in \cite{HelMetrFrQMod} as follows:

$i)$ any retract of $\square$-projective object is $\square$-projective;

$ii)$ any $\square$-admissible epimorphism into $\square$-projective object is a retraction;

$iii)$ any $\square$-free object is $\square$-projective;

$iv)$ if $(\mathbf{C},\square)$ is freedom-loving rigged category, then any object is $\square$-projective iff it is a retract of $\square$-free object;

$v)$ coproduct of the family of $\square$-projective objects is $\square$-projective.

The opposite rigged category of $(\mathbf{C}, \square)$ 
is a rigged category $(\mathbf{C}^{o},\square^{o}:\mathbf{C}^{o}\to\mathbf{D}^{o})$. 
Thus by passing to the opposite rigged category we may define admissible monomorphisms, injectivity and cofreedom. A morphism $\xi$ in called $\square$-admissible monomorphism if it is $\square^o$-admissible epimorphism. An object $J$ in $\mathbf{C}$ is called $\square$-injective if it is $\square^o$-projective. Finally, an object $F$ in $\mathbf{C}$ is called $\square$-cofree if it is $\square^o$-free. This gives us analogs of results as above for injectivity and cofreedom.

Now consider faithful functor $\square_{rel}:A-\mathbf{mod}\to\mathbf{Ban}$ that just `forgets'' the module structure. One can easily see that $(A-\mathbf{mod},\square_{rel})$ is a rigged category whose $\square_{rel}$-admissible $\langle$~epimorphisms / monomorphisms~$\rangle$ are exactly relatively admissible $\langle$~epimorphisms / monomorphisms~$\rangle$ and $\langle$~$\square_{rel}$-projective / $\square_{rel}$-injective~$\rangle$ objects are exactly relatively $\langle$~projective / injective~$\rangle$ $A$-modules. Even more all $\langle$~$\square_{rel}$-free / $\square_{rel}$-cofree~$\rangle$ objects are isomorphic in $A-\mathbf{mod}$ to $\langle$~$A_+\projtens E$ / $\mathcal{B}(A_+,E)$ ~$\rangle$ for some Banach space $E$. This example shows, that relative theory perfectly fits into the realm of rigged categories.

We shall apply this scheme for metric and topological theory in the next chapter. These two theories put much weaker restrictions on their admissible morphisms. The proverb ``all covet, all lose'' perfectly explains what will happen next.
	
% Chapter Template

\chapter{General theory} % Main chapter title

\label{ChapterGeneralTheory} % Change X to a consecutive number; for referencing this chapter elsewhere, use \ref{ChapterX}

\lhead{Chapter 2. \emph{General theory}} % Change X to a consecutive number; this is for the header on each page - perhaps a shortened title

%----------------------------------------------------------------------------------------
%	Projectivity, injectivity and flatness
%----------------------------------------------------------------------------------------

\section{Projectivity, injectivity and flatness}
\label{SectionProjectivityInjectivityAndFlatness}


%----------------------------------------------------------------------------------------
%	Metric and topological projectivity
%----------------------------------------------------------------------------------------

\subsection{Metric and topological projectivity}
\label{SubSectionMetricAndTopologicalProjectivity}

In what follows $A$ denotes a not necessary unital Banach algebra. We immediately start from the most important definitions in this thesis.

\begin{definition}[\cite{HelMetrFrQMod}, definition 1.4]\label{MetProjMod} An $A$-module $P$ is called metrically projective if for any strictly coisometric $A$-morphism $\xi:X\to Y$ and for any $A$-morphism $\phi:P\to Y$ there exists an $A$-morphism $\psi:P\to X$ such that $\xi\psi=\phi$  and $\Vert\psi\Vert=\Vert\phi\Vert$.
\end{definition}

\begin{definition}[\cite{HelMetrFrQMod}, definition 1.2]\label{TopProjMod} An $A$-module $P$ is called  topologically projective if for any topologically surjective $A$-morphism $\xi:X\to Y$ and for any $A$-morphism $\phi:P\to Y$ there exists an $A$-morphism $\psi:P\to X$ such that $\xi\psi=\phi$.
\end{definition}

A short but more involved definition is the following: an $A$-module $P$ is called $\langle$~metrically / topologically~$\rangle$ projective, if the functor $\langle$~$\operatorname{Hom}_{A-\mathbf{mod}_1}(P,-):A-\mathbf{mod}_1\to\mathbf{Ban}_1$ / $\operatorname{Hom}_{A-\mathbf{mod}}(P,-):A-\mathbf{mod}\to\mathbf{Ban}$~$\rangle$ maps $\langle$~strictly coisometric / topologically surjective~$\rangle$ $A$-morphisms into $\langle$~strictly coisometric / surjective~$\rangle$ operators.

Now we are aiming to apply the general scheme of rigged categories to metric and topological projectivity. Before doing that we need to introduce the category of hemi normed spaces defined by Shteiner in \cite{ShtTopFrClassicQuantMod}. A hemi linear space over $\mathbb{C}$ is a set $E$, whose elements are called vectors, with a binary operation $\cdot:\mathbb{C}\times E\to E$ satisfying three axioms:

$i)$ $\alpha\cdot(\beta\cdot x)=\alpha\beta\cdot x$ for all $\alpha,\beta\in\mathbb{C}$ and $x\in E$; 

$ii)$ $1\cdot x=x$ for all $x\in E$; 

$iii)$ there exists a zero vector $0\in E$ such that $0\cdot x=0$ for all $x\in E$. 

A map $T:E\to F$ between hemi linear spaces is called a hemi linear operator if $T(\alpha\cdot x)=\alpha\cdot T(x)$ for all $\alpha\in\mathbb{C}$ and $x\in E$. A hemi normed space $E$ is a hemi linear space endowed with a function (called a norm) $\Vert\cdot\Vert:E\to\mathbb{R}_+$ such that

$i)$ $\Vert x\Vert=0$ iff $x=0$;

$ii)$ $\Vert\alpha\cdot x\Vert=|\alpha|\Vert x\Vert$ for all $\alpha\in\mathbb{C}$ and $x\in E$. 

A hemi linear operator $T:E\to F$ between hemi normed spaces $E$ and $F$ is called bounded if there exists $c\geq 0$ such that $\Vert\phi(x)\Vert\leq c\Vert x\Vert$ for all $x\in E$. The least such constant is called the norm of $\phi$ and denoted by $\Vert \phi\Vert$. Finally, we define the category of hemi normed spaces $\mathbf{HNor}$: its objects are hemi normed spaces, its morphisms are bounded hemi linear operators. In fact hemi normed spaces and bounded hemi linear operators are just leftovers of normed spaces and bounded linear operators if one erases  all axioms mentioning addition operation. Here is a typical example of hemi normed space. For a given non empty set $\Lambda$ we consider a bouquet $\mathbb{C}^\Lambda:=\bigvee \{\mathbb{C}:\lambda\in\Lambda\}$ of copies of $\mathbb{C}$ with common zero vector. Multiplication by scalars and norm in $\mathbb{C}^\Lambda$ are defined in a most natural fashion. We put by definition $\mathbb{C}^\varnothing=\{0\}$. Any hemi normed space is isomorphic in $\mathbf{HNor}$ to $\mathbb{C}^\Lambda$ for some set $\Lambda$ [\cite{ShtTopFrClassicQuantMod}, proposition 1.1.9].

In \cite{HelMetrFrQMod} and \cite{ShtTopFrClassicQuantMod} there were constructed two faithful functors 
$$
\square_{met}:A-\mathbf{mod}_1\to\mathbf{Set}:X\mapsto B_X,\phi\mapsto\phi|_{B_X}^{B_Y}
$$
$$
\square_{top}:A-\mathbf{mod}\to\mathbf{HNor}:X\mapsto X,\phi\mapsto\phi
$$
The first one sends a Banach $A$-module to its unit ball and any contractive $A$-morphism to the respective birestriction, while the second one ``forgets'' about module and additive structure.
In the same papers it was proved that an $A$-morphism $\xi$ is $\langle$~strictly coisometric / topologically surjective~$\rangle$ iff it is $\langle$~$\square_{met}$-admissible / $\square_{top}$-admissible~$\rangle$ epimorphism and an $A$-module $P$ is $\langle$~metrically / topologically~$\rangle$ projective iff it is $\langle$~$\square_{met}$-projective / $\square_{top}$-projective~$\rangle$. Thus we immediately get the following proposition.

\begin{proposition}\label{RetrMetTopProjIsMetTopProj} Any retract of $\langle$~metrically / topologically~$\rangle$ projective module in $\langle$~$A-\mathbf{mod}_1$ / $A-\mathbf{mod}$~$\rangle$ is again $\langle$~metrically / topologically~$\rangle$ projective.
\end{proposition}

It was also shown that the rigged category $\langle$~$(A-\mathbf{mod}_1,\square_{met})$ / $(A-\mathbf{mod},\square_{top})$~$\rangle$ is freedom loving and that $\langle$~$\square_{met}$-free / $\square_{top}$-free~$\rangle$ modules are isomorphic in $\langle$~$A-\mathbf{mod}_1$ / $A-\mathbf{mod}$~$\rangle$ to $A_+\projtens \ell_1(\Lambda)$ for some set $\Lambda$. Even more, for any $A$-module $X$ there exists a $\langle$~$\square_{met}$-admissible / $\square_{top}$-admissible~$\rangle$ epimorphism
$$
\pi_X^+:A_+\projtens \ell_1(B_X):a\projtens \delta_x\mapsto a\cdot x
$$
As the consequence of general result on rigged categories we get

\begin{proposition}\label{MetTopProjModViaCanonicMorph}
An $A$-module $P$ is $\langle$~metrically / topologically~$\rangle$ projective iff $\pi_P^+$ is a retraction in $\langle$~$A-\mathbf{mod}_1$ / $A-\mathbf{mod}$~$\rangle$.
\end{proposition}

Since $\langle$~$\square_{met}$-free / $\square_{top}$-free~$\rangle$ modules are the same up to isomorphisms in $A-\mathbf{mod}$ and any retraction in $A-\mathbf{mod}_1$ is a retraction in $A-\mathbf{mod}$, then from proposition \ref{RetrMetTopProjIsMetTopProj} we see that any metrically projective $A$-module is topologically projective. Recall that every relatively projective module is a retract in $A-\mathbf{mod}$ of $A_+\projtens E$ for some Banach space $E$, therefore every topologically projective $A$-module is relatively projective. We summarize all these in the following proposition.

\begin{proposition}\label{MetProjIsTopProjAndTopProjIsRelProj} Every metrically projective module is topologically projective and every topologically projective module is relatively projective.
\end{proposition}

A quantitative version of the definition of topological projectivity was given by White, though he used a somewhat different terminology.

\begin{definition}[\cite{WhiteInjmoduAlg}, definition 2.4]\label{CTopProjMod} An $A$-module $P$ is called  $C$-topologically projective if for any strictly $c$-topologically surjective $A$-morphism $\xi:X\to Y$ and for any $A$-morphism $\phi:P\to Y$ there exists an $A$-morphism $\psi:P\to X$ such that $\xi\psi=\phi$ and $\Vert \psi\Vert\leq c C\Vert\phi\Vert$.
\end{definition}

We shall need the following facts about this type of projectivity.

\begin{proposition}[\cite{WhiteInjmoduAlg}, lemma 2.7]\label{RetrCTopProjIsCTopProj} Any $c$-retract of $C$-topologically projective module is $cC$-topologically projective.
\end{proposition}

\begin{proposition}[\cite{WhiteInjmoduAlg}, proposition 2.10]\label{CTopProjModViaCanonicMorph} An $A$-module $P$ is $C$-topologically projective iff $\pi_P^+$ is a $C$-retraction in $A-\mathbf{mod}$.
\end{proposition}

As the consequence, a Banach module is topologically projective iff it is $C$-topologically projective for some $C$. One more important consequence of proposition \ref{CTopProjModViaCanonicMorph} is the following.

\begin{proposition}\label{MetProjIsOneTopProj} An $A$-module is metrically projective iff it is $1$-topologically projective.
\end{proposition}

We shall exploit both definitions \ref{TopProjMod} and \ref{CTopProjMod} without further reference for their equivalence. 

Let us proceed to examples. Note that the category of Banach spaces may be regarded as the category of left Banach modules over zero algebra. As the results we get the definition of $\langle$~metrically / topologically~$\rangle$ projective Banach space. All the results mentioned above hold for this type of projectivity. Both types of projective objects are described by now. In \cite{KotheTopProjBanSp} K{\"o}the proved that all topologically projective Banach spaces are topologically isomorphic to $\ell_1(\Lambda)$ for some index set $\Lambda$. Using result of Grothendieck from \cite{GrothMetrProjFlatBanSp} Helemskii showed that metrically projective Banach spaces are isometrically isomorphic to $\ell_1(\Lambda)$ for some index set $\Lambda$ [\cite{HelMetrFrQMod}, proposition 3.2]. Thus the zoo of projective Banach spaces is wide but conformed.

\begin{proposition}\label{UnitalAlgIsMetTopProj} The left $A$-module $A_\times$ is metrically and $1$-topologically projective.
\end{proposition} 
\begin{proof} Consider arbitrary $A$-morphism $\phi:A_\times\to Y$ and a strictly coisometric $A$-morphism $\xi:X\to Y$. From the definition of strictly coisometric operator it follows that there exists an $x_0\in X$ such that $\xi(x_0)=\phi(e_{A_\times})$ and $\Vert x_0\Vert=\Vert\phi(e_{A_\times})\Vert$. Consider $A$-morphism $\psi:A_\times\to X:a\mapsto a\cdot x_0$. Clearly, $\Vert\psi\Vert\leq\Vert x_0\Vert\leq\Vert\phi\Vert\Vert e_{A_\times}\Vert=\Vert\phi\Vert$. On the other hand $\xi\psi=\phi$, so $\Vert\phi\Vert\leq\Vert\xi\Vert\Vert\psi\Vert=\Vert\psi\Vert$. Thus $\Vert\phi\Vert=\Vert\psi\Vert$. Therefore we proved by definition, that $A_\times$ is metrically projective $A$-module. By proposition \ref{MetProjIsOneTopProj} it is $1$-topologically projective $A$-module.
\end{proof}

\begin{proposition}\label{NonDegenMetTopProjCharac}  Let $P$ be an essential $A$-module. Then $P$ is $\langle$~metrically / $C$-topologically~$\rangle$ projective iff the map $\pi_P:A\projtens\ell_1(B_P):a\projtens\delta_x\mapsto a\cdot x$ is a $\langle$~$1$-retraction / $C$-retraction~$\rangle$ in $A-\mathbf{mod}$.
\end{proposition} 
\begin{proof}
If $P$ is $\langle$~metrically / $C$-topologically~$\rangle$ projective, then by proposition $\langle$~\ref{MetTopProjModViaCanonicMorph} / \ref{CTopProjModViaCanonicMorph}~$\rangle$ the morphism $\pi_P^+$ has a right inverse morphism $\sigma^+$ of norm $\langle$~at most $1$ / at most $C$~$\rangle$. Then 
$$
	\sigma^+(P)=\sigma^+(\operatorname{cl}_{A_+\projtens\ell_1(B_P)}(AP))\subset \operatorname{cl}_{A_+\projtens\ell_1(B_P)}(A\cdot\sigma(P))=
$$
$$\operatorname{cl}_{A_+\projtens\ell_1(B_P)}(A\cdot(A_+\projtens\ell_1(B_P)))=A\projtens\ell_1(B_P).
$$ 
So we have well a defined corestriction $\sigma:P\to A\projtens\ell_1(B_P)$ which is also an $A$-morphism with norm $\langle$~at most $1$ / at most $C$~$\rangle$. Clearly, $\pi_P\sigma=1_P$, so $\pi_P$ is a $\langle$~$1$-retraction / $C$-retraction~$\rangle$ in $A-\mathbf{mod}$.

Conversely, assume $\pi_P$ has a right inverse morphism $\sigma$ of norm $\langle$~at most $1$ / at most $C$~$\rangle$. Then its coextension $\sigma^+$ also is a right inverse morphism to $\pi_P^+$ with the same norm. Again, by proposition $\langle$~\ref{MetTopProjModViaCanonicMorph} / \ref{CTopProjModViaCanonicMorph}~$\rangle$ the module $P$ is $\langle$~metrically / $C$-topologically~$\rangle$ projective. 
\end{proof}

It is worth to mention that $\langle$~an arbitrary / only finite~$\rangle$ family of objects in $\langle$~$A-\mathbf{mod}_1$ / $A-\mathbf{mod}$~$\rangle$ have the categorical coproduct which is in fact their $\bigoplus_1$-sum. This is reason why we make additional assumption in the second paragraph of the next proposition.

\begin{proposition}\label{MetTopProjModCoprod} Let $(P_\lambda)_{\lambda\in\Lambda}$ be a family of $A$-modules. Then 

$i)$ $\bigoplus_1\{P_\lambda:\lambda\in\Lambda\}$ is metrically projective iff for all $\lambda\in\Lambda$ the $A$-module $P_\lambda$ is metrically projective;

$ii)$ $\bigoplus_1\{P_\lambda:\lambda\in\Lambda\}$ is $C$-topologically projective iff for all $\lambda\in\Lambda$ the $A$-module $P_\lambda$ is $C$-topologically projective.
\end{proposition}
\begin{proof} Denote $P:=\bigoplus_1\{P_\lambda:\lambda\in\Lambda\}$.

$i)$ The proof is literally the same as in paragraph $ii)$.

$ii)$ Assume that $P$ is $C$-topologically projective. Note that, for any $\lambda\in\Lambda$ the $A$-module $P_\lambda$ is a $1$-retract of $P$ via natural projection $p_\lambda:P\to P_\lambda$. By proposition \ref{RetrCTopProjIsCTopProj} the $A$-module $P_\lambda$ is $C$-topologically projective.

Conversely, let each $A$-module $P_\lambda$ be $C$-topologically projective. By proposition \ref{CTopProjModViaCanonicMorph} we have a family of $C$-retractions $\pi_\lambda:A_+\projtens\ell_1(S_\lambda)\to P_\lambda$. It follows that $\bigoplus_1\{\pi_\lambda:\lambda\in\Lambda\}$ is a $C$-retraction in $A-\mathbf{mod}$. As the result $P$ is a $C$-retract of 
$$
\bigoplus\nolimits_1\left\{A_+\projtens \ell_1(S_\lambda):\lambda\in\Lambda\right\}
\isom{A-\mathbf{mod}_1}
\bigoplus\nolimits_1\left\{\bigoplus\nolimits_1\{A_+:s\in S_\lambda\}:\lambda\in\Lambda\right\}
$$
$$
\isom{A-\mathbf{mod}_1}
\bigoplus\nolimits_1\{A_+:s\in S\}
$$
in $A-\mathbf{mod}$ where $S=\bigsqcup_{\lambda\in\Lambda}S_\lambda$. Clearly, the latter module is $1$-topologically projective, so by proposition \ref{RetrCTopProjIsCTopProj} the $A$-module $P$ is $C$-topologically projective.
\end{proof}

\begin{corollary}\label{MetTopProjTensProdWithl1} Let $P$ be an $A$-module and $\Lambda$ be an arbitrary set. Then $P\projtens \ell_1(\Lambda)$ is $\langle$~metrically / $C$-topologically~$\rangle$ projective iff $P$ is $\langle$~metrically / $C$-topologically~$\rangle$ projective.
\end{corollary}
\begin{proof} 
Note that $P\projtens \ell_1(\Lambda)\isom{A-\mathbf{mod}_1}\bigoplus_1\{P:\lambda\in\Lambda\}$. It remains to set $P_\lambda=P$ for all $\lambda\in\Lambda$ and apply proposition \ref{MetTopProjModCoprod}.
\end{proof}

%----------------------------------------------------------------------------------------
%	Metric and topological projectivity of ideals and cyclic modules
%----------------------------------------------------------------------------------------

\subsection{Metric and topological projectivity of ideals and cyclic modules}
\label{SubSectionMetricAndTopologicalProjectivityOfIdealsAndCyclicModules}

As we shall see idempotents plays a significant role in the study of metric and topological projectivity, so we shall recall one of the corollaries of Shilov's idempotent theorem [\cite{KaniBanAlg}, section 3.5]: every semisimple commutative Banach algebra with compact spectrum admits an identity, but not necessarily of norm 1. 

\begin{proposition}\label{UnIdeallIsMetTopProj}
Let $I$ be a left ideal of a Banach algebra $A$. Then

$i)$ if $I=Ap$ for some $\langle$~norm one idempotent / idempotent~$\rangle$ $p\in I$, then $I$ is $\langle$~metrically / $\Vert p\Vert$-topologically~$\rangle$ projective $A$-module;

$ii)$ if $I$ is commutative, semisimple and $\operatorname{Spec}(I)$ is compact then $I$ is topologically projective $A$-module.
\end{proposition}
\begin{proof} 
$i)$ For $A$-module maps $\pi:A_\times\to I:x\mapsto xp$ and $\sigma:I\to A_\times:x\mapsto x$ we clearly have $\pi\sigma=1_I$. Therefore $I$ is a $\langle$~$1$-retract / $\Vert p\Vert$-retract~$\rangle$ of $A_\times$. Now the result follows from propositions $\langle$~\ref{RetrMetTopProjIsMetTopProj} / \ref{RetrCTopProjIsCTopProj}~$\rangle$ and \ref{UnitalAlgIsMetTopProj}.

$ii)$ By Shilov's idempotent theorem the ideal $I$ is unital. In general the norm of this unit is not less than $1$. By paragraph $i)$ the ideal $I$ is topologically projective.
\end{proof}

The assumption of semisimplicity in \ref{UnIdeallIsMetTopProj} is not necessary.
From [\cite{DalesIntroBanAlgOpHarmAnal}, exercise 2.3.7] we know, that there exists a commutative non semisimple unital Banach algebra $A$. By proposition \ref{UnIdeallIsMetTopProj} it is topologically projective as $A$-module. 
To prove the main result of this section we need two preparatory lemmas.

\begin{lemma}\label{ImgOfAMorphFromBiIdToA} Let $I$ be a two-sided ideal of a Banach algebra $A$ which is essential as left $I$-module and let $\phi:I\to A$ be an $A$-morphism. Then $\operatorname{Im}(\phi)\subset I$.
\end{lemma}
\begin{proof} Since $I$ is a right ideal, then $\phi(ab)=a\phi(b)\in I$ for all $a,b\in I$. So $\phi(I\cdot I)\subset I$. Since $I$ is an essential left $I$-module then $I=\operatorname{cl}_A(\operatorname{span}(I\cdot I))$ and $\operatorname{Im}(\phi)\subset\operatorname{cl}_A(\operatorname{span}\phi(I\cdot I))=\operatorname{cl}_A(\operatorname{span}I)=I$.
\end{proof}

\begin{lemma}\label{GoodIdealMetTopProjIsUnital} Let $I$ be a left ideal of a Banach algebra $A$, which is $\langle$~metrically / $C$-topologically~$\rangle$ projective as $A$-module. Then the following holds:

$i)$ Assume $I$ has a left $\langle$~contractive / $c$-bounded~$\rangle$ approximate identity and for each  morphism $\phi:I\to A$ of left $A$-modules there exists a morphism $\psi:I\to I$ of right $I$-modules such that $\phi(x)y=x\psi(y)$ for all $x,y\in I$. Then $I$ has the identity of norm $\langle$~at most $1$ / at most $c$~$\rangle$;

$ii)$ Assume $I$ has a right $\langle$~contractive / $c$-bounded~$\rangle$ approximate identity and for $\langle$~$k=1$ / some $k\geq 1$~$\rangle$ and each morphism $\phi:I\to A$ of left $A$-modules there exists a morphism $\psi:I\to I$ of right $I$-modules such that $\Vert\psi\Vert\leq k\Vert\phi\Vert$ and $\phi(x)y=x\psi(y)$ for all $x,y\in I$. Then $I$ has a right identity of norm $\langle$~at most $1$ / at most $ckC$~$\rangle$.
\end{lemma} 
\begin{proof} If either $i)$ or $ii)$ holds then $I$ has a one-sided bounded approximate identity. So $I$ is an essential left $I$-module, and a fortiori an essential $A$-module. By proposition \ref{NonDegenMetTopProjCharac} we have a right inverse $A$-morphism $\sigma:I\to A\projtens \ell_1(B_I)$ of $\pi_I$ with norm  $\langle$~at most $1$ / at most $C$~$\rangle$. For each $d\in B_I$ consider $A$-morphisms $p_d:A\projtens \ell_1(B_I)\to A:a\projtens \delta_x\mapsto \delta_x(d)a$ and $\sigma_d=p_d\sigma$. Then $\sigma(x)=\sum_{d\in B_I}\sigma_d(x)\projtens \delta_d$ for all $x\in I$. From identification $A\projtens\ell_1(B_I)\isom{\mathbf{Ban}_1}\bigoplus_1\{A:d\in B_I\}$ we have $\Vert\sigma(x)\Vert=\sum_{d\in B_I} \Vert\sigma_d(x)\Vert$ for all $x\in I$. Since $\sigma$ is a right inverse morphism of $\pi_I$, then $x=\pi_I(\sigma(x))=\sum_{d\in B_I}\sigma_d(x)d$ for all $x\in I$. 

From assumption, for each $d\in B_I$ there exists a morphism of right $I$-modules $\tau_d:I\to I$ such that $\sigma_d(x)d=x\tau_d(d)$ for all $x\in I$. 

Assume $i)$ holds. From assumption, for each $d\in B_I$ there exists a morphism of right $I$-modules $\tau_d:I\to I$ such that $\sigma_d(x)d=x\tau_d(d)$ for all $x\in I$.  Let $(e_\nu)_{\nu\in N}$ be a left $\langle$~contractive / bounded~$\rangle$ approximate identity of $I$ bounded in norm by constant $\langle$~$D=1$ / $D=c$~$\rangle$. Since $\tau_d(d)\in I$ for all $d\in B_I$, then for all $S\in\mathcal{P}_0(B_I)$ holds
$$
\sum_{d\in S}\Vert \tau_d(d)\Vert
=\sum_{d\in S}\lim_{\nu}\Vert e_\nu \tau_d(d) \Vert
=\lim_{\nu}\sum_{d\in S}\Vert e_\nu \tau_d(d)\Vert
=\lim_{\nu}\sum_{d\in S}\Vert \sigma_d(e_\nu)d \Vert
$$
$$
\leq\liminf_{\nu}\sum_{d\in S}\Vert\sigma_d(e_\nu)\Vert\Vert d\Vert 
\leq\liminf_{\nu}\sum_{d\in S}\Vert\sigma_d(e_\nu)\Vert
\leq\liminf_{\nu}\sum_{d\in B_I}\Vert\sigma_d(e_\nu)\Vert
$$
$$
=\liminf_{\nu}\Vert\sigma(e_\nu)\Vert
\leq\Vert\sigma\Vert\liminf_{\nu}\Vert e_\nu\Vert
\leq D\Vert\sigma\Vert 
$$
Since $S\in \mathcal{P}_0(B_I)$ is arbitrary we have well defined element $p=\sum_{d\in B_I}\tau_d(d)$ with norm $\langle$ at most $1$ / at most $cC$~$\rangle$. For all $x\in I$ we have $x=\sum_{d\in B_I}\sigma_d(x)d=\sum_{d\in B_I}x\tau_d(d)=xp$, i.e. $p$ is a right identity for $I$. But $I$ admits a left $\langle$~contractive / $c$-bounded~$\rangle$ approximate identity, so $p$ is the identity of $I$ with $\Vert p\Vert=\lim_\nu\Vert e_\nu\Vert$. Therefore the norm of $p$ is $\langle$~at most $1$ / at most $c$~$\rangle$.

Assume $ii)$ holds. From assumption, for each $d\in B_I$ there exists a morphism of right $I$-modules $\tau_d:I\to I$ such that $\sigma_d(x)d=x\tau_d(d)$ for all $x\in I$ and $\Vert\tau_d\Vert\leq k\Vert\sigma_d\Vert$.  Let $(e_\nu)_{\nu\in N}$ be a right $\langle$~contractive / bounded~$\rangle$ approximate identity of $I$ bounded in norm by constant $\langle$~$D=1$ / $D=c$~$\rangle$. For all $x\in I$ we have
$$
\Vert\sigma_d(x)\Vert
=\Vert\sigma_d(\lim_\nu x e_\nu)\Vert
=\lim_\nu\Vert x\sigma_d(e_\nu)\Vert
\leq\Vert x\Vert\liminf_\nu\Vert\sigma_d(e_\nu)\Vert
$$
so $\Vert\sigma_d\Vert\leq \liminf_\nu\Vert\sigma_d(e_\nu)\Vert$. Then for all $S\in\mathcal{P}_0(B_I)$ holds
$$
\sum_{d\in S}\Vert \tau_d(d)\Vert
\leq \sum_{d\in S}\Vert \tau_d\Vert\Vert d\Vert
\leq k\sum_{d\in S}\Vert \sigma_d\Vert
\leq k\sum_{d\in S}\liminf_\nu \Vert \sigma_d(e_\nu)\Vert
\leq k\liminf_{\nu}\sum_{d\in S}\Vert \sigma_d(e_\nu) \Vert
$$
$$
\leq k\liminf_{\nu}\sum_{d\in B_I}\Vert \sigma_d(e_\nu) \Vert
=k\liminf_{\nu}\Vert\sigma(e_\nu)\Vert
\leq k\Vert\sigma\Vert\liminf_{\nu}\Vert e_\nu\Vert
\leq kD\Vert\sigma\Vert
$$
Since $S\in \mathcal{P}_0(B_I)$ is arbitrary we have well defined element $p=\sum_{d\in B_I}\tau_d(d)$ with norm $\langle$ at most $1$ / at most $ckC$~$\rangle$. For all $x\in I$ we have $x=\sum_{d\in B_I}\sigma_d(x)d=\sum_{d\in B_I}x\tau_d(d)=xp$, i.e. $p$ is a right identity for $I$.
\end{proof}

\begin{theorem}\label{GoodCommIdealMetTopProjIsUnital} Let $I$ be an ideal of a commutative Banach algebra $A$ and $I$ has a $\langle$~contractive / $c$-bounded~$\rangle$ approximate identity. Then $I$ is $\langle$~metrically / $c$-topologically~$\rangle$ projective as $A$-module iff $I$ has the identity of norm $\langle$~at most $1$ / at most $c$~$\rangle$.
\end{theorem} 
\begin{proof} Assume $I$ is $\langle$~metrically / $c$-topologically~$\rangle$ projective as $A$-module. Since $A$ is commutative, then for any $A$-morphism $\phi:I\to A$ and $x,y\in I$ we have $\phi(x)y=x\phi(y)$. Since $I$ has bounded approximate identity and $I$ is commutative we can apply lemma \ref{ImgOfAMorphFromBiIdToA} to get that $\phi(y)\in I$. Now by paragraph $i)$ of lemma \ref{GoodIdealMetTopProjIsUnital} we get that $I$ has the identity of norm $\langle$~at most $1$ / at most $c$~$\rangle$.

The converse immediately follows from proposition \ref{UnIdeallIsMetTopProj}.
\end{proof}

There is no analogous criterion of this theorem in relative theory. The most general result of this kind gives only a necessary condition: any ideal in a commutative Banach algebra $A$ which is relatively projective as $A$-module has a paracompact spectrum. This result is due to Helemskii [\cite{HelHomolBanTopAlg}, theorem IV.3.6]. 

Note that existence of bounded approximate identity is not necessary for the ideal of a commutative Banach algebra to be even topologically projective. Indeed, consider Banach algebra  $A_0(\mathbb{D})$ --- the ideal of disk algebra consisting of functions vanishing at zero. By combination of propositions 4.3.5  and 4.3.13 paragraph $iii)$ from \cite{DalBanAlgAutCont} we get that $A_0(\mathbb{D})$ has no bounded approximate identities. On the other hand, from [\cite{HelBanLocConvAlg}, example IV.2.2] we know that $A_0(\mathbb{D})\isom{A_0(\mathbb{D})-\mathbf{mod}} A_0(\mathbb{D})_+$, so $A_0(\mathbb{D})$ is topologically projective by proposition \ref{UnitalAlgIsMetTopProj}.

Next proposition is an obvious adaptation of purely algebraic argument on projective cyclic modules. It is almost identical to [\cite{WhiteInjmoduAlg}, proposition 2.11].

\begin{proposition}\label{MetTopProjCycModCharac} Let $I$ be a left ideal in $A_\times $. Assume the natural projection $\pi:A_\times\to A_\times/I$ is $\langle$~strictly coisometric / strictly $c$-topologically surjective~$\rangle$. Then the following holds:

$i)$ If $A_\times /I$ is $\langle$~metrically / $C$-topologically~$\rangle$ projective as $A$-module, then there exists an idempotent $p\in I$ such that $I=Ap$ and $\Vert e_{A_\times}-p\Vert$ is $\langle$~at most $1$ / at most $cC$~$\rangle$;

$ii)$ If there exists an idempotent $p\in I$ such that $I=A_\times  p$ and $\Vert e_{A_\times }-p\Vert$ is $\langle$~at most $1$ / at most $C$~$\rangle$, then $A_\times/I$ is $\langle$~metrically / $C$-topologically~$\rangle$ projective.
\end{proposition}
\begin{proof} $i)$ Since the natural quotient map $\pi$ is $\langle$~strictly coisometric / strictly $c$-topologically surjective~$\rangle$ and $A_\times /I$ is $\langle$~metrically / $C$-topologically~$\rangle$ projective, then $\pi$ has a right inverse $A$-morphism $\sigma$ with norm $\langle$~at most $1$ / at most $cC$~$\rangle$. We set $e_{A_\times }-p=(\sigma\pi)(e_{A_\times })$, then $(\sigma\pi)(a)=a(e_{A_\times }-p)$. By construction, $\pi\sigma=1_{A_\times }$, so  
$$
e_{A_\times }-p=(\sigma\pi)(e_{A_\times })=(\sigma\pi)(\sigma\pi)(e_{A_\times })=(\sigma\pi)(e_{A_\times }-p)=(e_{A_\times }-p)(\sigma\pi)(e_{A_\times })=(e_{A_\times }-p)^2
$$
This equality shows that $p^2=p$. Therefore $A_\times p=\operatorname{Ker}(\sigma\pi)$ because $(\sigma\pi)(a)=a-ap$. Since $\sigma$ is injective this is equivalent to $A_\times p=\operatorname{Ker}(\pi)$ which equals to $I$. Finally, note that $\Vert e_{A_\times }-p\Vert=\Vert(\sigma\pi)(e_{A_\times})\Vert\leq\Vert\sigma\Vert\Vert\pi\Vert\Vert e_{A_\times }\Vert=\Vert\sigma\Vert$.

$ii)$ Since $p^2=p$, then we have a well defined left ideal $I=A_\times p$ and an $A$-module map $\sigma:A_\times /I\to A_\times:a+I\mapsto a-ap$. It is easy to check that  $\pi\sigma=1_{A_\times/I}$ and $\Vert\sigma\Vert\leq\Vert e_{A_\times }-p\Vert$. This means that $\pi:A_\times \to A_\times /I$ is a $\langle$~$1$-retraction / $C$-retraction~$\rangle$. From propositions \ref{UnitalAlgIsMetTopProj} and $\langle$~\ref{RetrMetTopProjIsMetTopProj} / \ref{RetrCTopProjIsCTopProj}~$\rangle$ it follows that $A_\times /I$ is $\langle$~metrically / $C$-topologically~$\rangle$ projective.
\end{proof} 

In contrast with topological theory, there is no description of relatively projective cyclic modules. There are partial answers under additional assumptions. For example, if an ideal $I$ is complemented as Banach space in $A_\times$, then almost the same criterion as in previous proposition holds in relative theory [\cite{HelBanLocConvAlg}, proposition 7.1.29]. There are other characterizations of relatively projective cyclic modules under more mild assumptions on Banach geometry. For example, Selivanov proved that if $I$ is a two-sided ideal and either $A/I$ has the approximation property or all irreducible $A$-modules have the approximation property, then $A/I$ is relatively projective iff $A_\times\isom{A-\mathbf{mod}}I\bigoplus_1 I'$ for some left ideal $I'$ of $A$. For details see [\cite{HelHomolBanTopAlg}, chapter IV, \S 4].



%----------------------------------------------------------------------------------------
%	Metric and topological injectivity
%----------------------------------------------------------------------------------------

\subsection{Metric and topological injectivity}
\label{SubSectionMetricAndTopologicalInjectivity}

Unless otherwise stated we shall consider injectivity of right modules.

\begin{definition}[\cite{HelMetrFrQMod}, definition 4.3]\label{MetInjMod} An $A$-module $J$ is called metrically injective if for any isometric $A$-morphism $\xi:Y\to X$ and any $A$-morphism $\phi:Y\to J$ there exists an $A$-morphism $\psi:X\to J$ such that $\psi\xi=\phi$  and $\Vert\psi\Vert=\Vert\phi\Vert$.
\end{definition}

\begin{definition}[\cite{HelMetrFrQMod}, definition 4.3]\label{TopInjMod} An $A$-module $J$ is called  topologically injective if for any topologically injective $A$-morphism $\xi:Y\to X$ and any $A$-morphism $\phi:Y\to J$ there exists an $A$-morphism $\psi:X\to J$ such that $\psi\xi=\phi$.
\end{definition}

A short but more involved equivalent definition is the following: an $A$-module $J$ is called $\langle$~metrically / topologically~$\rangle$ injective, if the functor $\langle$~$\operatorname{Hom}_{\mathbf{mod}_1-A}(-,J):\mathbf{mod}_1-A\to\mathbf{Ban}_1$ / $\operatorname{Hom}_{\mathbf{mod}-A}(-,J):\mathbf{mod}-A\to\mathbf{Ban}$~$\rangle$ maps $\langle$~isometric / topologically injective~$\rangle$ $A$-morphisms into $\langle$~strictly coisometric / surjective~$\rangle$ operators.

In \cite{HelMetrFrQMod} and \cite{ShtTopFrClassicQuantMod} there were constructed two faithful functors:
$$
\square_{met}^d:A-\mathbf{mod}_1\to\mathbf{Set}:X\mapsto B_{X^*},\phi\mapsto\phi^*|_{B_{Y^*}}^{B_{X^*}},
$$
$$
\square_{top}^d:A-\mathbf{mod}\to\mathbf{HNor}:X\mapsto X^*,\phi\mapsto\phi^*
$$
The first one sends a Banach $A$-module to the unit ball of its dual and any contractive $A$-morphism to the respective birestriction of its adjoint, while the second one ``forgets'' about module and additive structure of the dual module and adjoint $A$-morphism.
In the same papers it was proved that an $A$-morphism $\xi$ is $\langle$~isometric / topologically injective~$\rangle$ iff it is $\langle$~$\square_{met}^d$-admissible / $\square_{top}^d$-admissible~$\rangle$ monomorphism and an $A$-module $J$ is $\langle$~metrically / topologically~$\rangle$ injective iff it is $\langle$~$\square_{met}^d$-injective / $\square_{top}^d$-injective~$\rangle$. Thus, from general categorical scheme we immediately get the following proposition.

\begin{proposition}\label{RetrMetTopInjIsMetTopInj} Any retract of $\langle$~metrically / topologically~$\rangle$ injective module in $\langle$~$A-\mathbf{mod}_1$ / $A-\mathbf{mod}$~$\rangle$ is again $\langle$~metrically / topologically~$\rangle$ injective.
\end{proposition}

It was also shown that the rigged category $\langle$~$(\mathbf{mod}_1-A,\square_{met}^d)$ / $(\mathbf{mod}-A,\square_{top}^d)$~$\rangle$ is cofreedom loving and that $\langle$~$\square_{met}^d$-cofree / $\square_{top}^d$-cofree~$\rangle$ modules are isomorphic in $\langle$~$\mathbf{mod}_1-A$ / $\mathbf{mod}-A$~$\rangle$ to $\mathcal{B}(A_+, \ell_\infty(\Lambda))$ for some set $\Lambda$. Even more, for any $A$-module $X$ there exists a $\langle$~$\square_{met}^d$-admissible / $\square_{top}^d$-admissible~$\rangle$ monomorphism
$$
\rho_X^+:X\to\mathcal{B}(A_+,\ell_\infty(B_{X^*})):x\mapsto(a\mapsto(f\mapsto f(x\cdot a)))
$$
As the consequence of general result on rigged categories we get

\begin{proposition}\label{MetTopInjModViaCanonicMorph}
The module $J$ is $\langle$~metrically / topologically~$\rangle$ injective iff $\rho_J^+$ is a coretraction in $\langle$~$\mathbf{mod}_1-A$ / $\mathbf{mod}-A$~$\rangle$.
\end{proposition}

Since $\langle$~$\square_{met}^d$-cofree / $\square_{top}^d$-cofree~$\rangle$ modules are the same up to isomorphisms in $\mathbf{mod}-A$ and any retraction in $\mathbf{mod}_1-A$ is a retraction in $\mathbf{mod}-A$, then from proposition \ref{RetrMetTopInjIsMetTopInj} we see that any metrically injective $A$-module is topologically injective. Recall that every relatively injective module is a retract in $\mathbf{mod}-A$ of $\mathcal{B}(A_+,E)$ for some Banach space $E$, therefore every topologically injective $A$-module is relatively injective. We summarize all these in the following proposition.

\begin{proposition}\label{MetInjIsTopInjAndTopInjIsRelInj} Every metrically injective module is topologically injective and every topologically injective module is relatively injective.
\end{proposition}

A quantitative version of the definition of topological injectivity was given by White,  though he used a somewhat different terminology.

\begin{definition}[\cite{WhiteInjmoduAlg}, definition 3.4]\label{CTopInjMod} An $A$-module $J$ is called  $C$-topologically injective if for any $c$-topologically injective $A$-morphism $\xi:Y\to X$ and any $A$-morphism $\phi:Y\to J$ there exists an $A$-morphism $\psi:X\to J$ such that $\psi\xi=\phi$ and $\Vert\psi\Vert\leq c C\Vert\phi\Vert$.
\end{definition}

We shall need the following facts about this type of injectivity.

\begin{proposition}[\cite{WhiteInjmoduAlg}, lemma 3.7]\label{RetrCTopInjIsCTopInj} Any $c$-retract of $C$-topologically injective module is $cC$-topologically injective.
\end{proposition}

\begin{proposition}[\cite{WhiteInjmoduAlg}, proposition 3.10]\label{CTopInjModViaCanonicMorph} An $A$-module $J$ is $C$-topologically injective iff $\rho_J^+$ is a $C$-coretraction in $\mathbf{mod}-A$.
\end{proposition}

As the consequence, a Banach module is topologically injective iff it is $C$-topologically injective for some $C$. One more important consequence of proposition \ref{CTopInjModViaCanonicMorph} is the following.

\begin{proposition}\label{MetInjIsOneTopInj} An $A$-module is metrically injective iff it is $1$-topologically injective.
\end{proposition}

We shall exploit both definitions \ref{TopInjMod} and \ref{CTopInjMod} without further reference for their equivalence. 

Let us proceed to examples. If we regard the category of Banach spaces as the category of right Banach modules over zero algebra, we may speak of $\langle$~metrically / topologically~$\rangle$ injective Banach spaces. All results mentioned above hold for this type of injectivity. An equivalent definition says that a Banach space is $\langle$~metrically / topologically~$\rangle$ injective if it is $\langle$~contractively complemented / complemented ~$\rangle$ in any ambient Banach space. The typical examples of metrically injective Banach spaces are $L_\infty$-spaces. Only metrically injective Banach spaces are completely understood --- these spaces are isometrically isomorphic to $C(K)$-space for some extremely disconnected compact Hausdorff space $K$ [\cite{LaceyIsomThOfClassicBanSp}, theorem 3.11.6]. Usually such topological spaces are referred to as Stonean spaces.  For the contemporary results on topologically injective Banach spaces see [\cite{JohnLinHandbookGeomBanSp}, chapter 40].

\begin{proposition}\label{DualOfUnitalAlgIsMetTopInj} The right $A$-module $A_\times^*$ is metrically and $1$-topologically injective.
\end{proposition}
\begin{proof} Consider arbitrary $A$-morphism $\phi:Y\to A_\times^*$ and isometric $A$-morphism $\xi:Y\to X$. Define a bounded linear functional $f:Y\to\mathbb{C}:y\mapsto \phi(y)(e_{A_\times})$. Since $\xi$ is an isometry, then by Hahn-Banach theorem we can extend $f$ to some bounded linear functional $g:X\to\mathbb{C}$ with the same norm. Consider $A$-morphism $\psi:X\to A_\times^*:x\mapsto (a\mapsto g(x\cdot a))$. Clearly, $\Vert\psi\Vert\leq\Vert g\Vert=\Vert f\Vert\leq\Vert\phi\Vert$. On the other hand $\psi\xi=\phi$, so $\Vert\phi\Vert\leq\Vert\psi\Vert\Vert\xi\Vert=\Vert\psi\Vert$. Thus $\Vert\phi\Vert=\Vert\psi\Vert$. Therefore we proved by definition, that $A_\times^*$ is metrically injective $A$-module. By proposition \ref{MetInjIsOneTopInj} it is $1$-topologically injective $A$-module.
\end{proof}

\begin{proposition}\label{NonDegenMetTopInjCharac}  Let $J$ be a faithful $A$-module. Then $J$ is $\langle$~metrically / $C$-topologically~$\rangle$ injective iff the map $\rho_J:J\to\mathcal{B}(A,\ell_\infty(B_{J^*})):x\mapsto(a\mapsto(f\mapsto f(x\cdot a)))$ is a $\langle$~$1$-coretraction / $C$-coretraction~$\rangle$ in $\mathbf{mod}-A$.
\end{proposition} 
\begin{proof}
If $J$ is $\langle$~metrically / $C$-topologically~$\rangle$ injective, then by proposition $\langle$~\ref{MetTopInjModViaCanonicMorph} / \ref{CTopInjModViaCanonicMorph}~$\rangle$ the $A$-morphism $\rho_J^+$ has right inverse morphism $\tau^+$, with norm $\langle$~at most $1$ / at most $C$~$\rangle$. Assume we are given an operator $T\in \mathcal{B}(A_+,\ell_\infty(B_{J^*}))$, such that $T|_A=0$. Fix $a\in A$, then $T\cdot a=0$, and so $\tau^+(T)\cdot a=\tau^+(T\cdot a)=0$. Since $J$ is faithful and $a\in A$ is arbitrary, then $\tau^+(T)=0$. Define $p:A_+\to A$ be the natural projection from $A_+$ onto $A$, then define $A$-morphisms $j=\mathcal{B}(p,\ell_\infty(B_{J^*}))$ and $\tau =\tau^+ j$. For any $a\in A$ and $T\in\mathcal{B}(A,\ell_\infty(B_{J^*}))$ we have $\tau (T\cdot a)-\tau (T)\cdot a=\tau^+(j(T\cdot a)-j(T)\cdot a)=0$, because $j(T\cdot a)-j(T)\cdot a|_A=0$. Therefore $\tau $ is an $A$-morphism. Note that $\Vert\tau \Vert\leq\Vert\tau^+\Vert\Vert j\Vert\leq \Vert\tau^+\Vert$. Therefore $\tau$ has norm $\langle$~at most $1$ / at most $C$~$\rangle$. Obviously, for all $x\in J$ we have $\rho_J^+(x)-j(\rho_J(x))|_A=0$, so $\tau^+(\rho_J^+(x)-j(\rho_J(x)))=0$. As a consequence $\tau (\rho_J(x))=\tau^+(j(\rho_J(x)))=\tau^+(\rho_J^+(x))=x$ for all $x\in J$. Since $\tau \rho_J=1_J$, then $\rho_J$ is a  $\langle$~$1$-coretraction / $C$-coretraction~$\rangle$  in $\mathbf{mod}-A$.

Conversely, assume $\rho_J$ is a $\langle$~$1$-coretraction / $C$-coretraction~$\rangle$, that is has a right inverse morphism $\tau $ with norm $\langle$~at most $1$ / at most $C$~$\rangle$. Define $i:A\to A_+$ to be the natural embedding of $A$ into $A_+$ and define $A$-morphism $q=\mathcal{B}(i,\ell_\infty(B_{J^*}))$. Obviously, $\rho_J=q\rho_J^+$. Consider $A$-morphism $\tau^+=\tau q$. Note that $\Vert\tau^+\Vert\leq\Vert\tau \Vert\Vert q\Vert\leq \Vert\tau \Vert$. Therefore $\tau^+$ has norm $\langle$~at most $1$ / at most $C$~$\rangle$. Clearly $\tau^+\rho_J^+=\tau q\rho_J^+=\tau \rho_J=1_J$. So $\rho_J^+$ is a $\langle$~$1$-coretraction / $C$-coretraction~$\rangle$ and by proposition $\langle$~\ref{MetTopInjModViaCanonicMorph} / \ref{CTopInjModViaCanonicMorph}~$\rangle$ the $A$-module $J$ is $\langle$~metrically / $C$-topologically~$\rangle$ injective.
\end{proof}

It is worth to mention that $\langle$~arbitrary / only finite~$\rangle$ family of objects in $\langle$~$\mathbf{mod}_1-A$ / $\mathbf{mod}-A$~$\rangle$ have the categorical product which in fact is their $\bigoplus_\infty$-sum. This is the reason why we make additional assumption in the second paragraph of the next proposition.

\begin{proposition}\label{MetTopInjModProd} Let $(J_\lambda)_{\lambda\in\Lambda}$ be a family of $A$-modules. Then 

$i)$ $\bigoplus_\infty\{J_\lambda:\lambda\in\Lambda\}$ is metrically injective iff for all $\lambda\in\Lambda$ the $A$-module $J_\lambda$ is metrically injective;

$ii)$ $\bigoplus_\infty\{J_\lambda:\lambda\in\Lambda\}$ is $C$-topologically injective iff for all $\lambda\in\Lambda$ the $A$-module $J_\lambda$ is a $C$-topologically injective.
\end{proposition}
\begin{proof} Denote $J:=\bigoplus_\infty\{J_\lambda:\lambda\in\Lambda\}$.

$i)$ The proof is literally the same as in paragraph $ii)$.

$ii)$ Assume that $J$ is $C$-topologically injective. Note that, for any $\lambda\in\Lambda$ the $A$-module $J_\lambda$ is a $1$-retract of $J$ via natural projection $p_\lambda:J\to J_\lambda$. By proposition \ref{RetrCTopInjIsCTopInj} the $A$-module $J_\lambda$ is $C$-topologically injective.

Conversely, let each $A$-module $J_\lambda$ be $C$-topologically injective. By proposition \ref{CTopInjModViaCanonicMorph} we have a family of $C$-coretractions $\rho_\lambda:J_\lambda\to\mathcal{B}(A_+,\ell_\infty(S_\lambda))$. It follows that $\bigoplus_\infty\{\rho_\lambda:\lambda\in\Lambda\}$ is a $C$-coretraction in $A-\mathbf{mod}$. As the result $J$ is a $C$-retract of 
$$
\bigoplus\nolimits_\infty\{\mathcal{B}(A_+,\ell_\infty(S_\lambda)):\lambda\in\Lambda\}
\isom{\mathbf{mod}_1-A}
\bigoplus\nolimits_\infty\left\{\bigoplus\nolimits_\infty\{ A_+^*:s\in S_\lambda\}:\lambda\in\Lambda\right\}
\isom{\mathbf{mod}_1-A}
$$
$$
\bigoplus\nolimits_\infty\{A_+^*:s\in S\}
\isom{\mathbf{mod}_1-A}
\mathcal{B}(A_+,\ell_\infty(S))
$$
in $\mathbf{mod}-A$, where $S=\bigsqcup_{\lambda\in\Lambda}S_\lambda$. Clearly, the latter module is $1$-topologically injective, so by proposition \ref{RetrCTopInjIsCTopInj} the $A$-module $J$ is $C$-topologically injective.
\end{proof}

\begin{corollary}\label{MetTopInjlInftySum} Let $J$ be an $A$-module and $\Lambda$ be an arbitrary set. Then $\bigoplus_\infty\{J:\lambda\in\Lambda\}$ is $\langle$~metrically / $C$-topologically~$\rangle$ injective iff $J$ is $\langle$~metrically / $C$-topologically~$\rangle$ injective.
\end{corollary}
\begin{proof} The result immediately follows from proposition \ref{MetTopInjModProd} if one set $J_\lambda=J$ for all $\lambda\in\Lambda$.
\end{proof}

\begin{proposition}\label{MapsFroml1toMetTopInj} Let $J$ be an $A$-module and $\Lambda$ be an arbitrary set. Then $\mathcal{B}(\ell_1(\Lambda),J)$ is $\langle$~metrically / $C$-topologically~$\rangle$ injective iff $J$ is $\langle$~metrically / $C$-topologically~$\rangle$ injective.
\end{proposition}
\begin{proof} 
Assume $\mathcal{B}(\ell_1(\Lambda), J)$ is $\langle$~metrically / $C$-topologically~$\rangle$ injective. Take any $\lambda\in\Lambda$ and consider contractive $A$-morphisms $i_\lambda:J\to\mathcal{B}(\ell_1(\Lambda),J):x\mapsto(f\mapsto f(\lambda)x)$ and $p_\lambda:\mathcal{B}(\ell_1(\Lambda),J)\to J:T\mapsto T(\delta_\lambda)$. Clearly, $p_\lambda i_\lambda=1_J$, so by proposition $\langle$~\ref{RetrMetTopInjIsMetTopInj} / \ref{RetrCTopInjIsCTopInj}~$\rangle$ the $A$-module $J$ is $\langle$~metrically / $C$-topologically~$\rangle$ injective as $1$-retract of $\langle$~metrically / $1$-topologically~$\rangle$ injective $A$-module $\mathcal{B}(\ell_1(\Lambda),J)$.

Conversely, since $J$ is $\langle$~metrically / $C$-topologically~$\rangle$ injective, by proposition $\langle$~\ref{MetTopInjModViaCanonicMorph} / \ref{CTopInjModViaCanonicMorph}~$\rangle$ the $A$-morphism $\rho_J^+$ is a $\langle$~$1$-coretraction / $C$-coretraction~$\rangle$. Apply the functor $\mathcal{B}(\ell_1(\Lambda),-)$ to this coretraction to get another $\langle$~$1$-coretraction / $C$-coretraction~$\rangle$ denoted by $\mathcal{B}(\ell_1(\Lambda),\rho_J^+)$. Note that 
$$
\mathcal{B}(\ell_1(\Lambda),\ell_\infty(B_{J^*}))\isom{\mathbf{Ban}_1}(\ell_1(\Lambda)\projtens \ell_1(B_{J^*}))^*\isom{\mathbf{Ban}_1}\ell_1(\Lambda\times B_{J^*})^*\isom{\mathbf{Ban}_1}\ell_\infty(\Lambda\times B_{J^*}),
$$ 
so we have isometric isomorphisms of Banach modules
$$
\mathcal{B}(\ell_1(\Lambda),\mathcal{B}(A_+,\ell_\infty(B_{J^*})))\isom{\mathbf{mod}_1-A}\mathcal{B}(A_+,\mathcal{B}(\ell_1(\Lambda),\ell_\infty(B_{J^*}))\isom{\mathbf{mod}_1-A}\mathcal{B}(A_+,\ell_\infty(\Lambda\times B_{J^*})).
$$ 
Therefore $\mathcal{B}(\ell_1(\Lambda),J)$ is a $\langle$~$1$-retract / $C$-retract~$\rangle$ of $\langle$~metrically / $1$-topologically~$\rangle$ injective $A$-module $\mathcal{B}(A_+,\ell_\infty(\Lambda\times B_{J^*}))$. By proposition $\langle$~\ref{RetrMetTopInjIsMetTopInj} / \ref{RetrCTopInjIsCTopInj}~$\rangle$ the $A$-module $\mathcal{B}(\ell_1(\Lambda), J)$ is $\langle$~metrically / $C$-topologically~$\rangle$ injective.
\end{proof}

%----------------------------------------------------------------------------------------
%	Metric and topological flatness
%----------------------------------------------------------------------------------------

\subsection{Metric and topological flatness}
\label{SubSectionMetricAndTopologicalFlatness}

To save the homogeneity of notation we call metrically flat $A$-modules of \cite{HelMetrFlatNorMod} where they were named extremely flat.

\begin{definition}[\cite{HelMetrFlatNorMod}, I]\label{MetFlatMod} A left $A$-module $F$ is called metrically flat if for each isometric $A$-morphism $\xi:X\to Y$ of right $A$-modules the operator $\xi\projmodtens{A} 1_F:X\projmodtens{A} F\to Y\projmodtens{A} F$ is isometric.
\end{definition}

\begin{definition}[\cite{HelMetrFlatNorMod}, definition I]\label{TopFlatMod} A left $A$-module $F$ is called topologically flat if for each topologically injective $A$-morphism $\xi:X\to Y$ of right $A$-modules the operator $\xi\projmodtens{A} 1_F:X\projmodtens{A} F\to Y\projmodtens{A} F$ is topologically injective.
\end{definition}

A short but more involved definition is the following: an $A$-module $F$ is called $\langle$~metrically / topologically~$\rangle$ flat, if the functor $\langle$~$-\projmodtens{A}F:A-\mathbf{mod}_1\to\mathbf{Ban}_1$ / $-\projmodtens{A}F:A-\mathbf{mod}\to\mathbf{Ban}$~$\rangle$ maps $\langle$~isometric / topologically injective~$\rangle$ $A$-morphisms into $\langle$~isometric / topologically injective~$\rangle$ operators.

Again, regard the category of Banach spaces as the category of left Banach modules over zero algebra, then we get the definition of $\langle$~metrically / topologically~$\rangle$ flat Banach space. From Grothendieck's paper \cite{GrothMetrProjFlatBanSp} it follows that any metrically flat Banach space is isometrically isomorphic to $L_1(\Omega,\mu)$ for some measure space $(\Omega,\Sigma,\mu)$. For topologically flat Banach spaces, in contrast with topologically injective ones, we also have a criterion [\cite{DefFloTensNorOpId}, corollary 23.5(1)]: a Banach space is topologically flat iff it is an $\mathscr{L}_1^g$-space.


It is well known that an $A$-module $F$ is relatively flat iff $F^*$ is relatively injective [\cite{HelBanLocConvAlg}, theorem 7.1.42]. Next proposition is an obvious analog of this result.

\begin{proposition}\label{MetTopFlatCharac} The $A$-module $F$ is $\langle$~metrically / topologically~$\rangle$ flat iff $F^*$ is $\langle$~metrically / topologically~$\rangle$ injective.
\end{proposition}
\begin{proof} Consider any $\langle$~isometric / topologically injective~$\rangle$ morphism of right $A$-modules, call it $\xi:X\to Y$. The operator $\xi\projmodtens{A} 1_F$ is $\langle$~isometric / topologically injective~$\rangle$ iff the adjoint operator $(\xi\projmodtens{A} 1_F)^*$ is $\langle$~strictly coisometric / topologically surjective~$\rangle$  [\cite{HelLectAndExOnFuncAn}, exercises 4.4.6, 4.4.7]. Since operators $(\xi\projmodtens{A} 1_F)^*$ and $\mathcal{B}_A(\xi,F^*)$ are equivalent in $\mathbf{Ban}_1$ via universal property of projective module tensor product, then we get that $\xi\projmodtens{A} 1_F$ is $\langle$~isometric / topologically injective~$\rangle$ iff $\mathcal{B}_A(\xi,F^*)$ is $\langle$~strictly coisometric / topologically surjective~$\rangle$. Since $\xi$ is arbitrary we conclude that $F$ is  $\langle$~metrically / topologically~$\rangle$ flat iff $F^*$ is $\langle$~metrically / topologically~$\rangle$ injective.
\end{proof}

Combining proposition \ref{MetTopFlatCharac} with propositions  \ref{RetrMetTopInjIsMetTopInj} and \ref{MetInjIsTopInjAndTopInjIsRelInj} we get the following.

\begin{proposition}\label{RetrMetTopFlatIsMetTopFlat} Any retract of $\langle$~metrically / topologically~$\rangle$ flat module in $\langle$~$A-\mathbf{mod}_1$ / $A-\mathbf{mod}$~$\rangle$ is again $\langle$~metrically / topologically~$\rangle$ flat.
\end{proposition}

\begin{proposition}\label{MetFlatIsTopFlatAndTopFlatIsRelFlat} Every metrically flat module is topologically flat and every topologically flat module is relatively flat.
\end{proposition}

A quantitative version of the definition of topological flatness was given by White, though he used a somewhat different terminology. Unfortunately, his definition was erroneous, meanwhile the results were correct. So we take responsibility to fix the definition.

\begin{definition}[\cite{WhiteInjmoduAlg}, definition 4.8]\label{CTopFlatMod} A left $A$-module $F$ is called $C$-topologically flat if for each $c$-topologically injective $A$-morphism $\xi:X\to Y$ of right $A$-modules the operator $\xi\projmodtens{A} 1_F:X\projmodtens{A} F\to Y\projmodtens{A} F$ is $cC$-topologically injective.
\end{definition}

The key result in the study of this type of flatness is the following.

\begin{proposition}[\cite{WhiteInjmoduAlg}, lemma 4.10]\label{CTopFlatCharac} An $A$-module $F$ is $C$-topologically flat iff $F^*$ is $C$-topologically injective.
\end{proposition}

As the consequence, a Banach module is topologically flat iff it is $C$-topologically flat for some $C$. One more important consequence of propositions \ref{MetInjIsOneTopInj} and \ref{CTopFlatCharac} is the following.

\begin{proposition}\label{MetFlatIsOneTopFlat} An $A$-module is metrically flat iff it is $1$-topologically flat.
\end{proposition}


We shall exploit both definitions \ref{TopInjMod} and \ref{CTopInjMod} without further reference for their equivalence. From propositions \ref{CTopFlatCharac} and \ref{RetrCTopInjIsCTopInj} we also get the following.

\begin{proposition}\label{RetrCTopFlatIsCTopFlat} Any $c$-retract of $C$-topologically flat module is $cC$-topologically flat.
\end{proposition}

\begin{proposition}\label{DualMetTopProjIsMetrInj} Let $P$ be a $\langle$~metrically / $C$-topologically~$\rangle$ projective $A$-module, and $\Lambda$ be an arbitrary set. Then $\mathcal{B}(P,\ell_\infty(\Lambda))$ is $\langle$~metrically / $C$-topologically~$\rangle$ injective $A$-module. In particular, $P^*$ is $\langle$~metrically / $C$-topologically~$\rangle$ injective $A$-module.
\end{proposition}
\begin{proof} From proposition $\langle$~\ref{MetTopProjModViaCanonicMorph} / \ref{RetrCTopProjIsCTopProj}~$\rangle$ we know that $\pi_P^+$ is a $\langle$~$1$-retraction / $C$-retraction~$\rangle$. Then the $A$-morphism $\rho^+=\mathcal{B}(\pi_P^+,\ell_\infty(\Lambda))$ is a $\langle$~$1$-coretraction / $C$-coretraction~$\rangle$. Note that, 
$$
\mathcal{B}(A_+\projtens\ell_1(B_P),\ell_\infty(\Lambda))\isom{\mathbf{mod}_1-A}\mathcal{B}(A_+,\mathcal{B}(\ell_1(B_P),\ell_\infty(\Lambda)))\isom{\mathbf{mod}_1-A}\mathcal{B}(A_+,\ell_\infty(B_P\times\Lambda)).
$$ 
Thus we showed that $\rho^+$ is a $\langle$~$1$-coretraction / $C$-coretraction~$\rangle$ from $\mathcal{B}(P,\ell_\infty(\Lambda))$ into $\langle$~metrically / $1$-topologically~$\rangle$ injective $A$-module. By proposition $\langle$~\ref{RetrMetTopInjIsMetTopInj} / \ref{RetrCTopInjIsCTopInj}~$\rangle$ the $A$-module $\mathcal{B}(P,\ell_\infty(\Lambda))$ is $\langle$~metrically / $C$-topologically~$\rangle$ injective. To prove the last claim, just set $\Lambda=\mathbb{N}_1$.
\end{proof}

As the consequence of propositions \ref{CTopFlatCharac} and \ref{DualMetTopProjIsMetrInj} we get

\begin{proposition}\label{MetTopProjIsMetTopFlat} Every $\langle$~metrically / $C$-topologically~$\rangle$ projective module is $\langle$~metrically / $C$-topologically~$\rangle$ flat.
\end{proposition}

As we shall see later $\langle$~metric / topological~$\rangle$ flatness is a weaker property than $\langle$~metric / topological~$\rangle$ projectivity.

\begin{proposition}\label{MetTopFlatModCoProd} Let $(F_\lambda)_{\lambda\in\Lambda}$ be family of $A$-modules. Then 

$i)$ $\bigoplus_1\{F_\lambda:\lambda\in\Lambda\}$ is metrically flat iff for all $\lambda\in\Lambda$ the $A$-module $F_\lambda$ is metrically flat;

$ii)$ $\bigoplus_1\{F_\lambda:\lambda\in\Lambda\}$ is $C$-topologically flat iff for all $\lambda\in\Lambda$ the $A$-module $F_\lambda$ is $C$-topologically flat.
\end{proposition}
\begin{proof} By proposition $\langle$~\ref{MetTopFlatCharac} / \ref{CTopFlatCharac}~$\rangle$ an $A$-module $F$ is $\langle$~metrically / $C$-topologically~$\rangle$ flat iff $F^*$ is $\langle$~metrically / $C$-topologically~$\rangle$ injective. It is remains to apply proposition \ref{MetTopInjModProd} with $J_\lambda=F_\lambda^*$ for all $\lambda\in\Lambda$ and recall that $\left(\bigoplus_1\{ F_\lambda:\lambda\in\Lambda\}\right)^*\isom{\mathbf{mod}_1-A}\bigoplus_\infty\{ F_\lambda^*:\lambda\in\Lambda\}$.
\end{proof}

As we have seen flatness is a weakened version of projectivity. Loosely speaking flatness is ``projectivity with respect to second duals''. We can give this statement a precise meaning.

\begin{proposition}\label{MetTopFlatSecondDualCharac} Let $F$ be a left Banach $A$-module. Then the following are equivalent:

$i)$ $F$ is $\langle$~metrically / $C$-topologically~$\rangle$ flat;

$ii)$ for any $\langle$~strictly coisometric / strictly $c$-topologically surjective~$\rangle$ $A$-morphism $\xi:X\to Y$ and for any $A$-morphism $\phi:F\to Y$ there exists an $A$-morphism $\psi:P\to X^{**}$ such that $\xi^{**}\psi=\iota_Y\phi$ and $\langle$~$\Vert\psi\Vert=\Vert\phi\Vert$ / $\Vert\psi\Vert\leq cC\Vert\phi\Vert$~$\rangle$;

$iii)$ there exists an $A$-morphism $\rho:F\to (A_+\projtens \ell_1(B_F))^{**}$ with norm $\langle$~at most $1$ / at most $C$~$\rangle$ such that $(\pi_F^+)^{**}\rho=\iota_F$.
\end{proposition}
\begin{proof} $i)\implies ii)$ Again, consider arbitrary $\langle$~strictly coisometric / strictly $c$-topologically surjective~$\rangle$ $A$-morphism $\xi:X\to Y$ and arbitrary $A$-morphism $\phi:F\to Y$. By [\cite{HelLectAndExOnFuncAn}, exercise 4.4.6] we know that $\xi^*$ is $\langle$~isometric / $c$-topologically injective~$\rangle$. Since $F$ is $\langle$~metrically / $C$-topologically~$\rangle$ flat then $(\xi^*\projmodtens{A}1_F)$ is $\langle$~isometric / $cC$-topologically injective~$\rangle$ too. Therefore $(\xi^*\projmodtens{A}1_F)^*$ is $\langle$~strictly coisometric / strictly $cC$-topologically surjective~$\rangle$ by [\cite{HelLectAndExOnFuncAn}, exercise 4.4.7]. Note that $(\xi^*\projmodtens{A}1_F)^*$ and $\mathcal{B}_A(F,\xi^{**})$ are equivalent in $\mathbf{Ban}_1$ thanks to the law of adjoint associativity. So $\mathcal{B}_A(F,\xi^{**})$ is $\langle$~strictly coisometric / strictly $cC$-topologically surjective~$\rangle$ too. The latter implies that for the operator $\iota_Y\phi$ we can find an $A$-morphism $\psi:Y\to X^{**}$ such that $\xi^{**}\psi=\iota_Y\phi$ and  $\langle$~$\Vert\psi\Vert=\Vert\iota_Y\phi\Vert=\Vert\phi\Vert$ / $\Vert\psi\Vert\leq cC\Vert\iota_Y\phi\Vert=cC\Vert\phi\Vert$~$\rangle$

$ii)\implies iii)$ Set $\xi=\pi_F^+$ and $\phi=1_F$. Since $\xi$ is $\langle$~strictly coisometric/ strictly $1$-topologically surjective~$\rangle$, then from assumption we get an $A$-morphism $\rho:F\to (A_+\projtens\ell_1(B_F))^{**}$ such that $(\pi_F^+)^{**}\rho=\iota_F 1_F=\iota_F$ and $\langle$~$\Vert\rho\Vert\leq \Vert\phi\Vert=1$ / $\Vert\rho\Vert\leq 1\cdot C\Vert\phi\Vert=C$~$\rangle$.

$iii)\implies i)$ Let $\rho$ be a right inverse $A$-morphism for $(\pi_F^+)^{**}$ with norm $\langle$~at most $1$ / at most $C$~$\rangle$. Consider $A$-morphism $\tau=\rho^*\iota_{(A_+\projtens\ell_1(B_F))^*}$. Clearly, its norm is $\langle$~at most $1$ / at most $C$~$\rangle$. For any $f\in F^*$ and $x\in F$ we have
$$
(\tau(\pi_F^+)^*)(f)(x)
=\rho^*(\iota_{(A_+\projtens\ell_1(B_F))^*}((\pi_F^+)^*(f)))(x)
=\iota_{(A_+\projtens\ell_1(B_F))^*}((\pi_F^+)^*(f))(\rho(x))
$$
$$
=\rho(x)((\pi_F^+)^*(f))
=(\pi_F^+)^{**}(\rho(x)))(f)
=\iota_F(x)(f)
=f(x)
$$
So $\tau(\pi_F^+)^*=1_{F^*}$, which means $F^*$ is a with norm $\langle$~$1$-retract / $C$-retract~$\rangle$ of $(A_+\projtens\ell_1(B_F))^*$. The latter module is $\langle$~metrically / $1$-topologically~$\rangle$ injective, because $(A_+\projtens\ell_1(B_F))^*\isom{\mathbf{mod}_1-A}\mathcal{B}(A_+,\ell_\infty(B_F))$. By proposition $\langle$~\ref{RetrMetTopInjIsMetTopInj} / \ref{RetrCTopInjIsCTopInj}~$\rangle$ the $A$-module $F^*$ is $\langle$~metrically / $C$-topologically~$\rangle$ injective. By proposition $\langle$~\ref{MetTopFlatCharac} / \ref{CTopFlatCharac}~$\rangle$ this is equivalent to $\langle$~metric / $C$-topological~$\rangle$ flatness of $F$.
\end{proof}


%----------------------------------------------------------------------------------------
%	Metric and topological flatness of ideals and cyclic modules
%----------------------------------------------------------------------------------------

\subsection{Metric and topological flatness of ideals and cyclic modules}
\label{SubSectionMetricAndTopologicalFlatnessOfIdealsAndCyclicModules}

In this section we study conditions under which ideals and cyclic modules are metrically or topologically flat. The proofs are somewhat similar to those used in the study of relative flatness of ideals and cyclic modules.

\begin{proposition}\label{MetTopFlatIdealsInUnitalAlg} Let $I$ be a left ideal of $A_\times $ and $I$ has a right $\langle$~contractive / $c$-bounded~$\rangle$ approximate identity. Then $I$ is $\langle$~metrically / $c$-topologically~$\rangle$ flat.
\end{proposition}
\begin{proof} Let $\mathfrak{F}$ be the section filter on $N$ and let $\mathfrak{U}$ be an ultrafilter dominating $\mathfrak{F}$. For a fixed $f\in I^*$ and $a\in A_\times $ we have $|f(a e_\nu)|\leq\Vert f\Vert\Vert a\Vert\Vert e_\nu\Vert\leq c\Vert f\Vert\Vert a\Vert$ i.e. $(f(ae_\nu))_{\nu\in N}$ is a bounded net of complex numbers. Therefore we have a well defined limit $\lim_{\mathfrak{U}}f(ae_\nu)$ along ultrafilter $\mathfrak{U}$. Now it is routine to check that $\sigma:A_\times ^*\to I^*:f\mapsto (a\mapsto \lim_{\mathfrak{U}}f(ae_\nu))$ is an $A$-morphism with norm $\langle$~at most $1$ / at most $c$~$\rangle$. Let $\rho:I\to A_\times$ be the natural embedding, then for all $f\in A_\times^*$ and $a\in I$ holds
$$
\rho^*(\sigma(f))(a)
=\sigma(f)(\rho(a))
=\sigma(f)(a)
=\lim_{\mathfrak{U}}f(a e_\nu)
=\lim_{\nu}f(a e_\nu)
=f(\lim_{\nu}a e_\nu)
=f(a)
$$
i.e. $\sigma:I^*\to A_\times^*$ is a $\langle$~$1$-coretraction / $c$-coretraction~$\rangle$. The right $A$-module $A_\times ^*$ is $\langle$~metrically / $1$-topologically~$\rangle$ injective by proposition \ref{DualOfUnitalAlgIsMetTopInj}, hence its $\langle$~$1$-retract / $c$-retract~$\rangle$ $I^*$ is $\langle$~metrically / $c$-topologically~$\rangle$ injective. Now from proposition $\langle$~\ref{MetTopFlatCharac} / \ref{CTopFlatCharac}~$\rangle$ we conclude that the $A$-module $I$ is $\langle$~metrically / $c$-topologically~$\rangle$ flat.
\end{proof}

Note that the same sufficient condition holds for relative flatness for ideals of $A_\times$ [\cite{HelBanLocConvAlg}, proposition 7.1.45]. Now we are able to give an example of a metrically flat module which is not even topologically projective. Clearly $\ell_\infty(\mathbb{N})$-module $c_0(\mathbb{N})$ is not unital as ideal but admits a contractive approximate identity. By theorem \ref{GoodCommIdealMetTopProjIsUnital} it is not topologically projective, but it is metrically flat by proposition \ref{MetTopFlatIdealsInUnitalAlg}.

The ``metric'' part of the following proposition is a slight modification of [\cite{WhiteInjmoduAlg}, proposition 4.11]. The case of topological flatness was solved by Helemskii in [\cite{HelHomolBanTopAlg}, theorem VI.1.20].

\begin{proposition}\label{MetTopFlatCycModCharac} Let $I$ be a left proper ideal of $A_\times $. Then the following are equivalent:

$i)$ $A_\times /I$ is $\langle$~metrically / $C$-topologically~$\rangle$ flat $A$-module;

$ii)$ $I$ has a right bounded approximate identity $(e_\nu)_{\nu\in N}$ with $\sup_{\nu\in N}\Vert e_{A_\times }-e_\nu\Vert$ $\langle$~at most $1$ / at most $C$~$\rangle$
\end{proposition}
\begin{proof} $i)$$\implies$$ ii)$  Since $A_\times /I$ is $\langle$~metrically / $C$-topologically~$\rangle$ flat, then by proposition $\langle$~\ref{MetTopFlatCharac} / \ref{CTopFlatCharac}~$\rangle$ the right $A$-module $(A_\times /I)^*$ is $\langle$~metrically / $C$-topologically~$\rangle$ injective. Let $\pi:A_\times \to A_\times /I$ be the natural quotient map, then $\pi^*:(A_\times /I)^*\to A_\times ^*$ is an isometry. Since $(A_\times /I)^*$ is $\langle$~metrically / $C$-topologically~$\rangle$ injective, then $\pi^*$ is a coretraction, i.e. there exists a $\langle$~strictly coisometric / topologically surjective~$\rangle$ $A$-morphism $\tau:A_\times ^*\to (A_\times /I)^*$ of norm $\langle$~at most $1$ / at most $C$~$\rangle$ such that $\tau\pi^*=1_{(A_\times /I)^*}$. Consider $p\in A^{**}$ such that $\iota_{A_\times }(e_{A_\times })-p=\tau^*(\pi^{**}(\iota_{A_\times }(e_{A_\times })))$. Fix $f\in I^\perp$. Since $I^\perp=\pi^*((A_\times /I)^*)$, then there exists $g\in (A_\times /I)^*$ such that $f=\pi^*(g)$. Thus,
$$
(\iota_{A_\times }(e_{A_\times })-p)(f)
=\tau^*(\pi^{**}(\iota_{A_\times }(e_{A_\times })))(\pi^*(g))
=\pi^{**}(\iota_{A_\times }(e_{A_\times }))(\tau(\pi^*(g)))
$$
$$
=\pi^{**}(\iota_{A_\times }(e_{A_\times }))(g)
=\iota_{A_\times }(e_{A_\times })(\pi^*(g))
=\iota_{A_\times }(e_{A_\times })(f).
$$
Therefore $p(f)=0$ for all $f\in I^\perp$, i.e. $p\in I^{\perp\perp}$. Recall that $I^{\perp\perp}$ is the weak${}^*$ closure of $I$ in $A^{**}$, so we can choose a net $(e_\nu'')_{\nu\in N''}\subset I$ such that $(\iota_I(e_\nu''))_{\nu\in N''}$ converges to $p$ in weak${}^*$ topology. Clearly $(\iota_{A_\times }(e_{A_\times }-e_\nu''))_{\nu\in N''}$ converges to $\iota_{A_\times }(e_{A_\times })-p$ in the same topology. By [\cite{PosAndApproxIdinBanAlg}, lemma 1.1] there exists a net in the convex hull $\operatorname{conv}(\iota_{A_\times }(e_{A_\times }-e_\nu''))_{\nu\in N''}=\iota_{A_\times }(e_{A_\times })-\operatorname{conv}(\iota_{A_\times }(e_\nu''))_{\nu\in N''}$ that weak${}^*$ converges to $\iota_{A_\times }(e_{A_\times })-p$ with norm bound $\Vert \iota_{A_\times }(e_{A_\times })-p\Vert$. Denote this net as $(\iota_{A_\times }(e_{A_\times })-\iota_{A_\times }(e_\nu'))_{\nu\in N'}$, then $(\iota_{A_\times }(e_\nu'))_{\nu\in N'}$ weak${}^*$ converges to $p$. For any $a\in I$ and $f\in I^*$ we have
$$
\lim_{\nu}f(ae_\nu')
=\lim_{\nu}\iota_{A_\times }(e_\nu')(f\cdot a)
=p(f\cdot a)
=\iota_{A_\times }(e_{A_\times })(f\cdot a)-\tau^*(\pi^{**}(\iota_{A_\times }(e_{A_\times })))(f\cdot a)
$$
$$
=f(a)-\iota_{A_\times }(e_{A_\times })(\pi^*(\tau(f\cdot a)))
=f(a)-\pi^*(\tau(f)\cdot a)(e_{A_\times })
=f(a)-\tau(f)(\pi(a))
=f(a)
$$
hence $(e_\nu')_{\nu\in N'}$ is a weak right bounded approximate identity for $I$. By [\cite{AppIdAndFactorInBanAlg}, proposition 33.2] there is a net $(e_\nu)_{\nu\in N}\subset\operatorname{conv}(e_\nu')_{\nu\in N'}$ which is a right bounded approximate identity for $I$. For any $\nu\in N$ we have $e_{A_\times }-e_\nu\in\operatorname{conv}(e_{A_\times }-e_\nu')_{\nu\in N'}$, so taking into account the norm bound on $(\iota_{A_\times }(e_{A_\times }-e_\nu'))_{\nu\in N'}$ we get 
$$
\sup_{\nu\in N}\Vert e_{A_\times }-e_\nu\Vert
\leq\Vert \iota_{A_\times }(e_{A_\times })-p\Vert
\leq\Vert\tau^*(\pi^{**}(\iota_{A_\times }(e_{A_\times })))\Vert
\leq\Vert\tau^*\Vert\Vert\pi^{**}\Vert\Vert\iota_{A_\times }(e_{A_\times })\Vert=\Vert\tau\Vert
$$
Since $\tau$ has norm $\langle$~at most $1$ / at most $C$~$\rangle$ we get the desired bound. By construction, $(e_\nu)_{\nu\in N}$ is a right bounded approximate identity for $I$.

$ii)$$\implies$$ i)$ Denote $D=\sup_{\nu\in N}\Vert e_{A_\times }-e_\nu\Vert$. Let $\mathfrak{F}$ be the section filter on $N$ and let $\mathfrak{U}$ be an ultrafilter dominating $\mathfrak{F}$. For a fixed $f\in A_\times ^*$ and $a\in A_\times $ we have $|f(a-a e_\nu)|=|f(a(e_{A_\times }-e_\nu))|\leq\Vert f\Vert\Vert a\Vert\Vert e_{A_\times }-e_\nu\Vert\leq D\Vert f\Vert\Vert a\Vert$ i.e. $(f(a-ae_\nu))_{\nu\in N}$ is a bounded net of complex numbers. Therefore we have a well defined limit $\lim_{\mathfrak{U}}f(a-ae_\nu)$ along ultrafilter $\mathfrak{U}$. Since $(e_\nu)_{\nu\in N}$ is a right approximate identity for $I$ and $\mathfrak{U}$ contains section filter then for all $a\in I$ we have $\lim_{\mathfrak{U}}f(a-ae_\nu)=\lim_{\nu}f(a-ae_\nu)=0$. Therefore for each $f\in A_\times ^*$ we have well a defined map $\tau(f):A_\times /I\to \mathbb{C}:a+I\mapsto \lim_{\mathfrak{U}} f(a-ae_\nu)$. Clearly, this is a linear functional and from inequalities above we see its norm bounded by $D\Vert f\Vert$. Now it is routine to check that $\tau:A_\times ^*\to (A_\times /I)^*:f\mapsto \tau(f)$ is an $A$-morphism with norm $\langle$~at most $1$ / at most $C$~$\rangle$. For all $g\in(A_\times /I)^*$ and $a+I\in A_\times /I$ holds
$$
\tau(\pi^*(g))(a+I)
=\lim_{\mathfrak{U}}\pi^*(g)(a-ae_\nu)
=\lim_{\mathfrak{U}} g(\pi(a-ae_\nu))
=\lim_{\mathfrak{U}} g(a+I)
=g(a+I)
$$
i.e. $\tau:A_\times ^*\to (A_\times /I)^*$ is a retraction. The right $A$-module $A_\times ^*$ is $\langle$~metrically / $1$-topologically~$\rangle$ injective by proposition \ref{DualOfUnitalAlgIsMetTopInj}, hence its $\langle$~$1$-retract / $C$-retract~$\rangle$ $(A_\times /I)^*$ is $\langle$~metrically / $C$-topologically~$\rangle$ injective. Proposition $\langle$~\ref{MetTopFlatCharac} / \ref{CTopFlatCharac}~$\rangle$ gives that the module $A_\times /I$ is $\langle$~metrically / $C$-topologically~$\rangle$ flat.
\end{proof}

It is worth to mention that every operator algebra $A$ (not necessary self adjoint) with contractive approximate identity has a contractive approximate identity $(e_\nu)_{\nu\in N}$ such that $\sup_{\nu\in N}\Vert e_{A_\#}-e_\nu\Vert\leq 1$ and even $\sup_{\nu\in N}\Vert e_{A_\#}-2e_\nu\Vert\leq 1$. Here $A_\#$ is a unitization of $A$ as operator algebra. For details see \cite{PosAndApproxIdinBanAlg}, \cite{BleContrAppIdInOpAlg}.

Again we shall compare our result on metric and topological flatness of cyclic modules with their relative counterpart. Helemeskii and Sheinberg showed [\cite{HelHomolBanTopAlg}, theorem VII.1.20] that a cyclic module is relatively flat if $I$ admits a right bounded approximate identity. In case when $I^\perp$ is complemented in $A_\times^*$ the converse is also true. In topological theory we don't need this assumption, so we have a criterion. Metric flatness of cyclic modules is a much stronger property due to specific restriction on the norm of approximate identity. As we shall see in the next section, it is so restrictive that it doesn't allow to construct any non zero annihilator metrically projective, injective or flat module over a non zero Banach algebra.

%----------------------------------------------------------------------------------------
%	The impact of Banach geometry
%----------------------------------------------------------------------------------------

\section{The impact of Banach geometry}
\label{SectionTheImpactOfBanachGeometry}


%----------------------------------------------------------------------------------------
%	Homologically trivial annihilator modules
%----------------------------------------------------------------------------------------

\subsection{Homologically trivial annihilator modules}
\label{SubSectionHomoligicallyTrivialAnnihilatorModules}

In this section we concentrate on the study of metrically and topologically projective, injective and flat annihilator modules. Unless otherwise stated, all Banach spaces in this section are regarded as annihilator modules. Note the obvious fact that we shall often use in this section: any bounded linear operator between annihilator $A$-modules is an $A$-morphism.

\begin{proposition}\label{AnnihCModIsRetAnnihMod} Let $X$ be a non zero annihilator $A$-module. Then $\mathbb{C}$ is a $1$-retract of $X$ in $A-\mathbf{mod}_1$.
\end{proposition}
\begin{proof} Take any $x_0\in X$ with $\Vert x_0\Vert=1$ and using Hahn-Banach theorem choose $f_0\in X^*$ such that $\Vert f_0\Vert=f_0(x_0)=1$. Consider contractive linear operators $\pi:X\to \mathbb{C}:x\mapsto f_0(x)$, $\sigma:\mathbb{C}\to X:z\mapsto zx_0$. It is easy to check that $\pi$ and $\sigma$ are contractive $A$-morphisms and what is more $\pi\sigma=1_\mathbb{C}$. In other words $\mathbb{C}$ is a $1$-retract of $X$ in $A-\mathbf{mod}_1$.
\end{proof}

Now it is time to recall that any Banach algebra $A$ can always be regarded as proper maximal ideal of $A_+$, and what is more $\mathbb{C}\isom{A-\mathbf{mod}_1} A_+/A$. If we regard $\mathbb{C}$ as a right annihilator $A$-module we also have $\mathbb{C}\isom{\mathbf{mod}_1-A}(A_+/A)^*$. 

\begin{proposition}\label{MetTopProjModCCharac} An annihilator $A$-module $\mathbb{C}$ is $\langle$~metrically / $C$-topologically~$\rangle$ projective iff $\langle$~$A=\{0\}$ / $A$ has a right identity of norm at most $C-1$~$\rangle$.
\end{proposition}
\begin{proof} 
It is enough to study $\langle$~metric / $C$-topological~$\rangle$ projectivity of $A_+/A$. Since the natural quotient map $\pi:A_+\to A_+/A$ is a strict coisometry, then by proposition \ref{MetTopProjCycModCharac} $\langle$~metric / $C$-topological~$\rangle$ projectivity of $A_+/A$ is equivalent to existence of idempotent $p\in A$ such that $A=A_+p$ and $e_{A_+}-p$ has norm $\langle$~at most $1$ / at most $C$~$\rangle$. It remains to note that this norm bound holds iff $\langle$~$p=0$ and therefore $A=A_+p=\{0\}$ / $p$ has norm at most $C-1$~$\rangle$.
\end{proof}

\begin{proposition}\label{MetTopProjOfAnnihModCharac} Let $P$ be a non zero  annihilator $A$-module. Then the following are equivalent:

$i)$ $P$ is $\langle$~metrically / $C$-topologically~$\rangle$ projective $A$-module;

$ii)$ $\langle$~$A=\{0\}$ / $A$ has a right identity of norm at most $C-1$~$\rangle$ and $P$ is a $\langle$~metrically / $C$-topologically~$\rangle$ projective Banach space. As the consequence $\langle$~$P\isom{\mathbf{Ban}_1}\ell_1(\Lambda)$ / $P\isom{\mathbf{Ban}}\ell_1(\Lambda)$~$\rangle$ for some set $\Lambda$.
\end{proposition}
\begin{proof} $i)$$\implies$$ ii)$ By propositions $\langle$~\ref{RetrMetTopProjIsMetTopProj} / \ref{RetrCTopProjIsCTopProj}~$\rangle$ and \ref{AnnihCModIsRetAnnihMod} the $A$-module $\mathbb{C}$ is $\langle$~metrically / $C$-topologically~$\rangle$ projective as $1$-retract of $\langle$~metrically / $C$-topologically~$\rangle$ projective module $P$. Proposition \ref{MetTopProjModCCharac} gives that $\langle$~$A=\{0\}$ / $A$ has right identity of norm at most $C-1$~$\rangle$.  By corollary \ref{MetTopProjTensProdWithl1} the annihilator $A$-module $\mathbb{C}\projtens\ell_1(B_P)\isom{A-\mathbf{mod}_1}\ell_1(B_P)$ is $\langle$~metrically / $C$-topologically~$\rangle$ projective. Consider strict coisometry $\pi:\ell_1(B_P)\to P$ well defined by equality $\pi(\delta_x)=x$. Since $P$ and $\ell_1(B_P)$ are annihilator modules, then $\pi$ is also an $A$-module map. Since $P$ is $\langle$~metrically / $C$-topologically~$\rangle$ projective, then the $A$-morphism $\pi$ has a right inverse morphism $\sigma$ of norm  $\langle$~at most $1$ / at most $C-1$~$\rangle$. Therefore $P$ is a $\langle$~$1$-retract / $C$-retract~$\rangle$ of $\langle$~metrically / $1$-topologically~$\rangle$ projective Banach space $\ell_1(B_P)$. By proposition $\langle$~\ref{RetrMetTopProjIsMetTopProj} / \ref{RetrCTopProjIsCTopProj}~$\rangle$ the Banach space $P$ is $\langle$~metrically / $C$-topologically~$\rangle$ projective. Now from $\langle$~[\cite{HelMetrFrQMod}, proposition 3.2] / results of \cite{KotheTopProjBanSp}~$\rangle$ we get that $P$ is isomorphic to $\ell_1(\Lambda)$ in $\langle$~$\mathbf{Ban}_1$ / $\mathbf{Ban}$~$\rangle$ for some set $\Lambda$. 

$ii)$$\implies$$ i)$ By proposition \ref{MetTopProjModCCharac} the annihilator $A$-module $\mathbb{C}$ is $\langle$~metrically / $C$-topologically~$\rangle$ projective. Therefore by corollary \ref{MetTopProjTensProdWithl1} the annihilator $A$-module $\mathbb{C}\projtens\ell_1(\Lambda)\isom{A-\mathbf{mod}_1}\ell_1(\Lambda)$ is $\langle$~metrically / $C$-topologically~$\rangle$ projective too.
\end{proof}

\begin{proposition}\label{MetTopInjModCCharac} A right annihilator $A$-module $\mathbb{C}$ is $\langle$~metrically / $C$-topologically~$\rangle$ injective iff $\langle$~$A=\{0\}$ / $A$ has right $(C-1)$-bounded approximate identity~$\rangle$.
\end{proposition}
\begin{proof} Because of proposition $\langle$~\ref{MetTopFlatCharac} / \ref{CTopFlatCharac}~$\rangle$ it is enough to study $\langle$~metric / $C$-topological~$\rangle$ flatness of $A_+/A$. By proposition \ref{MetTopFlatCycModCharac} this is equivalent to existence of right bounded approximate identity $(e_\nu)_{\nu\in N}$ in $A$ with $\langle$~$\sup_{\nu\in N}\Vert e_{A_+}-e_\nu\Vert\leq 1$ / $\sup_{\nu\in N}\Vert e_{A_+}-e_\nu\Vert\leq C$~$\rangle$. It remains to note that the latter inequality holds iff $\langle$~iff $e_\nu=0$ and therefore $A=\{0\}$. / $(e_\nu)_{n\in N}$ is a right $C-1$-bounded approximate identity~$\rangle$.
\end{proof}

\begin{proposition}\label{MetTopInjOfAnnihModCharac} Let $J$ be a non zero right annihilator $A$-module. Then the following are equivalent:

$i)$ $J$ is $\langle$~metrically / $C$-topologically~$\rangle$ injective $A$-module;

$ii)$ $\langle$~$A=\{0\}$ / $A$ has a right $(C-1)$-bounded approximate identity~$\rangle$ and $J$ is a $\langle$~metrically / $C$-topologically~$\rangle$ injective Banach space. $\langle$~As the consequence $J\isom{\mathbf{Ban}_1}C(K)$ for some Stonean space $K$ /~$\rangle$.
\end{proposition}
\begin{proof} $i)$$\implies$$ ii)$  By propositions $\langle$~\ref{RetrMetTopInjIsMetTopInj} / \ref{RetrCTopInjIsCTopInj}~$\rangle$ and \ref{AnnihCModIsRetAnnihMod} the $A$-module $\mathbb{C}$ is $\langle$~metrically / $C$-topologically~$\rangle$ injective as $1$-retract of $\langle$~metrically / $C$-topologically~$\rangle$ injective module $J$. Proposition \ref{MetTopInjModCCharac} gives that $\langle$~$A=\{0\}$ / $A$ has a right $(C-1)$-bounded approximate identity~$\rangle$. By proposition \ref{MapsFroml1toMetTopInj} the annihilator $A$-module $\mathcal{B}(\ell_1(B_{J^*}),\mathbb{C})\isom{\mathbf{mod}_1-A}\ell_\infty(B_{J^*})$ is $\langle$~metrically / $C$-topologically~$\rangle$ injective. Consider isometry $\rho:J\to\ell_\infty(B_{J^*})$ well defined by $\rho(x)(f)=f(x)$. Since $J$ and $\ell_\infty(B_{J^*})$ are annihilator modules, then $\rho$ is also an $A$-module map. Since $J$ is $\langle$~metrically / $C$-topologically~$\rangle$ injective, then the $A$-morphism $\rho$ has a left inverse morphism $\tau$ with norm $\langle$~at most $1$ / at most $C$~$\rangle$. Therefore $J$ is $\langle$~$1$-retract / $C$-retract~$\rangle$ of $\langle$~metrically / $1$-topologically~$\rangle$ injective Banach space $\ell_\infty(B_{J^*})$. By proposition $\langle$~\ref{RetrMetTopInjIsMetTopInj} / \ref{RetrCTopInjIsCTopInj}~$\rangle$ the Banach space $J$ is $\langle$~metrically / $C$-topologically~$\rangle$ injective. $\langle$~From  [\cite{LaceyIsomThOfClassicBanSp}, theorem 3.11.6] the Banach space $J$ is isometrically isomorphic to $C(K)$ for some Stonean space $K$. /~$\rangle$ 

$ii)$$\implies$$ i)$ By proposition \ref{MetTopInjModCCharac} the annihilator $A$-module $\mathbb{C}$ is $\langle$~metrically / $C$-topologically~$\rangle$ injective. By proposition \ref{MapsFroml1toMetTopInj} the annihilator $A$-module $\mathcal{B}(\ell_1(B_{J^*}),\mathbb{C})\isom{\mathbf{mod}_1-A}\ell_\infty(B_{J^*})$ is $\langle$~metrically / $C$-topologically~$\rangle$ injective too. Since $J$ is a $\langle$~metrically / $C$-topologically~$\rangle$ injective Banach space and there an isometric embedding $\rho:J\to \ell_\infty(B_{J^*})$, then as Banach space $J$ is a $\langle$~$1$-retract / $C$-retract~$\rangle$ of $\ell_\infty(B_{J^*})$. Recall, that $J$ and $\ell_\infty(B_{J^*})$ are annihilator modules, so in fact we have a retraction in $\langle$~$\mathbf{mod}_1-A$ / $\mathbf{mod}-A$~$\rangle$. By proposition $\langle$~\ref{RetrMetTopInjIsMetTopInj} / \ref{RetrCTopInjIsCTopInj}~$\rangle$ the $A$-module $J$ is $\langle$~metrically / $C$-topologically~$\rangle$ injective.
\end{proof}

\begin{proposition}\label{MetTopFlatAnnihModCharac} Let $F$ be a non zero annihilator $A$-module. Then the following are equivalent:

$i)$ $F$ is $\langle$~metrically / $C$-topologically~$\rangle$ flat $A$-module;

$ii)$ $\langle$~$A=\{0\}$ / $A$ has a right $(C-1)$-bounded approximate identity~$\rangle$ and $F$ is a $\langle$~metrically / $C$-topologically~$\rangle$ flat Banach space, that is $\langle$~$F\isom{\mathbf{Ban}_1}L_1(\Omega,\mu)$ for some measure space $(\Omega, \Sigma, \mu)$ / $F$ is an $\mathscr{L}_{1,C}^g$-space~$\rangle$.
\end{proposition}
\begin{proof} By $\langle$~[\cite{GrothMetrProjFlatBanSp}, theorem 1] / [\cite{DefFloTensNorOpId}, corollary 23.5(1)]~$\rangle$ the Banach space $F$ is $\langle$~metrically / $C$-topologically~$\rangle$ flat iff $\langle$~$F\isom{\mathbf{Ban}_1}L_1(\Omega,\mu)$ for some measure space $(\Omega, \Sigma, \mu)$ / $F$ is an $\mathscr{L}_{1,C}^g$-space~$\rangle$. Now the equivalence follows from propositions \ref{MetTopInjOfAnnihModCharac} and \ref{MetTopFlatCharac}.
\end{proof}

We obliged to compare these results with similar ones in relative theory. From $\langle$~[\cite{RamsHomPropSemgroupAlg}, proposition 2.1.7] / [\cite{RamsHomPropSemgroupAlg}, proposition 2.1.10]~$\rangle$ we know that an annihilator $A$-module over Banach algebra $A$ is relatively $\langle$~projective / flat~$\rangle$ iff $A$ has $\langle$~a right identity / a right bounded approximate identity~$\rangle$.   
In metric and topological theory, in comparison with relative one, homological triviality of annihilator modules puts restrictions not only on the algebra itself but on the geometry of the module too. These geometric restrictions forbid existence of certain homologically excellent algebras. One of the most important properties of relatively $\langle$~contractible / amenable~$\rangle$ Banach algebra is $\langle$~projectivity / flatness~$\rangle$ of all (and in particular of all annihilator) left Banach modules over it. In a sharp contrast in metric and topological theories such algebras can't exist.

\begin{proposition} There is no Banach algebra $A$ such that all $A$-modules are  $\langle$~metrically / topologically~$\rangle$ flat. A fortiori, there is no such Banach algebras that all $A$-modules are $\langle$~metrically / topologically~$\rangle$ projective.
\end{proposition}
\begin{proof} Consider any infinite dimensional $\mathscr{L}_\infty^g$-space $X$ (say $\ell_\infty(\mathbb{N})$) as an annihilator $A$-module. From remark right after [\cite{DefFloTensNorOpId}, corollary 23.3] we know that $X$ is not an $\mathscr{L}_1^g$-space. Therefore by proposition \ref{MetTopFlatAnnihModCharac} the $A$-module $X$ is not topologically flat. By proposition \ref{MetFlatIsTopFlatAndTopFlatIsRelFlat} it is not metrically flat. Now from proposition \ref{MetTopProjIsMetTopFlat} we see that $X$ is neither metrically nor topologically projective.
\end{proof}

%----------------------------------------------------------------------------------------
%	Homologically trivial modules over Banach algebras with specific geometry
%----------------------------------------------------------------------------------------

\subsection{Homologically trivial modules over Banach algebras with specific geometry}
\label{SubSectionHomologicallyTrivialModulesOverBanachAlgebrasWithSpecificGeometry}

The purpose of this section is to convince our reader that homologically trivial modules over certain Banach algebras have similar geometric structure of those algebras. For the case of metric theory the following proposition was proved by Graven in \cite{GravInjProjBanMod}.

\begin{proposition}\label{TopProjInjFlatModOverL1Charac} Let $A$ be a Banach algebra which is as Banach space isometrically isomorphic to $L_1(\Theta,\nu)$ for some measure space $(\Theta,\Sigma,\nu)$. Then

$i)$ if $P$ is a $\langle$~metrically / topologically~$\rangle$ projective $A$-module, then $P$ is $\langle$~an $L_1$-space / complemented in some $L_1$-space~$\rangle$.

$ii)$ if $J$ is a $\langle$~metrically / topologically~$\rangle$ injective $A$-module, then  $J$ is a $\langle$~$C(K)$-space for some Stonean space $K$ / topologically injective Banach space~$\rangle$.

$iii)$ if $F$ is a $\langle$~metrically / topologically~$\rangle$ flat $A$-module, then $F$ is an $\langle$~$L_1$-space / $\mathscr{L}_1^g$-space~$\rangle$.
\end{proposition}
\begin{proof} 

Denote by $(\Theta',\Sigma',\nu')$ the measure space $(\Theta,\Sigma,\nu)$ with singleton atom adjoined, then $A_+\isom{\mathbf{Ban}_1} L_1(\Theta',\nu')$.

$i)$ Since $P$ is a $\langle$~metrically / topologically~$\rangle$ projective $A$-module, then by proposition \ref{MetTopProjModViaCanonicMorph} it is a retract of $A_+\projtens \ell_1(B_P)$ in $\langle$~$A-\mathbf{mod}_1$ / $A-\mathbf{mod}$~$\rangle$. Let $\mu_c$ be the counting measure on $B_P$, then by Grothendieck's theorem [\cite{HelLectAndExOnFuncAn}, theorem 2.7.5]
$$
A_+\projtens\ell_1(B_P)
\isom{\mathbf{Ban}_1}L_1(\Theta',\nu')\projtens L_1(B_P,\mu_c)
\isom{\mathbf{Ban}_1}L_1(\Theta'\times B_P,\nu'\times \mu_c)
$$
Therefore $P$ is $\langle$~$1$-complemented / complemented~$\rangle$ in some $L_1$-space. It remains to recall that any $1$-complemented subspace of $L_1$-space is again an $L_1$-space [\cite{LaceyIsomThOfClassicBanSp}, theorem 6.17.3].

$ii)$ Since $J$ is $\langle$~metrically / topologically~$\rangle$ injective $A$-module, then by proposition \ref{MetTopInjModViaCanonicMorph} it is a retract of $\mathcal{B}(A_+,\ell_\infty(B_{J^*}))$ in $\langle$~$\mathbf{mod}_1-A$ / $\mathbf{mod}-A$~$\rangle$. Let $\mu_c$ be the counting measure on $B_{J^*}$, then by Grothendieck's theorem [\cite{HelLectAndExOnFuncAn}, theorem 2.7.5]
$$
\mathcal{B}(A_+,\ell_\infty(B_{J^*}))
\isom{\mathbf{Ban}_1}(A_+\projtens \ell_1(B_{J^*}))^*
\isom{\mathbf{Ban}_1}(L_1(\Theta',\nu')\projtens L_1(B_P,\mu_c))^*
$$
$$
\isom{\mathbf{Ban}_1}L_1(\Theta'\times B_P,\nu'\times \mu_c)^*
\isom{\mathbf{Ban}_1}L_\infty(\Theta'\times B_P,\nu'\times \mu_c)
$$
Therefore $J$ is $\langle$~$1$-complemented / complemented~$\rangle$ in some $L_\infty$-space. Since $L_\infty$-space is $\langle$~metrically / topologically~$\rangle$ injective Banach space, then so does $J$. It remains to recall that every metrically injective Banach space is a $C(K)$-space for some Stonean space $K$ [\cite{LaceyIsomThOfClassicBanSp}, theorem 3.11.6].

$iii)$  By $\langle$~[\cite{GrothMetrProjFlatBanSp}, theorem 1] / remark after [\cite{DefFloTensNorOpId}, corollary 23.5(1)]~$\rangle$ the Banach space $F^*$ is $\langle$~metrically / topologically~$\rangle$ injective iff $F$ is an $\langle$~$L_1$-space / $\mathscr{L}_1^g$-space~$\rangle$. Now the implication follows from paragraph $ii)$ and proposition \ref{MetTopFlatCharac}.
\end{proof}

\begin{proposition}\label{TopProjInjFlatModOverMthscrL1SpCharac} Let $A$ be a Banach algebra which is topologically isomorphic as Banach space to some $\mathscr{L}_1^g$-space. Then any topologically $\langle$~projective / injective / flat~$\rangle$ $A$-module is an $\langle$~$\mathscr{L}_1^g$-space / $\mathscr{L}_\infty^g$-space / $\mathscr{L}_1^g$-space~$\rangle$.
\end{proposition}
\begin{proof} If $A$ is an $\mathscr{L}_1^g$-space, then so does $A_+$. 

Let $P$ be a topologically projective $A$-module. Then by proposition \ref{MetTopProjModViaCanonicMorph} it is a retract of $A_+\projtens \ell_1(B_P)$ in $A-\mathbf{mod}$ and a fortiori in $\mathbf{Ban}$. Since $\ell_1(B_P)$ is an $\mathscr{L}_1^g$-space, then so does $A_+\projtens\ell_1(B_P)$ as projective tensor product of $\mathscr{L}_1^g$-spaces [\cite{DefFloTensNorOpId}, exercise 23.17(c)]. Therefore $P$ is an $\mathscr{L}_1^g$-space as complemented subspace of $\mathscr{L}_1^g$-space [\cite{DefFloTensNorOpId}, corollary 23.2.1(2)].

Let $J$ be a topologically injective $A$-module, then by proposition \ref{MetTopInjModViaCanonicMorph} it is a retract of $\mathcal{B}(A_+,\ell_\infty(B_{J^*}))\isom{\mathbf{mod}_1-A}(A_+\projtens\ell_1(B_{J^*}))^*$ in $\mathbf{mod}-A$ and a fortiori in $\mathbf{Ban}$. As we showed in the previous paragraph $A_+\projtens\ell_1(B_{J^*})$ is an $\mathscr{L}_1^g$-space, therefore its dual $\mathcal{B}(A_+,\ell_\infty(B_{J^*}))$ is an $\mathscr{L}_\infty^g$-space [\cite{DefFloTensNorOpId}, corollary 23.2.1(1)]. It remains to recall that any complemented subspace of $\mathscr{L}_\infty^g$-space is again an $\mathscr{L}_\infty^g$-space [\cite{DefFloTensNorOpId}, corollary 23.2.1(2)].

Finally, let $F$ be a topologically flat $A$-module, then $F^*$ is topologically injective $A$-module by proposition \ref{MetTopFlatCharac}. From previous paragraph it follows that $F^*$ is an $\mathscr{L}_\infty^g$-space. By [\cite{DefFloTensNorOpId}, corollary 23.5(1)] we get that $F$ is an $\mathscr{L}_1^g$-space.
\end{proof}

We proceed to the discussion of the Dunford-Pettis property for homologically trivial modules.   

\begin{proposition}\label{C0SumOfL1SpHaveDPP} Let $(\Omega, \Sigma, \mu)$ be a measure spaces and $\Lambda$ be an arbitrary set. Then the Banach space $\bigoplus_0\{L_1(\Omega,\mu):\lambda\in\Lambda\}$ and all its duals has the Dunford-Pettis property.
In particular, $\bigoplus_1\{L_\infty(\Omega,\mu):\lambda\in\Lambda\}$ and $\bigoplus_\infty\{L_1(\Omega,\mu):\lambda\in\Lambda\}$ have this property.
\end{proposition}
\begin{proof} Consider one point compactification $\alpha\Lambda:=\Lambda\cup\{\Lambda\}$ of the set $\Lambda$ with discrete topology. From [\cite{BourgOnTheDPP}, corollary 7] we know that $C(\alpha\Lambda, L_1(\Omega, \mu))$ and all its duals have the Dunford-Pettis property. Since $c_0(\Lambda)$ is complemented in $C(\alpha\Lambda)$ via projection $P:C(\alpha\Lambda)\to C(\alpha\Lambda):x\mapsto x(\lambda)-x(\{\Lambda\})$, then $E:=c_0(\Lambda, L_1(\Omega,\mu))$ is complemented in $C(\alpha\Lambda, L_1(\Omega, \mu))$. The same statement holds for any $n$-th dual of $E$, because we can take $n$-th adjoint of $P$ in the role of projection. Now it remains to note that $E=\bigoplus_0\{L_1(\Omega,\mu):\lambda\in\Lambda\}$ and that the Dunford-Pettis property is inherited by complemented subspaces [\cite{FabHabBanSpTh}, proposition 13.44]. 

As the consequence of previous paragraph the Banach spaces $E^*\isom{\mathbf{Ban}_1}\bigoplus_1\{L_
\infty(\Omega,\mu):\lambda\in\Lambda\}$ and $E^{**}\isom{\mathbf{Ban}_1}\bigoplus_\infty\{L_1(\Omega,\mu)^{**}:\lambda\in\Lambda\}$ have the Dunford-Pettis property. From [\cite{DefFloTensNorOpId}, proposition B10] we know that any $L_1$-space is contractively complemented in its second dual. By $Q$ we denote the respective projection. Therefore the Banach space $\bigoplus_\infty\{L_1(\Omega,\mu):\lambda\in\Lambda\}$ is contractively complemented in $E^{**}$ via projection $\bigoplus_\infty \{Q:\lambda\in\Lambda\}$. Since $E^{**}$ has the Dunford-Pettis property, then by [\cite{FabHabBanSpTh}, proposition 13.44] so does its complemented subspace $\bigoplus_\infty\{L_1(\Omega,\mu):\lambda\in\Lambda\}$.
\end{proof}

\begin{proposition}\label{ProdOfL1SpHaveDPP} Let $\{(\Omega_\lambda,\Sigma_\lambda,\mu_\lambda):\lambda\in\Lambda\}$ be a family of measure spaces. Then $\bigoplus_0\{L_1(\Omega_\lambda, \mu_\lambda):\lambda\in\Lambda\}$, $\bigoplus_1\{L_\infty(\Omega_\lambda, \mu_\lambda):\lambda\in\Lambda\}$ and $\bigoplus_\infty\{L_1(\Omega_\lambda,\mu_\lambda):\lambda\in\Lambda\}$ have the Dunford-Pettis property.
\end{proposition}
\begin{proof} Let $(\Omega, \Sigma, \mu)$ be a direct sum of $\{(\Omega_\lambda,\Sigma_\lambda,\mu_\lambda):\lambda\in\Lambda\}$. By construction each Banach space $L_1(\Omega_\lambda,\mu_\lambda)$ for $\lambda\in\Lambda$ is $1$-complemented in $L_1(\Omega, \mu)$. Therefore, their $\bigoplus_0$-, $\bigoplus_1$- and $\bigoplus_\infty$-sums are contractively complemented in $\bigoplus_0\{L_1(\Omega, \mu):\lambda\in\Lambda\}$, $\bigoplus_1\{L_\infty(\Omega, \mu):\lambda\in\Lambda\}$ and $\bigoplus_\infty\{L_1(\Omega,\mu):\lambda\in\Lambda\}$ respectively. It remains to combine proposition \ref{C0SumOfL1SpHaveDPP} and the fact that the Dunford-Pettis property is preserved by complemented subspaces [\cite{FabHabBanSpTh}, proposition 13.44].
\end{proof}

\begin{proposition}\label{ProdOfDualsOfMthscrLInftySpHaveDPP} Let $E$ be an $\mathscr{L}_\infty^g$-space and $\Lambda$ be an arbitrary set. Then $\bigoplus_\infty\{E^*:\lambda\in\Lambda\}$ has the Dunford-Pettis property.
\end{proposition}
\begin{proof} Since $E$ is an $\mathscr{L}_\infty^g$-space, then $E^*$ is a $\mathscr{L}_{1}^g$-space [\cite{DefFloTensNorOpId}, corollary 23.2.1(1)]. Then from [\cite{DefFloTensNorOpId}, corollary 23.2.1(3)] it follows that $E^{***}$ is a retract of $L_1$-space. Recall that $E^*$ is complemented in $E^{***}$ via Dixmier projection, so $E^*$ is complemented in some $L_1$-space too. Thus we have a bounded linear projection $P:L_1(\Omega,\mu)\to L_1(\Omega,\mu)$ with image topologically isomorphic to $E$. In this case $\bigoplus_\infty\{ E^*:\lambda\in\Lambda\}$ is complemented in $\bigoplus_\infty\{ L_1(\Omega,\mu):\lambda\in\Lambda\}$ via projection $\bigoplus_\infty\{P:\lambda\in\Lambda\}$. The space $\bigoplus_\infty\{ L_1(\Omega,\mu):\lambda\in\Lambda\}$ has the Dunford-Pettis property by proposition \ref{ProdOfL1SpHaveDPP}. By proposition 13.44 in \cite{FabHabBanSpTh} so does $\bigoplus_\infty\{ E^*:\lambda\in\Lambda\}$ as complemented subspace of $\bigoplus_\infty\{ L_1(\Omega,\mu):\lambda\in\Lambda\}$.
\end{proof}

\begin{proposition}\label{MthscrL1LInftyHaveDPP} Any $\langle$~$\mathscr{L}_1^g$-space / $\mathscr{L}_\infty^g$-space~$\rangle$ admits the Dunford-Pettis property.
\end{proposition}
\begin{proof} Assume $E$ is an $\langle$~$\mathscr{L}_1^g$-space / $\mathscr{L}_\infty^g$-space~$\rangle$. Then $E^{**}$ is complemented in some $\langle$~$L_1$-space / $L_\infty$-space~$\rangle$ [\cite{DefFloTensNorOpId}, corollary 23.2.1(3)]. Since any $\langle$~$L_1$-space / $L_\infty$-space~$\rangle$ admits the Dunford-Pettis property \cite{GrothApllFaiblCompSpCK}, then so does $E^{**}$ as complemented subspace [\cite{FabHabBanSpTh}, proposition 13.44]. It remains to recall that a Banach space have the Dunford-Pettis property whenever so does its dual.
\end{proof}

\begin{theorem}\label{TopProjInjFlatModOverMthscrL1OrLInftySpHaveDPP} Let $A$ be a Banach algebra which is an $\mathscr{L}_1^g$-space or $\mathscr{L}_\infty^g$-space as Banach space. Then any topologically projective, injective or flat $A$-module has the Dunford-Pettis property.
\end{theorem}
\begin{proof} If $A$ is an $\mathscr{L}_1^g$-space, we just need to combine propositions \ref{MthscrL1LInftyHaveDPP} and \ref{TopProjInjFlatModOverMthscrL1SpCharac}. 

Assume $A$ is an $\mathscr{L}_\infty^g$-space, then so does $A_+$. Let $J$ be a topologically injective $A$-module, then by proposition \ref{MetTopInjModViaCanonicMorph} it is a retract of 
$$
\mathcal{B}(A_+,\ell_\infty(B_{J^*}))\isom{\mathbf{mod}_1-A}(A_+\projtens\ell_1(B_{J^*}))^*\isom{\mathbf{mod}_1-A}
\left(\bigoplus\nolimits_1\{ A_+:\lambda\in B_{J^*}\}\right)^*\isom{\mathbf{mod}_1-A}
$$
$$
\bigoplus\nolimits_\infty\{ A_+^*:\lambda\in B_{J^*}\}
$$ 
in $\mathbf{mod}-A$ and a fortiori in $\mathbf{Ban}$. By proposition \ref{ProdOfDualsOfMthscrLInftySpHaveDPP} this space has the Dunford-Pettis property. As $J$ is its retract, then $J$ also has this property [\cite{FabHabBanSpTh}, proposition 13.44]. 

If $F$ is a topologically flat $A$-module, then $F^*$ is a topologically injective $A$-module by proposition \ref{MetTopFlatCharac}. By previous paragraph $F^*$ has the Dunford-Pettis property and so does $F$.

If $P$ is a topologically projective $A$-module, it is also topologically flat by proposition \ref{MetTopProjIsMetTopFlat}. From previous paragraph it follows that $P$ has the Dunford-Pettis property.
\end{proof}

\begin{corollary}\label{NoInfDimRefMetTopProjInjFlatModOverMthscrL1OrLInfty} Let $A$ be a Banach algebra which $\mathscr{L}_1^g$-space or $\mathscr{L}_\infty^g$-space as Banach space. Then there is no topologically projective, injective or flat infinite dimensional reflexive $A$-modules. A fortiori there is no metrically projective, injective or flat infinite dimensional reflexive $A$-modules.
\end{corollary}
\begin{proof} From theorem \ref{TopProjInjFlatModOverMthscrL1OrLInftySpHaveDPP} we know that any topologically injective $A$-module has the Dunford-Pettis property. On the other hand there is no infinite dimensional reflexive Banach spaces with the Dunford-Pettis property. Thus we get the desired result regarding topological injectivity. Since dual of reflexive module is reflexive, from proposition \ref{MetTopFlatCharac} we get the result for topological flatness. It remains to recall that by proposition \ref{MetTopProjIsMetTopFlat}  every topologically projective module is topologically flat. To prove the last claim note that metric $\langle$~projectivity / injectivity / flatness~$\rangle$ implies topological $\langle$~projectivity / injectivity / flatness~$\rangle$ by proposition $\langle$~\ref{MetProjIsTopProjAndTopProjIsRelProj} / \ref{MetInjIsTopInjAndTopInjIsRelInj} / \ref{MetFlatIsTopFlatAndTopFlatIsRelFlat}~$\rangle$.
\end{proof}

Note that in relative theory there are examples of infinite dimensional relatively projective injective and flat reflexive modules over Banach algebras that are $\mathscr{L}_1^g$- or $\mathscr{L}_\infty^g$-spaces. Here are two examples. The first one is about convolution algebra $L_1(G)$ on a locally compact group $G$ with Haar measure. It is an $\mathscr{L}_1^g$-space. In [\cite{DalPolHomolPropGrAlg}, \S6] and \cite{RachInjModAndAmenGr} it was proved that for $1<p<+\infty$ the $L_1(G)$-module $L_p(G)$ is relatively $\langle$~projective / injective / flat~$\rangle$ iff $G$ is $\langle$~compact / amenable / amenable~$\rangle$. Note that any compact group is amenable [\cite{PierAmenLCA}, proposition 3.12.1], so in case $G$ is compact $L_p(G)$ is relatively projective injective and flat for all $1<p<+\infty$.  The second example is about $\mathscr{L}_\infty^g$-spaces $c_0(\Lambda)$ and $\ell_\infty(\Lambda)$ for an infinite set $\Lambda$. In proposition \ref{c0AndlInftyModsRelTh} we shall show that $\ell_p(\Lambda)$ for $1<p<\infty$ is relatively projective injective and flat as $c_0(\Lambda)$- or $\ell_\infty(\Lambda)$-module. 

We finalize this lengthy section by short remark on the l.u.st. property of topologically projective, injective and flat modules. 

\begin{proposition} Let $A$ be a Banach algebra which as Banach space has the l.u.st. property. Then any topologically projective, injective or flat $A$-module has the l.u.st. property.
\end{proposition}
\begin{proof} 

If $J$ is topologically injective $A$-module, then by proposition \ref{MetTopInjModViaCanonicMorph} it is a retract of $\mathcal{B}(A_+,\ell_\infty(B_{J^*}))\isom{\mathbf{mod}_1-A}\bigoplus_\infty\{ A_+^*:\lambda\in B_{J^*}\}$ in $\mathbf{mod}-A$ and a fortiori in $\mathbf{Ban}$. If $A$ has the l.u.st. property, then $A^{**}$ is complemented in some Banach lattice $E$ [\cite{DiestAbsSumOps}, theorem 17.5]. As the consequence $A_+^{***}$ is complemented in the Banach lattice $F:=\left(E\bigoplus_1\mathbb{C}\right)^*$ via some bounded projection $P:F\to F$. Thus $\bigoplus_\infty\{A_+^{***}:\lambda\in B_{J^*}\}$ is complemented in the Banach lattice $\bigoplus_\infty\{F:\lambda\in B_{J^*}\}$ via projection $\bigoplus_\infty\{ P:\lambda\in B_{J^*}\}$. Any Banach lattice has the l.u.st. property [\cite{DiestAbsSumOps}, theorem 17.1]. This property is inherited by complemented subspaces, so $\bigoplus_\infty\{A_+^{***}:\lambda\in B_{J^*}\}$ has the l.u.st. property too. Note that $A_+^*$ is contractively complemented in $A_+^{***}$ by Dixmier projection $Q$, therefore $\bigoplus_\infty\{A_+^*:\lambda\in B_{J^*}\}$ is complemented in $\bigoplus_\infty\{A_+^{***}:\lambda\in B_{J^*}\}$ via contractive projection $\bigoplus_\infty\{Q:\lambda\in B_{J^*}\}$. Since the latter space has the l.u.st. property, then so does its retract $\bigoplus_\infty\{A_+^*:\lambda\in B_{J^*}\}$. Finally $J$ is a retract of the $\bigoplus_\infty\{A_+^*:\lambda\in B_{J^*}\}$, therefore also has this property.

If $F$ is topologically flat $A$-module, then $F^*$ is topologically injective by proposition \ref{MetTopFlatCharac}. By previous paragraph $F^*$ has the l.u.st. property. Corollary 17.6 from \cite{DiestAbsSumOps} gives that $F$ has this property too.

If $P$ is topologically projective $A$-module, it is topologically flat by proposition \ref{MetTopProjIsMetTopFlat}. So $P$ has the l.u.st. property by previous paragraph.
\end{proof}

%----------------------------------------------------------------------------------------
%	Further properties of projective injective and flat modules
%----------------------------------------------------------------------------------------

\section{Further properties of projective injective and flat modules}
\label{SectionFurtherPropertiesOfProjectiveInjectiveAndFlatModules}

%----------------------------------------------------------------------------------------
%	Homological triviality of modules under change of algebra
%----------------------------------------------------------------------------------------

\subsection{Homological triviality of modules under change of algebra}
\label{SubSectionHomologicalTrivialityOfModulesUnderChangeOfAlgebra}

The following three propositions are metric and topological versions of propositions 2.3.2, 2.3.3 and 2.3.4 in \cite{RamsHomPropSemgroupAlg}.

\begin{proposition}\label{MorphCoincide} Let $X$ and $Y$ be $\langle$~left / right~$\rangle$ $A$-modules. Assume one of the following holds

$i)$ $I$ is a $\langle$~left / right~$\rangle$ ideal of $A$ and $X$ is an essential $I$-module;

$ii)$ $I$ is a $\langle$~right / left~$\rangle$ ideal of $A$ and $Y$ is a faithful $I$-module. 

Then $\langle$~${}_A\mathcal{B}(X,Y)={}_I\mathcal{B}(X,Y)$ / $\mathcal{B}_A(X,Y)=\mathcal{B}_I(X,Y)$~$\rangle$.
\end{proposition}
\begin{proof} We shall prove both statements only for left modules, since their right counterparts are proved with minimal modifications. Fix $\phi\in {}_I\mathcal{B}(X,Y)$.

$i)$ Take $x\in I\cdot X$, then $x=a'\cdot x'$ for some $a'\in I$, $x'\in X$. For any $a\in A$ we have $\phi(a\cdot x)=\phi(aa'\cdot x')=aa'\cdot\phi(x')=a\cdot\phi(a'\cdot x')=a\cdot\phi(x)$. Therefore $\phi(a\cdot x)=a\cdot\phi(x)$ for all $a\in A$ and $x\in \operatorname{cl}_X(IX)=X$. Hence $\phi\in {}_A\mathcal{B}(X,Y)$.

$ii)$ For any $a\in I$ and $a'\in A$, $x\in X$ we have $a\cdot(\phi(a'\cdot x)-a'\cdot\phi(x))=\phi(aa'\cdot x)-aa'\cdot\phi(x)=0$. Since $Y$ is faithful $I$-module we have $\phi(a'\cdot x)=a'\cdot \phi(x)$ for all $x\in X$, $a'\in A$. Hence $\phi\in{}_A\mathcal{B}(X,Y)$.

In both cases we proved that $\phi\in{}_A\mathcal{B}(X,Y)$ for any $\phi\in{}_I\mathcal{B}(X,Y)$, therefore ${}_I\mathcal{B}(X,Y)\subset {}_A\mathcal{B}(X,Y)$. The reverse inclusion is obvious.
\end{proof}

\begin{proposition}\label{MetTopProjUnderChangeOfAlg} Let $I$ be a closed subalgebra of $A$ and $P$ be an $A$-module which is essential as $I$-module. Then

$i)$ if $I$ is a left ideal of $A$ and $P$ is $\langle$~metrically / $C$-topologically~$\rangle$  projective $I$-module, then $P$ is $\langle$~metrically / $C$-topologically~$\rangle$ projective $A$-module;

$ii)$ if $I$ is a $\langle$~$1$-complemented / $c$-complemented~$\rangle$ right ideal of $A$ and $P$ is $\langle$~metrically / $C$-topologically~$\rangle$ projective $A$-module, then $P$ is $\langle$~metrically / $cC$-topologically~$\rangle$ projective $I$-module.
\end{proposition}
\begin{proof} By $\widetilde{\pi}_P: I\projtens \ell_1(B_P)\to P$ and $\pi_P:A\projtens \ell_1(B_P)\to P$ we will denote the standard epimorphisms.

$i)$ By proposition \ref{NonDegenMetTopProjCharac} the morphism $\widetilde{\pi}_P$ has a right inverse morphism in $\langle$~$I-\mathbf{mod}_1$ / $I-\mathbf{mod}$~$\rangle$, say $\widetilde{\sigma}$ of norm $\langle$~at most $1$ / at most $C$~$\rangle$. Let $i:I\to A$ be the natural embedding, then consider $\langle$~contractive / bounded~$\rangle$ $I$-morphism $\sigma=(i\projtens 1_{\ell_1(B_P)})\widetilde{\sigma}$. By paragraph $i)$ of proposition \ref{MorphCoincide} we have that $\sigma$ is an $A$-morphism. Clearly, $\sigma$ has norm $\langle$~at most $1$ / at most $C$~$\rangle$. For $\pi_P:A\projtens \ell_1(B_P)\to P$ we obviously have $\pi_P(i\projtens 1_{\ell_1(B_P)})=\widetilde{\pi}_P$, hence $\pi_P\sigma=\pi_P(i\projtens 1_{\ell_1(B_P)})\widetilde{\sigma}=\widetilde{\pi}_P\widetilde{\sigma}=1_P$. Thus $\pi_P$ is a $\langle$~$1$-retraction / $C$-retraction~$\rangle$ in $\langle$~$A-\mathbf{mod}_1$ / $A-\mathbf{mod}$~$\rangle$. So by proposition \ref{NonDegenMetTopProjCharac} the $A$-module $P$ is $\langle$~metrically / $C$-topologically~$\rangle$ projective.

$ii)$ Since $P$ is an essential $I$-module it is a fortiori an essential $A$-module. By proposition \ref{NonDegenMetTopProjCharac} the morphism $\pi_P$ has a right inverse morphism $\sigma$ in $\langle$~$A-\mathbf{mod}_1$ / $A-\mathbf{mod}$~$\rangle$ with norm $\langle$~at most $1$ / at most $C$~$\rangle$. Obviously $\sigma$ is a right inverse for $\pi_P$ in $\langle$~$I-\mathbf{mod}_1$ / $I-\mathbf{mod}$~$\rangle$ too. By $i:I\to A$ we denote the natural embedding, and by $r:A\to I$ the $\langle$~contractive / bounded~$\rangle$ left inverse. By assumption $\Vert r\Vert\leq c$. Consider $\langle$~contractive / bounded~$\rangle$ linear operator $\widetilde{\sigma}=(r\projtens 1_{\ell_1(B_P)})\sigma$. Clearly, its norm is $\langle$~at most $1$ / at most $cC$~$\rangle$. Since $I$ is a right ideal of $A$ and $P$ is an essential $I$-module then $\sigma(P)=\sigma(\operatorname{cl}_P(IP))=\operatorname{cl}_{A\projtens \ell_1(B_P)}(I\cdot (A\projtens \ell_1(B_P)))=I\projtens \ell_1(B_P)$, so $\sigma=(ir\projtens 1_{\ell_1(B_P)})\sigma$. Even more, since $\sigma(P)\subset I\projtens\ell_1(B_P)$ and $r|_I=1_I$, then $\sigma$ is an $I$-morphism. Clearly, $\pi_P(i\projtens 1_{\ell_1(B_P)})=\widetilde{\pi}_P$, so $\widetilde{\pi}_P\widetilde{\sigma}=\pi_P(i\projtens 1_{\ell_1(B_P)})(r\projtens 1_{\ell_1(B_P)})\sigma=\pi_P(ir\projtens 1_{\ell_1(B_P)})\sigma=\pi_P\sigma=1_P$. Thus $\widetilde{\pi}_P$ is a $\langle$~$1$-retraction / $cC$-retraction~$\rangle$ in $\langle$~$I-\mathbf{mod}_1$ / $I-\mathbf{mod}$~$\rangle$, so by proposition \ref{NonDegenMetTopProjCharac} the $I$-module $P$ is $\langle$~metrically / $cC$-topologically~$\rangle$ projective.
\end{proof}

\begin{proposition}\label{MetTopInjUnderChangeOfAlg} Let $I$ be a closed subalgebra of $A$ and $J$ be a right $A$-module which is faithful as $I$-module. Then

$i)$ if $I$ is a left ideal of $A$ and $J$ is $\langle$~metrically / $C$-topologically~$\rangle$  injective $I$-module, then $J$ is $\langle$~metrically / $C$-topologically~$\rangle$ injective $A$-module;

$ii)$ if $I$ is a $\langle$~$1$-complemented / $c$-complemented~$\rangle$ right ideal of $A$ and $J$ is $\langle$~metrically / $C$-topologically~$\rangle$ injective $A$-module, then $J$ is $\langle$~metrically / $cC$-topologically~$\rangle$ injective $I$-module.
\end{proposition}
\begin{proof} By $\widetilde{\rho}_J:J\to\mathcal{B}(I,\ell_\infty(B_{J^*}))$ and $\rho_J:J\to\mathcal{B}(A,\ell_\infty(B_{J^*}))$ we will denote the standard monomorphisms.

$i)$ By proposition \ref{NonDegenMetTopInjCharac} the morphism $\widetilde{\rho}_J: J\to\mathcal{B}(I,\ell_\infty(B_{J^*}))$ has a left inverse morphism in $\langle$~$\mathbf{mod}_1-I$ / $\mathbf{mod}-I$~$\rangle$, say $\widetilde{\tau}$ of norm $\langle$~at most $1$ / at most $C$~$\rangle$. Let $i:I\to A$ be the natural embedding, and define $I$-morphism $q=\mathcal{B}(i,\ell_\infty(B_{J^*}))$. Obviously  $\widetilde{\rho}_J=q\rho_J$. Consider $I$-morphism $\tau =\widetilde{\tau} q$. By paragraph $ii)$ of proposition \ref{MorphCoincide} it is also an $A$-morphism. Note that $\Vert\tau \Vert\leq\Vert\widetilde{\tau}\Vert\Vert q\Vert\leq\Vert\widetilde{\tau}\Vert$, so $\tau$ has norm $\langle$~at most $1$ / at most $C$~$\rangle$. Clearly, $\tau \rho_J=\widetilde{\tau} q\rho_J=\widetilde{\tau}\widetilde{\rho}_J=1_J$. Thus $\rho_J$ is a $\langle$~$1$-coretraction / $C$-coretraction~$\rangle$, so by proposition \ref{NonDegenMetTopInjCharac} the $A$-module $J$ is $\langle$~metrically / $C$-topologically~$\rangle$ injective.

$ii)$ If $J$ is $\langle$~metrically / $C$-topologically~$\rangle$ injective as $A$-module, then by proposition \ref{NonDegenMetTopInjCharac} the $A$-morphism $\rho_J$ has a left inverse in $\langle$~$\mathbf{mod}_1-A$ / $\mathbf{mod}-A$~$\rangle$, say $\tau $ of norm $\langle$~at most $1$ / at most $C$~$\rangle$. Assume we are given an operator $T\in \mathcal{B}(A,\ell_\infty(B_{J^*}))$, such that $T|_I=0$. Fix $a\in I$, then $T\cdot a=0$, and so $\tau (T)\cdot a=\tau (T\cdot a)=0$. Since $J$ is faithful $I$-module and $a\in I$ is arbitrary, then $\tau (T)=0$. By $r:A\to I$  we denote the left inverse of $i$. By assumption $\Vert r\Vert\leq c$. Define bounded linear operators $j=\mathcal{B}(r,\ell_\infty(B_{J^*}))$ and $\widetilde{\tau}=\tau  j$. For any $a\in I$ and $T\in\mathcal{B}(I,\ell_\infty(B_{J^*}))$ we have $\widetilde{\tau}(T\cdot a)-\widetilde{\tau}(T)\cdot a=\tau (j(T\cdot a)-j(T)\cdot a)=0$, because $j(T\cdot a)-j(T)\cdot a|_I=0$. Therefore $\widetilde{\tau}$ is an $I$-morphism. Note that $\Vert\widetilde{\tau}\Vert\leq\Vert\tau \Vert\Vert j\Vert$, so $\widetilde{\tau}$ has norm $\langle$~at most $1$ / at most $cC$~$\rangle$. Obviously, for all $x\in J$ we have $\rho_J(x)-j(\widetilde{\rho}_J(x))|_I=0$, so $\tau (\rho_J(x)-j(\widetilde{\rho}_J(x)))=0$. As a consequence $\widetilde{\tau}(\widetilde{\rho}_J(x))=\tau (j(\widetilde{\rho}_J(x)))=\tau (\rho_J(x))=x$ for all $x\in J$. Since $\widetilde{\tau}\widetilde{\rho}_J=1_J$, then $\widetilde{\rho}_J$ is a  $\langle$~$1$-coretraction / $cC$-coretraction~$\rangle$ in $\langle$~$\mathbf{mod}_1-I$ / $\mathbf{mod}-I$~$\rangle$, so by proposition \ref{NonDegenMetTopInjCharac} the $I$-module $J$ is $\langle$~metrically / $cC$-topologically~$\rangle$ injective.
\end{proof}


\begin{proposition}\label{MetTopFlatUnderChangeOfAlg} Let $I$ be a closed subalgebra of $A$ and $F$ be an $A$-module which is essential as $I$-module. Then

$i)$ if $I$ is a left ideal of $A$ and $F$ is $\langle$~metrically / $C$-topologically~$\rangle$  flat $I$-module, then $F$ is $\langle$~metrically / $C$-topologically~$\rangle$ flat $A$-module;

$ii)$ if $I$ is a $\langle$~$1$-complemented / $c$-complemented~$\rangle$ right ideal of $A$ and $F$ is $\langle$~metrically / $C$-topologically~$\rangle$ flat $A$-module, then $F$ is $\langle$~metrically / $cC$-topologically~$\rangle$ flat $I$-module.
\end{proposition}
\begin{proof} Note that the dual of essential module is faithful. Now the result follows from propositions $\langle$~\ref{MetTopFlatCharac} / \ref{CTopFlatCharac}~$\rangle$ and \ref{MetTopInjUnderChangeOfAlg}.
\end{proof}	

\begin{proposition}\label{MetTopProjInjFlatUnderSumOfAlg} Let $(A_\lambda)_{\lambda\in\Lambda}$ be a family of Banach algebras and for each $\lambda\in\Lambda$ let $X_\lambda$ be $\langle$~an essential / a faithful / an essential~$\rangle$ $A_\lambda$-module. Denote $A=\bigoplus_p\{A_\lambda:\lambda\in\Lambda\}$ for $1\leq p\leq +\infty$ or $p=0$. Let $X$ denote $\langle$~$\bigoplus_1\{X_\lambda:\lambda\in\Lambda\}$ / $\bigoplus_\infty\{X_\lambda:\lambda\in\Lambda\}$ / $\bigoplus_1\{X_\lambda:\lambda\in\Lambda\}$~$\rangle$. Then

$i)$ $X$ is metrically $\langle$~projective / injective / flat~$\rangle$ $A$-module iff for all $\lambda\in\Lambda$ the $A_\lambda$-module $X_\lambda$ is metrically $\langle$~projective / injective / flat~$\rangle$;

$ii)$ $X$ is $C$-topologically $\langle$~projective / injective / flat~$\rangle$ $A$-module iff for all $\lambda\in\Lambda$ the $A_\lambda$-module $X_\lambda$ is $C$-topologically $\langle$~projective / injective / flat~$\rangle$.
\end{proposition}
\begin{proof} Note that for each $\lambda\in\Lambda$ the natural embedding $i_\lambda:A_\lambda\to A$ allows to regard $A_\lambda$ as complemented in $\mathbf{Ban}_1$ two sided ideal of $A$.

$i)$ The proof is literally the same as in paragraph $ii)$.

$ii)$ Assume $X_\lambda$ is $C$-topologically $\langle$~projective / injective / flat~$\rangle$ $A_\lambda$-module for all $\lambda\in\Lambda$, then by paragraph $i)$ of proposition $\langle$~\ref{MetTopProjUnderChangeOfAlg} / \ref{MetTopInjUnderChangeOfAlg} / \ref{MetTopFlatUnderChangeOfAlg}~$\rangle$ it is $C$-topologically $\langle$~projective / injective / flat~$\rangle$ as $A$-module. It is remains to apply the proposition $\langle$~\ref{MetTopProjModCoprod} / \ref{MetTopInjModProd} / \ref{MetTopFlatModCoProd}~$\rangle$. 

Conversely, assume that $X$ is $C$-topologically $\langle$~projective / injective / flat~$\rangle$ as $A$-module. Fix arbitrary $\lambda\in\Lambda$. Clearly, we may regard $X_\lambda$ as $A$-module and even more $X_\lambda$ is a $1$-retract of $X$ in $\langle$~$A-\mathbf{mod}_1$ / $\mathbf{mod}_1-A$ / $A-\mathbf{mod}_1$~$\rangle$. By proposition $\langle$~\ref{RetrMetTopProjIsMetTopProj} / \ref{RetrMetTopInjIsMetTopInj} / \ref{RetrMetTopFlatIsMetTopFlat}~$\rangle$ we get that $X_\lambda$ is $C$-topologically $\langle$~projective / injective / flat~$\rangle$ as $A$-module. It is remains to apply paragraph $ii)$ of proposition $\langle$~\ref{MetTopProjUnderChangeOfAlg} / \ref{MetTopInjUnderChangeOfAlg} / \ref{MetTopFlatUnderChangeOfAlg}~$\rangle$.
\end{proof} 

%----------------------------------------------------------------------------------------
%	Further properties of flat modules
%----------------------------------------------------------------------------------------

\subsection{Further properties of flat modules}
\label{SubSectionFurtherPropertiesOfFlatModules}

Based on results obtained above, we collect more interesting facts on metric and topological injectivity and flatness of Banach modules.

\begin{proposition}\label{DualBanModDecomp} Let $B$ be a unital Banach algebra, $A$ be its subalgebra with two-sided bounded approximate identity $(e_\nu)_{\nu\in N}$ and $X$ be a left $B$-module. Denote $c_1=\sup_{\nu\in N}\Vert e_\nu\Vert$, $c_2=\sup_{\nu\in N}\Vert e_B-e_\nu\Vert$ and $X_{ess}=\operatorname{cl}_X(AX)$. Then 

$i)$ $X^*$ is $c_2(c_1+1)$-isomorphic as a right $A$-module to $X_{ess}^*\bigoplus_\infty (X/X_{ess})^*$;

$ii)$ $\langle$~$X_{ess}^*$ / $(X/X_{ess})^*$~$\rangle$ is a $\langle$ $c_1$-retract / $c_2$-retract~$\rangle$ of $A$-module $X^*$;

$iii)$ if $X$ is an $\mathscr{L}_{1,C}^g$-space, then $\langle$~$X_{ess}$ / $X/X_{ess}$~$\rangle$ is an $\langle$~$\mathscr{L}_{1,c_1C}^g$-space / $\mathscr{L}_{1,c_2C}^g$-space~$\rangle$.

\end{proposition}
\begin{proof} $i)$ Define the natural embedding $\rho:X_{ess}\to X:x\mapsto x$ and the quotient map  $\pi:X\to X/X_{ess}:x\mapsto x+X_{ess}$. Let $\mathfrak{F}$ be the section filter on $N$ and let $\mathfrak{U}$ be an ultrafilter dominating $\mathfrak{F}$. For a fixed $f\in X ^*$ and $x\in X $ we have $|f(x-e_\nu\cdot x)|\leq\Vert f\Vert\Vert e_B - e_\nu\Vert\Vert x\Vert\leq c_2\Vert f\Vert\Vert x\Vert$ i.e. $(f(x-e_\nu\cdot x))_{\nu\in N}$ is a bounded net of complex numbers. Therefore we have a well defined limit $\lim_{\mathfrak{U}}f(x-e_\nu\cdot x)$ along ultrafilter $\mathfrak{U}$. Since $(e_\nu)_{\nu\in N}$ is a two-sided approximate identity for $A$ and $\mathfrak{U}$ contains section filter then for all $x\in X_{ess}$ we have $\lim_{\mathfrak{U}}f(x-e_\nu\cdot x)=\lim_{\nu}f(x-e_\nu\cdot x)=0$. Therefore for each $f\in X ^*$ we have a well defined map $\tau(f):X /X_{ess}\to \mathbb{C}:x+X_{ess}\mapsto \lim_{\mathfrak{U}} f(x-e_\nu\cdot x)$. Clearly this is a linear functional, and from inequalities above we see its norm is bounded by $c_2\Vert f\Vert$. Now it is routine to check that $\tau:X^*\to (X/ X_{ess})^*:f\mapsto \tau(f)$ is an $A$-morphism with norm not greater than $c_2$. Similarly, one can show that $\sigma:X_{ess}^*\to X^*:h\mapsto(x\mapsto \lim_{\mathfrak{U}}h(e_\nu\cdot x))$ is an $A$-morphism with norm not greater than $c_1$. For any $f\in X^*$, $g\in (X/X_{ess})^*$, $h\in X_{ess}^*$ and $x\in X$, $y\in X_{ess}$ we have
$$
\sigma(h)(y)
=\lim_{\mathfrak{U}}h(e_\nu\cdot y)
=\lim_{\nu}h(e_\nu\cdot y)
=h(y),
\qquad
(\rho^*\sigma)(h)(y)
=\sigma(h)(\rho(y))
\sigma(h)(y)
=h(y),
$$
$$
(\tau\pi^*)(g)(x+X_{ess})
=\lim_{\mathfrak{U}}\pi^*(g)(x-e_\nu\cdot x)
=\lim_{\mathfrak{U}}g(x+X_{ess})
=g(x+X_{ess}),
$$
$$
(\tau\sigma)(h)(x+X_{ess})
=\lim_{\mathfrak{U}}\sigma(h)(x-e_\nu\cdot x)
=\lim_{\mathfrak{U}}(\sigma(h)(x)-h(e_\nu\cdot x))
=\sigma(h)(x)-\lim_{\mathfrak{U}}h(e_\nu\cdot x)=0,
$$
$$
(\pi^*\tau + \sigma\rho^*)(f)(x)
=\tau(f)(x+X_{ess})+\lim_{\mathfrak{U}}\rho^*(f)(e_\nu\cdot x)
=\lim_{\mathfrak{U}}f(x - e_\nu\cdot x)+\lim_{\mathfrak{U}}f(e_\nu\cdot x)
=f(x).
$$
Therefore $\tau \pi^*=1_{(X/X_{ess})^*}$, $\rho^*\sigma=1_{X_{ess}^*}$ and  $\pi^*\tau+\sigma\rho^*=1_{X^*}$. Now it is easy to check that the linear maps
$$
\xi:X^*\to X_{ess}^*\bigoplus{}_\infty (X/X_{ess})^*:f\mapsto \rho^*(f)\bigoplus{}_\infty \tau(f)
$$
$$
\eta:X_{ess}^*\bigoplus{}_\infty (X/X_{ess})^*\to X^*:h\bigoplus{}_\infty g\mapsto \pi^*(h)+\sigma(g)
$$
are isomorphism of right $A$-modules with $\Vert\xi \Vert\leq c_2$ and $\Vert \eta\Vert\leq c_1+1$. Hence $X^*$ is $c_2(c_1+1)$-isomorphic in $\mathbf{mod}-A$ to $X_{ess}^*\bigoplus_\infty (X/X_{ess})^*$.

$ii)$ Both results immediately follow from equalities $\rho^*\sigma=1_{X_{ess}^*}$, $\tau \pi^*=1_{(X/X_{ess})^*}$ and estimates $\Vert \rho^*\Vert\Vert \sigma\Vert\leq c_1$, $\Vert\tau\Vert\Vert \pi^*\Vert\leq c_2$.

$iii)$ Now consider case when $X$ is an $\mathscr{L}_{1,C}^g$-space. Then $X^*$ is an $\mathscr{L}_{\infty,C}^g$-space [\cite{DefFloTensNorOpId}, corollary 23.2.1(1)]. As $\langle$~$X_{ess}^*$ / $(X/X_{ess})^*$~$\rangle$ is $\langle$~$c_1$-complemented / $c_2$-complemented~$\rangle$ in $X^*$ it is an $\langle$~$\mathscr{L}_{\infty,c_1C}^g$-space / $\mathscr{L}_{\infty,c_2C}^g$-space~$\rangle$ by [\cite{DefFloTensNorOpId}, corollary 23.2.1(1)]. Again we apply [\cite{DefFloTensNorOpId}, corollary 23.2.1(1)] to conclude that $\langle$~$X_{ess}$  / $X/X_{ess}$~$\rangle$ is an $\langle$~$\mathscr{L}_{1,c_1C}^g$-space / $\mathscr{L}_{1,c_2C}^g$-space~$\rangle$.
\end{proof}

The following proposition is an analog of [\cite{RamsHomPropSemgroupAlg}, proposition 2.1.11].

\begin{proposition}\label{TopFlatModCharac} Let $A$ be a Banach algebra with two-sided $c$-bounded approximate identity, and $F$ be a left $A$-module. Then

$i)$ if $F$ is $C$-topologically flat $A$-module, then $F_{ess}$ is $(1+c)C$-topologically flat $A$-module and $F/F_{ess}$ is an $\mathscr{L}_{1,(1+c)C}^g$-space;

$ii)$ if $F_{ess}$ is $C_1$-topologically flat $A$-module and $F/F_{ess}$ is an $\mathscr{L}_{1,C_1}^g$-space, then $F$ is $(1+c)^2\max(C_1, C_2)$-topologically flat $A$-module.

$iii)$ $F$ is topologically flat $A$-module iff $F_{ess}$  is topologically flat $A$-module and $F/F_{ess}$ is an $\mathscr{L}_1^g$-space.
\end{proposition}
\begin{proof} We regard $A$ as closed subalgebra of unital Banach algebra $B:=A_+$. Then $F$ is unital left $B$-module. Using notation of proposition \ref{DualBanModDecomp} we may say that $c_1=c$ and $c_2=1+c$, so the right $A$-modules $F_{ess}^*$ and $(F/F_{ess})^*$ are $(1+c)$-retracts of $F^*$.

$i)$ By proposition \ref{CTopFlatCharac} the right $A$-module $F^*$ is $C$-topologically injective. Therefore from propositions \ref{RetrCTopInjIsCTopInj}, \ref{CTopFlatCharac} the modules $F_{ess}$ and $F/F_{ess}$ are $(1+c)C$-topologically flat. It remains to note that $F/F_{ess}$ is an annihilator $A$-module, so by proposition \ref{MetTopFlatAnnihModCharac} it is an $\mathscr{L}_{1,(1+c)C}^g$-space.

$ii)$ Again, by proposition \ref{CTopFlatCharac} the right $A$-modules $F_{ess}^*$ and $(F/F_{ess})^*$ are $C_1$- and $C_2$-topologically injective respectively. So from proposition \ref{MetTopInjModProd} their product is $\max(C_1,C_2)$-topologically injective. By proposition \ref{DualBanModDecomp} this product is $(1+c)^2$-isomorphic to $F^*$ in $\mathbf{mod}-A$. Therefore $F^*$ is $(1+c)^2\max(C_1, C_2)$-topologically injective $A$-module. Now the result follows from proposition \ref{CTopFlatCharac}.

$iii)$ The result immediately follows from paragraphs $i)$ and $ii)$.
\end{proof}

\begin{proposition}\label{MetTopEssL1FlatModAoverAmenBanAlg} Let $A$ be a relatively $\langle$~$1$-amenable / $c$-amenable~$\rangle$ Banach algebra and $F$ be an essential Banach $A$-module which is an $\langle$~$L_1$-space / $\mathscr{L}_{1,C}^g$-space~$\rangle$. Then $F$ is a $\langle$~metrically / $c^2C$-topologically~$\rangle$ flat $A$-module.
\end{proposition}
\begin{proof} We may assume that $A$ is relatively $c$-amenable for $\langle$~$c=1$ / $c\geq 1$~$\rangle$. Let $(d_\nu)_{\nu\in N}$ be an approximate diagonal for $A$ with norm bound at most $c$. Recall, that $(\Pi_A(d_\nu))_{\nu\in N}$ is a two-sided $\langle$~contractive / bounded~$\rangle$ approximate identity for $A$. Since $F$ is essential left $A$-module, then $\lim_{\nu}\Pi_A(d_\nu)\cdot x=x$ for all $x\in F$ [\cite{HelHomolBanTopAlg}, proposition 0.3.15]. As the consequence $c\pi_F(B_{A\projtens\ell_1(B_F)})$ is dense in $B_F$. Then for all $f\in F^*$ we have
$$
\Vert\pi_F^*(f)\Vert
=\sup\{|f(\pi_F(u))|:u\in B_{A\projtens\ell_1(B_F)}\}
=\sup\{|f(x)|:x\in \operatorname{cl}_F(\pi_F(B_{A\projtens\ell_1(B_F)}))\}
$$
$$
\geq\sup\{c^{-1}|f(x)|:x\in B_F\}=c^{-1}\Vert f\Vert
$$
This means, that $\pi_F^*$ is $c$-topologically injective. By assumption $F$ is an $\langle$~$L_1$-space / $\mathscr{L}_{1,C}^g$-space~$\rangle$, then by $\langle$~[\cite{GrothMetrProjFlatBanSp}, theorem 1] / remark after [\cite{DefFloTensNorOpId}, corollary 23.5(1)]~$\rangle$ the Banach space $F^*$ is $\langle$~metrically / $C$-topologically~$\rangle$ injective. Since operator $\pi_F^*$ is $\langle$~isometric / $c$-topologically injective~$\rangle$, then there exists a linear operator $R:(A\projtens\ell_1(B_F))^*\to F^*$ of norm $\langle$~at most $1$ / at most $cC$~$\rangle$ such that $R\pi_F^*=1_{F^*}$.

Fix $h\in (A\projtens\ell_1(B_F))^*$ and $x\in F$. Consider bilinear functional $M_{h,x}:A\times A\to\mathbb{C}:(a,b)\mapsto R(h\cdot a)(b\cdot x)$. Clearly, $\Vert M_{h,x}\Vert\leq\Vert R\Vert\Vert h\Vert\Vert x\Vert$. By universal property of projective tensor product we have a bounded linear functional $m_{h,x}:A\projtens A\to\mathbb{C}:a\projtens b\mapsto R(h\cdot a)(b\cdot x)$. Note that $m_{h,x}$ is linear in $h$ and $x$. Even more, for any $u\in A\projtens A$, $a\in A$ and $f\in F^*$ we have $m_{\pi_F^*(f),x}(u)=f(\Pi_A(u)\cdot x)$, $m_{h\cdot a,x}(u)=m_{h,x}(a\cdot u)$, $m_{h,a\cdot x}(u)=m_{h,x}(u\cdot a)$. It easily checked for elementary tensors. Then it is enough to recall that their linear span is dense in $A\projtens A$.

Let $\mathfrak{F}$ be the section filter on $N$ and let $\mathfrak{U}$ be an ultrafilter dominating $\mathfrak{F}$. For any $h\in (A\projtens\ell_1(B_F))^*$ and $x\in F$ we have $|m_{h,x}(d_\nu)|\leq c\Vert R\Vert\Vert h\Vert\Vert x\Vert$, i.e. $(m_{h,x}(d_\nu))_{\nu\in N}$ is a bounded net of complex numbers. Therefore we have a well defined limit $\lim_{\mathfrak{U}}m_{h,x}(d_\nu)$ along ultrafilter $\mathfrak{U}$. Consider linear operator $\tau:(A\projtens\ell_1(B_F))^*\to F^*:h\mapsto(x\mapsto\lim_{\mathfrak{U}}m_{h,x}(d_\nu))$. From norm estimates for $m_{h,x}$ it follows that $\tau$ is bounded with $\Vert\tau\Vert\leq c\Vert R\Vert$. For all $a\in A$, $x\in F$ and $h\in (A\projtens\ell_1(B_F))^*$ we have
$$
\tau(h\cdot a)(x)-(\tau(h)\cdot a)(x)
=\tau(h\cdot a)(x)-\tau(h)(a\cdot x)
=\lim_{\mathfrak{U}}m_{h\cdot a,x}(d_\nu)-\lim_{\mathfrak{U}}m_{h,a\cdot x}(d_\nu).
$$
$$
=\lim_{\mathfrak{U}}m_{h,x}(a\cdot d_\nu)-m_{h,x}(d_\nu\cdot a)
=m_{h,x}\left(\lim_{\mathfrak{U}}(a\cdot d_\nu-d_\nu\cdot a)\right)
$$
$$
=m_{h,x}\left(\lim_{\nu}(a\cdot d_\nu-d_\nu\cdot a)\right)
=m_{h,x}(0)
=0.
$$
Therefore $\tau$ is a morphism of right $A$-modules. Now for all $f\in F^*$ and $x\in F$ we have
$$
(\tau(\pi_F^*)(f))(x)
=\lim_{\mathfrak{U}}m_{\pi_F^*(f),x}(d_\nu)
=\lim_{\mathfrak{U}}f(\Pi_A(d_\nu)\cdot x)
=\lim_{\nu}f(\Pi_A(d_\nu)\cdot x)
$$
$$
=f\left(\lim_{\nu}\Pi_A(d_\nu)\cdot x\right)
=f(x).
$$
So $\tau\pi_F^*=1_{F^*}$. This means that $F^*$ is a $\langle$~$1$-retract / $c^2 C$-retract~$\rangle$ of $(A\projtens\ell_1(B_F))^*$
 in $\langle$~$\mathbf{mod}_1-A$ / $\mathbf{mod}-A$~$\rangle$. The latter $A$-module is $\langle$~metrically / topologically~$\rangle$ injective, because $(A_+\projtens\ell_1(B_F))^*\isom{\mathbf{mod}_1-A}\mathcal{B}(A_+,\ell_\infty(B_F))$ and by proposition $\langle$~\ref{RetrMetTopInjIsMetTopInj} / \ref{RetrCTopInjIsCTopInj}~$\rangle$ so does its retract $F^*$. By proposition $\langle$~\ref{MetTopFlatCharac} / \ref{CTopFlatCharac}~$\rangle$ this is equivalent to $\langle$~metric / $c^2 C$-topological~$\rangle$ flatness of $F$.
\end{proof}

\begin{theorem}\label{TopL1FlatModAoverAmenBanAlg} Let $A$ be a relatively $c$-amenable Banach algebra and $F$ be a left Banach $A$-module which as Banach space is an $\mathscr{L}_{1, C}^g$-space. Then $F$ is a $(1+c)^2C\max(c^2,(1+c))$-topologically flat $A$-module.
\end{theorem}
\begin{proof} Since $A$ is amenable it admits a two-sided $c$-bounded approximate identity. By proposition \ref{DualBanModDecomp} the annihilator $A$-module $F/F_{ess}$ is an $\mathscr{L}_{1,1+c}^g$-space. From proposition \ref{MetTopEssL1FlatModAoverAmenBanAlg} we get that the essential $A$-module $F_{ess}$ is $c^2 C$-topologically flat. Now the result follows from proposition \ref{TopFlatModCharac}.
\end{proof}

We must point out here that in relative Banach homology any left Banach module over relatively amenable Banach algebra is relatively flat [\cite{HelBanLocConvAlg}, theorem 7.1.60]. Even topological theory is so restrictive that in some cases, as the following proposition shows, we can obtain complete characterization of all flat modules.

\begin{proposition}\label{TopFlatModAoverAmenL1BanAlgCharac} Let $A$ be a relatively amenable Banach algebra which as Banach space is an $\mathscr{L}_1^g$-space. Then for a Banach $A$-module $F$ the following are equivalent:

$i)$ $F$ is topologically flat $A$-module; 

$ii)$ $F$ is an $\mathscr{L}_1^g$-space.
\end{proposition}
\begin{proof} The equivalence follows from proposition \ref{TopProjInjFlatModOverMthscrL1SpCharac} and theorem \ref{TopL1FlatModAoverAmenBanAlg}.
\end{proof}

Finally we are able to give an example of relatively flat, but not topologically flat ideal in a Banach algebra. Consider $A=L_1(\mathbb{T})$. It is known, that $A$ has a translation invariant infinite dimensional closed subspace $I$ isomorphic to a Hilbert space [\cite{RosProjTransInvSbspLpG}, p.52]. By [\cite{KaniBanAlg}, proposition 1.4.7] we have that $I$ is a two-sided ideal of $A$, as any translation invariant subspace of $A$. By [\cite{DefFloTensNorOpId}, section 23.3] this ideal is not an $\mathscr{L}_1^g$-space. So from proposition \ref{TopFlatModAoverAmenL1BanAlgCharac} we get that $I$ is not topologically flat as $A$-module. We claim it is still relatively flat. Since $\mathbb{T}$ is a compact group, then it is amenable [\cite{PierAmenLCA}, proposition 3.12.1]. Thus $A$ is relatively amenable [\cite{HelBanLocConvAlg}, proposition VII.1.86], so all left ideals of $A$ are relatively flat [\cite{HelBanLocConvAlg}, proposition VII.1.60(I)]. In particular, $I$ is relatively flat.

%----------------------------------------------------------------------------------------
%	Injectivity of ideals
%----------------------------------------------------------------------------------------

\subsection{Injectivity of ideals}
\label{SubSectionInjectivityOfIdeals}

Injective ideals are rare creatures but we need to say a few words about them. Results of this section needed for the study of metric and topological injectivity of $C^*$-algebras.

\begin{proposition}\label{MetTopInjOfId} Let $I$ be a right ideal of a Banach algebra $A$. Assume $I$ is $\langle$~metrically / $C$-topologically~$\rangle$ injective $A$-module. Then $I$ has a left identity of norm $\langle$~at most $1$ / at most $C$~$\rangle$ and is a $\langle$~$1$-retract / $C$-retract~$\rangle$ of $A$ in $\mathbf{mod}-A$.
\end{proposition}
\begin{proof} Consider isometric embedding $\rho^+:I\to A_+$ of $I$ into $A_+$. Clearly, this is an $A$-morphism. Since $I$ is $\langle$~metrically / $C$-topologically~$\rangle$ injective, then $\rho^+$ has left inverse $A$-morphism $\tau^+:A_+\to I$ with norm $\langle$~at most $1$ / at most $C$~$\rangle$. Now for all $x\in I$ we have $x=\tau^+(\rho^+(x))=\tau^+(e_{A_+}\rho^+(x))=\tau(e_{A_+})\rho^+(x)=\tau^+(e_{A_+})x$. In other words $p=\tau^+(e_{A_+})\in I$ is a left unit for $I$. Clearly, $\Vert p\Vert\leq\Vert\tau^+\Vert\Vert e_{A_+}\Vert\leq\Vert\tau^+\Vert$. Consider maps $\rho:I\to A:x\mapsto x$ and $\tau:A\to I:x\mapsto p x$. Clearly, they are morphisms of right $A$-modules and $\tau\rho=1_I$. Hence $I$ is a $\langle$~$1$-retract / $C$-retract~$\rangle$ of $A$ in $\mathbf{mod}-A$.
\end{proof}

\begin{proposition}\label{ReduceInjIdToInjAlg} Let $I$ be a two-sided ideal of Banach algebra $A$, which is faithful as right $I$-module. Then

$i)$ if $I$ is $\langle$~metrically / $C$-topologically~$\rangle$ injective $I$-module, then $I$ is $\langle$~metrically / $C$-topologically~$\rangle$ injective $A$-module; 

$ii)$ if $I$ is $\langle$~metrically / $C$-topologically~$\rangle$ injective $A$-module, then $I$ is $\langle$~metrically / $C^2$-topologically~$\rangle$ injective $I$-module.
\end{proposition}
\begin{proof} $i)$ The result immediately follows from paragraph $i)$ of proposition \ref{MetTopInjUnderChangeOfAlg}.

$ii)$ By proposition \ref{MetTopInjOfId} the $A$-module $I$ is $\langle$~$1$-complemented / $C$-complemented~$\rangle$ in $A$. By paragraph $ii)$ of proposition \ref{MetTopInjUnderChangeOfAlg} the $I$-module $I$ is $\langle$~metrically / $C^2$-topologically~$\rangle$ injective.
\end{proof}
 
% Chapter Template

\chapter{Applications to algebras of analysis} % Main chapter title

\label{ChapterApplicationsToAlgebrasOfAnalysis} % Change X to a consecutive number; for referencing this chapter elsewhere, use \ref{ChapterX}

\lhead{Chapter 3. \emph{Applications to algebras of analysis}} % Change X to a consecutive number; this is for the header on each page - perhaps a shortened title

Vaguely speaking there are three types of Banach modules depending on the type of module action: modules with pointwise multiplication, modules with composition of operators in the role of module action and modules with convolution. We shall investigate main examples of these types. Following the style of Dales and Polyakov from \cite{DalPolHomolPropGrAlg} we shall systematize all results on classical modules of analysis, but this time for metric and topological theory. We shall consider modules over operator algebras, sequence algebras, algebras of continuous functions and, finally, classical modules of harmonic analysis.


%----------------------------------------------------------------------------------------
%	Applications to operator algebras
%----------------------------------------------------------------------------------------


\section{Applications to modules over \texorpdfstring{$C^*$}{C*}-algebras}
\label{SectionApplicationsToModulesOverCStarAlgebras}

%----------------------------------------------------------------------------------------
%	Spatial modules
%----------------------------------------------------------------------------------------

\subsection{Spatial modules}
\label{SubSectionSpatialModules}

We start from the simplest  examples of modules over operator algebras --- the spatial modules. By Gelfand-Naimark's theorem (see e.g. [\cite{HelBanLocConvAlg}, theorem 4.7.57]) for any $C^*$-algebra $A$ there exists a Hilbert space $H$ and an isometric ${}^*$-homomorphism $\varrho:A\to\mathcal{B}(H)$. For Hilbert spaces that admit such homomorphism we may consider the left $A$-module $H_\varrho$ with module action defined as $a\cdot x=\varrho(a)(x)$. Automatically we get the structure of right $A$-module on $H^*$ which is by Riesz's theorem is isometrically isomorphic to $H^{cc}$. This isomorphism allows one to define the right $A$-module structure on $H^{cc}$ by $\overline{x}\cdot a=\overline{\varrho(a^*)(x)}$. For a given $x_1,x_2\in H$ we define the rank one operator $x_1\bigcirc x_2:H\to H:x\mapsto \langle x, x_2\rangle x_1$. 

\begin{proposition}\label{SpatModOverCStarAlgProp} Let $A$ be a $C^*$-algebra and $\varrho:A\to\mathcal{B}(H)$ be an isometric ${}^*$-homomorphism, such that its image contains a subspace of rank one operators of the form $\{x\bigcirc x_0:x\in H\}$ for some non zero $x_0\in H$. Then the left $A$-module $H_\varrho$ is metrically projective and flat, while the  right $A$-module $H_\varrho^{cc}$ is metrically injective.
\end{proposition}
\begin{proof} Without loss of generality we may assume that $\Vert x_0\Vert=1$. Consider linear operators $\pi:A_+\to H_\varrho:a\oplus_1 z\mapsto \varrho(a)(x_0)+zx_0$ and $\sigma:H_\varrho\to A_+:x\mapsto \varrho^{-1}(x\bigcirc x_0)$. It is straightforward to check that $\pi$ and $\sigma$ are contractive $A$-morphisms such that $\pi\sigma=1_{H_\varrho}$. Therefore $H_\varrho$ is a retract of $A_+$ in $A-\mathbf{mod}_1$. From propositions \ref{UnitalAlgIsMetTopProj} and \ref{RetrMetTopProjIsMetTopProj} it follows that $H_\varrho$ is metrically projective $A$-module. From proposition \ref{MetTopProjIsMetTopFlat} it follows that $H_\varrho$ is metrically flat too. Since $H_\varrho^{cc}\isom{\mathbf{mod}_1-A}H_\varrho^*$, proposition \ref{DualMetTopProjIsMetrInj} gives that $H_\varrho^{cc}$ is metrically injective.
\end{proof}

In what follows we shall use the following simple application of the above result.

\begin{proposition}\label{FinDimNHModTopProjFlat} Let $H$ be a finite dimensional Hilbert space. Then $\mathcal{N}(H)$ is topologically projective and hence flat as $\mathcal{B}(H)$-module.
\end{proposition}
\begin{proof} From [\cite{HelBanLocConvAlg}, proposition 0.3.38] we know that $\mathcal{N}(H)\isom{\mathbf{Ban}_1}H\projtens H^*$. Let $\varrho=1_{\mathcal{B}(H)}$, then we can claim a little bit more: $\mathcal{N}(H)\isom{\mathcal{B}(H)-\mathbf{mod}_1} H_\varrho\projtens H^*$. Since $H^*$ is finite dimensional, then  $H^*\isom{\mathbf{Ban}}\ell_1(\mathbb{N}_n)$ for $n=\dim(H)$ and as the result $\mathcal{N}(H)\isom{\mathcal{B}(H)-\mathbf{mod}} H_\varrho\projtens\ell_1(\mathbb{N}_n)$. By proposition \ref{SpatModOverCStarAlgProp} the module $H_\varrho$ it topologically projective, so from corollary \ref{MetTopProjTensProdWithl1} we get that $\mathcal{N}(H)$ is topologically projective as $\mathcal{B}(H)$-module. The last claim of theorem follows from proposition \ref{MetTopProjIsMetTopFlat}.
\end{proof}

%----------------------------------------------------------------------------------------
%	Projective ideals of C^*-algebras
%----------------------------------------------------------------------------------------

\subsection{Projective ideals of \texorpdfstring{$C^*$}{C*}-algebras}
\label{SubSectionProjectiveIdealsOfCStarAlgebras}

The study of homologically trivial ideals of $C^*$-algebras we start from projectivity, but before stating the main result we need a preparatory lemma.

\begin{lemma}\label{ContFuncCalcOnIdealOfCStarAlg} Let $I$ be a left ideal of a unital $C^*$-algebra $A$. Assume $a\in I$ is a self-adjoint element and let $E$ be the real subspace of real valued functions in $C(\operatorname{sp}_A(a))$ vanishing at zero. Then there is an isometric homomorphism $\operatorname{RCont}_a^0:E\to I$ well defined by $\operatorname{RCont}_a^0(f)=a$, where $f:\operatorname{sp}_A(a)\to\mathbb{C}:t\mapsto t$.
\end{lemma}
\begin{proof} By $\mathbb{R}_0[t]$ we denote the real linear subspace of $E$ consisting of polynomials vanishing at zero. Since $I$ is an ideal of $A$ and and $p\in\mathbb{R}_0[t]$ has no constant term then $p(a)\in I$.  Hence we have well defined $\mathbb{R}$-linear homomorphism of algebras $\operatorname{RPol}_a^0:\mathbb{R}_0[t]\to I:p\mapsto p(a)$. By continuous functional calculus for any polynomial $p$ holds $\Vert p(a)\Vert=\Vert p|_{\operatorname{sp}_A(a)}\Vert_\infty$, so $\Vert\operatorname{RPol}_a^0(p)\Vert=\Vert p|_{\operatorname{sp}_A(a)}\Vert_\infty$. Thus $\operatorname{RPol}_a^0$ is isometric. As $\mathbb{R}_0[t]$ is dense in $E$ and $I$ is complete, then $\operatorname{RPol}_a^0$ has an isometric extension $\operatorname{RCont}_a^0:E\to I$ which is also an $\mathbb{R}$-linear homomorphism. 
\end{proof}

The following proof is inspired by ideas of D. P. Blecher and T. Kania. In [\cite{BleKanFinGenCStarAlgHilbMod}, lemma 2.1] they proved that any algebraically finitely generated left ideal of $C^*$-algebras is principal.  

\begin{theorem}\label{LeftIdealOfCStarAlgMetTopProjCharac} Let $I$ be a left ideal of a $C^*$-algebra $A$. Then the following are equivalent:

$i)$ $I=Ap$ for some self-adjoint idempotent $p\in I$;

$ii)$ $I$ is metrically projective $A$-module;

$iii)$ $I$ is topologically projective $A$-module.
\end{theorem}
\begin{proof} $i)$ $\implies$ $ii)$ Since $p$ is a self-adjoint idempotent, then $\Vert p\Vert=1$, so by proposition \ref{UnIdeallIsMetTopProj} paragraph $i)$ the ideal $I$ is metrically projective as $A$-module.

$ii)$ $\implies$ $iii)$ See proposition \ref{MetProjIsTopProjAndTopProjIsRelProj}.

$iii)$ $\implies$ $i)$ Let $(e_\nu)_{\nu\in N}$ be a right contractive approximate identity of ideal $I$ [\cite{HelBanLocConvAlg}, theorem 4.7.79]. Since $I$ admits a right approximate identity, then it is an essential left $I$-module, and a fortiori an essential $A$-module. By proposition \ref{NonDegenMetTopProjCharac} we have a right inverse $A$-morphism $\sigma:I\to A\projtens \ell_1(B_I)$ of $\pi_I$ in $A-\mathbf{mod}$. For each $d\in B_I$ consider $A$-morphisms $p_d:A\projtens \ell_1(B_I)\to A:a\projtens \delta_x\mapsto \delta_x(d)a$ and $\sigma_d=p_d\sigma$. Then $\sigma(x)=\sum_{d\in B_I}\sigma_d(x)\projtens \delta_d$ for all $x\in I$. From identification $A\projtens\ell_1(B_I)\isom{\mathbf{Ban}_1}\bigoplus_1\{ A:d\in B_I\}$, for all $x\in I$ we have $\Vert\sigma(x)\Vert=\sum_{d\in B_I} \Vert\sigma_d(x)\Vert$. Since $\sigma$ is a right inverse morphism of $\pi_I$ we have $x=\pi_I(\sigma(x))=\sum_{d\in B_I}\sigma_d(x)d$ for all $x\in I$. 

For all $x\in I$ we have
$\Vert\sigma_d(x)\Vert=\Vert\sigma_d(\lim_\nu xe_\nu)\Vert=\lim_\nu\Vert x\sigma_d(e_\nu)\Vert \leq\Vert x\Vert\liminf_\nu\Vert\sigma_d(e_\nu)\Vert$, so $\Vert\sigma_d\Vert\leq \liminf_\nu\Vert\sigma_d(e_\nu)\Vert$. Then for all $S\in\mathcal{P}_0(B_I)$ holds
$$
\sum_{d\in S}\Vert \sigma_d\Vert
\leq \sum_{d\in S}\liminf_\nu\Vert \sigma_d(e_\nu)\Vert
\leq \liminf_\nu\sum_{d\in S}\Vert \sigma_d(e_\nu)\Vert
\leq \liminf_\nu\sum_{d\in B_I}\Vert \sigma_d(e_\nu) \Vert
$$
$$
=\liminf_{\nu}\Vert\sigma(e_\nu)\Vert
\leq \Vert\sigma\Vert\liminf_{\nu}\Vert e_\nu\Vert
\leq \Vert\sigma\Vert
$$
Since $S\in \mathcal{P}_0(B_I)$ is arbitrary, then the sum $\sum_{d\in B_I}\Vert\sigma_d\Vert$ is finite. As the consequence, the sum $\sum_{d\in B_I}\Vert\sigma_d\Vert^2$ is finite too. 

Now we regard $A$ as an ideal in its unitization $A_\#$, then $I$ is an ideal of $A_\#$ too. Fix a natural number $m\in\mathbb{N}$ and a real number $\epsilon>0$. Then there exists a set $S\in\mathcal{P}_0(B_I)$ such that $\sum_{d\in B_I\setminus S}\Vert\sigma_d\Vert<\epsilon$. Denote its cardinality by $N$. Consider positive element $b=\sum_{d\in B_I}\Vert\sigma_d\Vert^2 d^*d\in I$. Now we perform a ``power trick'' by considering different powers $b^{1/m}$ of positive element $b$, where $m\in\mathbb{N}$. By lemma \ref{ContFuncCalcOnIdealOfCStarAlg} we have that $b^{1/m}\in I$, so $b^{1/m}=\sum_{d\in B_I}\sigma_d(b^{1/m})d$. By continuous functional calculus we have $\Vert b^{1/m}\Vert=\sup_{t\in\operatorname{sp}_{A_\#}(b)} t^{1/m}\leq\Vert b\Vert^{1/m}$, then $\limsup_{m\to\infty}\Vert b^{1/m}\Vert\leq 1$. Therefore $\Vert b^{1/m}\Vert\leq 2$ for sufficiently big $m$. Denote $\varsigma_d=\sigma_d(b^{1/m})$, $u=\sum_{d\in S}\varsigma_dd$ and $v=\sum_{d\in B_I\setminus S}\varsigma_d d$, so 
$$
b^{2/m}=(b^{1/m})^*b^{1/m}=u^*u+u^*v+v^*u+v^*v
$$
Clearly, $\varsigma_d^*\varsigma_d\leq \Vert \varsigma_d\Vert^2 e_{A_\#}\leq \Vert \sigma_d\Vert^2\Vert b^{1/m}\Vert^2 e_{A_\#}\leq 4\Vert \sigma_d\Vert^2 e_{A_\#}$. For any $x,y\in A$ we have $x^*x+y^*y-(x^*y+y^*x)=(x-y)^*(x-y)\geq 0$, therefore 
$$
d^*\varsigma_d^* \varsigma_c c+c^*\varsigma_c^* \varsigma_d d
\leq d^*\varsigma_d^*\varsigma_d d + c^*\varsigma_c^*\varsigma_c c
\leq 4\Vert \sigma_d\Vert^2 d^*d+4\Vert \sigma_c\Vert^2 c^*c
$$
for all $c,d\in B_I$. We sum up these inequalities over $c\in S$ and $d\in S$, then 
$$
\begin{aligned}
\sum_{c\in S}\sum_{d\in S}c^*\varsigma_c^* \varsigma_d d
&=\frac{1}{2}\left(\sum_{c\in S}\sum_{d\in S}d^*\varsigma_d^* \varsigma_c c+\sum_{c\in S}\sum_{d\in S}c^*\varsigma_c^* \varsigma_d d\right)\\
&\leq\frac{1}{2}\left(4 N\sum_{d\in S} \Vert \sigma_d\Vert^2 d^*d+
4 N\sum_{c\in S} \Vert \sigma_c\Vert^2 c^*c\right)\\
&=4 N\sum_{d\in S} \Vert \sigma_d\Vert^2 d^*d.
\end{aligned}
$$
Therefore
$$
u^*u
=\left(\sum_{c\in S}\varsigma_c c\right)^*\left(\sum_{d\in S}\varsigma_d d\right)
=\sum_{c\in S}\sum_{d\in S}c^*\varsigma_c^* \varsigma_d d
\leq 4N\sum_{d\in S} \Vert \sigma_d\Vert^2 d^*d
= 4N b
$$
Note that
$$
\Vert u\Vert
\leq \sum_{d\in S}\Vert\varsigma_d\Vert\Vert d\Vert
\leq \sum_{d\in S}2\Vert\sigma_d\Vert
\leq 2\Vert\sigma\Vert,
\qquad
\Vert v\Vert
\leq \sum_{d\in B_I\setminus S}\Vert\varsigma_d\Vert\Vert d\Vert
\leq \sum_{d\in B_I\setminus S}2\Vert\sigma_d\Vert
\leq 2\epsilon,
$$
so $\Vert u^*v+v^*u\Vert\leq 8\Vert\sigma\Vert\epsilon$ and $\Vert v^*v\Vert\leq 4\epsilon^2$. Since $u^*v+v^*u$ and $v^*v$ are self adjoint, then $u^*v+v^*u\leq 8\Vert\sigma\Vert\epsilon e_{A_\#}$ and $v^*v\leq 4\epsilon^2 e_{A_\#}$
Therefore for any $\epsilon>0$ and sufficiently big $m$ we have 
$$
b^{2/m}
=u^*u+u^*v+v^*u+v^*v
\leq 4Nb+\epsilon(8\Vert\sigma\Vert+4\epsilon)e_{A_\#}.
$$


In other words $g_m(b)\geq 0$ for continuous function $g_m:\mathbb{R}_+\to\mathbb{R}:t\mapsto 4Nt+\epsilon(8\Vert\sigma\Vert+4\epsilon)-t^{2/m}$. Now choose $\epsilon>0$ such that $M:=\epsilon(8\Vert\sigma\Vert+4\epsilon)<1$.  By spectral mapping theorem [\cite{HelLectAndExOnFuncAn}, theorem 6.4.2] we get $g_m(\operatorname{sp}_{A_\#}(b))=\operatorname{sp}_{A_\#}(g_m(b))\subset\mathbb{R}_+$. It is routine to check that $g_m$ has the only one extreme point $t_{0,m}=(2Nm)^{\frac{m}{2-m}}$ where the minimum of $g_m$ is attained. Since $\lim_{m\to\infty} g_m(t_{0,m})=M-1<0$, $g_m(0)=M>0$ and $\lim_{t\to\infty} g_m(t)=+\infty$, then for sufficiently big $m$ the function $g_m$ has exactly two roots: $t_{1,m}\in(0,t_{0,m})$ and $t_{2,m}\in(t_{0,m},+\infty)$. Therefore solution of the inequality $g_m(t)\geq 0$ is $t\in[0,t_{1,m}]\cup[t_{2,m},+\infty)$. Hence $\operatorname{sp}_{A_\#}(b)\subset[0,t_{1,m}]\cup[t_{2,m},+\infty)$ for all sufficiently large $m$. Since $\lim_{m\to\infty} t_{0,m}=0$ then $\lim_{m\to\infty} t_{1,m}=0$ also. Note that $g_m(1)=4N+M-1>0$, so for sufficiently large $m$ we also have $t_{2,m}\leq 1$. Consequently, $\operatorname{sp}_{A_\#}(b)\subset\{0\}\cup[1,+\infty)$.

Consider a continuous function $h:\mathbb{R}_+\to\mathbb{R}:t\mapsto\min(1, t)$, then from lemma \ref{ContFuncCalcOnIdealOfCStarAlg} we get an idempotent $p=h(b)=\operatorname{RCont}_b^0(h)\in I$ such that $\Vert p\Vert=\sup_{t\in\operatorname{sp}_{A_\#}(b)}|h(t)|\leq 1$. Therefore $p$ is a self-adjoint idempotent. Since $h(t)t=th(t)=t$ for all $t\in \operatorname{sp}_{A_\#}(b)$ we have $bp=pb=b$. The last equality implies
$$
0=(e_{A_\#}-p)b(e_{A_\#}-p)=\sum_{d\in B_I}(\Vert\sigma_d\Vert d(e_{A_\#}-p))^*\Vert\sigma_d\Vert d(e_{A_\#}-p).
$$
Since the right hand side of this equality is non negative, then $d=dp$ for all $d\in B_I$ with $\sigma_d\neq 0$. Finally, for any $x\in I$ we obtain $xp=\sum_{d\in B_I}\sigma_d(x)dp=\sum_{d\in B_I}\sigma_d(x)d=x$, i.e. $I=Ap$, for self-adjoint idempotent $p\in I$.
\end{proof}

It is worth to point out here that in relative theory there no such description for relative projectivity of left ideals of $C^*$-algebras. Though for the case of separable $C^*$-algebras (that is $C^*$-algebras that are separable as Banach spaces) all left ideals are relatively projective. See [\cite{LykProjOfBanAndCStarAlgsOfContFld}, section 6] for a nice overview of the current state of the problem.

\begin{corollary}\label{BiIdealOfCStarAlgMetTopProjCharac} Let $I$ be a two-sided ideal of a $C^*$-algebra $A$. Then the following are equivalent:

$i)$ $I$ is unital;

$ii)$ $I$ is metrically projective $A$-module;

$iii)$ $I$ is topologically projective $A$-module.
\end{corollary}
\begin{proof} The ideal $I$ has a contractive approximate identity [\cite{HelBanLocConvAlg}, theorem 4.7.79]. Therefore $I$ has a right identity iff $I$ is unital. Now all equivalences follow from theorem \ref{LeftIdealOfCStarAlgMetTopProjCharac}. 
\end{proof}

\begin{corollary}\label{IdealofCommCStarAlgMetTopProjCharac} Let $S$ be a locally compact Hausdorff space, and $I$ be an ideal of $C_0(S)$. Then the following are equivalent:

$i)$ $\operatorname{Spec}(I)$ is compact;

$ii)$ $I$ is metrically projective $C_0(S)$-module;

$iii)$ $I$ is topologically projective $C_0(S)$-module.
\end{corollary}
\begin{proof} By Gelfand-Naimark's theorem $I\isom{\mathbf{Ban}_1}C_0(\operatorname{Spec}(I))$, therefore $I$ is semisimple. Now by Shilov's idempotent theorem $I$ is unital iff $\operatorname{Spec}(I)$ is compact. It remains to apply corollary \ref{BiIdealOfCStarAlgMetTopProjCharac}. 
\end{proof}

It is worth to mention that the class of relatively projective ideals of $C_0(S)$ is much larger. In fact a closed ideal $I$ of $C_0(S)$ is relatively projective iff $\operatorname{Spec}(I)$ is paracompact [\cite{HelHomolBanTopAlg}, chapter IV,\S\S 2-3].

%----------------------------------------------------------------------------------------
%	Injective ideals of C^*-algebras
%----------------------------------------------------------------------------------------

\subsection{Injective ideals of \texorpdfstring{$C^*$}{C*}-algebras}
\label{SubSectionInjectiveIdealsOfCStarAlgebras}

Now we proceed to injectivity of two-sided ideals of $C^*$-algebras. Unfortunately we don't have a complete characterization at hand, but some necessary conditions and several examples. The following proposition shows that we may restrict investigation of injective ideals to the case of $C^*$-algebras that $\langle$~metrically / topologically~$\rangle$ injective over themselves as right modules.

\begin{proposition}\label{MetTopInjOvrAlgIffOvrItslf} Let $I$ be a two-sided ideal of a $C^*$-algebra $A$, then $I$ is $\langle$~metrically / topologically~$\rangle$ injective as $A$-module iff $I$ is $\langle$~metrically / topologically~$\rangle$ injective as $I$-module.
\end{proposition}
\begin{proof} Note that any two-sided ideal $I$ of a $C^*$-algebra $A$ is again a $C^*$-algebra with contractive approximate identity [\cite{HelBanLocConvAlg}, theorem 4.7.79]. Therefore $I$ is faithful as right $I$-module. Now proposition \ref{ReduceInjIdToInjAlg} gives the desired equivalence.
\end{proof}

We shall say a few words on so called $AW^*$-algebras, since they are key players here. In attempts to find a purely algebraic definition of $W^*$-algebras Kaplanski introduced this class of $C^*$-algebras in \cite{KaplProjInBanAlg}. A $C^*$-algebra $A$ is called an $AW^*$-algebra if for any subset $S\subset A$ the right algebraic annihilator $\operatorname{r.ann}_A(S)=\{y\in A: Sy=\{0\}\}$ is of the form $pA$ for some self-adjoint idempotent $p\in A$. This class contains all $W^*$-algebras, but strictly larger. Note that for the case of commutative $C^*$-algebras the property of being an $AW^*$-algebra is equivalent to  $\operatorname{Spec}(A)$ being a Stonean space [\cite{BerbBaerStarRings}, theorem 1.7.1]. The main reference to $AW^*$-algebras and more general Baer ${}^*$-rings is \cite{BerbBaerStarRings}. 

The following proposition is a combination of results by M. Hamana and M. Takesaki.

\begin{proposition}[Hamana, Takesaki]\label{MetInjCStarAlgCharac} Let $A$ be a $C^*$-algebra, then it is metrically injective right $A$-module iff it is a commutative $AW^*$-algebra, that is $\operatorname{Spec}(A)$ is a Stonean space.
\end{proposition}
\begin{proof} 

If $A$ is metrically injective as $A$-module, then it has norm one left identity by proposition \ref{MetTopInjOfId}. But $A$ also has a contractive approximate identity  [\cite{HelBanLocConvAlg}, theorem 4.7.79], therefore $A$ is unital. Now by result of Hamana  [\cite{HamInjEnvBanMod}, proposition 2] the $C^*$-algebra $A$ is a commutative $AW^*$-algebra. Hamana's argument was for left modules, but one can easily modify his proof to get the result for right modules.

The converse proved by Takesaki in [\cite{TakHanBanThAndJordDecomOfModMap}, theorem 2]. Although only two-sided Banach modules were treated there, the reasoning is essentially the same for right modules.
\end{proof}

It remains to consider topological injectivity. As the following proposition shows topologically injective $C^*$-algebras are not so far from commutative ones. This proposition exploits the l.u.st. property, for its definition see section  \ref{SubSectionHomologicallyTrivialModulesOverBanachAlgebrasWithSpecificGeometry}.

\begin{proposition}\label{TopInjIdHaveLUST} Let $A$ be a $C^*$-algebra which is topologically injective as an $A$-module. Then $A$ has the l.u.st. property and as the consequence it can't contain  $\mathcal{B}(\ell_2(\mathbb{N}_n))$ as ${}^*$-subalgebra for arbitrary $n\in\mathbb{N}$.
\end{proposition}
\begin{proof} By Gelfand-Naimark's theorem [\cite{HelBanLocConvAlg}, theorem 4.7.57] there exists a Hilbert space $H$ and an isometric ${}^*$-homomorphism $\varrho:A\to\mathcal{B}(H)$. Denote $\Lambda:=B_{H_\varrho^{cc}}$. It is easy to check that 
$$
\rho:A\to\bigoplus\nolimits_\infty\{H_\varrho^{cc}:\overline{x}\in \Lambda\}:a\mapsto \bigoplus\nolimits_\infty\{\overline{x}\cdot a:\overline{x}\in \Lambda\}
$$
is an isometric $A$-morphism of right $A$-modules. Since $A$ is topologically injective $A$-module, then $\rho$ has a left inverse $A$-morphism $\tau$. Therefore $A$ is complemented in $E:=\bigoplus_\infty\{H_\varrho^{cc}:\overline{x}\in \Lambda\}$ via projection $\rho\tau$. Note that $H_{\varrho}^{cc}$ is a Banach lattice as any Hilbert space, then so does $E$. As any Banach lattice $E$ has the l.u.st. property [\cite{DiestAbsSumOps}, theorem 17.1], then so does $A$, because the l.u.st. property is inherited by complemented subspaces.

Assume $A$ contains $\mathcal{B}(\ell_2(\mathbb{N}_n))$ as ${}^*$-subalgebra for arbitrarily large $n\in\mathbb{N}$. In fact this copy of $\mathcal{B}(\ell_2(\mathbb{N}_n))$ is $1$-complemented in $A$ [\cite{LauLoyWillisAmnblOfBanAndCStarAlgsOfLCG}, lemma 2.1]. Therefore we have an inequality for local unconditional constants $\kappa_u(\mathcal{B}(\ell_2(\mathbb{N}_n)))\leq \kappa_u(A)$. By theorem 5.1 in \cite{GorLewAbsSmOpAndLocUncondStrct} we know that $\lim_n \kappa_u(\mathcal{B}(\ell_2(\mathbb{N}_n)))=+\infty$, so $\kappa_u(A)=+\infty$. This contradicts the l.u.st. property of $A$. Hence $A$ can't contain $\mathcal{B}(\ell_2(\mathbb{N}_n))$ as ${}^*$-subalgebra for arbitrary $n\in\mathbb{N}$.
\end{proof}

As the proposition \ref{TopInjIdHaveLUST} shows $C^*$-algebras that are topologically injective over themselves can't contain $\mathcal{B}(\ell_2(\mathbb{N}_n))$ as ${}^*$-subalgebra for arbitrary $n\in\mathbb{N}$. Such $C^*$-algebras are called subhomogeneous, and in fact [\cite{BlackadarOpAlg}, proposition IV.1.4.3] they can be treated as closed ${}^*$-subalgebras of $M_n(C(K))$ for some compact Hausdorff space $K$ and some natural number $n$. For more on subhomogeneous $C^*$-algebras see [\cite{BlackadarOpAlg}, section IV.1.4]. 

We shall give two important examples of non commutative $C^*$-algebras that are topologically injective over themselves.

\begin{proposition}\label{FinDimBHModTopInj} Let $H$ be a finite dimensional Hilbert space. Then $\mathcal{B}(H)$ is topologically injective as $\mathcal{B}(H)$-module. 
\end{proposition}
\begin{proof} Note that $\mathcal{B}(H)\isom{\mathbf{mod}_1-\mathcal{B}(H)}\mathcal{N}(H)^*$, and the the result immediately follows from propositions \ref{FinDimNHModTopProjFlat} and \ref{DualMetTopProjIsMetrInj}.
\end{proof}

\begin{proposition}\label{CKMatrixModTopInj} Let $K$ be a Stonean space and $n\in\mathbb{N}$, then $M_n(C(K))$ is topologically injective $M_n(C(K))$-module.
\end{proposition}
\begin{proof} For a fixed $s\in K$ by $\mathbb{C}_s$ we denote the right $C(K)$-module $\mathbb{C}$ with outer action defined by $z\cdot a=a(s)z$ for all $a\in C(K)$ and $z\in\mathbb{C}$. By $M_n(\mathbb{C}_s)$ we denote the right Banach $M_n(C(K))$-module $M_n(\mathbb{C})$ with outer action defined by $(x\cdot a)_{i,j}=\sum_{k=1}^n x_{i,k}a_{k,j}(s)$ for $a\in M_n(C(K))$ and $x\in M_n(\mathbb{C}_s)$. The $C^*$-algebra $M_n(C(K))$ is nuclear [\cite{BroOzaCStarAlgFinDimApprox}, corollary 2.4.4], then by [\cite{HaaNucCStarAlgAmen}, theorem 3.1] this algebra is relatively amenable and even $1$-amenable [\cite{RundeAmenConstFour}, example 2]. Since $M_n(\mathbb{C}_s)$ is finite dimensional, it is an $\mathscr{L}_{1, C}^g$-space for some constant $C\geq 1$ independent of $s$. Thus, by proposition \ref{MetTopEssL1FlatModAoverAmenBanAlg} the $M_n(C(K))$-module $M_n(\mathbb{C}_s)^*$ is $C$-topologically flat. Since the latter module is essential, by proposition \ref{CTopFlatCharac} the right $M_n(C(K))$-module $M_n(\mathbb{C}_s)^{**}$ is $C$-topologically injective. Note that $M_n(\mathbb{C}_s)^{**}$ is isometrically isomorphic to $M_n(\mathbb{C}_s)$ as right $M_n(C(K))$-module, so from proposition \ref{MetTopInjModProd} we get that $\bigoplus_\infty\{M_n(\mathbb{C}_s):s\in K\}$ is topologically injective as $M_n(C(K))$-module.

Note that by proposition \ref{MetInjCStarAlgCharac} the $C(K)$-module $C(K)$ is metrically injective, therefore an isometric $C(K)$-morphism $\widetilde{\rho}:C(K)\to\bigoplus_\infty\{ \mathbb{C}_s:s\in K\}:x\mapsto \bigoplus_\infty\{x(s):s\in K\}$ admits a left inverse contractive $C(K)$-morphism $\widetilde{\tau}:\bigoplus_\infty\{ \mathbb{C}_s:s\in K\} \to C(K)$. It is routine to check now that linear operators
$$
\rho:M_n(C(K))\to\bigoplus\nolimits_\infty\{M_n(\mathbb{C}_s):s\in K\}:x\mapsto \bigoplus\nolimits_\infty\{(x_{i,j}(s))_{i,j\in\mathbb{N}_n}:s\in K\}
$$
$$
\tau:\bigoplus\nolimits_\infty\{M_n(\mathbb{C}_s):s\in K\}\to M_n(C(K)):y\mapsto \left(\widetilde{\tau}\left(\bigoplus\nolimits_\infty\{y_{s,i,j}:s\in K\}\right)\right)_{i,j\in\mathbb{N}_n}
$$
are bounded $M_n(C(K))$-morphisms such that $\tau \rho=1_{M_n(C(K))}$. That is $M_n(C(K))$ is a retract of topologically injective $M_n(C(K))$-module $\bigoplus_\infty\{M_n(\mathbb{C}_s):s\in K\}$ in $\mathbf{mod}_1-M_n(C(K))$. Finally, from proposition \ref{RetrMetTopInjIsMetTopInj} we conclude that $M_n(C(K))$ is topologically injective $M_n(C(K))$-module.
\end{proof}

\begin{theorem}\label{TopInjAWStarAlgCharac} Let $A$ be a $C^*$-algebra. Then the following are equivalent:

$i)$ $A$ is an $AW^*$-algebra which is topologically injective as $A$-module;

$ii)$ $A=\bigoplus_\infty\{M_{n_\lambda}(C(K_\lambda)):\lambda\in\Lambda\}$ for some finite families of natural numbers $(n_\lambda)_{\lambda\in\Lambda}$ and Stonean spaces $(K_\lambda)_{\lambda\in\Lambda}$.
\end{theorem}
\begin{proof}$i)\implies ii)$ From proposition 6.6 in \cite{SmithDecompPropCStarAlg} we know that an $AW^*$-algebra is either isomorphic as $C^*$-algebra to $\bigoplus_\infty\{M_{n_\lambda}(C(K_\lambda)):\lambda\in\Lambda\}$ for some finite families of natural numbers $(n_\lambda)_{\lambda\in\Lambda}$ and Stonean spaces $(K_\lambda)_{\lambda\in\Lambda}$ or contains a ${}^*$-subalgebra $\bigoplus_\infty\{ \mathcal{B}(\ell_2(\mathbb{N}_n)):n\in\mathbb{N}\}$. The second option is canceled out by proposition \ref{TopInjIdHaveLUST}.

$ii)\implies i)$ For each $\lambda\in\Lambda$ the algebra $M_{n_\lambda}(C(K_\lambda))$ is unital because $K_\lambda$ is compact. Therefore $M_{n_\lambda}(C(K_\lambda))$ is faithful as $M_{n_\lambda}(C(K_\lambda))$-module. It is also topologically injective as $M_{n_\lambda}(C(K_\lambda))$-module by proposition  \ref{CKMatrixModTopInj}. Now the topological injectivity of $A$ as $A$-module follows from paragraph $ii)$ of proposition \ref{MetTopProjInjFlatUnderSumOfAlg} with $p=\infty$ and $X_\lambda=A_\lambda=M_{n_\lambda}(C(K_\lambda))$ for all $\lambda\in\Lambda$. 

For all $\lambda\in\Lambda$ the algebra $C(K_\lambda)$ is an $AW^*$-algebra, because $K_\lambda$ is a Stonean space [\cite{BerbBaerStarRings}, theorem 1.7.1]. Therefore $M_{n_\lambda}(C(K_\lambda))$ is an $AW^*$-algebra too [\cite{BerbBaerStarRings}, corollary 9.62.1]. Finally $A$ is an $AW^*$-algebra as $\bigoplus_\infty$-sum of such algebras [\cite{BerbBaerStarRings}, proposition 1.10.1].
\end{proof}

It is desirable to prove that any topologically injective over itself $C^*$-algebra is an $AW^*$-algebra, but it seems to be a challenge even in the commutative case.

%----------------------------------------------------------------------------------------
%	Flat ideals of C^*-algebras
%----------------------------------------------------------------------------------------

\subsection{Flat ideals of \texorpdfstring{$C^*$}{C*}-algebras}
\label{SubSectionFlatIdealsOfCStarAlgebras}

By considering flatness we finalize this lengthy investigations of ideals of $C^*$-algebras.

\begin{proposition}\label{IdealofCstarAlgisMetTopFlat} Let $I$ be a left ideal of a $C^*$-algebra $A$. Then $I$ is metrically and topologically flat as $A$-module.
\end{proposition}
\begin{proof} From [\cite{HelBanLocConvAlg}, proposition 4.7.78] it follows that $I$ has a right contractive identity. It remains to apply proposition \ref{MetTopFlatIdealsInUnitalAlg}.
\end{proof}

\begin{proposition}\label{CStarAlgIsL1IfFinDim} Let $A$ be a $C^*$-algebra, then $A$ is an $\langle$~$L_1$-space / $\mathscr{L}_1^g$-space~$\rangle$ iff $\langle$~$\operatorname{dim}(A)\leq 1$ / $A$ is finite dimensional~$\rangle$.
\end{proposition}
\begin{proof} Assume $A$ is an $\mathscr{L}_1^g$-space, then $A^{**}$ is complemented in some $L_1$-space [\cite{DefFloTensNorOpId}, corollary 23.2.1(2)]. Since $A$ isometrically embeds in its second dual via $\iota_{A}$ we may regard $A$ as closed subspace of some $L_1$-space. The latter space is weakly sequentially complete [\cite{WojBanSpForAnalysts}, corollary III.C.14]. The property of being weakly sequentially complete is preserved by closed subspaces, therefore $A$ is weakly sequentially complete too. By proposition 2 in \cite{SakWeakCompOpOnOpAlg} every weakly sequentially complete $C^*$-algebra is finite dimensional, hence $A$ is finite dimensional. Conversely, if $A$ is finite dimensional it is an $\mathscr{L}_1^g$-space as any finite dimensional Banach space.

Assume $A$ is an $L_1$-space and, a fortiori, an $\mathscr{L}_1^g$-space. As was noted above $A$ is a finite dimensional, so $A\isom{\mathbf{Ban}_1}\ell_1(\mathbb{N}_n)$ for $n=\operatorname{dim}(A)$. On the other hand, $A$ is a finite dimensional $C^*$-algebra, so it is isometrically isomorphic to $\bigoplus_\infty\{ \mathcal{B}(\ell_2(\mathbb{N}_{n_k})):k\in\mathbb{N}_m\}$ for some natural numbers $(n_k)_{k\in\mathbb{N}_m}$ [\cite{DavCSatrAlgByExmpl}, theorem III.1.1]. Assume $\operatorname{dim}(A)>1$, then $A$ contains an isometric copy of $\ell_\infty(\mathbb{N}_2)$. Therefore we have an isometric embedding of $\ell_\infty(\mathbb{N}_2)$ into $\ell_1(\mathbb{N}_n)$. This is impossible by theorem 1 from \cite{LyubIsomEmdbFinDimLp}. Therefore $\operatorname{dim}(A)\leq 1$. 
\end{proof}

\begin{proposition}\label{CStarAlgIsTopFlatOverItsIdeal} Let $I$ be a proper two-sided ideal of a  $C^*$-algebra $A$. Then the following are equivalent:

$i)$ $A$ is $\langle$~metrically / topologically~$\rangle$ flat $I$-module;

$ii)$ $\langle$~$\operatorname{dim}(A)=1$, $I=\{0\}$ / $A/I$ is finite dimensional ~$\rangle$.

\end{proposition}
\begin{proof} We may regard $I$ as an  ideal of unitazation $A_\#$ of $A$. Since $I$ is a two-sided ideal, then it has a contractive approximate identity $(e_\nu)_{\nu\in N}$ such that $0\leq e_\nu\leq e_{A_\#}$ [\cite{HelBanLocConvAlg}, proposition 4.7.79]. As a corollary $\sup_{\nu\in N}\Vert e_{A_\#}-e_\nu\Vert\leq 1$. Since $I$ has an approximate identity we also have $A_{ess}:=\operatorname{cl}_A(IA)=I$. Since $I$ is a two sided ideal then $A/I$ is a $C^*$-algebra [\cite{HelBanLocConvAlg}, theorem 4.7.81].

Assume, $A$ is a metrically flat $I$-module. Since $\sup_{\nu\in N}\Vert e_{A_\#}-e_\nu\Vert\leq 1$, then paragraph $ii)$ of proposition \ref{DualBanModDecomp} tells us that $(A/A_{ess})^*=(A/I)^*$ is a retract of $A^*$ in $\mathbf{mod}_1-I$. Now from propositions \ref{MetTopFlatCharac} and \ref{RetrMetTopInjIsMetTopInj} it follows that $A/I$ is metrically flat $I$-module. Since this is an annihilator module, then from proposition \ref{MetTopFlatAnnihModCharac} it follows that $I=\{0\}$ and $A/I$ is an $L_1$-space. Now from proposition \ref{CStarAlgIsL1IfFinDim} we get that $\operatorname{dim}(A/I)\leq 1$. Since $A$ contains a proper ideal $I=\{0\}$, then $\operatorname{dim}(A)=1$. Conversely, if $I=\{0\}$ and $\operatorname{dim}(A)=1$, then we have an annihilator $I$-module $A$ which is isometrically isomorphic to $\ell_1(\mathbb{N}_1)$. By proposition \ref{MetTopFlatAnnihModCharac} it is metrically flat. 

By proposition \ref{TopFlatModCharac} the $I$-module $A$ is topologically flat iff $A_{ess}=I$ and $A/A_{ess}=A/I$ are topologically flat $I$-modules. By proposition \ref{IdealofCstarAlgisMetTopFlat} the ideal $I$ is topologically flat $I$-module. By proposition \ref{MetTopFlatAnnihModCharac} the annihilator $I$-module $A/I$ is topologically flat iff it is an $\mathscr{L}_1^g$-space. By proposition \ref{CStarAlgIsL1IfFinDim} this is equivalent to $A/I$ being finite dimensional.
\end{proof}

%----------------------------------------------------------------------------------------
%	\mathcal{K}(H)- and \mathcal{B}(H)-modules
%----------------------------------------------------------------------------------------

\subsection{\texorpdfstring{$\mathcal{K}(H)$}{K(H)}- and \texorpdfstring{$\mathcal{B}(H)$}{B(H)}-modules}
\label{SubSectionKHAndBHModules}

In this section we apply general results on ideals obtained above to classical modules over $C^*$-algebras.  For a given Hilbert space $H$ we consider $\mathcal{B}(H)$, $\mathcal{K}(H)$ and $\mathcal{N}(H)$ as left and right Banach modules over $\mathcal{B}(H)$ and $\mathcal{K}(H)$. For all modules the module action is just the composition of operators. The Schatten-von Neumann isomorphisms $\mathcal{N}(H)\isom{\mathbf{Ban}_1}\mathcal{K}(H)^*$, $\mathcal{B}(H)\isom{\mathbf{Ban}_1}\mathcal{N}(H)^*$ (see [\cite{TakThOpAlgVol1}, theorems II.1.6, II.1.8]) will be of use here. They are in fact isomorphisms of left and right $\mathcal{B}(H)$-modules and a fortiori of $\mathcal{K}(H)$-modules.

\begin{proposition}\label{KHAndBHModBH} Let $H$ be a Hilbert space. Then

$i)$ $\mathcal{B}(H)$ is metrically and topologically projective and flat as $\mathcal{B}(H)$-module;

$ii)$ $\mathcal{B}(H)$ is metrically or topologically projective or flat as $\mathcal{K}(H)$-module iff $H$ is finite dimensional;

$iii)$ $\mathcal{B}(H)$ is topologically injective as $\mathcal{B}(H)$- or $\mathcal{K}(H)$-module iff $H$ is finite dimensional;

$iv)$ $\mathcal{B}(H)$ is metrically injective as $\mathcal{B}(H)$- or $\mathcal{K}(H)$-module iff $\dim(H)\leq 1$.
\end{proposition}
\begin{proof} $i)$ Since $\mathcal{B}(H)$ is a unital algebra it is metrically and topologically projective as $\mathcal{B}(H)$-module by proposition \ref{UnitalAlgIsMetTopProj}. Both results regarding flatness follow from proposition \ref{MetTopProjIsMetTopFlat}.

$ii)$ For infinite dimensional $H$ the Banach space $\mathcal{B}(H)/\mathcal{K}(H)$ is of infinite dimension, so by proposition \ref{CStarAlgIsTopFlatOverItsIdeal} the module $\mathcal{B}(H)$ neither topologically nor metrically flat as $\mathcal{K}(H)$-module. Both claims regarding projectivity follow from proposition \ref{MetTopProjIsMetTopFlat}. If $H$ is finite dimensional, then $\mathcal{K}(H)=\mathcal{B}(H)$, so the result follows from paragraph $i)$.

$iii)$ If $H$ is infinite dimensional, then $\mathcal{B}(H)$ contains $\mathcal{B}(\ell_2(\mathbb{N}_n))$ as ${}^*$-subalgebra for all $n\in\mathbb{N}$. By proposition \ref{TopInjIdHaveLUST} we get that $\mathcal{B}(H)$ is not topologically injective as $\mathcal{B}(H)$-module. The rest follows from paragraph $i)$ of proposition \ref{MetTopInjUnderChangeOfAlg}. If $H$ is finite dimensional, then $\mathcal{K}(H)=\mathcal{B}(H)$, so the result follows from proposition \ref{FinDimBHModTopInj}.

$iv)$ If $\dim(H)>1$, then $C^*$-algebra $\mathcal{B}(H)$ is not commutative. By  proposition \ref{MetInjCStarAlgCharac} we get that it is not metrically injective as $\mathcal{B}(H)$-module. Now from paragraph $i)$ of \ref{MetTopInjUnderChangeOfAlg} we get that $\mathcal{B}(H)$ is not metrically injective as $\mathcal{K}(H)$-module. If $\dim(H)\leq 1$ both claims obviously follow from \ref{MetInjCStarAlgCharac}.
\end{proof}

\begin{proposition}\label{KHAndBHModKH} Let $H$ be a Hilbert space. Then 

$i)$ $\mathcal{K}(H)$ is metrically and topologically flat as $\mathcal{B}(H)$- or $\mathcal{K}(H)$-module;

$ii)$ $\mathcal{K}(H)$ is metrically or topologically projective as $\mathcal{B}(H)$- or $\mathcal{K}(H)$-module iff $H$ is finite dimensional;

$iii)$ $\mathcal{K}(H)$ is topologically injective as $\mathcal{B}(H)$- or $\mathcal{K}(H)$-module iff $H$ is finite dimensional;

$iv)$ $\mathcal{K}(H)$ is metrically injective as $\mathcal{B}(H)$- or $\mathcal{K}(H)$-module iff $\dim(H)\leq 1$.
\end{proposition}
\begin{proof} Let $A$ be either $\mathcal{B}(H)$ or $\mathcal{K}(H)$. Note that $\mathcal{K}(H)$ is a two-sided ideal of $A$. 

$i)$ Recall that $\mathcal{K}(H)$ has a contractive approximate identity consisting of orthogonal projections onto all finite-dimensional subspaces of $H$. Since $\mathcal{K}(H)$ is a two-sided ideal of $A$, then the result follows from proposition \ref{IdealofCstarAlgisMetTopFlat}.

$ii)$, $iii)$, $iv)$ If $H$ is infinite dimensional, then $\mathcal{K}(H)$ is not unital as Banach algebra. From corollary \ref{BiIdealOfCStarAlgMetTopProjCharac} and proposition \ref{MetTopInjOfId} the $A$-module $\mathcal{K}(H)$ is neither metrically nor topologically projective or injective. If $H$ is finite dimensional, then $\mathcal{K}(H)=\mathcal{B}(H)$, so both results follow from paragraphs $i)$, $iii)$ and $iv)$ of proposition \ref{KHAndBHModBH}.
\end{proof}

\begin{proposition}\label{KHAndBHModNH} Let $H$ be a Hilbert space. Then

$i)$ $\mathcal{N}(H)$ is metrically and topologically injective as $\mathcal{B}(H)$- or $\mathcal{K}(H)$-module;

$ii)$ $\mathcal{N}(H)$ is topologically projective or flat as $\mathcal{B}(H)$- or $\mathcal{K}(H)$-module iff $H$ is finite dimensional;

$iii)$ $\mathcal{N}(H)$ is metrically projective or flat as $\mathcal{B}(H)$- or $\mathcal{K}(H)$-module iff $\dim(H)\leq 1$.
\end{proposition}
\begin{proof} Let $A$ be either $\mathcal{B}(H)$ or $\mathcal{K}(H)$.

$i)$ Note that $\mathcal{N}(H)\isom{\mathbf{mod}_1-A}\mathcal{K}(H)^*$, so the result follows from proposition \ref{MetTopFlatCharac} and paragraph $i)$ of proposition \ref{KHAndBHModKH}.

$ii)$ Assume $H$ is infinite dimensional. Note that $\mathcal{B}(H)\isom{\mathbf{mod}_1-A}\mathcal{N}(H)^*$, so from proposition \ref{DualMetTopProjIsMetrInj} and paragraph $iii)$ of proposition \ref{KHAndBHModBH} we get that $\mathcal{N}(H)$ is not topologically projective as $A$-module. Both results regarding flatness follow from proposition \ref{MetTopProjIsMetTopFlat}. If $H$ is finite dimensional we use proposition \ref{FinDimNHModTopProjFlat}.

$iii)$ Assume $\dim(H)>1$, then by paragraph $iv)$ of proposition \ref{KHAndBHModBH} the $A$-module $\mathcal{B}(H)$ is not metrically injective. Since $\mathcal{B}(H)\isom{\mathbf{mod}_1-A}\mathcal{N}(H)^*$, then from proposition \ref{MetTopFlatCharac} we get that $\mathcal{N}(H)$ is not metrically flat as $A$-module. By proposition \ref{MetTopProjIsMetTopFlat}, it is not metrically projective as $A$-module. If $\dim(H)\leq 1$, then $\mathcal{N}(H)=\mathcal{K}(H)=\mathcal{B}(H)$, so both results follow from paragraph $i)$ of proposition \ref{KHAndBHModBH}.
\end{proof}

\begin{proposition}\label{KHAndBHModsRelTh} Let $H$ be a Hilbert space. Then

$i)$ as $\mathcal{K}(H)$-modules $\mathcal{N}(H)$ is relatively projective injective and flat, $\mathcal{K}(H)$ is relatively projective and flat, but relatively injective only for finite dimensional $H$, $\mathcal{B}(H)$ is relatively injective and flat, but relatively projective only for finite dimensional $H$;

$ii)$ as $\mathcal{B}(H)$-modules $\mathcal{N}(H)$ is relatively projective injective and flat, $\mathcal{K}(H)$ is relatively projective and flat, $\mathcal{B}(H)$ is relatively projective, injective and flat.

\end{proposition}
\begin{proof} $i)$ Note that $H$ is relatively projective as $\mathcal{K}(H)$-module [\cite{HelBanLocConvAlg}, theorem 7.1.27], so from proposition 7.1.13 in \cite{HelBanLocConvAlg} we get that $\mathcal{N}(H)\isom{\mathcal{K}(H)-\mathbf{mod}_1}H\projtens H^*$ is also relatively projective as $\mathcal{K}(H)$-module. By theorem IV.2.16 in \cite{HelHomolBanTopAlg} the $\mathcal{K}(H)$-module $\mathcal{K}(H)$ is relatively projective. A fortiori $\mathcal{N}(H)$ and $\mathcal{K}(H)$ are relatively flat $\mathcal{K}(H)$-modules [\cite{HelBanLocConvAlg}, proposition 7.1.40], so $\mathcal{N}(H)\isom{\mathbf{mod}_1-\mathcal{K}(H)}\mathcal{K}(H)^*$ and  $\mathcal{B}(H)\isom{\mathbf{mod}_1-\mathcal{K}(H)}\mathcal{N}(H)^*$ are relatively injective $\mathcal{K}(H)$-modules. From [\cite{RamsHomPropSemgroupAlg}, proposition 2.2.8  (i)] we know that a Banach algebra relatively injective over itself as a right module, necessarily has a left identity. Therefore $\mathcal{K}(H)$ is not relatively injective $\mathcal{K}(H)$-module for infinite dimensional $H$. If $H$ is finite dimensional, then $\mathcal{K}(H)$-module $\mathcal{K}(H)$ is relatively injective because $\mathcal{K}(H)=\mathcal{B}(H)$ and $\mathcal{B}(H)$ is relatively injective $\mathcal{K}(H)$-module as was shown above. By corollary 5.5.64 from \cite{DalBanAlgAutCont} the algebra $\mathcal{K}(H)$ is relatively amenable, so all its left modules are relatively flat [\cite{HelBanLocConvAlg}, theorem 7.1.60]. In particular $\mathcal{B}(H)$ is relatively flat $\mathcal{K}(H)$-module. From [\cite{HelHomolBanTopAlg}, exercise V.2.20] we know that $\mathcal{B}(H)$ is not relatively projective as $\mathcal{K}(H)$-module when $H$ is infinite dimensional. If $H$ is finite dimensional then $\mathcal{B}(H)$ is relatively projective $\mathcal{K}(H)$-module because $\mathcal{B}(H)=\mathcal{K}(H)$ and $\mathcal{K}(H)$ is relatively projective $\mathcal{K}(H)$-module as was shown above.

$ii)$ From proposition \ref{KHAndBHModBH} paragraph $i)$ and proposition \ref{MetProjIsTopProjAndTopProjIsRelProj} it follows that $\mathcal{B}(H)$ is relatively projective $\mathcal{B}(H)$-module. From [\cite{RamsHomPropSemgroupAlg}, propositions 2.3.3, 2.3.4] we know that $\langle$~an essential relatively projective / a faithful relatively injective~$\rangle$ module over ideal of Banach algebra is $\langle$~relatively projective / relative injective~$\rangle$ over algebra itself. Since $\mathcal{K}(H)$ and $\mathcal{N}(H)$ are essential and faithful $\mathcal{K}(H)$-modules, then from results of previous paragraph $\mathcal{N}(H)$ is relatively projective and injective, while $\mathcal{K}(H)$ is relatively projective as $\mathcal{B}(H)$-modules. Now, by [\cite{HelBanLocConvAlg}, proposition 7.1.40] all aforementioned modules are relatively flat $\mathcal{B}(H)$-modules. In particular $\mathcal{B}(H)\isom{\mathbf{mod}_1-\mathcal{B}(H)}\mathcal{N}(H)^*$ is relatively injective $\mathcal{B}(H)$-module.
\end{proof}

Results of this section are summarized in the following three tables. Each cell contains a condition under which the respective module has the respective property and propositions where this is proved. We use ??? symbol to indicate open problems. These tables confirm that the property of being homologically trivial in metric and topological theory is too restrictive. It is easier to mention cases where metric and topological properties coincide with relative ones: flatness of $\mathcal{K}(H)$ as $\mathcal{B}(H)$- or $\mathcal{K}(H)$-module, injectivity of $\mathcal{N}(H)$ as $\mathcal{B}(H)$- or $\mathcal{K}(H)$-module, projectivity and flatness of $\mathcal{B}(H)$-module $\mathcal{B}(H)$. In the remaining cases $H$ needs to be at least finite dimensional in order to these properties be equivalent in metric, topological and relative theory.


\begin{scriptsize}
\begin{longtable}{|c|c|c|c|c|c|c|} 
\multicolumn{7}{c}{\mbox{Homologically trivial $\mathcal{K}(H)$- and $\mathcal{B}(H)$-modules in metric theory}}                                                                                                                                                                                                                                                                                                                                                                                                                                                    \\
				 
\hline          & \multicolumn{3}{c|}{$\mathcal{K}(H)$-modules}                                                                                                                                                                                                                     & \multicolumn{3}{c|}{$\mathcal{B}(H)$-modules}                                                                                                                                                                                                                       \\
\hline
                & \mbox{Projectivity}                                                                   & \mbox{Injectivity}                                                                    & \mbox{Flatness}                                                                        & \mbox{Projectivity}                                                                    & \mbox{Injectivity}                                                                     & \mbox{Flatness}                                                                        \\ 
\hline
$\mathcal{N}(H)$  & \begin{tabular}{@{}c@{}}$\dim(H)\leq 1$ \\ \ref{KHAndBHModNH}\end{tabular}            & \begin{tabular}{@{}c@{}}$H$\mbox{ is any }  \\ \ref{KHAndBHModNH}\end{tabular}        & \begin{tabular}{@{}c@{}}$\dim(H)\leq 1$ \\ \ref{KHAndBHModNH}\end{tabular}             & \begin{tabular}{@{}c@{}}$\dim(H)\leq 1$ \\ \ref{KHAndBHModNH}\end{tabular}             & \begin{tabular}{@{}c@{}}$H$\mbox{ is any }  \\ \ref{KHAndBHModNH}\end{tabular}         & \begin{tabular}{@{}c@{}}$\dim(H)\leq 1$ \\ \ref{KHAndBHModNH}\end{tabular}             \\
\hline
$\mathcal{B}(H)$  & \begin{tabular}{@{}c@{}}$\dim(H)<\aleph_0$ \\ \ref{KHAndBHModBH}\end{tabular}         & \begin{tabular}{@{}c@{}}$\dim(H)\leq 1$ \\ \ref{KHAndBHModBH}\end{tabular}            & \begin{tabular}{@{}c@{}}$\dim(H)<\aleph_0$ \\ \ref{KHAndBHModBH}\end{tabular}          & \begin{tabular}{@{}c@{}}$H$\mbox{ is any } \\ \ref{KHAndBHModBH}\end{tabular}          & \begin{tabular}{@{}c@{}}$\dim(H)\leq 1$ \\ \ref{KHAndBHModBH}\end{tabular}             & \begin{tabular}{@{}c@{}}$H$\mbox{ is any } \\ \ref{KHAndBHModBH}\end{tabular}          \\ 
\hline
$\mathcal{K}(H)$  & \begin{tabular}{@{}c@{}}$\dim(H)<\aleph_0$ \\ \ref{KHAndBHModKH}\end{tabular}         & \begin{tabular}{@{}c@{}}$\dim(H)\leq 1$ \\ \ref{KHAndBHModKH}\end{tabular}            & \begin{tabular}{@{}c@{}}$H$\mbox{ is any } \\ \ref{KHAndBHModKH}\end{tabular}          & \begin{tabular}{@{}c@{}}$\dim(H)<\aleph_0$ \\ \ref{KHAndBHModKH}\end{tabular}          & \begin{tabular}{@{}c@{}}$\dim(H)\leq 1$ \\ \ref{KHAndBHModKH}\end{tabular}             & \begin{tabular}{@{}c@{}}$H$\mbox{ is any } \\ \ref{KHAndBHModKH}\end{tabular}          \\ 
\hline

\multicolumn{7}{c}{\mbox{Homologically trivial $\mathcal{K}(H)$- and $\mathcal{B}(H)$-modules in topological theory}}                                                                                                                                                                                                                                                                                                                                                                                                                                               \\
					 
\hline          & \multicolumn{3}{c|}{$\mathcal{K}(H)$-modules}                                                                                                                                                                                                                     & \multicolumn{3}{c|}{$\mathcal{B}(H)$-modules}                                                                                                                                                                                                                       \\
\hline
                & \mbox{Projectivity}                                                                   & \mbox{Injectivity}                                                                    & \mbox{Flatness}                                                                        & \mbox{Projectivity}                                                                    & \mbox{Injectivity}                                                                     & \mbox{Flatness}                                                                        \\ 
\hline
$\mathcal{N}(H)$  & \begin{tabular}{@{}c@{}}$\dim(H)<\aleph_0$ \\ \ref{KHAndBHModNH}\end{tabular}         & \begin{tabular}{@{}c@{}}$H$\mbox{ is any } \\ \ref{KHAndBHModNH}\end{tabular}         & \begin{tabular}{@{}c@{}}$\dim(H)<\aleph_0$ \\ \ref{KHAndBHModNH}\end{tabular}          & \begin{tabular}{@{}c@{}}$\dim(H)<\aleph_0$ \\ \ref{KHAndBHModNH}\end{tabular}          & \begin{tabular}{@{}c@{}}$H$\mbox{ is any } \\ \ref{KHAndBHModNH}\end{tabular}          & \begin{tabular}{@{}c@{}}$\dim(H)<\aleph_0$ \\ \ref{KHAndBHModNH}\end{tabular}          \\
\hline
$\mathcal{B}(H)$  & \begin{tabular}{@{}c@{}}$\dim(H)<\aleph_0$ \\ \ref{KHAndBHModBH}\end{tabular}         & \begin{tabular}{@{}c@{}}$\dim(H)<\aleph_0$ \\ \ref{KHAndBHModBH}\end{tabular}         & \begin{tabular}{@{}c@{}}$\dim(H)<\aleph_0$ \\ \ref{KHAndBHModBH}\end{tabular}          & \begin{tabular}{@{}c@{}}$H$\mbox{ is any } \\ \ref{KHAndBHModBH}\end{tabular}          & \begin{tabular}{@{}c@{}}$\dim(H)<\aleph_0$ \\ \ref{KHAndBHModBH}\end{tabular}          & \begin{tabular}{@{}c@{}}$H$\mbox{ is any } \\ \ref{KHAndBHModBH}\end{tabular}          \\ 
\hline
$\mathcal{K}(H)$  & \begin{tabular}{@{}c@{}}$\dim(H)<\aleph_0$ \\ \ref{KHAndBHModKH}\end{tabular}         & \begin{tabular}{@{}c@{}}$\dim(H)<\aleph_0$ \\ \ref{KHAndBHModKH}\end{tabular}         & \begin{tabular}{@{}c@{}}$H$\mbox{ is any } \\ \ref{KHAndBHModKH}\end{tabular}          & \begin{tabular}{@{}c@{}}$\dim(H)<\aleph_0$ \\ \ref{KHAndBHModKH}\end{tabular}          & \begin{tabular}{@{}c@{}}$\dim(H)<\aleph_0$ \\ \ref{KHAndBHModKH}\end{tabular}          & \begin{tabular}{@{}c@{}}$H$\mbox{ is any } \\ \ref{KHAndBHModKH}\end{tabular}          \\ 
\hline

\multicolumn{7}{c}{\mbox{Homologically trivial $\mathcal{K}(H)$- and $\mathcal{B}(H)$-modules in relative theory}}                                                                                                                                                                                                                                                                                                                                                                                                                                                  \\

\hline          & \multicolumn{3}{c|}{$\mathcal{K}(H)$-modules}                                                                                                                                                                                                                     & \multicolumn{3}{c|}{$\mathcal{B}(H)$-modules}                                                                                                                                                                                                                       \\
\hline
                & \mbox{Projectivity}                                                                   & \mbox{Injectivity}                                                                    & \mbox{Flatness}                                                                        & \mbox{Projectivity}                                                                    & \mbox{Injectivity}                                                                     & \mbox{Flatness}                                                                        \\ 
\hline
$\mathcal{N}(H)$  & \begin{tabular}{@{}c@{}}$H$\mbox{ is any } \\ \ref{KHAndBHModsRelTh}, i)\end{tabular} & \begin{tabular}{@{}c@{}}$H$\mbox{ is any } \\ \ref{KHAndBHModsRelTh}, i)\end{tabular} & \begin{tabular}{@{}c@{}}$H$\mbox{ is any } \\ \ref{KHAndBHModsRelTh}, i)\end{tabular}  & \begin{tabular}{@{}c@{}}$H$\mbox{ is any } \\ \ref{KHAndBHModsRelTh}, ii)\end{tabular} & \begin{tabular}{@{}c@{}}$H$\mbox{ is any } \\ \ref{KHAndBHModsRelTh}, ii)\end{tabular} & \begin{tabular}{@{}c@{}}$H$\mbox{ is any } \\ \ref{KHAndBHModsRelTh}, ii)\end{tabular} \\
\hline
$\mathcal{B}(H)$  & \begin{tabular}{@{}c@{}}$\dim(H)<\aleph_0$ \\ \ref{KHAndBHModsRelTh}, i)\end{tabular} & \begin{tabular}{@{}c@{}}$H$\mbox{ is any } \\ \ref{KHAndBHModsRelTh}, i)\end{tabular} & \begin{tabular}{@{}c@{}}$H$\mbox{ is any } \\ \ref{KHAndBHModsRelTh}, i)\end{tabular}  & \begin{tabular}{@{}c@{}}$H$\mbox{ is any } \\ \ref{KHAndBHModsRelTh}, ii)\end{tabular} & \begin{tabular}{@{}c@{}}$H$\mbox{ is any } \\ \ref{KHAndBHModsRelTh}, ii)\end{tabular} & \begin{tabular}{@{}c@{}}$H$\mbox{ is any } \\ \ref{KHAndBHModsRelTh}, ii)\end{tabular} \\
\hline
$\mathcal{K}(H)$  & \begin{tabular}{@{}c@{}}$H$\mbox{ is any } \\ \ref{KHAndBHModsRelTh}, i)\end{tabular} & \begin{tabular}{@{}c@{}}$\dim(H)<\aleph_0$ \\ \ref{KHAndBHModsRelTh}, i)\end{tabular} & \begin{tabular}{@{}c@{}}$H$\mbox{ is any } \\ \ref{KHAndBHModsRelTh}, i)\end{tabular}  & \begin{tabular}{@{}c@{}}$H$\mbox{ is any } \\ \ref{KHAndBHModsRelTh}, ii)\end{tabular} & \begin{tabular}{@{}c@{}} ??? \end{tabular}                                             & \begin{tabular}{@{}c@{}}$H$\mbox{ is any } \\ \ref{KHAndBHModsRelTh}, ii)\end{tabular} \\
\hline
\end{longtable}
\end{scriptsize}

%----------------------------------------------------------------------------------------
%	c_0(\Lambda)- and l_infty(\Lambda)-modules
%----------------------------------------------------------------------------------------

\subsection{\texorpdfstring{$c_0(\Lambda)$}{c0(Lambda)}- and \texorpdfstring{$\ell_\infty(\Lambda)$}{lInfty(Lambda)}-modules}
\label{SubSectionc0AndlInftyModules}

We continue our study of modules over $C^*$-algebras and move to commutative examples. For a given index set $\Lambda$ we consider spaces $c_0(\Lambda)$ and $\ell_p(\Lambda)$ for $1\leq p\leq+\infty$ as left and right modules over algebras $c_0(\Lambda)$ and $\ell_\infty(\Lambda)$. For all these modules the module action is just the pointwise multiplication. It is well known that $c_0(\Lambda)^*\isom{\mathbf{Ban}_1}\ell_1(\Lambda)$ and  $\ell_p(\Lambda)^*\isom{\mathbf{Ban}_1}\ell_{p^*}(\Lambda)$ for $1\leq p<+\infty$. In fact these isomorphisms are isomorphisms of $\ell_\infty(\Lambda)$- and $c_0(\Lambda)$-modules. 

For a given $\lambda\in\Lambda$ we define $\mathbb{C}_\lambda$ as left or right $\ell_\infty(\Lambda)$- or $c_0(\Lambda)$-module $\mathbb{C}$ with module action defined by
$$
a\cdot_\lambda z=a(\lambda)z,\qquad z\cdot_\lambda a=a(\lambda) z.
$$
for $a\in \ell_\infty(\Lambda)$ and $z\in\mathbb{C}_s$. 

\begin{proposition}\label{OneDimlInftyc0ModMetTopProjIngFlat} Let $\Lambda$ be a set and $\lambda\in\Lambda$. Then $\mathbb{C}_\lambda$ is metrically and topologically projective injective and flat $\ell_\infty(\Lambda)$- or $c_0(\Lambda)$-module.
\end{proposition}
\begin{proof} Let $A$ be either $\ell_\infty(\Lambda)$ or $c_0(\Lambda)$. One can easily check that the maps $\pi:A_+\to\mathbb{C}_\lambda:a\oplus_1 z\mapsto a(\lambda)+z$ and $\sigma:\mathbb{C}_\lambda\to A_+:z\mapsto z\delta_\lambda\oplus_1 0$ are contractive $A$-morphisms of left $A$-modules. Since $\pi\sigma=1_{\mathbb{C}_\lambda}$, then $\mathbb{C}_\lambda$ is retract of $A_+$ in $A-\mathbf{mod}_1$. From proposition \ref{UnitalAlgIsMetTopProj} and \ref{RetrMetTopProjIsMetTopProj} it follows that $\mathbb{C}_\lambda$ is metrically and topologically projective left $A$-module and a fortiori metrically and topologically flat by proposition \ref{MetTopProjIsMetTopFlat}. By proposition \ref{DualMetTopProjIsMetrInj} we have that $\mathbb{C}_\lambda^*$ is metrically and topologically injective as $A$-module. Now metric and topological injectivity of $\mathbb{C}_\lambda$ follow from isomorphism $\mathbb{C}_\lambda\isom{\mathbf{mod}_1-A}\mathbb{C}_\lambda^*$.
\end{proof}

\begin{proposition}\label{c0AndlInftyModlIfty} Let $\Lambda$ be a set. Then

$i)$ $\ell_\infty(\Lambda)$ is metrically and topologically projective and flat as $\ell_\infty(\Lambda)$-module;

$ii)$ $\ell_\infty(\Lambda)$ is metrically or topologically projective or flat as $c_0(\Lambda)$-module iff $\Lambda$ is finite;

$iii)$ $\ell_\infty(\Lambda)$ is metrically and topologically injective as $\ell_\infty(\Lambda)$- and $c_0(\Lambda)$-module.
\end{proposition}
\begin{proof} $i)$ Since $\ell_\infty(\Lambda)$ is a unital algebra, then it is metrically and topologically projective as $\ell_\infty(\Lambda)$-module by proposition \ref{UnitalAlgIsMetTopProj}. Results on flatness follow from proposition \ref{MetTopProjIsMetTopFlat}.

$ii)$ For infinite $\Lambda$ the Banach space $\ell_\infty(\Lambda)/c_0(\Lambda)$ is of infinite dimension, so by proposition \ref{CStarAlgIsTopFlatOverItsIdeal} the module $\ell_\infty(\Lambda)$ neither topologically nor metrically flat as $c_0(\Lambda)$-module. Both claims regarding projectivity follow from proposition \ref{MetTopProjIsMetTopFlat}. If $\Lambda$ is finite, then $c_0(\Lambda)=\ell_\infty(\Lambda)$, so the result follows from paragraph $i)$.

$iii)$ Let $A$ be either $\ell_\infty(\Lambda)$ or $c_0(\Lambda)$. Note that $\ell_\infty(\Lambda)\isom{A-\mathbf{mod}_1}\bigoplus_\infty\{\mathbb{C}_\lambda:\lambda\in\Lambda\}$, then from propositions \ref{OneDimlInftyc0ModMetTopProjIngFlat} and \ref{MetTopInjModProd} it follows that $\ell_\infty(\Lambda)$ is metrically injective as $A$-module. Topological injectivity follows from proposition \ref{MetInjIsTopInjAndTopInjIsRelInj}.
\end{proof}

\begin{proposition}\label{c0AndlInftyModc0} Let $\Lambda$ be a set. Then 

$i)$ $c_0(\Lambda)$ is metrically and topologically flat as $\ell_\infty(\Lambda)$- or $c_0(\Lambda)$-module;

$ii)$ $c_0(\Lambda)$ is metrically or topologically projective as $\ell_\infty(\Lambda)$- or $c_0(\Lambda)$-module iff $\Lambda$ is finite;

$iii)$ $c_0(\Lambda)$ is metrically or topologically injective as $\ell_\infty(\Lambda)$- or $c_0(\Lambda)$-module iff $\Lambda$ is finite.
\end{proposition}
\begin{proof} Let $A$ be either $\ell_\infty(\Lambda)$ or $c_0(\Lambda)$. Note that $c_0(\Lambda)$ is a two-sided ideal of $A$. 

$i)$ Recall that $c_0(\Lambda)$ has a contractive approximate identity of the form $(\sum_{\lambda\in S}\delta_\lambda)_{S\in\mathcal{P}_0(\Lambda)}$. Since $c_0(\Lambda)$ is a two-sided ideal of $A$, then the result follows from proposition \ref{IdealofCstarAlgisMetTopFlat}.

$ii)$, $iii)$ If $\Lambda$ is infinite, then $c_0(\Lambda)$ is not unital as Banach algebra. From corollary \ref{BiIdealOfCStarAlgMetTopProjCharac} and proposition \ref{MetTopInjOfId} the $A$-module $c_0(\Lambda)$ is neither metrically nor topologically projective or injective. If $\Lambda$ is finite, then $c_0(\Lambda)=\ell_\infty(\Lambda)$, so both results follow from paragraphs $i)$ and $iii)$ of proposition \ref{c0AndlInftyModlIfty}.
\end{proof}

\begin{proposition}\label{c0AndlInftyModl1} Let $\Lambda$ be a set. Then

$i)$ $\ell_1(\Lambda)$ is metrically and topologically injective as $\ell_\infty(\Lambda)$- or $c_0(\Lambda)$-module;

$ii)$ $\ell_1(\Lambda)$ is metrically and topologically projective and flat as $\ell_\infty(\Lambda)$- or $c_0(\Lambda)$-module;
\end{proposition}
\begin{proof} Let $A$ be either $\ell_\infty(\Lambda)$ or $c_0(\Lambda)$.

$i)$ Note that $\ell_1(\Lambda)\isom{\mathbf{mod}_1-A}c_0(\Lambda)^*$, so the result follows from proposition \ref{MetTopFlatCharac} and paragraph $i)$ of proposition \ref{c0AndlInftyModc0}.

$ii)$ Note that $\ell_1(\Lambda)\isom{A-\mathbf{mod}_1}\bigoplus_1\{\mathbb{C}_\lambda:\lambda\in\Lambda\}$, then from propositions \ref{OneDimlInftyc0ModMetTopProjIngFlat} and \ref{MetTopProjModCoprod} it follows that $\ell_1(\Lambda)$ is metrically projective as $A$-module. Topological projectivity follows from proposition \ref{MetProjIsTopProjAndTopProjIsRelProj}. Metric and topological flatness follow from proposition \ref{MetTopProjIsMetTopFlat}.
\end{proof}

\begin{proposition}\label{c0AndlInftyModlp} Let $\Lambda$ be a set and $1<p<+\infty$. Then $\ell_p(\Lambda)$ is metrically or topologically projective, injective or flat as $\ell_\infty(\Lambda)$- or $c_0(\Lambda)$-module iff $\Lambda$ is finite.
\end{proposition}
\begin{proof} Let $A$ be either $\ell_\infty(\Lambda)$ or $c_0(\Lambda)$, then $A$ is an $\mathscr{L}_\infty^g$-space. Since $\ell_p(\Lambda)$ is reflexive for $1<p<+\infty$, then from corollary  \ref{NoInfDimRefMetTopProjInjFlatModOverMthscrL1OrLInfty} it follows that $\ell_p(\Lambda)$ is necessarily finite dimensional if it is metrically or topologically projective injective or flat. This is equivalent to $\Lambda$ being finite. If $\Lambda$ is finite then $\ell_p(\Lambda)\isom{A-\mathbf{mod}}\ell_1(\Lambda)$ and $\ell_p(\Lambda)\isom{\mathbf{mod}-A}\ell_1(\Lambda)$, so topological projectivity injectivity and flatness follow from proposition \ref{c0AndlInftyModl1}.
\end{proof}

\begin{proposition}\label{c0AndlInftyModsRelTh} Let $\Lambda$ be a set. Then

$i)$ as $c_0(\Lambda)$-modules $\ell_p(\Lambda)$ for $1\leq p<+\infty$ and $\mathbb{C}_\lambda$ for $\lambda\in\Lambda$ are relatively projective, injective and flat, $c_0(\Lambda)$ relatively projective and flat, but relatively injective only for finite $\Lambda$, $\ell_\infty(\Lambda)$ is relatively injective and flat, but relatively projective only for finite $\Lambda$;

$ii)$ as $\ell_\infty(\Lambda)$-modules $\ell_p(\Lambda)$ for $1\leq p\leq+\infty$ and $\mathbb{C}_\lambda$ for $\lambda\in\Lambda$ are relatively projective, injective and flat, $c_0(\Lambda)$ is relatively projective and flat.
\end{proposition}
\begin{proof} $i)$ The algebra $c_0(\Lambda)$ is relatively biprojective [\cite{HelHomolBanTopAlg}, theorem IV.5.26] and admits a contractive approximate identity, so by [\cite{HelBanLocConvAlg}, theorem 7.1.60] all essential $c_0(\Lambda)$-modules are projective. Thus $c_0(\Lambda)$ and $\ell_p(\Lambda)$ for $1\leq p<+\infty$ are relatively projective $c_0(\Lambda)$-modules. A fortiori they are relatively flat as $c_0(\Lambda)$-modules [\cite{HelBanLocConvAlg}, proposition 7.1.40]. By the same proposition $\ell_1(\Lambda)\isom{\mathbf{mod}_1-c_0(\Lambda)}c_0(\Lambda)^*$ and $\ell_{p^*}(\Lambda)\isom{\mathbf{mod}_1-c_0(\Lambda)}\ell_p(\Lambda)^*$ for $1\leq p<+\infty$ are relatively injective $c_0(\Lambda)$-modules. From [\cite{RamsHomPropSemgroupAlg}, proposition 2.2.8 (i)] we know that a Banach algebra relatively injective over itself as right module, necessarily has a left identity. Therefore $c_0(\Lambda)$ is not relatively injective $c_0(\Lambda)$-module for infinite $\Lambda$. If $\Lambda$ is finite, then $c_0(\Lambda)$-module $c_0(\Lambda)$ is relatively injective because $c_0(\Lambda)=\ell_\infty(\Lambda)$ and $\ell_\infty(\Lambda)$ is relatively injective $c_0(\Lambda)$-module as was shown above. From [\cite{HelHomolBanTopAlg}, corollary V.2.16(II)] we know that $\ell_\infty(\Lambda)$ is not relatively projective as $c_0(\Lambda)$-module provided $\Lambda$ is infinite. If $\Lambda$ is finite then $\ell_\infty(\Lambda)$ is relatively projective $c_0(\Lambda)$-module because $\ell_\infty(\Lambda)=c_0(\Lambda)$ and $c_0(\Lambda)$ is relatively projective $c_0(\Lambda)$-module as was shown above. Propositions \ref{OneDimlInftyc0ModMetTopProjIngFlat}, \ref{MetProjIsTopProjAndTopProjIsRelProj}, \ref{MetInjIsTopInjAndTopInjIsRelInj} and \ref{MetFlatIsTopFlatAndTopFlatIsRelFlat} give the result for modules $\mathbb{C}_\lambda$, where $\lambda\in\Lambda$.

$ii)$ From proposition \ref{c0AndlInftyModlIfty} paragraph $i)$ and proposition \ref{MetProjIsTopProjAndTopProjIsRelProj} it follows that $\ell_\infty(\Lambda)$ is relatively projective $\ell_\infty(\Lambda)$-module. From [\cite{RamsHomPropSemgroupAlg}, propositions 2.3.3, 2.3.4] we know that $\langle$~an essential relatively projective / a  faithful relatively injective~$\rangle$ module over ideal of Banach algebra is $\langle$~relatively projective / relatively injective~$\rangle$ over algebra itself. Since $c_0(\Lambda)$ and $\ell_p(\Lambda)$ for $1\leq p<+\infty$ are essential and faithful $c_0(\Lambda)$-modules then from results of previous paragraph $\ell_p(\Lambda)$ for $1\leq p<+\infty$ are relatively projective and injective as $\ell_\infty(\Lambda)$-modules. Also we get that $c_0(\Lambda)$ is relatively projective $\ell_\infty(\Lambda)$-module. Therefore all these $\ell_\infty(\Lambda)$-modules are relatively flat [\cite{HelBanLocConvAlg}, proposition 7.1.40]. As the consequence $\ell_\infty(\Lambda)\isom{\mathbf{mod}_1-\ell_\infty(\Lambda)}\ell_1(\Lambda)^*$ is relatively injective $\ell_\infty(\Lambda)$-module.  Propositions \ref{OneDimlInftyc0ModMetTopProjIngFlat}, \ref{MetProjIsTopProjAndTopProjIsRelProj}, \ref{MetInjIsTopInjAndTopInjIsRelInj} and \ref{MetFlatIsTopFlatAndTopFlatIsRelFlat} give the result for modules $\mathbb{C}_\lambda$, where $\lambda\in\Lambda$.
\end{proof}

Results of this section are summarized in the following three tables. Each cell contains a condition under which the respective module has the respective property and propositions where this is proved. We use ??? symbol to indicate open problems. For the case of $\ell_\infty(\Lambda)$- and $c_0(\Lambda)$-modules $\ell_p(\Lambda)$ for $1<p<+\infty$ we don't have a criterion of homological triviality in metric theory, just a necessary condition. We indicate this fact via symbol $\implies$. From these table one can easily see that for modules over commutative $C^*$-algebras, there is much more in common between relative, metric and topological  theory. For example $\ell_1(\Lambda)$ is projective injective and flat as $\ell_\infty(\Lambda)$- or $c_0(\Lambda)$-module in all three theories.


\begin{scriptsize}
\begin{longtable}{|c|c|c|c|c|c|c|} 
\multicolumn{7}{c}{\mbox{Homologically trivial $c_0(\Lambda)$- and $\ell_\infty(\Lambda)$-modules in metric theory}}                                                                                                                                                                                                                                                                                                                                                                                                                                                                                                                                                                                                                                     \\
				 
\hline                 & \multicolumn{3}{c|}{$c_0(\Lambda)$-modules}                                                                                                                                                                                                                                                                                                                     & \multicolumn{3}{c|}{$\ell_\infty(\Lambda)$-modules}                                                                                                                                                                                                                                                                                                        \\
\hline
                       & \mbox{Projectivity}                                                                                                 & \mbox{Injectivity}                                                                                                  & \mbox{Flatness}                                                                                                     & \mbox{Projectivity}                                                                                                 & \mbox{Injectivity}                                                                                                  & \mbox{Flatness}                                                                                                     \\ 
\hline
$\ell_1(\Lambda)$      & \begin{tabular}{@{}c@{}}$\Lambda$\mbox{ is any } \\ \ref{c0AndlInftyModl1}\end{tabular}                             & \begin{tabular}{@{}c@{}}$\Lambda$\mbox{ is any }  \\ \ref{c0AndlInftyModl1}\end{tabular}                            & \begin{tabular}{@{}c@{}}$\Lambda$\mbox{ is any } \\ \ref{c0AndlInftyModl1}\end{tabular}                             & \begin{tabular}{@{}c@{}}$\Lambda$\mbox{ is any } \\ \ref{c0AndlInftyModl1}\end{tabular}                             & \begin{tabular}{@{}c@{}}$\Lambda$\mbox{ is any }  \\ \ref{c0AndlInftyModl1}\end{tabular}                            & \begin{tabular}{@{}c@{}}$\Lambda$\mbox{ is any } \\ \ref{c0AndlInftyModl1}\end{tabular}                             \\
\hline
$\ell_p(\Lambda)$      & \begin{tabular}{@{}c@{}}$\operatorname{Card}(\Lambda)<\aleph_0$ \\ ($\implies$) \ref{c0AndlInftyModlp}\end{tabular} & \begin{tabular}{@{}c@{}}$\operatorname{Card}(\Lambda)<\aleph_0$ \\ ($\implies$) \ref{c0AndlInftyModlp}\end{tabular} & \begin{tabular}{@{}c@{}}$\operatorname{Card}(\Lambda)<\aleph_0$ \\ ($\implies$) \ref{c0AndlInftyModlp}\end{tabular} & \begin{tabular}{@{}c@{}}$\operatorname{Card}(\Lambda)<\aleph_0$ \\ ($\implies$) \ref{c0AndlInftyModlp}\end{tabular} & \begin{tabular}{@{}c@{}}$\operatorname{Card}(\Lambda)<\aleph_0$ \\ ($\implies$) \ref{c0AndlInftyModlp}\end{tabular} & \begin{tabular}{@{}c@{}}$\operatorname{Card}(\Lambda)<\aleph_0$ \\ ($\implies$) \ref{c0AndlInftyModlp}\end{tabular} \\
\hline
$\ell_\infty(\Lambda)$ & \begin{tabular}{@{}c@{}}$\operatorname{Card}(\Lambda)<\aleph_0$ \\ \ref{c0AndlInftyModlIfty}\end{tabular}           & \begin{tabular}{@{}c@{}}$\Lambda$\mbox{ is any } \\ \ref{c0AndlInftyModlIfty}\end{tabular}                          & \begin{tabular}{@{}c@{}}$\operatorname{Card}(\Lambda)<\aleph_0$ \\ \ref{c0AndlInftyModlIfty}\end{tabular}           & \begin{tabular}{@{}c@{}}$\Lambda$\mbox{ is any } \\ \ref{c0AndlInftyModlIfty}\end{tabular}                          & \begin{tabular}{@{}c@{}}$\Lambda$\mbox{ is any } \\ \ref{c0AndlInftyModlIfty}\end{tabular}                          & \begin{tabular}{@{}c@{}}$\Lambda$\mbox{ is any } \\ \ref{c0AndlInftyModlIfty}\end{tabular}                          \\ 
\hline
$c_0(\Lambda)$         & \begin{tabular}{@{}c@{}}$\operatorname{Card}(\Lambda)<\aleph_0$ \\ \ref{c0AndlInftyModc0}\end{tabular}              & \begin{tabular}{@{}c@{}}$\operatorname{Card}(\Lambda)< \aleph_0$ \\ \ref{c0AndlInftyModc0}\end{tabular}             & \begin{tabular}{@{}c@{}}$\Lambda$\mbox{ is any } \\ \ref{c0AndlInftyModc0}\end{tabular}                             & \begin{tabular}{@{}c@{}}$\operatorname{Card}(\Lambda)<\aleph_0$ \\ \ref{c0AndlInftyModc0}\end{tabular}              & \begin{tabular}{@{}c@{}}$\operatorname{Card}(\Lambda)< \aleph_0$ \\ \ref{c0AndlInftyModc0}\end{tabular}             & \begin{tabular}{@{}c@{}}$\Lambda$\mbox{ is any } \\ \ref{c0AndlInftyModc0}\end{tabular}                             \\ 
\hline
$\mathbb{C}_\lambda$   & \begin{tabular}{@{}c@{}}$\lambda$\mbox{ is any } \\ \ref{OneDimlInftyc0ModMetTopProjIngFlat}\end{tabular}           & \begin{tabular}{@{}c@{}}$\lambda$\mbox{ is any } \\ \ref{OneDimlInftyc0ModMetTopProjIngFlat}\end{tabular}           & \begin{tabular}{@{}c@{}}$\lambda$\mbox{ is any } \\ \ref{OneDimlInftyc0ModMetTopProjIngFlat}\end{tabular}           & \begin{tabular}{@{}c@{}}$\lambda$\mbox{ is any } \\ \ref{OneDimlInftyc0ModMetTopProjIngFlat}\end{tabular}           & \begin{tabular}{@{}c@{}}$\lambda$\mbox{ is any } \\ \ref{OneDimlInftyc0ModMetTopProjIngFlat}\end{tabular}           & \begin{tabular}{@{}c@{}}$\lambda$\mbox{ is any } \\ \ref{OneDimlInftyc0ModMetTopProjIngFlat}\end{tabular}           \\
\hline

\multicolumn{7}{c}{\mbox{Homologically trivial $c_0(\Lambda)$- and $\ell_\infty(\Lambda)$-modules in topological theory}}                                                                                                                                                                                                                                                                                                                                                                                                                                                                                                                                                                                                                                \\
					 
\hline                 & \multicolumn{3}{c|}{$c_0(\Lambda)$-modules}                                                                                                                                                                                                                                                                                                                     & \multicolumn{3}{c|}{$\ell_\infty(\Lambda)$-modules}                                                                                                                                                                                                                                                                                                        \\
\hline
                       & \mbox{Projectivity}                                                                                                 & \mbox{Injectivity}                                                                                                  & \mbox{Flatness}                                                                                                     & \mbox{Projectivity}                                                                                                 & \mbox{Injectivity}                                                                                                  & \mbox{Flatness}                                                                                                     \\ 
\hline
$\ell_1(\Lambda)$      & \begin{tabular}{@{}c@{}}$\Lambda$\mbox{ is any }  \\ \ref{c0AndlInftyModl1}\end{tabular}                            & \begin{tabular}{@{}c@{}}$\Lambda$\mbox{ is any } \\ \ref{c0AndlInftyModl1}\end{tabular}                             & \begin{tabular}{@{}c@{}}$\Lambda$\mbox{ is any }  \\ \ref{c0AndlInftyModl1}\end{tabular}                            & \begin{tabular}{@{}c@{}}$\Lambda$\mbox{ is any }  \\ \ref{c0AndlInftyModl1}\end{tabular}                            & \begin{tabular}{@{}c@{}}$\Lambda$\mbox{ is any } \\ \ref{c0AndlInftyModl1}\end{tabular}                             & \begin{tabular}{@{}c@{}}$\Lambda$\mbox{ is any }  \\ \ref{c0AndlInftyModl1}\end{tabular}                            \\
\hline
$\ell_p(\Lambda)$      & \begin{tabular}{@{}c@{}}$\operatorname{Card}(\Lambda)<\aleph_0$ \\ \ref{c0AndlInftyModlp}\end{tabular}              & \begin{tabular}{@{}c@{}}$\operatorname{Card}(\Lambda)<\aleph_0$ \\ \ref{c0AndlInftyModlp}\end{tabular}              & \begin{tabular}{@{}c@{}}$\operatorname{Card}(\Lambda)<\aleph_0$ \\ \ref{c0AndlInftyModlp}\end{tabular}              & \begin{tabular}{@{}c@{}}$\operatorname{Card}(\Lambda)<\aleph_0$ \\ \ref{c0AndlInftyModlp}\end{tabular}              & \begin{tabular}{@{}c@{}}$\operatorname{Card}(\Lambda)<\aleph_0$ \\ \ref{c0AndlInftyModlp}\end{tabular}              & \begin{tabular}{@{}c@{}}$\operatorname{Card}(\Lambda)<\aleph_0$ \\ \ref{c0AndlInftyModlp}\end{tabular}              \\
\hline
$\ell_\infty(\Lambda)$ & \begin{tabular}{@{}c@{}}$\operatorname{Card}(\Lambda)<\aleph_0$ \\ \ref{c0AndlInftyModlIfty}\end{tabular}           & \begin{tabular}{@{}c@{}}$\Lambda$\mbox{ is any } \\ \ref{c0AndlInftyModlIfty}\end{tabular}                          & \begin{tabular}{@{}c@{}}$\operatorname{Card}(\Lambda)<\aleph_0$ \\ \ref{c0AndlInftyModlIfty}\end{tabular}           & \begin{tabular}{@{}c@{}}$\Lambda$\mbox{ is any } \\ \ref{c0AndlInftyModlIfty}\end{tabular}                          & \begin{tabular}{@{}c@{}}$\Lambda$\mbox{ is any } \\ \ref{c0AndlInftyModlIfty}\end{tabular}                          & \begin{tabular}{@{}c@{}}$\Lambda$\mbox{ is any } \\ \ref{c0AndlInftyModlIfty}\end{tabular}                          \\ 
\hline
$c_0(\Lambda)$         & \begin{tabular}{@{}c@{}}$\operatorname{Card}(\Lambda)<\aleph_0$ \\ \ref{c0AndlInftyModc0}\end{tabular}              & \begin{tabular}{@{}c@{}}$\operatorname{Card}(\Lambda)<\aleph_0$ \\ \ref{c0AndlInftyModc0}\end{tabular}              & \begin{tabular}{@{}c@{}}$\Lambda$\mbox{ is any } \\ \ref{c0AndlInftyModc0}\end{tabular}                             & \begin{tabular}{@{}c@{}}$\operatorname{Card}(\Lambda)<\aleph_0$ \\ \ref{c0AndlInftyModc0}\end{tabular}              & \begin{tabular}{@{}c@{}}$\operatorname{Card}(\Lambda)<\aleph_0$ \\ \ref{c0AndlInftyModc0}\end{tabular}              & \begin{tabular}{@{}c@{}}$\Lambda$\mbox{ is any } \\ \ref{c0AndlInftyModc0}\end{tabular}                             \\ 
\hline
$\mathbb{C}_\lambda$   & \begin{tabular}{@{}c@{}}$\lambda$\mbox{ is any } \\ \ref{OneDimlInftyc0ModMetTopProjIngFlat}\end{tabular}           & \begin{tabular}{@{}c@{}}$\lambda$\mbox{ is any } \\ \ref{OneDimlInftyc0ModMetTopProjIngFlat}\end{tabular}           & \begin{tabular}{@{}c@{}}$\lambda$\mbox{ is any } \\ \ref{OneDimlInftyc0ModMetTopProjIngFlat}\end{tabular}           & \begin{tabular}{@{}c@{}}$\lambda$\mbox{ is any } \\ \ref{OneDimlInftyc0ModMetTopProjIngFlat}\end{tabular}           & \begin{tabular}{@{}c@{}}$\lambda$\mbox{ is any } \\ \ref{OneDimlInftyc0ModMetTopProjIngFlat}\end{tabular}           & \begin{tabular}{@{}c@{}}$\lambda$\mbox{ is any } \\ \ref{OneDimlInftyc0ModMetTopProjIngFlat}\end{tabular}           \\
\hline

\multicolumn{7}{c}{\mbox{Homologically trivial $c_0(\Lambda)$- and $\ell_\infty(\Lambda)$-modules in relative theory}}                                                                                                                                                                                                                                                                                                                                                                                                                                                                                                                                                                                                                                   \\

\hline                 & \multicolumn{3}{c|}{$c_0(\Lambda)$-modules}                                                                                                                                                                                                                                                                                                                     & \multicolumn{3}{c|}{$\ell_\infty(\Lambda)$-modules}                                                                                                                                                                                                                                                                                                        \\
\hline
                       & \mbox{Projectivity}                                                                                                 & \mbox{Injectivity}                                                                                                  & \mbox{Flatness}                                                                                                     & \mbox{Projectivity}                                                                                                 & \mbox{Injectivity}                                                                                                  & \mbox{Flatness}                                                                                                     \\ 
\hline
$\ell_1(\Lambda)$      & \begin{tabular}{@{}c@{}}$\Lambda$\mbox{ is any } \\ \ref{c0AndlInftyModsRelTh}, i)\end{tabular}                     & \begin{tabular}{@{}c@{}}$\Lambda$\mbox{ is any }  \\ \ref{c0AndlInftyModsRelTh}, i)\end{tabular}                    & \begin{tabular}{@{}c@{}}$\Lambda$\mbox{ is any } \\ \ref{c0AndlInftyModsRelTh}, i)\end{tabular}                     & \begin{tabular}{@{}c@{}}$\Lambda$\mbox{ is any }  \\ \ref{c0AndlInftyModsRelTh}, ii)\end{tabular}                   & \begin{tabular}{@{}c@{}}$\Lambda$\mbox{ is any } \\ \ref{c0AndlInftyModsRelTh}, ii)\end{tabular}                    & \begin{tabular}{@{}c@{}}$\Lambda$\mbox{ is any }  \\ \ref{c0AndlInftyModlIfty}, ii)\end{tabular}                    \\
\hline
$\ell_p(\Lambda)$      & \begin{tabular}{@{}c@{}}$\Lambda$\mbox{ is any } \\ \ref{c0AndlInftyModsRelTh}, i)\end{tabular}                     & \begin{tabular}{@{}c@{}}$\Lambda$\mbox{ is any }  \\ \ref{c0AndlInftyModsRelTh}, i)\end{tabular}                    & \begin{tabular}{@{}c@{}}$\Lambda$\mbox{ is any } \\ \ref{c0AndlInftyModsRelTh}, i)\end{tabular}                     & \begin{tabular}{@{}c@{}}$\Lambda$\mbox{ is any }  \\ \ref{c0AndlInftyModsRelTh}, ii)\end{tabular}                   & \begin{tabular}{@{}c@{}}$\Lambda$\mbox{ is any } \\ \ref{c0AndlInftyModsRelTh}, ii)\end{tabular}                    & \begin{tabular}{@{}c@{}}$\Lambda$\mbox{ is any }  \\ \ref{c0AndlInftyModlIfty}, ii)\end{tabular}                    \\
\hline
$\ell_\infty(\Lambda)$ & \begin{tabular}{@{}c@{}}$\operatorname{Card}(\Lambda)<\aleph_0$ \\ \ref{c0AndlInftyModsRelTh}, i)\end{tabular}      & \begin{tabular}{@{}c@{}}$\Lambda$\mbox{ is any }  \\ \ref{c0AndlInftyModsRelTh}, i)\end{tabular}                    & \begin{tabular}{@{}c@{}}$\Lambda$\mbox{ is any } \\ \ref{c0AndlInftyModsRelTh}, i)\end{tabular}                     & \begin{tabular}{@{}c@{}}$\Lambda$\mbox{ is any }  \\ \ref{c0AndlInftyModsRelTh}, ii)\end{tabular}                   & \begin{tabular}{@{}c@{}}$\Lambda$\mbox{ is any } \\ \ref{c0AndlInftyModsRelTh}, ii)\end{tabular}                    & \begin{tabular}{@{}c@{}}$\Lambda$\mbox{ is any }  \\ \ref{c0AndlInftyModlIfty}, ii)\end{tabular}                    \\
\hline
$c_0(\Lambda)$         & \begin{tabular}{@{}c@{}}$\Lambda$\mbox{ is any } \\ \ref{c0AndlInftyModsRelTh}, i)\end{tabular}                     & \begin{tabular}{@{}c@{}}$\operatorname{Card}(\Lambda)<\aleph_0$  \\ \ref{c0AndlInftyModsRelTh}, i) \end{tabular}    & \begin{tabular}{@{}c@{}}$\Lambda$\mbox{ is any } \\ \ref{c0AndlInftyModsRelTh}, i)\end{tabular}                     & \begin{tabular}{@{}c@{}}$\Lambda$\mbox{ is any }  \\ \ref{c0AndlInftyModsRelTh}, ii)\end{tabular}                   & \begin{tabular}{@{}c@{}}\mbox{ ??? } \end{tabular}                                                                  & \begin{tabular}{@{}c@{}}$\Lambda$\mbox{ is any }  \\ \ref{c0AndlInftyModlIfty}, ii)\end{tabular}                    \\
\hline
$\mathbb{C}_\lambda$   & \begin{tabular}{@{}c@{}}$\lambda$\mbox{ is any } \\ \ref{c0AndlInftyModsRelTh}, i)\end{tabular}                     & \begin{tabular}{@{}c@{}}$\lambda$\mbox{ is any }  \\ \ref{c0AndlInftyModsRelTh}, i)\end{tabular}                    & \begin{tabular}{@{}c@{}}$\lambda$\mbox{ is any } \\ \ref{c0AndlInftyModsRelTh}, i)\end{tabular}                     & \begin{tabular}{@{}c@{}}$\lambda$\mbox{ is any }  \\ \ref{c0AndlInftyModsRelTh}, ii)\end{tabular}                   & \begin{tabular}{@{}c@{}}$\lambda$\mbox{ is any } \\ \ref{c0AndlInftyModsRelTh}, ii)\end{tabular}                    & \begin{tabular}{@{}c@{}}$\lambda$\mbox{ is any }  \\ \ref{c0AndlInftyModlIfty}, ii)\end{tabular}                    \\
\hline
\end{longtable}
\end{scriptsize}

%----------------------------------------------------------------------------------------
%	C_0(S)-modules
%----------------------------------------------------------------------------------------

\subsection{\texorpdfstring{$C_0(S)$}{C0(S)}-modules}
\label{SubSectionC0SModules}

This section is devoted to study of homological triviality of classical modules over algebra $C_0(S)$, where $S$ is a locally compact Hausdorff space. By classical we mean modules $C_0(S)$, $M(S)$ and $L_p(S,\mu)$ for positive measure $\mu\in M(S)$. The pointwise multiplication plays the role of outer action for these modules.

Further we give short preliminaries on these modules. Recall that $C_0(S)^*\isom{\mathbf{Ban}_1}M(S)$ and $L_p(S,\mu)^*\isom{\mathbf{Ban}_1}L_{p^*}(S,\mu)$ for $1\leq p<+\infty$. In fact these identifications are isomorphisms of left and right $C_0(S)$-modules. For a given positive measure $\mu\in M(S)$ by $M_s(S,\mu)$ we shall denote the closed $C_0(S)$-submodule of $M(S)$ consisting of measures strictly singular with respect to $\mu$. Then the well known Lebesgue decomposition theorem can be stated as $M(S)\isom{\mathbf{Ban}_1}L_1(S,\mu)\bigoplus_1 M_s(S,\mu)$. Even more, this identification is an isomorphism of left and right $C_0(S)$-modules. 

We obliged to emphasize here that we consider only finite Borel regular positive measures. This shall simplify many considerations. For example, any atom of regular measure on a locally compact Hausdorff spaces is a point [\cite{BourbElemMathIntegLivVI}, chapter 5, \S 5, exercise 7]. Since we consider only finite measures, one can not say that this section simply generalizes results of the previous one. Strictly speaking these sections are different, though their methods have much in common.

For a fixed point $s\in S$ by $\mathbb{C}_s$ we denote left or right Banach $C_0(S)$-module $\mathbb{C}$ with outer action defined by
$$
a\cdot_s z=a(s)z,\qquad z\cdot_s a=a(s)z
$$
\begin{proposition}\label{OneDimC0SModMetTopRelProjIngFlat} Let $S$ be a locally compact Hausdorff space and let $s\in S$. Then 

$i)$ $\mathbb{C}_s$ is metrically, topologically or relatively projective as $C_0(S)$-module iff $s$ is an isolated point of $S$;

$ii)$ $\mathbb{C}_s$ is metrically, topologically and relatively flat as $C_0(S)$-module.

$iii)$ $\mathbb{C}_s$ is metrically, topologically and relatively injective as $C_0(S)$-module;

\end{proposition}
\begin{proof} $i)$ If $\mathbb{C}_s$ is metrically or topologically or relatively projective, then by proposition \ref{MetProjIsTopProjAndTopProjIsRelProj} it is at least relatively projective. Now from [\cite{HelBanLocConvAlg}, proposition 7.1.31] we know that the latter forces $s$ to be an isolated point of $S$. Conversely, assume $s$ is an isolated point of $S$. One can easily check that the maps $\pi:C_0(S)_+\to\mathbb{C}_s:a\oplus_1 z\mapsto a(s)+z$ and $\sigma:\mathbb{C}_s\to C_0(S)_+:z\mapsto z\delta_s\oplus_1 0$ are contractive $C_0(S)$-morphisms. Since $\pi\sigma=1_{\mathbb{C}_s}$, then $\mathbb{C}_s$ is a retract of $C_0(S)_+$ in $C_0(S)-\mathbf{mod}_1$. From propositions \ref{RetrMetTopProjIsMetTopProj} and \ref{UnitalAlgIsMetTopProj} it follows that $\mathbb{C}_s$ is metrically and topologically projective left $C_0(S)$-module. From \ref{MetProjIsTopProjAndTopProjIsRelProj} we conclude that $\mathbb{C}_s$ is also relatively projective $C_0(S)$-module.

$ii)$ By [\cite{HelBanLocConvAlg}, theorem 7.1.87] the algebra $C_0(S)$ is relatively amenable. Since this algebra is a $C^*$-algebra it is $1$-amenable [\cite{RundeAmenConstFour}, example 3]. Clearly, $\mathbb{C}_s$ is a $1$-dimensional $L_1$-space and an essential $C_0(S)$-module. Therefore, by proposition \ref{MetTopEssL1FlatModAoverAmenBanAlg} this module is metrically flat. Now the result follows from proposition \ref{MetFlatIsTopFlatAndTopFlatIsRelFlat}.

$iii)$ From paragraph $ii)$ and proposition \ref{MetTopFlatCharac} it follows that $\mathbb{C}_s^*$ is metrically injective. By proposition \ref{MetFlatIsTopFlatAndTopFlatIsRelFlat} it is topologically and relatively injective too. It remains to note that $\mathbb{C}_s\isom{\mathbf{mod}_1-A}\mathbb{C}_s^*$. 
\end{proof}

\begin{proposition}\label{C0SC0SModMetTopRelProjIngFlat} Let $S$ be a locally compact Hausdorff space and let $s\in S$. Then

$i)$ $C_0(S)$ is $\langle$~metrically / topologically / relatively~$\rangle$ projective as $C_0(S)$-module iff $S$ is $\langle$~compact / compact / paracompact~$\rangle$;

$ii)$ $C_0(S)$ is metrically injective as $C_0(S)$-module iff $S$ is a Stonean space;

$iii)$ $C_0(S)$ is metrically, topologically and relatively flat as $C_0(S)$-module.
\end{proposition}
\begin{proof} We regard $C_0(S)$ as a two-sided ideal of $C_0(S)$. Recall that $\operatorname{Spec}(C_0(S))$ is homeomorphic to $S$ [\cite{HelHomolBanTopAlg}, corollary 3.1.6].

$i)$ It is enough to note that by $\langle$~proposition \ref{IdealofCommCStarAlgMetTopProjCharac} / proposition \ref{IdealofCommCStarAlgMetTopProjCharac} / [\cite{HelHomolBanTopAlg}, chapter IV,\S\S 2-3]~$\rangle$ the spectrum of $C_0(S)$ is $\langle$~compact / compact / paracompact~$\rangle$. 

$ii)$ This result is a weakened version of proposition \ref{MetInjCStarAlgCharac}.

$iii)$ From proposition \ref{IdealofCstarAlgisMetTopFlat} it immediately follows that $C_0(S)$-module $C_0(S)$ is metrically and topologically flat. By proposition \ref{MetFlatIsTopFlatAndTopFlatIsRelFlat} it is also relatively flat.
\end{proof}

\begin{proposition}\label{AtomsOfRelProjLpMod} Let $S$ be a locally compact Hausdorff space, $\mu$ be a finite Borel regular positive measure on $S$. Assume $1\leq p\leq+\infty$ and $C_0(S)$-module $L_p(S,\mu)$ is relatively projective. Then any atom of $\mu$ is an isolated point in $S$.
\end{proposition} 
\begin{proof} Assume $\mu$ has at least one atom, otherwise there is nothing to prove. From [\cite{BourbElemMathIntegLivVI}, chapter 5, \S 5, exercise 7] we know that any atom of $\mu$ is a point. Call it $s$. Consider well defined linear maps $\pi:L_p(\Omega,\mu)\to\mathbb{C}_s:f\mapsto f(s)$ and $\sigma:\mathbb{C}_s\to L_p(\Omega,\mu):z\mapsto z\delta_s$. One can easily check that these maps are $C_0(S)$-morphisms and $\pi\sigma=1_{\mathbb{C}_s}$. Therefore, $\mathbb{C}_s$ is a retract of $L_p(S,\mu)$ in $C_0(S)-\mathbf{mod}$. By assumption, the latter module is relatively projective, so by [\cite{HelBanLocConvAlg}, proposition 7.1.6] the $C_0(S)$-module $\mathbb{C}_s$ is relatively projective. By paragraph $i)$ of proposition \ref{C0SC0SModMetTopRelProjIngFlat} we see that $s$ is an isolated point of $S$.
\end{proof}

\begin{proposition}\label{SuppsOfSomeFuncInC0SAndLp} Let $S$ be a locally compact Hausdorff space, $\mu$ be a finite Borel regular positive measure on $S$. Let $1\leq p<+\infty$ and $V$ be an open subset of $S$, then

$i)$ there exists a net $(e_\nu^V)_{\nu\in N}$ of continuous functions that pointwise converge to $\chi_V$ and $0\leq e_\nu^V\leq 1$, $e_\nu^V|_{S\setminus V}=0$ for all $\nu\in N$. Even more $(e_\nu^V)_{\nu\in N}$ converges to $\chi_V$ in $L_p(S,\mu)$; 

$ii)$ for any $C_0(S)$-morphism $\phi:L_p(S,\mu)\to C_0(S)$ holds $\phi(\chi_V)|_{S\setminus V}=0$.
\end{proposition}
\begin{proof} $i)$ Let $(K_\nu)_{\nu\in N}$ be the net of all compact subsets of $V$. By Urysohn's lemma for each $\nu\in N$ we can construct a continuous function $e_\nu^V:S\to\mathbb{C}$ such that $0\leq e_\nu^V\leq 1$, $e_\nu^V|_{K_\nu}=1$ and $e_\nu^V|_{S\setminus V}=0$. Since $V=\bigcup_{\nu\in N} K_\nu$, then $(e_\nu^V)_{\nu\in N}$ pointwise converges to $\chi_V$. Recall that $\mu$ is a regular Borel measure, so $\lim_\nu\mu(V\setminus K_\nu)=0$. As the consequence $\lim_\nu e_\nu^V=\chi_V$ in $L_p(S,\mu)$.

$ii)$ Consider the net of functions $(e_\nu^V)_{\nu\in N}$ constructed in the paragraph $i)$. Clearly $e_\nu^V\chi_V=e_\nu^V$ for all $\nu\in N$, so $\phi(\chi_V)=\phi(\lim_\nu e_\nu^V\chi_V)=\lim_\nu e_\nu^V\phi(\chi_V)$. Since $e_\nu^V$ is zero outside $V$, then so does $\lim_\nu e_\nu^V\phi(\chi_V)$. Therefore $\phi(\chi_V)|_{S\setminus V}=0$.
\end{proof}

For the next proposition we need a short reminder on approximation property. Recall that $C(K)$-spaces have the approximation property [\cite{DefFloTensNorOpId}, section 5.2(3)], and this property is inherited by complemented subspaces [\cite{DefFloTensNorOpId}, exercise 5.5]. Therefore any space of the form $C_0(S)$ for some locally compact Hausdorff space has the approximation property because it is complemented in $C(\alpha S)$.

\begin{proposition}\label{RelProjLpModImplsPureAtomMeas} Let $S$ be a locally compact Hausdorff space, $\mu$ be a finite Borel regular positive measure on $S$. Assume $1\leq p<+\infty$ and $C_0(S)$-module $L_p(S,\mu)$ is relatively projective. Then $\mu$ is purely atomic and all its atoms are isolated points.
\end{proposition}
\begin{proof} By proposition \ref{AtomsOfRelProjLpMod} the set $S_a^{\mu}$ of atoms of measure $\mu$ consist of isolated points. Therefore $S\setminus S_a^{\mu}$ is a locally compact space and $\mu|_{S\setminus S_a^{\mu}}$ is a finite non atomic Borel regular measure. One more fact worth noting:  since $S_a^{\mu}$ is discrete, then a set $V$ is open in $S\setminus S_a^{\mu}$ iff it is open in $S$. Now assume there is an open set $V\subset S\setminus S_a^{\mu}$ with $\mu(V)>0$. Then $\chi_V\neq 0$ in $L_p(S,\mu)$. Recall that $C_0(S)$-module $L_p(S,\mu)$ is essential and $C_0(S)$ has the approximation property. Bearing all this in mind we can apply lemma 1.4 from \cite{SelivBiproBanAlg}. It guarantees existence of $C_0(S)$-morphism $\phi:L_p(S,\mu)\to C_0(S)$ such that $\phi(\chi_V)\neq 0$. 

Denote $f=\phi(\chi_V)$. By paragraph $ii)$ of proposition \ref{SuppsOfSomeFuncInC0SAndLp} we have $f|_{S\setminus V}=0$. Since $f\neq 0$, then there necessarily exists an open set $U\subset V$ such that $|f|_U|> 0$. Consider net $(e_\nu^U)_{\nu\in N}$ constructed from set $U$ in paragraph $i)$ of proposition \ref{SuppsOfSomeFuncInC0SAndLp}. Note that $e_\nu^U\chi_V=e_\nu^U$, so for any $t\in U$ we have
$$
|\phi(\chi_U)(t)|
=|\phi(\lim_\nu e_\nu^U)(t)|
=\lim_\nu|\phi(e_\nu^U\chi_V)(t)|
=\lim_\nu |e_\nu^U(t)f(t)|=|f(t)|
>0.
$$ 
Thus $\phi(\chi_U)\neq 0$. Hence $\chi_U\neq 0$ in $L_p(S,\mu)$ which is equivalent to $\mu(U)>0$. The latter implies existence of some point $r\in U\cap\operatorname{supp}(\mu)$. For the future note that $|f(r)|>0$.

Since $r\in U\cap\operatorname{supp}(\mu)\subset S\setminus S_a^{\mu}$ and $\mu|_{S\setminus S_a^{\mu}}$ is non atomic, then for the net $(W_\nu)_{\nu\in M}$ of all open neighborhoods of $r$ we have $\lim_\nu\mu(W_\nu)=0$. By Urysohn's lemma for each $\nu\in M$ we have a continuous function $h_\nu:S\to\mathbb{C}$ such that $0\leq h_\nu\leq 1$, $h_\nu|_{S\setminus (W_\nu\cap V)}=0$ and $h_\nu(r)=1$. By construction $h_\nu\chi_V=h_\nu$ for all $\nu\in N$. Now we have a long chain of inequalities
$$
0<|f(r)|
=|(h_\nu f)(r)|
=|(h_\nu \phi(\chi_V))(r)|
=|\phi(h_\nu\chi_V)(r)|
=|\phi(h_\nu)(r)|
\leq\Vert\phi(h_\nu)\Vert
$$
$$
\leq\limsup_\nu\Vert\phi(h_\nu)\Vert
\leq\limsup_\nu\Vert\phi\Vert\Vert h_\nu\Vert
\leq\Vert\phi\Vert\limsup_\nu\mu(W_\nu\cap V)^{1/p}
$$
$$
\leq\Vert\phi\Vert\limsup_\nu\mu(W_\nu)^{1/p}
=\Vert\phi\Vert\lim_\nu\mu(W_\nu)^{1/p}
=0.
$$
Contradiction, hence $\mu(V)=0$ for all open subsets $V$ of $S\setminus S_a^{\mu}$. Therefore by [\cite{FremMeasTh}, proposition 414L] the measure $\mu|_{S\setminus S_a^{\mu}}$ is zero. In other words $\mu$ is purely atomic.
\end{proof}

\begin{proposition}\label{L1C0SModMetTopRelProjInjFlat}
Let $S$ be a locally compact Hausdorff space and $\mu$ be a finite Borel regular positive measure on $S$. Then 

$i)$ $L_1(S,\mu)$ is metrically or topologically or relatively projective as $C_0(S)$-module iff $\mu$ is purely atomic and all its atoms are isolated points in $S$;

$ii)$ $L_1(S,\mu)$ is metrically, topologically and relatively injective as $C_0(S)$-module;

$iii)$ $L_1(S,\mu)$ is metrically, topologically and relatively flat as $C_0(S)$-module.
\end{proposition}
\begin{proof} $i)$ If $L_1(S,\mu)$ is metrically or topologically or relatively projective, then by proposition \ref{MetProjIsTopProjAndTopProjIsRelProj} it is at least relatively projective. Now from proposition \ref{RelProjLpModImplsPureAtomMeas} the measure $\mu$ is purely atomic and all atoms are isolated  points. Conversely, assume that $\mu$ is purely atomic and all atoms are isolated points. By $S_a^{\mu}$ we denote the set of these atoms. Now one can easily show that the linear map $i:L_1(S,\mu)\to\bigoplus_1\{\mathbb{C}_s:s\in S_a^{\mu}\}:f\mapsto \bigoplus_1\{\mu(\{s\})f(s):s\in S_a^{\mu}\}$ is an isometric isomorphism of $C_0(S)$-modules. By paragraphs $i)$ of propositions \ref{MetTopProjModCoprod} and \ref{OneDimC0SModMetTopRelProjIngFlat} the $C_0(S)$-module $\bigoplus_1\{\mathbb{C}_s:s\in S_a^{\mu}\}$ is metrically projective. Therefore so does $L_1(S,\mu)$. By proposition \ref{MetProjIsTopProjAndTopProjIsRelProj} it is also topologically and relatively projective.

$ii)$ By paragraph $iii)$ of proposition \ref{C0SC0SModMetTopRelProjIngFlat} the $C_0(S)$-module $C_0(S)$ is metrically flat. From proposition \ref{DualMetTopProjIsMetrInj} we get that $M(S)\isom{\mathbf{mod}_1-C_0(S)}C_0(S)^*$ is metrically injective. Since $M(S)\isom{\mathbf{mod}_1-C_0(S)}L_1(S,\mu)\bigoplus_1 M_s(S,\mu)$, then $L_1(S,\mu)$ is a retract of $M(S)$ in $\mathbf{mod}_1-C_0(S)$. So by proposition \ref{RetrMetTopInjIsMetTopInj} the $C_0(S)$-module $L_1(S,\mu)$ is metrically injective. Relative and topological injectivity of $L_1(S,\mu)$ follows from proposition \ref{MetInjIsTopInjAndTopInjIsRelInj}.

$iii)$ By [\cite{HelBanLocConvAlg}, theorem 7.1.87] the algebra $C_0(S)$ is relatively amenable. Since this algebra is a $C^*$-algebra it is $1$-amenable [\cite{RundeAmenConstFour}, example 3]. Since $L_1(S,\mu)$ is an essential $C_0(S)$-module which tautologically an $L_1$-space, then by proposition \ref{MetTopEssL1FlatModAoverAmenBanAlg} this module is metrically flat. From proposition \ref{MetFlatIsTopFlatAndTopFlatIsRelFlat} the $C_0(S)$-module $L_1(S,\mu)$ is also topologically and relatively flat.
\end{proof}

\begin{proposition}\label{LpC0SModMetTopRelProjIngFlat}
Let $S$ be a locally compact Hausdorff space and $\mu$ be a finite Borel regular positive measure on $S$. Assume $1<p<+\infty$, then 

$i)$ $L_p(S,\mu)$ is relatively injective and flat, but relatively projective iff $\mu$ is purely atomic and all atoms are isolated points;

$ii)$ $L_p(S,\mu)$ is topologically projective or injective or flat iff $\mu$ is purely atomic with finitely many atoms;

$iii)$ if $L_p(S,\mu)$ is metrically projective or injective or flat, then $\mu$ is purely atomic with finitely many atoms.
\end{proposition}
\begin{proof} $i)$ By [\cite{HelBanLocConvAlg}, theorem 7.1.87] the algebra $C_0(S)$ is relatively amenable. Now from [\cite{HelBanLocConvAlg}, theorem 7.1.60] it follows that $L_p(S,\mu)$ is relatively flat for all $1<p<+\infty$. Note that $L_p(S,\mu)\isom{\mathbf{mod}_1-C_0(S)}L_{p^*}(S,\mu)^*$. Then from [\cite{HelBanLocConvAlg}, proposition 7.1.42] we get that $L_p(S,\mu)$ is relatively injective for any $1<p<+\infty$. Now assume that $L_p(S,\mu)$ is relatively projective, then by proposition \ref{RelProjLpModImplsPureAtomMeas} the measure $\mu$ is purely atomic and all its atoms are isolated points. Conversely, let $\mu$ be purely atomic with all atoms isolated. Denote by $S_a^{\mu}$ the set of these atoms. Since $S_a^{\mu}$ is discrete, then $C_0(S_a^{\mu})$ is relatively biprojective [\cite{HelHomolBanTopAlg}, theorem 4.5.26]. Then $L_p(S,\mu)$ is relatively projective $C_0(S_a^{\mu})$-module because it is essential module over relatively biprojective algebra with two-sided bounded approximate identity. Clearly $C_0(S_a^{\mu})$ is a two-sided ideal of $C_0(S)$, so from [\cite{RamsHomPropSemgroupAlg}, proposition 2.3.2(i)] we get that $L_p(S,\mu)$ is relatively projective as $C_0(S)$-module.

$ii), iii)$ Assume that $L_p(S,\mu)$ is metrically or topologically projective or injective or flat $C_0(S)$-module. Since $L_p(S,\mu)$ is reflexive and $C_0(S)$ is an $\mathscr{L}_\infty^g$-space, then $L_p(S,\mu)$ is finite dimensional by corollary \ref{NoInfDimRefMetTopProjInjFlatModOverMthscrL1OrLInfty}. The latter is equivalent to measure $\mu$ being purely atomic with finitely many atoms. On the other hand, if $\mu$ is purely atomic with finitely many atoms, then $L_p(S,\mu)$ is topologically isomorphic to $L_1(S,\mu)$ as left or right $C_0(S)$-module. The latter module is topologically projective, injective and flat for our measure by proposition \ref{L1C0SModMetTopRelProjInjFlat}. Hence so does $L_p(S,\mu)$.
\end{proof}

\begin{proposition}\label{LinftyC0SModMetTopRelProjIngFlat}
Let $S$ be a locally compact Hausdorff space and $\mu$ be a finite Borel regular positive measure on $S$. Then

$i)$ $L_\infty(S,\mu)$ is metrically, topologically and relatively injective as $C_0(S)$-module;

$ii)$ $L_\infty(S,\mu)$ is relatively flat $C_0(S)$-module.
\end{proposition}
\begin{proof} $i)$ Since $L_\infty(S,\mu)\isom{\mathbf{mod}_1-C_0(S)}L_1(S,\mu)^*$, then the result immediately follows from proposition \ref{DualMetTopProjIsMetrInj} and paragraph $iii)$ of proposition \ref{L1C0SModMetTopRelProjInjFlat}.

$ii)$ By [\cite{HelBanLocConvAlg}, theorem 7.1.87] the algebra $C_0(S)$ is relatively amenable. Any left Banach module over relatively amenable Banach algebra is relatively flat [\cite{HelBanLocConvAlg}, theorem 7.1.60]. In particular $L_\infty(S,\mu)$ is relatively flat $C_0(S)$-module.
\end{proof}

\begin{proposition}\label{MSC0SModMetTopRelProjIngFlat}
Let $S$ be a locally compact Hausdorff space and $\mu$ be a finite Borel regular positive measure on $S$. Then

$i)$ $M(S)$ is metrically or topologically or relatively projective as $C_0(S)$-module iff $S$ is discrete; 

$ii)$ $M(S)$ is metrically, topologically and relatively injective as $C_0(S)$-module; 

$iii)$ $M(S)$ is metrically, topologically and relatively flat as $C_0(S)$-module.
\end{proposition}
\begin{proof} $i)$ If $M(S)$ is metrically or topologically or relatively projective, then by proposition \ref{MetProjIsTopProjAndTopProjIsRelProj} it is at least relatively projective. For arbitrary $s\in S$ consider measure $\mu=\delta_s$ and recall the decomposition $M(S)\isom{C_0(S)-\mathbf{mod}_1}L_1(S,\mu)\bigoplus_1 M_s(S,\mu)$. Then $L_1(S,\mu)$ is a retract of $M(S)$ in $C_0(S)-\mathbf{mod}_1$. So from [\cite{HelBanLocConvAlg}, proposition 7.1.6] we get that $L_1(S,\mu)$ is relatively projective $C_0(S)$-module. Since $s$ is the only atom of $\mu$, then from proposition \ref{L1C0SModMetTopRelProjInjFlat} it follows that $s$ is an isolated point in $S$. Since $s\in S$ is arbitrary, then $S$ is discrete. Conversely, assume $S$ is discrete. Then $C_0(S)=c_0(S)$, and $M(S)\isom{C_0(S)-\mathbf{mod}_1}C_0(S)^*\isom{C_0(S)-\mathbf{mod}_1}\ell_1(S)\isom{C_0(S)-\mathbf{mod}_1}\bigoplus_1\{\mathbb{C}_s:s\in S\}$. The latter $C_0(S)$-module is metrically projective by paragraphs $i)$ of propositions \ref{MetTopProjModCoprod} and \ref{OneDimC0SModMetTopRelProjIngFlat}. Therefore $M(S)$ is metrically projective $C_0(S)$-module too. By proposition \ref{MetProjIsTopProjAndTopProjIsRelProj} it is also topologically and relatively projective.

$ii)$ Since $M(S)\isom{\mathbf{mod}_1-C_0(S)}C_0(S)^*$, then the result immediately follows from proposition \ref{DualMetTopProjIsMetrInj} and paragraph $iii)$ of proposition \ref{C0SC0SModMetTopRelProjIngFlat}.

$iii)$ By [\cite{HelBanLocConvAlg}, theorem 7.1.87] the algebra $C_0(S)$ is relatively amenable. Since this algebra is a $C^*$-algebra it is $1$-amenable [\cite{RundeAmenConstFour}, example 2]. Note that $M(S)$ is an essential $C_0(S)$-module which as Banach space is an $L_1$-space [\cite{DalLauSecondDualOfMeasAlg}, discussion after proposition 2.14]. Then by proposition \ref{MetTopEssL1FlatModAoverAmenBanAlg} this module is metrically flat. From proposition \ref{MetFlatIsTopFlatAndTopFlatIsRelFlat} it is also topologically and relatively flat.
\end{proof}

Results of this section are summarized in the following three tables. Each cell contains a condition under which the respective module has the respective property and propositions where it is proved. We use ??? symbol to indicate open problems. Open problems of this section are divided into three parts: injectivity of $C_0(S)$, projectivity of $L_\infty(S,\mu)$ and flatness of $L_\infty(S,\mu)$. Complete description of relatively and topologically injective $C_0(S)$-modules $C_0(S)$ seems quite a challenge for one simple reason --- still there is no standard category of functional analysis where even topologically injective objects were fully understood. The question of relative projectivity of $C_0(S)$-module $L_\infty(S,\mu)$ is rather old. It seems that even relative projectivity of $L_\infty(S,\mu)$ is a rare property. Our conjecture that $\mu$ must be purely atomic with finitely many atoms. Finally we presume that a necessary condition for metric and topological flatness of $C_0(S)$-module $L_\infty(S,\mu)$ is compactness of $S$.

As for partial results, we don't have a criterion of homological triviality of $C_0(S)$-modules $L_p(S,\mu)$ in metric theory for $1<p<+\infty$. We indicate this fact via symbol $\implies$. Using advanced Banach geometric techniques on factorization constants through finite dimensional Hilbert spaces one may show that atoms count for metrically projective modules $L_p(S,\mu)$ doesn't exceed some universal constant. It seems that $L_p(S,\mu)$ is homologically trivial $C_0(S)$-module in metric theory only for purely atomic measures with unique atom. 

\begin{scriptsize}
\begin{longtable}{|c|c|c|c|} 
\multicolumn{4}{c}{\mbox{Homologically trivial $C_0(S)$-modules in metric theory}}                                                                                                                                                                                                                                                                                                                                                                                                                               \\
				 
\hline
                       & \mbox{Projectivity}                                                                                                                                         & \mbox{Injectivity}                                                                                                                                          & \mbox{Flatness}                                                                                                                                             \\
\hline
$L_1(S,\mu)$           & \begin{tabular}{@{}c@{}}$\mu$\mbox{ is purely atomic, all } \\ \mbox{ atoms are isolated points } \\ \ref{L1C0SModMetTopRelProjInjFlat}\end{tabular}        & \begin{tabular}{@{}c@{}}$\mu$\mbox{ is any }  \\ \ref{L1C0SModMetTopRelProjInjFlat}\end{tabular}                                                            & \begin{tabular}{@{}c@{}}$\mu$\mbox{ is any }  \\ \ref{L1C0SModMetTopRelProjInjFlat}\end{tabular}                                                            \\
\hline
$L_p(S,\mu)$           & \begin{tabular}{@{}c@{}}$\mu$\mbox{ is purely atomic } \\ \mbox{ with finitely many atoms } \\ ($\implies$) \ref{LpC0SModMetTopRelProjIngFlat}\end{tabular} & \begin{tabular}{@{}c@{}}$\mu$\mbox{ is purely atomic } \\ \mbox{ with finitely many atoms } \\ ($\implies$) \ref{LpC0SModMetTopRelProjIngFlat}\end{tabular} & \begin{tabular}{@{}c@{}}$\mu$\mbox{ is purely atomic } \\ \mbox{ with finitely many atoms } \\ ($\implies$) \ref{LpC0SModMetTopRelProjIngFlat}\end{tabular} \\
\hline
$L_\infty(S,\mu)$      & \begin{tabular}{@{}c@{}} ??? \end{tabular}                                                                                                                  & \begin{tabular}{@{}c@{}}$\mu$\mbox{ is any } \\ \ref{LinftyC0SModMetTopRelProjIngFlat}\end{tabular}                                                         & \begin{tabular}{@{}c@{}} ??? \end{tabular}                                                                                                                  \\
\hline
$M(S)$                 & \begin{tabular}{@{}c@{}}$S$\mbox{ is discrete } \\ \ref{MSC0SModMetTopRelProjIngFlat}\end{tabular}                                                          & \begin{tabular}{@{}c@{}}$S$\mbox{ is any } \\ \ref{MSC0SModMetTopRelProjIngFlat}\end{tabular}                                                             & \begin{tabular}{@{}c@{}}$S$\mbox{ is any } \\ \ref{MSC0SModMetTopRelProjIngFlat}\end{tabular}                                                               \\
\hline
$C_0(S)$               & \begin{tabular}{@{}c@{}}$S$\mbox{ is compact } \\ \ref{C0SC0SModMetTopRelProjIngFlat}\end{tabular}                                                          & \begin{tabular}{@{}c@{}}$S$\mbox{ is Stonean } \\ \ref{C0SC0SModMetTopRelProjIngFlat} \end{tabular}                                                          & \begin{tabular}{@{}c@{}}$S$\mbox{ is any } \\ \ref{C0SC0SModMetTopRelProjIngFlat}\end{tabular}                                                              \\
\hline
$\mathbb{C}_s$         & \begin{tabular}{@{}c@{}}$s$\mbox{ is an isolated point } \\ \ref{OneDimC0SModMetTopRelProjIngFlat}\end{tabular}                                             & \begin{tabular}{@{}c@{}}$s$\mbox{ is any } \\ \ref{OneDimC0SModMetTopRelProjIngFlat}\end{tabular}                                                           & \begin{tabular}{@{}c@{}}$s$\mbox{ is any } \\ \ref{OneDimC0SModMetTopRelProjIngFlat}\end{tabular}                                                           \\
\hline

\multicolumn{4}{c}{\mbox{Homologically trivial $C_0(S)$-modules in topological theory}}                                                                                                                                                                                                                                                                                                                                                                                                                          \\
					 
\hline
                       & \mbox{Projectivity}                                                                                                                                         & \mbox{Injectivity}                                                                                                                                          & \mbox{Flatness}                                                                                                                                             \\
\hline
$L_1(S,\mu)$           & \begin{tabular}{@{}c@{}}$\mu$\mbox{ is purely atomic, all } \\ \mbox{ atoms are isolated points } \\ \ref{L1C0SModMetTopRelProjInjFlat}\end{tabular}        & \begin{tabular}{@{}c@{}}$\mu$\mbox{ is any }  \\ \ref{L1C0SModMetTopRelProjInjFlat}\end{tabular}                                                            & \begin{tabular}{@{}c@{}}$\mu$\mbox{ is any }  \\ \ref{L1C0SModMetTopRelProjInjFlat}\end{tabular}                                                            \\
\hline
$L_p(S,\mu)$           & \begin{tabular}{@{}c@{}}$\mu$\mbox{ is purely atomic } \\ \mbox{ with finitely many atoms } \\ \ref{LpC0SModMetTopRelProjIngFlat}\end{tabular}              & \begin{tabular}{@{}c@{}}$\mu$\mbox{ is purely atomic } \\ \mbox{ with finitely many atoms } \\ \ref{LpC0SModMetTopRelProjIngFlat}\end{tabular}              & \begin{tabular}{@{}c@{}}$\mu$\mbox{ is purely atomic } \\ \mbox{ with finitely many atoms } \\ \ref{LpC0SModMetTopRelProjIngFlat}\end{tabular}              \\
\hline
$L_\infty(S,\mu)$      & \begin{tabular}{@{}c@{}} ??? \end{tabular}                                                                                                                  & \begin{tabular}{@{}c@{}}$\mu$\mbox{ is any } \\ \ref{LinftyC0SModMetTopRelProjIngFlat}\end{tabular}                                                         & \begin{tabular}{@{}c@{}} ??? \end{tabular}                                                                                                                  \\
\hline
$M(S)$                 & \begin{tabular}{@{}c@{}}$S$\mbox{ is discrete } \\ \ref{MSC0SModMetTopRelProjIngFlat}\end{tabular}                                                          & \begin{tabular}{@{}c@{}}$S$\mbox{ is any } \\ \ref{MSC0SModMetTopRelProjIngFlat}\end{tabular}                                                             & \begin{tabular}{@{}c@{}}$S$\mbox{ is any } \\ \ref{MSC0SModMetTopRelProjIngFlat}\end{tabular}                                                               \\
\hline
$C_0(S)$               & \begin{tabular}{@{}c@{}}$S$\mbox{ is compact } \\ \ref{C0SC0SModMetTopRelProjIngFlat}\end{tabular}                                                          & \begin{tabular}{@{}c@{}} ??? \end{tabular}                                                                                                                  & \begin{tabular}{@{}c@{}}$S$\mbox{ is any } \\ \ref{C0SC0SModMetTopRelProjIngFlat}\end{tabular}                                                              \\
\hline
$\mathbb{C}_s$         & \begin{tabular}{@{}c@{}}$s$\mbox{ is an isolated point } \\ \ref{OneDimC0SModMetTopRelProjIngFlat}\end{tabular}                                             & \begin{tabular}{@{}c@{}}$s$\mbox{ is any } \\ \ref{OneDimC0SModMetTopRelProjIngFlat}\end{tabular}                                                           & \begin{tabular}{@{}c@{}}$s$\mbox{ is any } \\ \ref{OneDimC0SModMetTopRelProjIngFlat}\end{tabular}                                                           \\
\hline

\multicolumn{4}{c}{\mbox{Homologically trivial $C_0(S)$-modules in relative theory}}                                                                                                                                                                                                                                                                                                                                                                                                                             \\

\hline
                       & \mbox{Projectivity}                                                                                                                                         & \mbox{Injectivity}                                                                                                                                          & \mbox{Flatness}                                                                                                                                             \\
\hline
$L_1(S,\mu)$           & \begin{tabular}{@{}c@{}}$\mu$\mbox{ is purely atomic, all } \\ \mbox{ atoms are isolated points } \\ \ref{L1C0SModMetTopRelProjInjFlat}\end{tabular}         & \begin{tabular}{@{}c@{}}$\mu$\mbox{ is any }  \\ \ref{L1C0SModMetTopRelProjInjFlat}\end{tabular}                                                            & \begin{tabular}{@{}c@{}}$\mu$\mbox{ is any } \\ \ref{L1C0SModMetTopRelProjInjFlat}, i)\end{tabular}                                                         \\
\hline
$L_p(S,\mu)$           & \begin{tabular}{@{}c@{}}$\mu$\mbox{ is purely atomic, all } \\ \mbox{ atoms are isolated points } \\ \ref{LpC0SModMetTopRelProjIngFlat}\end{tabular}         & \begin{tabular}{@{}c@{}}$\mu$\mbox{ is any } \\ \ref{LpC0SModMetTopRelProjIngFlat}\end{tabular}                                                             & \begin{tabular}{@{}c@{}}$\mu$\mbox{ is any } \\ \ref{LpC0SModMetTopRelProjIngFlat}, i)\end{tabular}                                                         \\
\hline
$L_\infty(S,\mu)$      & \begin{tabular}{@{}c@{}} ??? \end{tabular}                                                                                                                  & \begin{tabular}{@{}c@{}}$\mu$\mbox{ is any } \\ \ref{LinftyC0SModMetTopRelProjIngFlat}\end{tabular}                                                         & \begin{tabular}{@{}c@{}}$\mu$\mbox{ is any } \\ \ref{LinftyC0SModMetTopRelProjIngFlat}, i)\end{tabular}                                                     \\
\hline
$M(S)$                 & \begin{tabular}{@{}c@{}}$S$\mbox{ is discrete } \\ \ref{MSC0SModMetTopRelProjIngFlat}\end{tabular}                                                          & \begin{tabular}{@{}c@{}}$S$\mbox{ is any } \\ \ref{MSC0SModMetTopRelProjIngFlat}\end{tabular}                                                             & \begin{tabular}{@{}c@{}}$S$\mbox{ is any } \\ \ref{MSC0SModMetTopRelProjIngFlat}\end{tabular}                                                               \\
\hline
$C_0(S)$               & \begin{tabular}{@{}c@{}}$S$\mbox{ is paracompact } \\ \ref{C0SC0SModMetTopRelProjIngFlat}\end{tabular}                                                      & \begin{tabular}{@{}c@{}} ???  \end{tabular}                                                                                                                 & \begin{tabular}{@{}c@{}}$S$\mbox{ is any } \\ \ref{C0SC0SModMetTopRelProjIngFlat}, i)\end{tabular}                                                          \\
\hline
$\mathbb{C}_s$         & \begin{tabular}{@{}c@{}}$s$\mbox{ is an isolated point } \\ \ref{OneDimC0SModMetTopRelProjIngFlat}\end{tabular}                                             & \begin{tabular}{@{}c@{}}$s$\mbox{ is any } \\ \ref{OneDimC0SModMetTopRelProjIngFlat}\end{tabular}                                                           & \begin{tabular}{@{}c@{}}$s$\mbox{ is any } \\ \ref{OneDimC0SModMetTopRelProjIngFlat}\end{tabular}                                                           \\
\hline
\end{longtable}
\end{scriptsize}


%----------------------------------------------------------------------------------------
%	Applications to harmonic analysis
%----------------------------------------------------------------------------------------

\section{Applications to modules of harmonic analysis}
\label{SectionApplicationsToModulesOfHarmonicAnalysis}

%----------------------------------------------------------------------------------------
%	Preliminaries on harmonic analysis
%----------------------------------------------------------------------------------------

\subsection{Preliminaries on harmonic analysis}
\label{SectionPreliminariesOnHarmonicAnalysis} 

Let $G$ be a locally compact group. Its identity we shall denote by $e_G$. By well known Haar's theorem [\cite{HewRossAbstrHarmAnalVol1},section 15.8] there exists a unique up to positive constant Borel regular measure $m_G$ which is finite on all compact sets, positive on all open sets and left translation invariant, that is $m_G(sE)=m_G(E)$ for all $s\in G$ and $E\in Bor(G)$. It is called the left Haar measure of group $G$. If $G$ is compact we assume $m_G(G)=1$. If $G$ is infinite and discrete we choose $m_G$ as counting measure. For each $s\in G$ the map $m:Bor(G)\to[0,+\infty]:E\mapsto m_G(Es)$ is also a left Haar measure, so from uniqueness we infer that $m(E)=\Delta_G(s)m_G(E)$ for some $\Delta_G(s)>0$. The function $\Delta_G:G\to(0,+\infty)$ is called the modular function of the group $G$. It is clear that $\Delta_G(st)=\Delta_G(s)\Delta_G(t)$ for all $s,t\in G$. Groups with modular function equal to one are called unimodular. Examples of groups with unimodular function include compact groups, commutative groups and discrete groups. In what follows we use the notation $L_p(G)$ instead of $L_p(G,m_G)$ for $1\leq p\leq+\infty$. For a fixed $s\in G$ we define the left shift operator $L_s:L_1(G)\to L_1(G):f\mapsto(t\mapsto f(s^{-1}t))$ and the right shift operator $R_s:L_1(G)\to L_1(G):f\mapsto (t\mapsto f(ts))$. 

Group structure of $G$ allows us to introduce the Banach algebra structure on $L_1(G)$. For a given $f,g\in L_1(G)$ we define their convolution as
$$
(f\convol g)(s)=\int_G f(t)g(t^{-1}s)dm_G(t)=\int_G f(st)g(t^{-1})dm_G(t)
$$
$$=\int_G f(st^{-1})g(t)\Delta_G(t^{-1})dm_G(t)
$$
for almost all $s\in G$. In this case $L_1(G)$ endowed with convolution product becomes a Banach algebra. The Banach algebra $L_1(G)$ has a contractive two-sided approximate identity consisting of positive compactly supported continuous functions. The algebra $L_1(G)$ is unital iff $G$ is discrete, and in this case $\delta_{e_G}$ is the identity of $L_1(G)$. The group structure of $G$ allows us to turn the Banach space of complex finite Borel regular measures $M(G)$ into the Banach algebra too. We define convolution of two measures $\mu,\nu\in M(G)$ as
$$
(\mu\convol \nu)(E)=\int_G\nu(s^{-1}E)d\mu(s)=\int_G\mu(Es^{-1})d\nu(s)
$$
for all $E\in Bor(G)$. The Banach space $M(G)$ along with this convolution is a unital Banach algebra. The role of identity is played by Dirac delta measure $\delta_{e_G}$ supported on $e_G$. In fact $M(G)$ is a coproduct in $L_1(G)-\mathbf{mod}_1$ (but not in $M(G)-\mathbf{mod}_1$) of two-sided ideal $M_a(G)$ of measures absolutely continuous with respect to $m_G$ and subalgebra $M_s(G)$ of measures singular with respect to $m_G$. Note that $M_a(G)\isom{M(G)-\mathbf{mod}_1}L_1(G)$ and $M_s(G)$ is an annihilator $L_1(G)$-module. Finally, $M(G)=M_a(G)$ iff $G$ is discrete. 

Now we proceed to the discussion of standard left and right modules over $L_1(G)$ and $M(G)$. Since $L_1(G)$ can be regarded as two-sided ideal of $M(G)$ because of isometric left and right $M(G)$-morphism $i:L_1(G)\to M(G):f\mapsto f m_G$ it is enough to define module structure over $M(G)$. For $1\leq p<+\infty$ and any $f\in L_p(G)$, $\mu\in M(G)$ we define
$$
(\mu\convol_p f)(s)=\int_G f(t^{-1}s)d\mu(t),
\qquad\qquad
(f\convol_p \mu)(s)=\int_G f(st^{-1})\Delta_G(t^{-1})^{1/p}d\mu(t)
$$
These module actions turn any Banach space $L_p(G)$ for $1\leq p<+\infty$ into the left and right $M(G)$-module. Note that for $p=1$ and $\mu\in M_a(G)$ we get the usual definition of convolution. For $1<p\leq +\infty$ and any $f\in L_p(G)$, $\mu\in M(G)$ we define module actions
$$
(\mu\cdot_p f)(s)=\int_G \Delta_G(t)^{1/p}f(st)d\mu(t),
\qquad\qquad
(f\cdot_p \mu)(s)=\int_G f(ts)d\mu(t)
$$
These module actions turn any Banach space $L_p(G)$ for $1<p\leq+\infty$ into the left and right $M(G)$-module too. This special choice of module structure nicely interacts with duality. Indeed we have and $(L_p(G),\convol_p)^*\isom{\mathbf{mod}_1-M(G)}(L_{p^*}(G),\cdot_{p^*})$ for all $1\leq p<+\infty$. Finally, the Banach space $C_0(G)$ also becomes left and right $M(G)$-module when endowed with $\cdot_\infty$ in the role of module action. Even more, $C_0(G)$ is a closed left and right $M(G)$-submodule of $L_\infty(G)$ and $(C_0(G),\cdot_\infty)^*\isom{M(G)-\mathbf{mod}_1}(M(G),\convol)$.

A character on a locally compact group $G$ is by definition a continuous homomorphism from $G$ to $\mathbb{T}$. The set of characters on $G$ forms a group denoted by $\widehat{G}$. It becomes a locally compact group when considered with compact open topology. Any character $\gamma\in\widehat{G}$ gives rise to the continuous character $\varkappa_\gamma^L:L_1(G)\to\mathbb{C}:f\mapsto \int_G f(s)\overline{\gamma(s)}dm_G(s)$ on $L_1(G)$. In fact all characters of $L_1(G)$ arise this way. This result is due to Gelfand [\cite{KaniBanAlg}, theorems 2.7.2, 2.7.5]. Similarly, for each $\gamma\in\widehat{G}$ we have a character on $M(G)$ defined by $\varkappa_\gamma^M:M(G)\to\mathbb{C}:\mu\mapsto\int_{G} \overline{\gamma(s)}d\mu(s)$. By $\mathbb{C}_\gamma$ we denote the respective augmentation left and right $L_1(G)$- or $M(G)$-module. Their module actions are defined by
$$
f\cdot_{\gamma}z=z\cdot_{\gamma}f=\varkappa_\gamma^L(f)z
\qquad\qquad
\mu\cdot_{\gamma}z=z\cdot_{\gamma}\mu=\varkappa_\gamma^M(\mu)z
$$
for all $f\in L_1(G)$, $\mu\in M(G)$ and $z\in\mathbb{C}$. 

One of the numerous definitions of amenable group says, that a locally compact group $G$ is amenable if there exists an $L_1(G)$-morphism of right modules $M:L_\infty(G)\to\mathbb{C}_{e_{\widehat{G}}}$ such that $M(\chi_G)=1$ [\cite{HelBanLocConvAlg}, section VII.2.5]. We can even assume that $M$ is contractive [\cite{HelBanLocConvAlg}, remark 7.1.54].

Most of results of this section that not supported with references are presented in a full detail in [\cite{DalBanAlgAutCont}, section 3.3].

%----------------------------------------------------------------------------------------
%	L_1(G)-modules
%----------------------------------------------------------------------------------------

\subsection{\texorpdfstring{$L_1(G)$}{L1(G)}-modules}
\label{SubSectionL1GModules}

Metric homological properties of most of the standard $L_1(G)$-modules of harmonic analysis are studied in \cite{GravInjProjBanMod}. We borrow these ideas to unify approaches to metrical and topological homological properties of modules over group algebras.

\begin{proposition}\label{LInfIsL1MetrInj} Let $G$ be a locally compact group. Then $L_1(G)$ is metrically and topologically flat $L_1(G)$-module, i.e. $L_1(G)$-module $L_\infty(G)$ is metrically and topologically injective.
\end{proposition} 
\begin{proof} Since $L_1(G)$ has contractive approximate identity, then $L_1(G)$ is metrically and topologically flat $L_1(G)$-module by proposition \ref{MetTopFlatIdealsInUnitalAlg}. Since $L_\infty(G)\isom{\mathbf{mod}_1-L_1(G)}L_1(G)^*$, then by proposition \ref{MetTopFlatCharac} it is metrically and topologically injective.
\end{proof}

\begin{proposition}\label{OneDimL1ModMetTopProjCharac} Let $G$ be a locally compact group, and $\gamma\in\widehat{G}$. Then the following are equivalent:

$i)$ $G$ is compact;

$ii)$ $\mathbb{C}_\gamma$ is metrically projective $L_1(G)$-module;

$iii)$ $\mathbb{C}_\gamma$ is topologically projective $L_1(G)$-module.
\end{proposition}
\begin{proof} $i)$$\implies$$ ii)$ Consider $L_1(G)$-morphisms $\sigma^+:\mathbb{C}_\gamma\to L_1(G)_+:z\mapsto z\gamma \oplus_1 0$ and $\pi^+:L_1(G)_+\to\mathbb{C}_\gamma: f\oplus_1 w\to f\cdot_{\gamma}1+w$. One can easily check that $\Vert\pi^+\Vert=\Vert\sigma^+\Vert=1$ and $\pi^+\sigma^+=1_{\mathbb{C}_\gamma}$. Therefore $\mathbb{C}_\gamma$ is a retract of $L_1(G)_+$ in $L_1(G)-\mathbf{mod}_1$. From propositions \ref{UnitalAlgIsMetTopProj} and \ref{RetrMetTopProjIsMetTopProj} it follows that $\mathbb{C}_\gamma$ is metrically projective.

$ii)$$\implies$$ iii)$ See proposition \ref{MetProjIsTopProjAndTopProjIsRelProj}.

$iii)$$\implies$$i)$ Consider $L_1(G)$-morphism $\pi:L_1(G)\to\mathbb{C}_\gamma:f\mapsto f\cdot_{\gamma} 1$. It is easy to see that $\pi$ is strictly coisometric. Since $\mathbb{C}_\gamma$ is topologically projective, then there exists an $L_1(G)$-morphism $\sigma:\mathbb{C}_\gamma\to L_1(G)$ such that $\pi\sigma=1_{\mathbb{C}_\gamma}$. Let $f=\sigma(1)\in L_1(G)$ and $(e_\nu)_{\nu\in N}$ be a standard approximate identity of $L_1(G)$. Since $\sigma$ is an $L_1(G)$-morphism, then for all $s,t\in G$ we have 
$$
f(s^{-1}t)
=L_s(f)(t)
=\lim_\nu L_s(e_\nu\convol \sigma(1))(t)
=\lim_\nu((\delta_s\convol e_\nu)\convol \sigma(1))(t)
=\lim_\nu\sigma((\delta_s\convol e_\nu)\cdot_{\gamma} 1)(t)
$$
$$
=\lim_\nu\sigma(\varkappa_\gamma^L(\delta_s\convol e_\nu))(t)
=\lim_\nu\varkappa_\gamma^L(\delta_s\convol e_\nu)\sigma(1)(t)
=\lim_\nu(e_\nu\convol\gamma)(s^{-1})f(t)
=\gamma(s^{-1})f(t).
$$
Therefore, for the function $g(t):=\gamma(t^{-1})f(t)$ in $L_1(G)$ we have $g(st)=g(t)$ for all $s,t\in G$. Thus $g$ is a constant function in $L_1(G)$, which is possible only for compact group $G$.
\end{proof}

\begin{proposition}\label{OneDimL1ModMetTopInjFlatCharac} Let $G$ be a locally compact group, and $\gamma\in\widehat{G}$. Then the following are equivalent:

$i)$ $G$ is amenable;

$ii)$ $\mathbb{C}_\gamma$ is metrically injective $L_1(G)$-module;

$iii)$ $\mathbb{C}_\gamma$ is topologically injective $L_1(G)$-module.

$iv)$ $\mathbb{C}_\gamma$ is metrically flat $L_1(G)$-module;

$v)$ $\mathbb{C}_\gamma$ is topologically flat $L_1(G)$-module.
\end{proposition}
\begin{proof} $i)$$\implies$$ ii)$ Since $G$ is amenable, then we have contractive $L_1(G)$-morphism $M:L_\infty(G)\to\mathbb{C}_{e_{\widehat{G}}}$ with $M(\chi_G)=1$. Consider linear operators $\rho:\mathbb{C}_\gamma\to L_\infty(G):z\mapsto z\overline{\gamma}$ and $\tau:L_\infty(G)\to\mathbb{C}_\gamma:f\mapsto M(f\gamma)$. These are $L_1(G)$-morphisms of right $L_1(G)$-modules. We shall check this for operator $\tau$: for all $f\in L_\infty(G)$ and $g\in L_1(G)$ we have
$$
\tau(f\cdot_\infty g)
=M((f\cdot_\infty g)\gamma)
=M(f\gamma\cdot_\infty g\overline{\gamma})
=M(f\gamma)\cdot_{e_{\widehat{G}}} g\overline{\gamma}
=M(f\gamma)\varkappa_\gamma^L(g)
=\tau(f)\cdot_{\gamma} g.
$$  
It is easy to check that $\rho$ and $\tau$ are contractive and $\tau\rho=1_{\mathbb{C}_\gamma}$. Therefore $\mathbb{C}_\gamma$ is a retract of $L_\infty(G)$ in $\mathbf{mod}_1-L_1(G)$. From propositions \ref{LInfIsL1MetrInj} and \ref{RetrMetTopInjIsMetTopInj} it follows that $\mathbb{C}_\gamma$ is metrically injective as $L_1(G)$-module.

$ii)$$\implies$$ iii)$ See proposition \ref{MetInjIsTopInjAndTopInjIsRelInj}.

$iii)$$ \implies$$ i)$ Since $\rho$ is an isometric $L_1(G)$-morphism of right $L_1(G)$-modules and $\mathbb{C}_\gamma$ is topologically injective as $L_1(G)$-module, then $\rho$ is a coretraction in $\mathbf{mod}-L_1(G)$. Denote its left inverse morphism by $\pi$, then $\pi(\overline{\gamma})=\pi(\rho(1))=1$. Consider bounded linear functional $M:L_\infty(G)\to\mathbb{C}_\gamma:f\mapsto \pi(f\overline{\gamma})$. For all $f\in L_\infty(G)$ and $g\in L_1(G)$ we have
$$
M(f\cdot_\infty g)
=\pi((f\cdot_\infty g)\overline{\gamma})
=\pi(f\overline{\gamma}\cdot_\infty g\gamma)
=\pi(f\overline{\gamma})\cdot_{\gamma} g\gamma
=M(f)\varkappa_\gamma^L(g\gamma)
=M(f)\cdot_{e_{\widehat{G}}}g.
$$
Therefore $M$ is an $L_1(G)$-morphism, but we also have $M(\chi_G)=\pi(\overline{\gamma})=1$. Therefore $G$ is amenable.

$ii)$ $\Longleftrightarrow$ $iv)$, $iii)$ $\Longleftrightarrow$ $v)$ Note that $\mathbb{C}_\gamma^*\isom{\mathbf{mod}_1-L_1(G)}\mathbb{C}_\gamma$, so all equivalences  follow from three previous paragraphs and proposition \ref{MetTopFlatCharac}.
\end{proof}

In the next proposition we shall study specific ideals of Banach algebra $L_1(G)$. They are of the form $L_1(G)\convol\mu$ for some idempotent measure $\mu$. In fact, this class of ideals in case of commutative compact groups $G$ coincides with those left ideals of $L_1(G)$ that admit a right bounded approximate identity.

\begin{proposition}\label{CommIdealByIdemMeasL1MetTopProjCharac} Let $G$ be a locally compact group and  $\mu\in M(G)$ be an idempotent measure, that is $\mu\convol\mu=\mu$. If the left ideal $I=L_1(G)\convol\mu$ of Banach algebra $L_1(G)$ is topologically projective $L_1(G)$-module, then $\mu=p m_G$, for some $p\in I$.
\end{proposition}
\begin{proof} Let $\phi:I\to L_1(G)$ be arbitrary morphism of left $L_1(G)$-modules. Consider $L_1(G)$-morphism $\phi':L_1(G)\to L_1(G):x\mapsto\phi(x\convol\mu)$. By Wendel's theorem [\cite{WendLeftCentrzrs}, theorem 1], there exists a measure $\nu\in M(G)$ such that $\phi'(x)=x\convol\nu$ for all $x\in L_1(G)$. In particular, $\phi(x)=\phi(x\convol\mu)=\phi'(x)=x\convol\nu$ for all $x\in I$. It is now clear that $\psi:I\to I:x\mapsto\nu\convol x$ is a morphism of right $I$-modules satisfying $\phi(x)y=x\psi(y)$ for all $x,y\in I$. By paragraph $ii)$ of lemma \ref{GoodIdealMetTopProjIsUnital} the ideal $I$ has a right identity, say $e\in I$. Then $x\convol\mu=x\convol\mu\convol e$ for all $x\in L_1(G)$. Two measures are equal if their convolutions with all functions of $L_1(G)$ coincide [\cite{DalBanAlgAutCont}, corollary 3.3.24], so $\mu=\mu\convol e m_G$. Since $e\in I\subset L_1(G)$, then $\mu=\mu\convol e m_G\in M_a(G)$. Set $p=\mu\convol e\in I$, then $\mu=p m_G$.
\end{proof}

We conjecture that the left ideal $L_1(G)\convol \mu$ for idempotent measure $\mu$ is metrically projective $L_1(G)$-module iff $\mu=p m_G$ where $p\in I$ and $\Vert p\Vert=1$.

\begin{theorem}\label{L1ModL1MetTopProjCharac} Let $G$ be a locally compact group. Then the following are equivalent:

$i)$ $G$ is discrete;

$ii)$ $L_1(G)$ is metrically projective $L_1(G)$-module;

$iii)$ $L_1(G)$ is topologically projective $L_1(G)$-module.
\end{theorem}
\begin{proof} $i)$$\implies$$ ii)$ If $G$ is discrete, then $L_1(G)$ is unital with unit of norm $1$. By  proposition \ref{UnIdeallIsMetTopProj} we see that $L_1(G)$ is metrically projective as $L_1(G)$-module.

$ii)$$\implies$$ iii)$ See proposition \ref{MetProjIsTopProjAndTopProjIsRelProj}.

$iii)$$ \implies$$ i)$ Clearly, $\delta_{e_G}$ is an idempotent measure. Since $L_1(G)=L_1(G)\convol \delta_{e_G}$ is topologically projective, then by proposition \ref{CommIdealByIdemMeasL1MetTopProjCharac} we have $\delta_{e_G}=f m_G$ for some $f\in L_1(G)$. This is possible only if $G$ is discrete.
\end{proof}

Note that $L_1(G)$-module $L_1(G)$ is relatively projective for any locally compact group $G$ [\cite{HelBanLocConvAlg}, exercise 7.1.17].

\begin{proposition}\label{L1MetTopProjAndMetrFlatOfMeasAlg} Let $G$ be a locally compact group. Then the following are equivalent:

$i)$ $G$ is discrete;

$ii)$ $M(G)$ is metrically projective $L_1(G)$-module;

$iii)$ $M(G)$ is topologically projective $L_1(G)$-module;

$iv)$ $M(G)$ is metrically flat $L_1(G)$-module.
\end{proposition}
\begin{proof} 
$i)$$\implies$$ ii)$ We have $M(G)\isom{L_1(G)-\mathbf{mod}_1}L_1(G)$ for discrete $G$, so the result follows from theorem \ref{L1ModL1MetTopProjCharac}. 

$ii)$$\implies$$ iii)$ See proposition \ref{MetProjIsTopProjAndTopProjIsRelProj}.

$ii)$$\implies$$ iv)$ See proposition \ref{MetTopProjIsMetTopFlat}.

$iii)$$\implies$$ i)$ Note that $M(G)\isom{L_1(G)-\mathbf{mod}_1} L_1(G)\bigoplus_1 M_s(G)$, so $M_s(G)$ is topologically projective by proposition \ref{MetTopProjModCoprod}. Note that $M_s(G)$ is an annihilator $L_1(G)$-module, then by proposition \ref{MetTopProjOfAnnihModCharac} the algebra $L_1(G)$ has a right identity. Recall that $L_1(G)$ also has a two-sided bounded approximate identity, so $L_1(G)$ is unital. The last is equivalent to $G$ being discrete.

$iv)$$\implies$$ i)$ Note that $M(G)\isom{L_1(G)-\mathbf{mod}_1} L_1(G)\bigoplus_1 M_s(G)$, so $M_s(G)$ is metrically flat by proposition \ref{MetTopFlatModCoProd}. Note that $M_s(G)$ is an annihilator $L_1(G)$-module, then by proposition \ref{MetTopFlatAnnihModCharac} it is equal to zero. The last is equivalent to $G$ being discrete.
\end{proof}

\begin{proposition}\label{MeasAlgIsL1TopFlat} Let $G$ be a locally compact group. Then $M(G)$ is topologically flat $L_1(G)$-module.
\end{proposition}
\begin{proof} Since $M(G)$ is an $L_1$-space it is a fortiori an $\mathscr{L}_1^g$-space. Since $M_s(G)$ is complemented in $M(G)$, then $M_s(G)$ is an $\mathscr{L}_1^g$-space too [\cite{DefFloTensNorOpId}, corollary 23.2.1(2)]. Since $M_s(G)$ is an annihilator $L_1(G)$-module, then from proposition \ref{MetTopFlatAnnihModCharac} we have that $M_s(G)$ is topologically flat $L_1(G)$-module. The $L_1(G)$-module $L_1(G)$ is also topologically flat by proposition \ref{LInfIsL1MetrInj} . Since $M(G)\isom{L_1(G)-\mathbf{mod}_1}L_1(G)\bigoplus_1 M_s(G)$, then $M(G)$ is topologically flat $L_1(G)$-module by proposition \ref{MetTopFlatModCoProd}.
\end{proof}

%----------------------------------------------------------------------------------------
%	M(G)-modules
%----------------------------------------------------------------------------------------

\subsection{\texorpdfstring{$M(G)$}{M(G)}-modules}
\label{SubSectionMGModules}

We turn to the study of standard $M(G)$-modules of harmonic analysis. As we shall see most of results can be derived from results on $L_1(G)$-modules.

\begin{proposition}\label{MGMetTopProjInjFlatRedToL1} Let $G$ be a locally compact group, and $X$ be $\langle$~essential / faithful / essential~$\rangle$ $L_1(G)$-module. Then

$i)$ $X$ is metrically $\langle$~projective / injective / flat~$\rangle$ $M(G)$-module iff it is metrically $\langle$~projective / injective / flat~$\rangle$ $L_1(G)$-module;

$ii)$ $X$ is topologically $\langle$~projective / injective / flat~$\rangle$ $M(G)$-module iff it is topologically $\langle$~projective / injective / flat~$\rangle$ $L_1(G)$-module.
\end{proposition}
\begin{proof} Recall that $L_1(G)\isom{L_1(G)-\mathbf{mod}_1}M_a(G)$ is a two-sided complemented in $\mathbf{Ban}_1$ ideal of $M(G)$. Now $i)$ and $ii)$ follow from proposition $\langle$~\ref{MetTopProjUnderChangeOfAlg} / \ref{MetTopInjUnderChangeOfAlg}  / \ref{MetTopFlatUnderChangeOfAlg}~$\rangle$.
\end{proof} 

It is worth to mention here that the $L_1(G)$-modules $C_0(G)$, $L_p(G)$ for $1\leq p<\infty$ and $\mathbb{C}_\gamma$ for $\gamma\in\widehat{G}$ are essential and $L_1(G)$-modules $C_0(G)$, $M(G)$, $L_p(G)$ for $1\leq p\leq \infty$ and $\mathbb{C}_\gamma$ for $\gamma\in\widehat{G}$ are faithful. 

\begin{proposition}\label{MGModMGMetTopProjFlatCharac} Let $G$ be a locally compact group. Then $M(G)$ is metrically and topologically projective $M(G)$-module. As the consequence it is metrically and topologically flat $M(G)$-module.
\end{proposition} 
\begin{proof} Since $M(G)$ is a unital algebra, the metric and topological projectivity of $M(G)$ follow from proposition \ref{UnitalAlgIsMetTopProj}. It remains to apply proposition \ref{MetTopProjIsMetTopFlat}.
\end{proof}

%----------------------------------------------------------------------------------------
%	Banach geometric restrictions
%----------------------------------------------------------------------------------------

\subsection{Banach geometric restriction}
\label{SubSectionBanachGeometricRestriction}

In this section we shall show that many modules of harmonic analysis are fail to be metrically or topologically projective, injective or flat for purely Banach geometric reasons. 

\begin{proposition}\label{StdModAreNotRetrOfL1LInf} Let $G$ be an infinite locally compact group. Then

$i)$ $L_1(G)$, $C_0(G)$, $M(G)$, $L_\infty(G)^*$ are not topologically injective Banach spaces;

$ii)$ $C_0(G)$, $L_\infty(G)$ are not complemented in any $L_1$-space.
\end{proposition}
\begin{proof}
Since $G$ is infinite all modules in question are infinite dimensional.

$i)$ If an infinite dimensional Banach space is topologically injective, then it contains a copy of $\ell_\infty(\mathbb{N})$ [\cite{RosOnRelDisjFamOfMeas}, corollary 1.1.4], and consequently a copy of $c_0(\mathbb{N})$. The Banach space $L_1(G)$ is weakly sequentially complete [\cite{WojBanSpForAnalysts}, corollary III.C.14], so by corollary 5.2.11 in \cite{KalAlbTopicsBanSpTh} it can't contain a copy of $c_0(\mathbb{N})$. Therefore, $L_1(G)$ is not topologically injective Banach space.  If $M(G)$ is topologically injective Banach space, then so does its complemented subspace $M_a(G)\isom{\mathbf{Ban}_1}L_1(G)$. By previous argument this is impossible. So $M(G)$ is not topologically injective as Banach space. By corollary 3 of \cite{LauMingComplSubspInLInfOfG} the space $C_0(G)$ is not complemented in $L_\infty(G)$. Then $C_0(G)$ can't be topologically injective either. The Banach space $L_1(G)$ is complemented in $L_\infty(G)^*\isom{\mathbf{Ban}_1}L_1(G)^{**}$ [\cite{DefFloTensNorOpId}, proposition  B10]. Therefore if $L_\infty(G)^*$ is topologically injective as Banach space, then so does its retract $L_1(G)$. By previous arguments this is impossible, so $L_\infty(G)^*$ is not topologically injective Banach space.

$ii)$ If $C_0(G)$ is a retract of $L_1$-space, then $M(G)\isom{\mathbf{Ban}_1}C_0(G)^*$ is a retract of $L_\infty$-space, so it must be a topologically injective Banach space. This contradicts paragraph $i)$, so $C_0(G)$ is not a retract of $L_1$-space. Note that $\ell_\infty(\mathbb{N})$ embeds in $L_\infty(G)$, then so does $c_0(\mathbb{N})$. So if $L_\infty(G)$ is a retract of $L_1$-space, then there would exist an $L_1$-space containing a copy of $c_0(\mathbb{N})$. This is impossible as already showed in paragraph $i)$.
\end{proof}

From now on by $A$ we denote either $L_1(G)$ or $M(G)$. Recall that $L_1(G)$ and $M(G)$ are both $L_1$-spaces.

\begin{proposition}\label{StdModAreNotL1MGMetTopProjInjFlat} Let $G$ be an infinite locally compact group. Then

$i)$ $C_0(G)$, $L_\infty(G)$ are neither topologically nor metrically projective $A$-modules;

$ii)$ $L_1(G)$, $C_0(G)$, $M(G)$, $L_\infty(G)^*$ are neither topologically nor metrically injective $A$-modules;

$iii)$ $L_\infty(G)$, $C_0(G)$ are neither topologically nor metrically flat $A$-modules.
\end{proposition}

$iv)$ $L_p(G)$ for $1<p<\infty$ are neither topologically nor metrically projective, injective or flat $A$-flat.

\begin{proof} $i)$ The result follows from propositions \ref{TopProjInjFlatModOverL1Charac} paragraph $i)$ and  \ref{StdModAreNotRetrOfL1LInf} paragraph $ii)$.

$ii)$ The result follows from propositions \ref{TopProjInjFlatModOverL1Charac} paragraph $ii)$ and \ref{StdModAreNotRetrOfL1LInf}.

$iii)$ Note that $C_0(G)^*\isom{\mathbf{mod}_1-A}M(G)$. Now the result follows from paragraph $i)$ and proposition \ref{MetTopFlatCharac}.

$iv)$ Since $L_p(G)$ is reflexive for $1<p<\infty$ the result follows from \ref{NoInfDimRefMetTopProjInjFlatModOverMthscrL1OrLInfty}.
\end{proof}

It remains to consider metric and topological homological properties of $A$-modules when $G$ is finite.

\begin{proposition}\label{LpFinGrL1MGMetrInjProjCharac} Let $G$ be a non trivial finite group and $1\leq p\leq \infty$. Then the $A$-module $L_p(G)$ is metrically $\langle$~projective / injective~$\rangle$ iff $\langle$~$p=1$ / $p=\infty$~$\rangle$
\end{proposition}
\begin{proof} 
Assume $L_p(G)$ is metrically $\langle$~projective / injective~$\rangle$ as $A$-module. Since $L_p(G)$ is finite dimensional, then by paragraphs $i)$ and $ii)$ of proposition \ref{TopProjInjFlatModOverL1Charac} we have identifications $\langle$~$L_p(G)\isom{\mathbf{Ban}_1}\ell_1(\mathbb{N}_n)$ / $L_p(G)\isom{\mathbf{Ban}_1}C(\mathbb{N}_n)\isom{\mathbf{Ban}_1}\ell_\infty(\mathbb{N}_n)$ ~$\rangle$, where $n=\operatorname{Card}(G)>1$. Now we use the result of theorem 1 from \cite{LyubIsomEmdbFinDimLp} for Banach spaces over field $\mathbb{C}$: if for $2\leq m\leq k$ and $1\leq r,s\leq \infty$, there exists an isometric embedding from $\ell_r(\mathbb{N}_m)$ into $\ell_s(\mathbb{N}_k)$, then either $r=2$, $s\in 2\mathbb{N}$ or $r=s$. Therefore $\langle$~$p=1$ / $p=\infty$~$\rangle$. The converse easily follows from $\langle$~theorem \ref{L1ModL1MetTopProjCharac} / proposition \ref{LInfIsL1MetrInj}~$\rangle$
\end{proof}

\begin{proposition}\label{StdModFinGrL1MGMetrInjProjFlatCharac} Let $G$ be a finite group. Then

$i)$ $C_0(G)$, $L_\infty(G)$ are metrically injective $A$-modules;

$ii)$ $C_0(G)$, $L_p(G)$ for $1<p\leq\infty$ are metrically projective $A$-modules iff $G$ is trivial;

$iii)$ $M(G)$, $L_p(G)$ for $1\leq p<\infty$ are metrically injective  $A$-modules iff $G$ is trivial;

$iv)$ $C_0(G)$, $L_p(G)$ for $1<p\leq\infty$ are metrically flat $A$-modules iff $G$ is trivial.
\end{proposition}
\begin{proof}
$i)$ Since $G$ is finite then $C_0(G)=L_\infty(G)$. The result follows from proposition \ref{LInfIsL1MetrInj}.

$ii)$ If $G$ is trivial, that is $G=\{e_G\}$, then $L_p(G)=C_0(G)=L_1(G)$ and the result follows from paragraph $i)$. If $G$ is non trivial, then we recall that $C_0(G)=L_\infty(G)$ and use proposition \ref{LpFinGrL1MGMetrInjProjCharac}.

$iii)$ If $G=\{e_G\}$, then $M(G)=L_p(G)=L_\infty(G)$ and the result follows from paragraph $i)$. If $G$ is non trivial, then we note that $M(G)=L_1(G)$ and use proposition \ref{LpFinGrL1MGMetrInjProjCharac}.

$iv)$ From paragraph $iii)$ it follows that $L_p(G)$ for $1\leq p<\infty$ is metrically injective $A$-module iff $G$ is trivial. Now the result follows from proposition \ref{MetTopFlatCharac} and the facts that $C_0(G)^*\isom{\mathbf{mod}_1-L_1(G)}M(G)\isom{\mathbf{mod}_1-L_1(G)}L_1(G)$, $L_p(G)^*\isom{\mathbf{mod}_1-L_1(G)}L_{p^*}(G)$ for $1\leq p^*<\infty$.
\end{proof}

It is worth to mention here that if we would consider all Banach spaces over the field of real numbers, then $L_\infty(G)$ and $L_1(G)$ would be metrically projective and injective respectively,  additionally for $G$ consisting of two elements, because
$$
L_\infty(\mathbb{Z}_2)\isom{L_1(\mathbb{Z}_2)-\mathbf{mod}_1}\mathbb{R}_{\gamma_0}\bigoplus\nolimits_1\mathbb{R}_{\gamma_1},
\qquad
L_1(\mathbb{Z}_2)\isom{L_1(\mathbb{Z}_2)-\mathbf{mod}_1}\mathbb{R}_{\gamma_0}\bigoplus\nolimits_\infty\mathbb{R}_{\gamma_1}
$$
for $\gamma_0,\gamma_1\in\widehat{\mathbb{Z}_2}$ defined by $\gamma_0(0)=\gamma_0(1)=\gamma_1(0)=-\gamma_1(1)=1$. Here $\mathbb{Z}_2$ denotes the unique group of two elements.

\begin{proposition}\label{StdModFinGrL1MGTopInjProjFlatCharac} Let $G$ be a finite group. Then the $A$-modules $C_0(G)$, $M(G)$, $L_p(G)$ for $1\leq p\leq \infty$ are both topologically projective, injective and flat.
\end{proposition} 
\begin{proof}
For finite group $G$ we have $M(G)=L_1(G)$ and $C_0(G)=L_\infty(G)$, so these modules do not require special considerations. Since $M(G)=L_1(G)$, we can restrict our considerations to the case $A=L_1(G)$. The identity map $i:L_1(G)\to L_p(G):f\mapsto f$ is a topological isomorphism of Banach spaces, because $L_1(G)$ and $L_p(G)$ for $1\leq p<+\infty$ are of equal finite dimension. Since $G$ is finite, it is unimodular. Therefore, the module actions in $(L_1(G),\convol)$ and $(L_p(G),\convol_p)$ coincide for $1\leq p<+\infty$ and $i$ is an isomorphism in $L_1(G)-\mathbf{mod}$ and $\mathbf{mod}-L_1(G)$. Similarly one can show that $(L_\infty(G),\cdot_\infty)$ and $(L_p(G),\cdot_p)$ for $1<p\leq+\infty$ are isomorphic in $L_1(G)-\mathbf{mod}$ and $\mathbf{mod}-L_1(G)$. Finally, one can easily check that $(L_1(G),\convol)$ and $(L_\infty(G),\cdot_\infty)$ are isomorphic in $L_1(G)-\mathbf{mod}$ and $\mathbf{mod}-L_1(G)$ via the map $j:L_1(G)\to L_\infty(G):f\mapsto(s\mapsto f(s^{-1}))$. Therefore all the discussed modules are isomorphic. It remains to recall that $L_1(G)$ is topologically projective and flat by theorem \ref{L1ModL1MetTopProjCharac} and proposition \ref{LInfIsL1MetrInj}, while $L_\infty(G)$ is topologically injective by proposition \ref{LInfIsL1MetrInj}.
\end{proof}

Now we can summarize results on homological properties of modules of harmonic analysis into three tables. Each cell of the table contains a condition under which the respective module has respective property and propositions where this is proved. We shall mention that results for modules $L_p(G)$ are valid for both module actions $\convol_p$ and $\cdot_p$. Characterization and proofs for homologically trivial modules $\mathbb{C}_\gamma$ in case of relative theory is the same as in propositions \ref{OneDimL1ModMetTopProjCharac}, \ref{OneDimL1ModMetTopInjFlatCharac} and \ref{OneDimL1ModMetTopInjFlatCharac}. As usually, the arrow $\implies$ indicates that only a necessary conditions is known. As we showed above even topological theory is too restrictive for $L_1(G)$ to be projective as $L_1(G)$-module. Similarly a Banach space is topologically projective iff it is an $L_1$-space, and the underlying measure space is atomic. This analogy confirms important role of Banach geometry in metric and topological Banach homology.

\bigskip

\begin{scriptsize}
\begin{longtable}{|c|c|c|c|c|c|c|} 
\multicolumn{7}{c}{\mbox{Homologically trivial $L_1(G)$- and $M(G)$-modules in metric theory}}                                                                                                                                                                                                                                                                                                                                                                                                                                                                                                                                                                                                                                                                                                                                                                                                                                                                                                                                             \\
				 
\hline            & \multicolumn{3}{c|}{$L_1(G)$-modules}                                                                                                                                                                                                                                                                                                                                                                                                                                                                      & \multicolumn{3}{c|}{$M(G)$-modules}                                                                                                                                                                                                                                                                                                                                                                                                                                                                  \\
\hline
                  & \mbox{Projectivity}                                                                                                                                               & \mbox{Injectivity}                                                                                                                                                & \mbox{Flatness}                                                                                                                                                    & \mbox{Projectivity}                                                                                                                                               & \mbox{Injectivity}                                                                                                                                                & \mbox{Flatness}                                                                                                                                                   \\ 
\hline
 $L_1(G)$           & \begin{tabular}{@{}c@{}}$G$\mbox{ is discrete } \\ \ref{L1ModL1MetTopProjCharac}\end{tabular}                                                                     & \begin{tabular}{@{}c@{}}$G=\{e_G\}$ \\ \ref{StdModAreNotL1MGMetTopProjInjFlat}, \ref{StdModFinGrL1MGMetrInjProjFlatCharac}\end{tabular}                                  & \begin{tabular}{@{}c@{}}$G$\mbox{ is any } \\ \ref{LInfIsL1MetrInj}\end{tabular}                                                                                   & \begin{tabular}{@{}c@{}}$G$\mbox{ is discrete } \\ \ref{L1ModL1MetTopProjCharac},\ref{MGMetTopProjInjFlatRedToL1}\end{tabular}                                   & \begin{tabular}{@{}c@{}}$G=\{e_G\}$ \\ \ref{StdModAreNotL1MGMetTopProjInjFlat}, \ref{StdModFinGrL1MGMetrInjProjFlatCharac}\end{tabular}                                   & \begin{tabular}{@{}c@{}}$G$\mbox{ is any } \\ \ref{LInfIsL1MetrInj},\ref{MGMetTopProjInjFlatRedToL1}\end{tabular}                                                 \\ 
\hline
 $L_p(G)$           & \begin{tabular}{@{}c@{}}$G=\{e_G\}$ \\ \ref{StdModAreNotL1MGMetTopProjInjFlat},\ref{LpFinGrL1MGMetrInjProjCharac}\end{tabular}                  & \begin{tabular}{@{}c@{}}$G=\{e_G\}$ \\ \ref{StdModAreNotL1MGMetTopProjInjFlat},\ref{LpFinGrL1MGMetrInjProjCharac}\end{tabular}                  & \begin{tabular}{@{}c@{}}$G=\{e_G\}$ \\ \ref{StdModAreNotL1MGMetTopProjInjFlat},\ref{StdModFinGrL1MGMetrInjProjFlatCharac}\end{tabular}           & \begin{tabular}{@{}c@{}}$G=\{e_G\}$ \\ \ref{StdModAreNotL1MGMetTopProjInjFlat},\ref{LpFinGrL1MGMetrInjProjCharac}\end{tabular}                 & \begin{tabular}{@{}c@{}}$G=\{e_G\}$ \\ \ref{StdModAreNotL1MGMetTopProjInjFlat},\ref{LpFinGrL1MGMetrInjProjCharac}\end{tabular}                   & \begin{tabular}{@{}c@{}}$G=\{e_G\}$ \\ \ref{StdModAreNotL1MGMetTopProjInjFlat},\ref{StdModFinGrL1MGMetrInjProjFlatCharac}\end{tabular}          \\
\hline
 $L_\infty(G)$      & \begin{tabular}{@{}c@{}}$G=\{e_G\}$ \\ \ref{StdModAreNotL1MGMetTopProjInjFlat},\ref{LpFinGrL1MGMetrInjProjCharac}\end{tabular}                                           & \begin{tabular}{@{}c@{}}$G$\mbox{ is any } \\ \ref{LInfIsL1MetrInj}\end{tabular}                                                                                  & \begin{tabular}{@{}c@{}}$G=\{e_G\}$ \\ \ref{StdModAreNotL1MGMetTopProjInjFlat},\ref{StdModFinGrL1MGMetrInjProjFlatCharac}\end{tabular}                                    & \begin{tabular}{@{}c@{}}$G=\{e_G\}$ \\ \ref{StdModAreNotL1MGMetTopProjInjFlat},\ref{LpFinGrL1MGMetrInjProjCharac}\end{tabular}                                          & \begin{tabular}{@{}c@{}}$G$\mbox{ is any } \\ \ref{LInfIsL1MetrInj},\ref{MGMetTopProjInjFlatRedToL1}\end{tabular}                                                  & \begin{tabular}{@{}c@{}}$G=\{e_G\}$ \\ \ref{StdModAreNotL1MGMetTopProjInjFlat},\ref{StdModFinGrL1MGMetrInjProjFlatCharac}\end{tabular}                                   \\ 
\hline
$M(G)$              & \begin{tabular}{@{}c@{}}$G$\mbox{ is discrete } \\ \ref{L1MetTopProjAndMetrFlatOfMeasAlg}\end{tabular}                                                            & \begin{tabular}{@{}c@{}}$G=\{e_G\}$ \\ \ref{StdModAreNotL1MGMetTopProjInjFlat},\ref{StdModFinGrL1MGMetrInjProjFlatCharac}\end{tabular}                                   & \begin{tabular}{@{}c@{}}$G$\mbox{ is discrete } \\ \ref{MeasAlgIsL1TopFlat}\end{tabular}                                                                           & \begin{tabular}{@{}c@{}}$G$\mbox{ is any } \\ \ref{MGModMGMetTopProjFlatCharac}\end{tabular}                                                                     & \begin{tabular}{@{}c@{}}$G=\{e_G\}$ \\ \ref{StdModAreNotL1MGMetTopProjInjFlat},\ref{StdModFinGrL1MGMetrInjProjFlatCharac}\end{tabular}                                    & \begin{tabular}{@{}c@{}}$G$\mbox{ is any } \\ \ref{MGModMGMetTopProjFlatCharac}\end{tabular}                                                                      \\ 
\hline
$C_0(G)$            & \begin{tabular}{@{}c@{}}$G=\{e_G\}$ \\ \ref{StdModAreNotL1MGMetTopProjInjFlat},\ref{StdModFinGrL1MGMetrInjProjFlatCharac}\end{tabular}                                   & \begin{tabular}{@{}c@{}}$G$\mbox{ is finite } \\ \ref{StdModAreNotL1MGMetTopProjInjFlat},\ref{StdModFinGrL1MGMetrInjProjFlatCharac}\end{tabular}                         & \begin{tabular}{@{}c@{}}$G=\{e_G\}$ \\ \ref{StdModAreNotL1MGMetTopProjInjFlat},\ref{StdModFinGrL1MGMetrInjProjFlatCharac}\end{tabular}                                    & \begin{tabular}{@{}c@{}}$G=\{e_G\}$ \\ \ref{StdModAreNotL1MGMetTopProjInjFlat},\ref{StdModFinGrL1MGMetrInjProjFlatCharac}\end{tabular}                                  & \begin{tabular}{@{}c@{}}$G$\mbox{ is finite } \\ \ref{StdModAreNotL1MGMetTopProjInjFlat},\ref{StdModFinGrL1MGMetrInjProjFlatCharac}\end{tabular}                          & \begin{tabular}{@{}c@{}}$G=\{e_G\}$ \\ \ref{StdModAreNotL1MGMetTopProjInjFlat},\ref{StdModFinGrL1MGMetrInjProjFlatCharac}\end{tabular}                                   \\ 
\hline          
$\mathbb{C}_\gamma$ & \begin{tabular}{@{}c@{}}$G$\mbox{ is compact } \\ \ref{OneDimL1ModMetTopProjCharac}\end{tabular}                                                                  & \begin{tabular}{@{}c@{}}$G$\mbox{ is amenable } \\ \ref{OneDimL1ModMetTopInjFlatCharac}\end{tabular}                                                              & \begin{tabular}{@{}c@{}}$G$\mbox{ is amenable } \\ \ref{OneDimL1ModMetTopInjFlatCharac}\end{tabular}                                                               & \begin{tabular}{@{}c@{}}$G$\mbox{ is compact } \\ \ref{OneDimL1ModMetTopProjCharac},\ref{MGMetTopProjInjFlatRedToL1}\end{tabular}                                & \begin{tabular}{@{}c@{}}$G$\mbox{ is amenable } \\ \ref{OneDimL1ModMetTopInjFlatCharac},\ref{MGMetTopProjInjFlatRedToL1}\end{tabular}                              & \begin{tabular}{@{}c@{}}$G$\mbox{ is amenable } \\ \ref{OneDimL1ModMetTopInjFlatCharac},\ref{MGMetTopProjInjFlatRedToL1}\end{tabular}                             \\ 
\hline
\multicolumn{7}{c}{\mbox{Homologically trivial $L_1(G)$- and $M(G)$-modules in topological theory}}                                                                                                                                                                                                                                                                                                                                                                                                                                                                                                                                                                                                                                                                                                                                                                                                                                                                                                                                        \\
					 
\hline            & \multicolumn{3}{c|}{$L_1(G)$-modules}                                                                                                                                                                                                                                                                                                                                                                                                                                                                     & \multicolumn{3}{c|}{$M(G)$-modules}                                                                                                                                                                                                                                                                                                                                                                                                                                                                  \\
\hline
                  & \mbox{Projectivity}                                                                                                                                               & \mbox{Injectivity}                                                                                                                                                & \mbox{Flatness}                                                                                                                                                    & \mbox{Projectivity}                                                                                                                                               & \mbox{Injectivity}                                                                                                                                                & \mbox{Flatness}                                                                                                                                                   \\ 
\hline
$L_1(G)$            & \begin{tabular}{@{}c@{}}$G$\mbox{ is discrete } \\ \ref{L1ModL1MetTopProjCharac}\end{tabular}                                                                     & \begin{tabular}{@{}c@{}}$G$\mbox{ is finite } \\ \ref{StdModAreNotL1MGMetTopProjInjFlat}, \ref{StdModFinGrL1MGTopInjProjFlatCharac}\end{tabular}                         & \begin{tabular}{@{}c@{}}$G$\mbox{ is any } \\ \ref{LInfIsL1MetrInj}\end{tabular}                                                                                   & \begin{tabular}{@{}c@{}}$G$\mbox{ is discrete } \\ \ref{L1ModL1MetTopProjCharac},\ref{MGMetTopProjInjFlatRedToL1}\end{tabular}                                    & \begin{tabular}{@{}c@{}}$G$\mbox{ is finite } \\ \ref{StdModAreNotL1MGMetTopProjInjFlat}, \ref{StdModFinGrL1MGTopInjProjFlatCharac}\end{tabular}                         & \begin{tabular}{@{}c@{}}$G$\mbox{ is any } \\ \ref{LInfIsL1MetrInj},\ref{MGMetTopProjInjFlatRedToL1}\end{tabular}                                                 \\ 
\hline
 $L_p(G)$           & \begin{tabular}{@{}c@{}}$G$\mbox{ is finite } \\ \ref{StdModAreNotL1MGMetTopProjInjFlat},\ref{StdModFinGrL1MGTopInjProjFlatCharac}\end{tabular} & \begin{tabular}{@{}c@{}}$G$\mbox{ is finite } \\ \ref{StdModAreNotL1MGMetTopProjInjFlat},\ref{StdModFinGrL1MGTopInjProjFlatCharac}\end{tabular} & \begin{tabular}{@{}c@{}}$G$\mbox{ is finite } \\ \ref{StdModAreNotL1MGMetTopProjInjFlat},\ref{StdModFinGrL1MGTopInjProjFlatCharac}\end{tabular}  & \begin{tabular}{@{}c@{}}$G$\mbox{ is finite } \\ \ref{StdModAreNotL1MGMetTopProjInjFlat},\ref{StdModFinGrL1MGTopInjProjFlatCharac}\end{tabular} & \begin{tabular}{@{}c@{}}$G$\mbox{ is finite } \\ \ref{StdModAreNotL1MGMetTopProjInjFlat},\ref{StdModFinGrL1MGTopInjProjFlatCharac}\end{tabular} & \begin{tabular}{@{}c@{}}$G$\mbox{ is finite } \\ \ref{StdModAreNotL1MGMetTopProjInjFlat},\ref{StdModFinGrL1MGTopInjProjFlatCharac}\end{tabular} \\ 
\hline
 $L_\infty(G)$      & \begin{tabular}{@{}c@{}}$G$\mbox{ is finite } \\ \ref{StdModAreNotL1MGMetTopProjInjFlat},\ref{StdModFinGrL1MGTopInjProjFlatCharac}\end{tabular}                          & \begin{tabular}{@{}c@{}}$G$\mbox{ is any } \\ \ref{LInfIsL1MetrInj}\end{tabular}                                                                                  & \begin{tabular}{@{}c@{}}$G$\mbox{ is finite } \\ \ref{StdModAreNotL1MGMetTopProjInjFlat},\ref{StdModFinGrL1MGTopInjProjFlatCharac}\end{tabular}                           & \begin{tabular}{@{}c@{}}$G$\mbox{ is finite } \\ \ref{StdModAreNotL1MGMetTopProjInjFlat},\ref{StdModFinGrL1MGTopInjProjFlatCharac}\end{tabular}                          & \begin{tabular}{@{}c@{}}$G$\mbox{ is any } \\ \ref{LInfIsL1MetrInj},\ref{MGMetTopProjInjFlatRedToL1}\end{tabular}                                                 & \begin{tabular}{@{}c@{}}$G$\mbox{ is finite } \\ \ref{StdModAreNotL1MGMetTopProjInjFlat},\ref{StdModFinGrL1MGTopInjProjFlatCharac}\end{tabular}                          \\ 
\hline
$M(G)$              & \begin{tabular}{@{}c@{}}$G$\mbox{ is discrete } \\ \ref{L1MetTopProjAndMetrFlatOfMeasAlg}\end{tabular}                                                            & \begin{tabular}{@{}c@{}}$G$\mbox{ is finite } \\ \ref{StdModAreNotL1MGMetTopProjInjFlat},\ref{StdModFinGrL1MGTopInjProjFlatCharac}\end{tabular} & \begin{tabular}{@{}c@{}}$G$\mbox{ is any } \\ \ref{MeasAlgIsL1TopFlat}\end{tabular}                                                                                & \begin{tabular}{@{}c@{}}$G$\mbox{ is any } \\ \ref{MGModMGMetTopProjFlatCharac}\end{tabular}                                                                      & \begin{tabular}{@{}c@{}}$G$\mbox{ is finite } \\ \ref{StdModAreNotL1MGMetTopProjInjFlat},\ref{StdModFinGrL1MGTopInjProjFlatCharac}\end{tabular} & \begin{tabular}{@{}c@{}}$G$\mbox{ is any } \\ \ref{MGModMGMetTopProjFlatCharac}\end{tabular}                                                                      \\ 
\hline
$C_0(G)$            & \begin{tabular}{@{}c@{}}$G$\mbox{ is finite } \\ \ref{StdModAreNotL1MGMetTopProjInjFlat},\ref{StdModFinGrL1MGTopInjProjFlatCharac}\end{tabular}                          & \begin{tabular}{@{}c@{}}$G$\mbox{ is finite } \\ \ref{StdModAreNotL1MGMetTopProjInjFlat},\ref{StdModFinGrL1MGTopInjProjFlatCharac}\end{tabular}                          & \begin{tabular}{@{}c@{}}$G$\mbox{ is finite } \\ \ref{StdModAreNotL1MGMetTopProjInjFlat},\ref{StdModFinGrL1MGTopInjProjFlatCharac}\end{tabular}                           & \begin{tabular}{@{}c@{}}$G$\mbox{ is finite } \\ \ref{StdModAreNotL1MGMetTopProjInjFlat},\ref{StdModFinGrL1MGTopInjProjFlatCharac}\end{tabular}                          & \begin{tabular}{@{}c@{}}$G$\mbox{ is finite } \\ \ref{StdModAreNotL1MGMetTopProjInjFlat},\ref{StdModFinGrL1MGTopInjProjFlatCharac}\end{tabular}                          & \begin{tabular}{@{}c@{}}$G$\mbox{ is finite } \\ \ref{StdModAreNotL1MGMetTopProjInjFlat},\ref{StdModFinGrL1MGTopInjProjFlatCharac}\end{tabular}                          \\ 
\hline          
$\mathbb{C}_\gamma$ & \begin{tabular}{@{}c@{}}$G$\mbox{ is compact } \\ \ref{OneDimL1ModMetTopProjCharac}\end{tabular}                                                                  & \begin{tabular}{@{}c@{}}$G$\mbox{ is amenable } \\ \ref{OneDimL1ModMetTopInjFlatCharac}\end{tabular}                                                              & \begin{tabular}{@{}c@{}}$G$\mbox{ is amenable } \\ \ref{OneDimL1ModMetTopInjFlatCharac}\end{tabular}                                                               & \begin{tabular}{@{}c@{}}$G$\mbox{ is compact } \\ \ref{OneDimL1ModMetTopProjCharac},\ref{MGMetTopProjInjFlatRedToL1}\end{tabular}                                 & \begin{tabular}{@{}c@{}}$G$\mbox{ is amenable } \\ \ref{OneDimL1ModMetTopInjFlatCharac},\ref{MGMetTopProjInjFlatRedToL1}\end{tabular}                             & \begin{tabular}{@{}c@{}}$G$\mbox{ is amenable } \\ \ref{OneDimL1ModMetTopInjFlatCharac},\ref{MGMetTopProjInjFlatRedToL1}\end{tabular}                             \\ 
\hline
\multicolumn{7}{c}{\mbox{Homologically trivial $L_1(G)$- and $M(G)$-modules in relative theory}}                                                                                                                                                                                                                                                                                                                                                                                                                                                                                                                                                                                                                                                                                                                                                                                                                                                                                                                                           \\

\hline            & \multicolumn{3}{c|}{$L_1(G)$-modules}                                                                                                                                                                                                                                                                                                                                                                                                                                                                      & \multicolumn{3}{c|}{$M(G)$-modules}                                                                                                                                                                                                                                                                                                                                                                                                                                                                  \\
\hline
                  & \mbox{Projectivity}                                                                                                                                               & \mbox{Injectivity}                                                                                                                                                & \mbox{Flatness}                                                                                                                                                    & \mbox{Projectivity}                                                                                                                                               & \mbox{Injectivity}                                                                                                                                                & \mbox{Flatness}                                                                                                                                                   \\ 
\hline
$L_1(G)$            & \begin{tabular}{@{}c@{}}$G$\mbox{ is any } \\ \mbox{\cite{DalPolHomolPropGrAlg}, \S 6}\end{tabular}                                                               & \begin{tabular}{@{}c@{}}$G$\mbox{ is amenable } \\ \mbox{ and discrete } \\ \mbox{\cite{DalPolHomolPropGrAlg}, \S 6}\end{tabular}                                 & \begin{tabular}{@{}c@{}}$G$\mbox{ is any } \\ \mbox{\cite{DalPolHomolPropGrAlg}, \S 6}\end{tabular}                                                                & \begin{tabular}{@{}c@{}}$G$\mbox{ is any } \\ \mbox{\cite{RamsHomPropSemgroupAlg}, \S 3.5}\end{tabular}                                                           & \begin{tabular}{@{}c@{}}$G$\mbox{ is amenable } \\ \mbox{ and discrete } \\ \mbox{\cite{RamsHomPropSemgroupAlg}, \S 3.5}\end{tabular}                             & \begin{tabular}{@{}c@{}}$G$\mbox{ is any } \\ \mbox{\cite{RamsHomPropSemgroupAlg}, \S 3.5}\end{tabular}                                                           \\ 
\hline
 $L_p(G)$           & \begin{tabular}{@{}c@{}}$G$\mbox{ is compact } \\ \mbox{\cite{DalPolHomolPropGrAlg}, \S 6}\end{tabular}                                                           & \begin{tabular}{@{}c@{}}$G$\mbox{ is amenable } \\ \cite{RachInjModAndAmenGr}\end{tabular}                                                                        & \begin{tabular}{@{}c@{}}$G$\mbox{ is amenable } \\ \cite{RachInjModAndAmenGr}\end{tabular}                                                                         & \begin{tabular}{@{}c@{}}$G$\mbox{ is compact } \\ \mbox{\cite{RamsHomPropSemgroupAlg}, \S 3.5}\end{tabular}                                                       & \begin{tabular}{@{}c@{}}$G$\mbox{ is amenable } \\ \mbox{\cite{RamsHomPropSemgroupAlg}, \S 3.5}, \cite{RachInjModAndAmenGr}\end{tabular}                          & \begin{tabular}{@{}c@{}}$G$\mbox{ is amenable } \\ \mbox{\cite{RamsHomPropSemgroupAlg}, \S 3.5}\end{tabular}                                                      \\
\hline
 $L_\infty(G)$      & \begin{tabular}{@{}c@{}}$G$\mbox{ is finite } \\ \mbox{\cite{DalPolHomolPropGrAlg}, \S 6}\end{tabular}                                                            & \begin{tabular}{@{}c@{}}$G$\mbox{ is any } \\ \mbox{\cite{DalPolHomolPropGrAlg}, \S 6}\end{tabular}                                                               & \begin{tabular}{@{}c@{}}$G$\mbox{ is amenable } \\ \mbox{\cite{DalPolHomolPropGrAlg}, \S 6}\end{tabular}                                                           & \begin{tabular}{@{}c@{}}$G$\mbox{ is finite } \\ \mbox{\cite{RamsHomPropSemgroupAlg}, \S 3.5}\end{tabular}                                                        & \begin{tabular}{@{}c@{}}$G$\mbox{ is any } \\ \mbox{\cite{RamsHomPropSemgroupAlg}, \S 3.5}\end{tabular}                                                           & \begin{tabular}{@{}c@{}}$G$\mbox{ is amenable } \\ ($\implies$)\mbox{\cite{RamsHomPropSemgroupAlg}, \S 3.5}\end{tabular}                                          \\ 
\hline
$M(G)$              & \begin{tabular}{@{}c@{}}$G$\mbox{ is discrete } \\ \mbox{\cite{DalPolHomolPropGrAlg}, \S 6}\end{tabular}                                                          & \begin{tabular}{@{}c@{}}$G$\mbox{ is amenable }\\ \mbox{\cite{DalPolHomolPropGrAlg}, \S 6}\end{tabular}                                                           & \begin{tabular}{@{}c@{}}$G$\mbox{ is any } \\ \mbox{\cite{RamsHomPropSemgroupAlg}, \S 3.5}\end{tabular}                                                            & \begin{tabular}{@{}c@{}}$G$\mbox{ is any } \\ \mbox{\cite{RamsHomPropSemgroupAlg}, \S 3.5}\end{tabular}                                                           & \begin{tabular}{@{}c@{}}$G$\mbox{ is amenable } \\ \mbox{\cite{RamsHomPropSemgroupAlg}, \S 3.5}\end{tabular}                                                      & \begin{tabular}{@{}c@{}}$G$\mbox{ is any } \\ \mbox{\cite{RamsHomPropSemgroupAlg}, \S 3.5}\end{tabular}                                                           \\ 
\hline
$C_0(G)$            & \begin{tabular}{@{}c@{}}$G$\mbox{ is compact } \\ \mbox{\cite{DalPolHomolPropGrAlg}, \S 6}\end{tabular}                                                           & \begin{tabular}{@{}c@{}}$G$\mbox{ is finite } \\ \mbox{\cite{DalPolHomolPropGrAlg}, \S 6}\end{tabular}                                                            & \begin{tabular}{@{}c@{}}$G$\mbox{ is amenable } \\ \mbox{\cite{DalPolHomolPropGrAlg}, \S 6}\end{tabular}                                                           & \begin{tabular}{@{}c@{}}$G$\mbox{ is compact } \\ \mbox{\cite{RamsHomPropSemgroupAlg}, \S 3.5}\end{tabular}                                                       & \begin{tabular}{@{}c@{}}$G$\mbox{ is finite } \\ \mbox{\cite{RamsHomPropSemgroupAlg}, \S 3.5}\end{tabular}                                                        & \begin{tabular}{@{}c@{}}$G$\mbox{ is amenable } \\ \mbox{\cite{RamsHomPropSemgroupAlg}, \S 3.5}\end{tabular}                                                      \\ 
\hline          
$\mathbb{C}_\gamma$ & \begin{tabular}{@{}c@{}}$G$\mbox{ is compact } \\ \ref{OneDimL1ModMetTopProjCharac}\end{tabular}                                                                  & \begin{tabular}{@{}c@{}}$G$\mbox{ is amenable } \\ \ref{OneDimL1ModMetTopInjFlatCharac}\end{tabular}                                                              & \begin{tabular}{@{}c@{}}$G$\mbox{ is amenable } \\ \ref{OneDimL1ModMetTopInjFlatCharac}\end{tabular}                                                               & \begin{tabular}{@{}c@{}}$G$\mbox{ is compact } \\ \ref{OneDimL1ModMetTopProjCharac},\ref{MGMetTopProjInjFlatRedToL1}\end{tabular}                                 & \begin{tabular}{@{}c@{}}$G$\mbox{ is amenable } \\ \ref{OneDimL1ModMetTopInjFlatCharac},\ref{MGMetTopProjInjFlatRedToL1}\end{tabular}                             & \begin{tabular}{@{}c@{}}$G$\mbox{ is amenable } \\ \ref{OneDimL1ModMetTopInjFlatCharac},\ref{MGMetTopProjInjFlatRedToL1}\end{tabular}                             \\                   
\hline
\end{longtable}
\end{scriptsize}

% :1067,1124s/NoInfDimRefMetTopProjInjFlatModOverMthscrL1OrLInfty/StdModAreNotL1MGMetTopProjInjFlat/gc
%----------------------------------------------------------------------------------------
%	An example of small category.
%----------------------------------------------------------------------------------------

\section{An example of ``small'' category}
\label{SectionAnExampleOfSmallCategory}

As we have seen by a lot of examples above most of natural modules of algebras of analysis turn out to be homologically non trivial when considered with respect to big categories. The situation may change dramatically for comparatively ``small'' categories. This section is devoted to construction of meaningful example of this kind.

%----------------------------------------------------------------------------------------
%	The category of B(\Omega,\Sigma)-module L_p
%----------------------------------------------------------------------------------------

\subsection{The category of \texorpdfstring{$B(\Omega,\Sigma)$}{B(Omega,Sigma)}-modules \texorpdfstring{$L_p$}{Lp}}
\label{SubSectionTheCategoryOfBOmegaSigmaModulesLp}

An example of ``small'' category we shall construct is the category of $B(\Omega,\Sigma)$-modules of the form $L_p(\Omega,\mu)$ on some measurable space $(\Omega,\Sigma)$ with different $\sigma$-finite positive measures $\mu$. We denote it by $B(\Omega,\Sigma)-\mathbf{mod(L)}$. We shall show that all its modules are metrically and topologically projective, injective and flat. Along the way we shall give a complete characterization of topologically surjective, topologically injective, coisometric and isometric multiplication operators between $L_p$-spaces. In \cite{HelTensProdAndMultModLp} Helemskii described morphisms of $B(\Omega,\Sigma)-\mathbf{mod(L)}$, but only for a locally compact space $\Omega$, with Borel $\sigma$-algebra. Careful inspection of his proof shows that his characterization is valid for all $\sigma$-finite measure spaces. To properly state the result we need to introduce the notation. By $L_0(\Omega,\Sigma)$ we denote the linear space of measurable functions on $\Omega$. For a given $1\leq p,q\leq +\infty$ and positive $\sigma$-finite measures $\mu,\nu$ on a measurable space $(\Omega,\Sigma)$ denote $\Omega_+:=\{\omega\in\Omega_c^{\nu,\mu}:\rho_{\nu,\mu}(\omega)>0\}$ and
$$
L_{p,q,\mu,\nu}(\Omega):=
\begin{cases}
\{g\in L_0(\Omega,\Sigma):g\in L_{pq/(p-q)}(\Omega,\rho_{\nu,\mu}^{p/(p-q)}\mu),\quad g|_{\Omega\setminus\Omega_+}=0\}&\text{if}\quad p>q\\
\{g\in L_0(\Omega,\Sigma):g\rho_{\nu,\mu}^{1/p}\in L_{\infty}(\Omega,\mu),\quad g|_{\Omega\setminus\Omega_+}=0\}&\text{if}\quad p=q\\
\{g\in L_0(\Omega,\Sigma):g\rho_{\nu,\mu}^{1/p}\mu^{pq/(p-q)}\in L_{\infty}(\Omega,\mu),\quad g|_{\Omega\setminus\Omega_a^{\mu}}=0\}&\text{if}\quad p<q\\
\end{cases}
$$
$$
\Vert g\Vert_{L_{p,q,\mu,\nu}(\Omega)}:=
\begin{cases}
\Vert g\Vert_{L_{pq/(p-q)}(\Omega,\rho_{\nu,\mu}^{p/(p-q)}\mu)}&\text{if}\quad p>q\\
\Vert g\rho_{\nu,\mu}^{1/p}\Vert_{L_{\infty}(\Omega,\mu)}&\text{if}\quad p=q\\
\Vert g\rho_{\nu,\mu}^{1/p}\mu^{pq/(p-q)}\Vert_{L_{\infty}(\Omega,\mu)}&\text{if}\quad p<q\\
\end{cases}
$$


\begin{theorem}[\cite{HelTensProdAndMultModLp}, theorem 4.1]\label{LpModMorphCharac}
Let $(\Omega,\Sigma)$ be a measurable space, $1\leq p,q\leq +\infty$ and $\mu,\nu$ be two $\sigma$-finite measures on $(\Omega, \Sigma)$. Then there exists an isometric isomorphism
$$
\mathcal{I}_{p,q,\mu,\nu}:L_{p,q,\mu,\nu}(\Omega)\to\operatorname{Hom}_{B(\Omega,\Sigma)-\mathbf{mod(L)}}(L_p(\Omega,\mu),L_q(\Omega,\nu)):g\mapsto (f\mapsto g f)
$$
\end{theorem}

Simply speaking all morphisms in $B(\Omega,\Sigma)-\mathbf{mod(L)}$ are multiplication operators.

%----------------------------------------------------------------------------------------
%	Decomposition of Lp-spaces
%----------------------------------------------------------------------------------------

\subsection{Decomposition of \texorpdfstring{$L_p$}{Lp}-spaces}
\label{SubSectionDecompositionOfLpSpaces}

For further investigations we need to recall some obvious facts on decomposition of $L_p$-spaces induced by decomposition of their underlying measure spaces. Let $1\leq p\leq+\infty$ and $(\Omega,\Sigma,\mu)$ be a measure space. If the whole space $\Omega$ is a single atom, then its $L_p$-space is one dimensional and we have an isometric isomorphism
$$
J_p:L_p(\Omega,\mu|_{\Omega})\to \ell_p(\mathbb{N}_1):f\mapsto\left(1\mapsto \mu(\Omega)^{1/p-1}\int_{\Omega} f(\omega)d\mu(\omega)\right)
$$
If $\Omega=\bigcup_{\lambda\in\Lambda}\Omega_\lambda$ is a representation of $\Omega$ as disjoint union of measurable sets, then for all $1\leq p\leq+\infty$ we always have an isometric isomorphism
$$
I_p:L_p(\Omega,\mu)\to \bigoplus\nolimits_p\{ L_p(\Omega_\lambda,\mu|_{\Omega_\lambda}):\lambda\in\Lambda\}: f\mapsto (\lambda\mapsto f|_{\Omega_\lambda})
$$
If each set $\Omega_\lambda$ is an atom, then $\Omega$ is a purely atomic measure space and we have one more isometric isomorphism
$$
\widetilde{I}_p:L_p(\Omega,\mu)\to \ell_p(\Lambda):f\mapsto\left (\lambda\mapsto \mu(\Omega_\lambda)^{1/p-1}\int_{\Omega_\lambda} f(\omega)d\mu(\omega)\right)
$$
Note: when dealing with $p$ indexes we take by definition that $1/0=\infty$ and $1/\infty=0$. Another useful technique in the study of $L_p$-spaces is a so called change of density: if $\rho:\Omega\to(0,+\infty)$ is a measurable function, then
$$
\bar{I}_p:L_p(\Omega,\mu)\to L_p(\Omega,\rho\mu): f\mapsto\rho^{-1/p} f
$$
is an isometric isomorphism. For different values of $p$ infinite dimensional $L_p$-spaces are not topologically isomorphic. This can be proved via type and cotype techniques [\cite{KalAlbTopicsBanSpTh}, theorem 6.2.14]. Obviously, in finite dimensional setting we have an isomorphism only for spaces of equal dimension. More precisely: if $\Lambda$ is a finite set and $1\leq p,q\leq +\infty$, then there exists a $c_{p,q}>0$ such that $\Vert x\Vert_{\ell_p(\Lambda)}\leq c_{p,q}\Vert x\Vert_{\ell_q(\Lambda)}$ for all $x\in\mathbb{C}^\Lambda$.


%----------------------------------------------------------------------------------------
%	Multiplication operators
%----------------------------------------------------------------------------------------

\subsection{Multiplication operators}
\label{SubSectionMultiplicationOperators}

Let $(\Omega,\Sigma,\mu)$ and $(\Omega,\Sigma,\nu)$ be two measure spaces with the same $\sigma$-algebra of measurable sets. For a given $g\in L_0(\Omega,\Sigma)$ and $1\leq p,q\leq +\infty$ we define the multiplication operator
$$
M_g:L_p(\Omega,\mu)\to L_q(\Omega,\nu): f\mapsto g f
$$ 
Of course certain restrictions on $g$, $\mu$ and $\nu$ are required for $M_g$ to be well defined. For a given $E\in\Sigma$ by $M_g^E$ we denote the linear operator
$$
M_g^E:L_p(E,\mu|_E)\to L_q(E,\nu|_E):f\mapsto g|_E f
$$
It is well defined because $f|_{\Omega\setminus E}=0$ implies $M_g(f)|_{\Omega\setminus E}=0$. 

\begin{proposition}\label{MultpOpSurjInjDesc} Let $(\Omega,\Sigma,\mu)$ be a measure space and $g\in L_0(\Omega,\Sigma)$ and denote $Z_g:=g^{-1}(\{0\})$. Then for the operator $M_g:L_p(\Omega,\mu)\to L_q(\Omega,\mu)$ we have

$i)$ $\operatorname{Ker}(M_g)=\{f\in L_p(\Omega,\mu):f|_{\Omega\setminus {Z_g}}=0\}$, so $M_g$ is injective iff $\mu(Z_g)=0$

$ii)$ $\operatorname{Im}(M_g)\subset\{h\in L_q(\Omega,\mu): h|_{Z_g}=0\}$, so if $M_g$ is surjective then $\mu(Z_g)=0$.

\end{proposition}
\begin{proof}
$i)$ The desired equality follows from the chain of equivalences
$f\in\operatorname{Ker}(M_g)
\Longleftrightarrow g f=0
\Longleftrightarrow f|_{\Omega\setminus Z_g}=0$


$ii)$ Since $g|_{Z_g}=0$ then for all $f\in L_p(\Omega,\mu)$ we have $M_g(f)|_{Z_g}=(g f)|_{Z_g}=0$, thus we get the inclusion. If $M_g$ is surjective then, clearly,  $\mu(Z_g)=0$.
\end{proof}

For a given $E\in \Sigma$ and $f\in L_0(E,\Sigma|_{E})$ by $\widetilde{f}$ we will denote the function in $L_0(\Omega, \Sigma)$ such that $\widetilde{f}(\omega)=f(\omega)$ if $\omega\in E$ and $\widetilde{f}(\omega)=0$ otherwise.

\begin{proposition}\label{MultOpDecompDecomp} Let $(\Omega,\Sigma,\mu)$, $(\Omega,\Sigma,\nu)$ be measure spaces and $1\leq p,q\leq +\infty$. Assume we have a representation $\Omega=\bigcup_{\lambda\in\Lambda}\Omega_\lambda$ of $\Omega$ as finite disjoint union of measurable sets. Then 

$i)$ operator $M_g$ is $c$-topologically injective for some $c>0$ iff operators $M_g^{\Omega_\lambda}$ are $c'$-topologically injective for all $\lambda\in\Lambda$ and some $c'>0$

$ii)$ operator $M_g$ is $c$-topologically surjective for some $c>0$ iff operators $M_g^{\Omega_\lambda}$ are $c'$-topologically surjective for all $\lambda\in\Lambda$ and some $c'>0$

$iii)$ if operator $M_g$ is isometric then so does $M_g^{\Omega_\lambda}$ for all $\lambda\in\Lambda$

$iv)$ if operator $M_g$ is coisometric then so does $M_g^{\Omega_\lambda}$ for all $\lambda\in\Lambda$

\end{proposition}
\begin{proof}
$i)$ Let $M_g$ be $c$-topologically injective. Fix $\lambda\in\Lambda$ and $f\in L_p(\Omega_\lambda,\mu|_{\Omega_\lambda})$, then 
$$
\Vert M_g^{\Omega_\lambda}(f)\Vert_{L_q(\Omega_\lambda,\nu|_{\Omega_\lambda})}
=\Vert g \widetilde{f}\Vert_{L_q(\Omega,\nu)}
\geq c^{-1}\Vert\widetilde{f}\Vert_{L_p(\Omega,\mu)}
=c^{-1}\Vert f\Vert_{L_p(\Omega_\lambda,\mu|_{\Omega_\lambda})}
$$
So $M_g^{\Omega_\lambda}$ is $c$-topologically injective for all $\lambda\in\Lambda$. 

Conversely, assume that operators $\{M_g^{\Omega_\lambda}:\lambda\in\Lambda\}$ are $c'$-topologically injective. Let $f\in L_p(\Omega,\mu)$, then 
$$
\Vert M_g(f)\Vert_{L_q(\Omega,\nu)}
=\left\Vert\left(\Vert M_g^{\Omega_\lambda}(f|_{\Omega_\lambda})\Vert_{L_q(\Omega_\lambda,\nu|_{\Omega_\lambda})}:\lambda\in\Lambda\right)\right\Vert_{\ell_q(\Lambda)}
$$
$$
\geq (c')^{-1}\left\Vert\left(\Vert f|_{\Omega_\lambda}\Vert_{L_p(\Omega_\lambda,\mu|_{\Omega_\lambda})}:\lambda\in\Lambda\right)\right\Vert_{\ell_q(\Lambda)}
$$
$$
\geq (c')^{-1} c_{p,q}^{-1}\left\Vert\left(\Vert f|_{\Omega_\lambda}\Vert_{L_p(\Omega_\lambda,\mu|_{\Omega_\lambda})}:\lambda\in\Lambda\right)\right\Vert_{\ell_p(\Lambda)}
=(c')^{-1}c_{p,q}^{-1}\Vert f\Vert_{L_p(\Omega,\mu)}
$$
Since $f$ is arbitrary, then $M_g$ is $c$-topologically injective for $c=c'c_{p,q}>0$.

$ii)$ Let $M_g$ be $c$-topologically surjective. Fix $\lambda\in\Lambda$ and $h\in L_q(\Omega_\lambda,\nu|_{\Omega_\lambda})$. Then there exists $f\in L_p(\Omega,\mu)$ such that $M_g(f)=\widetilde{h}$ and $\Vert f\Vert_{L_p(\Omega,\mu)}< c\Vert \widetilde{h}\Vert_{L_q(\Omega,\nu)}$. Then $M_g^{\Omega_\lambda}(f|_{\Omega_\lambda})=\widetilde{h}|_{\Omega_\lambda}=h$ and $\Vert f|_{\Omega_\lambda}\Vert_{L_p(\Omega_\lambda,\mu|_{\Omega_\lambda})}\leq \Vert f\Vert_{L_p(\Omega,\mu)}< c\Vert\widetilde{h}\Vert_{L_q(\Omega,\nu)}=c\Vert h\Vert_{L_q(\Omega_\lambda,\nu|_{\Omega_\lambda})}$. Since $h$ is arbitrary, then $M_g^{\Omega_\lambda}$ is $c$-topologically surjective for all $\lambda\in\Lambda$.

Conversely, assume operators $\{M_g^{\Omega_\lambda}:\lambda\in\Lambda\}$ are $c'$-topologically surjective. Let $h\in L_q(\Omega,\nu)$. From assumption for each $\lambda\in\Lambda$ we have $f_\lambda\in L_p(\Omega_\lambda,\mu|_{\Omega_\lambda})$ such that $M_g^{\Omega_\lambda}(f_\lambda)=h|_{\Omega_\lambda}$ and $\Vert f_\lambda\Vert_{L_p(\Omega_\lambda,\mu|_{\Omega_\lambda})}< c'\Vert h|_{\Omega_\lambda}\Vert_{L_q(\Omega_\lambda,\nu|_{\Omega_\lambda})}$. Define $f\in L_0(\Omega,\Sigma)$ such that $f(\omega)=f_\lambda(\omega)$ if $\omega\in\Omega_\lambda$, then
$$
\Vert f\Vert_{L_p(\Omega,\mu)}
=\left\Vert\left(\Vert f_\lambda\Vert_{L_p(\Omega_\lambda,\mu|_{\Omega_\lambda})}:\lambda\in\Lambda\right)\right\Vert_{\ell_p(\Lambda)}
< c'\left\Vert\left(\Vert h|_{\Omega_\lambda}\Vert_{L_q(\Omega_\lambda,\nu|_{\Omega_\lambda})}:\lambda\in\Lambda\right)\right\Vert_{\ell_p(\Lambda)}
$$
$$
\leq c'c_{p,q}\left\Vert\left(\Vert h|_{\Omega_\lambda}\Vert_{L_q(\Omega_\lambda,\nu|_{\Omega_\lambda})}:\lambda\in\Lambda\right)\right\Vert_{\ell_q(\Lambda)}
=c'c_{p,q}\Vert h\Vert_{L_q(\Omega,\nu)}
$$
Obviously, $M_g(f)=h$. Since $h$ is arbitrary, then we get that $M_g$ is $c$-topologically surjective for $c=c'c_{p,q}>0$.

$iii)$ Fix $\lambda\in\Lambda$ and $f\in L_p(\Omega_\lambda,\mu|_{\Omega_\lambda})$, then 
$$
\Vert M_g^{\Omega_\lambda}(f)\Vert_{L_q(\Omega_\lambda,\nu|_{\Omega_\lambda})}
=\Vert g \widetilde{f}\Vert_{L_q(\Omega,\nu)}
=\Vert\widetilde{f}\Vert_{L_p(\Omega,\mu)}
=\Vert f\Vert_{L_p(\Omega_\lambda,\mu|_{\Omega_\lambda})}
$$
So $M_g^{\Omega_\lambda}$ is isometric for all $\lambda\in\Lambda$

$iv)$ Fix $\lambda\in\Lambda$. Since $M_g$ is coisometric, then it is $1$-topologically surjective and contractive. So from paragraph $ii)$ we see that $M_g^{\Omega_\lambda}$ is $1$-topologically surjective. Let $f\in L_p(\Omega_\lambda,\mu|_{\Omega_\lambda})$. Since $M_g$ is contractive we get
$$
\Vert M_g^{\Omega_\lambda}(f)\Vert_{L_q(\Omega_\lambda,\nu|_{\Omega_\lambda})}
=\Vert M_g(\widetilde{f})\chi_{\Omega_\lambda}\Vert_{L_q(\Omega,\nu)}
=\Vert M_g(\widetilde{f}\chi_{\Omega_\lambda})\Vert_{L_q(\Omega,\nu)}
$$
$$
\leq \Vert\widetilde{f}\chi_{\Omega_\lambda}\Vert_{L_p(\Omega,\mu)}
=\Vert f\Vert_{L_p(\Omega_{\lambda},\mu|_{\Omega_\lambda})}
$$
Since $M_g^{\Omega_\lambda}$ is contractive and $1$-topologically surjective it is coisometric.
\end{proof}


\begin{proposition}\label{MultOpCharacBtwnTwoSingMeasSp} Let $(\Omega,\Sigma,\mu)$ and $(\Omega,\Sigma,\nu)$ be two $\sigma$-finite measure spaces. Let $1\leq p,q\leq +\infty$ and $g\in L_0(\Omega,\Sigma)$. If $\mu\perp\nu$, then $M_g:L_p(\Omega,\mu)\to L_q(\Omega,\nu)$ is the zero operator.
\end{proposition}
\begin{proof} Since $\mu\perp\nu$, then there exists an $\Omega_s^{\nu,\mu}\in\Sigma$ such that $\mu(\Omega_s^{\nu,\mu})=\nu(\Omega_c^{\nu,\mu})=0$, where $\Omega_c^{\nu,\mu}=\Omega\setminus\Omega_s^{\nu,\mu}$. Since $\mu(\Omega_s^{\nu,\mu})=0$, then $\chi_{\Omega_c^{\nu,\mu}}=\chi_{\Omega}$ in $L_p(\Omega,\mu)$ and $\chi_{\Omega_c^{\nu,\mu}}=0$ in $L_q(\Omega,\nu)$. Now for all $f\in L_p(\Omega,\mu)$ we have $M_g(f)=M_g(f \chi_{\Omega})=M_g(f \chi_{\Omega_c^{\nu,\mu}})=g f\chi_{\Omega_c^{\nu,\mu}}=0$. Since $f$ is arbitrary, then $M_g=0$.
\end{proof}

From this point we start our main study of multiplication operators. We shall show that $\langle$~isometric / topologically injective~$\rangle$ morphisms in $B(\Omega,\Sigma)-\mathbf{mod(L)}$ are coretractions in $\langle$~$\mathbf{Ban}_1$ / $\mathbf{Ban}$~$\rangle$, while $\langle$~strictly coisometric / topologically surjective~$\rangle$ morphisms in $B(\Omega,\Sigma)-\mathbf{mod(L)}$ are retractions in $\langle$~$\mathbf{Ban}_1$ / $\mathbf{Ban}$~$\rangle$. Using this characterizations we shall easily describe metrically and topologically projective, injective and flat modules of the category $B(\Omega,\Sigma)-\mathbf{mod(L)}$.

\begin{proposition}\label{MultpOpPropIfPeqqualsQ} Let $(\Omega,\Sigma,\mu)$ be a measure space and $g\in L_0(\Omega,\Sigma)$. Then 

$i)$ $M_g\in\mathcal{B}(L_p(\Omega,\mu))$ iff $g\in L_\infty(\Omega,\mu)$;

$ii)$ $M_g$ is a topological isomorphism iff $c\leq |g|\leq c'$ for some $c,c'>0$.
\end{proposition}
\begin{proof}
$i)$ Consider $M_g\in\mathcal{B}(L_p(\Omega,\mu))$. Assume there exists an $E\in\Sigma$ with $\mu(E)>0$ such that $|g|_E|>\Vert M_g\Vert$, then
$$
\Vert M_g(\chi_E)\Vert_{L_p(\Omega,\mu)}
=\Vert g\chi_E\Vert_{L_p(\Omega,\mu)}
>\Vert M_g\Vert\Vert\chi_E\Vert_{L_p(\Omega,\mu)}
$$
Contradiction, hence for all $E\in\Sigma$ with $\mu(E)>0$ we have $|g|_E|\leq \Vert M_g\Vert$ i.e.  $|g|\leq \Vert M_g\Vert$. Thus $g\in L_\infty(\Omega,\mu)$. Conversely, let $g\in L_\infty(\Omega,\mu)$. Now for any $1\leq p\leq +\infty$ and $f\in L_p(\Omega,\mu)$ we have
$$
\Vert M_g(f)\Vert_{L_p(\Omega,\mu)}
=\Vert g  f\Vert_{L_p(\Omega,\mu)}
\leq \Vert g\Vert_{L_\infty(\Omega,\mu)}\Vert f\Vert_{L_p(\Omega,\mu)}
$$
Hence $M_g\in\mathcal{B}(L_p(\Omega,\mu))$.

$ii)$ Note that $M_g^{-1}=M_{1/g}$ as linear maps provided $1/g$ is well defined. Now $M_g$ is a topological isomorphism iff $M_g$ and $M_g^{-1}$ are bounded operators. From previous paragraph and equality $M_g^{-1}=M_{1/g}$ we see that it is equivalent to boundedness of $g$ and $1/g$. This is equivalent to $c\leq|g|\leq c'$ for some $c,c'>0$.
\end{proof}

\begin{proposition}\label{EquivMultOp} Let $(\Omega,\Sigma,\mu)$ be a $\sigma$-finite purely atomic measure space, $1\leq p,q\leq +\infty$ and $g\in L_0(\Omega,\Sigma)$. Then the operator $\widetilde{M}_{\widetilde{g}}:=\widetilde{I}_q M_g\widetilde{I}_p^{-1}\in\mathcal{B}(\ell_p(\Lambda),\ell_q(\Lambda))$ is a multiplication operator by the function $\widetilde{g}:\Lambda\to\mathbb{C}:\lambda\mapsto \mu(\Omega_\lambda)^{1/q-1/p-1}\int_{\Omega_\lambda}f(\omega)d\mu(\omega)$ where $\{\Omega_\lambda:\lambda\in\Lambda\}$ is at most countable decomposition of $\Omega$ into disjoint family of atoms.
\end{proposition}
\begin{proof} Let $1\leq p,q\leq +\infty$. For any $x\in\ell_p(\Lambda)$ we have
$$
\widetilde{M}_{\widetilde{g}}(x)(\lambda)
=(\widetilde{I}_q((M_g\widetilde{I}_p^{-1})(x))(\lambda)
=J_q(M_g(\widetilde{I}_p^{-1}(x))|_{\Omega_\lambda})(1)
$$
$$
=J_q((g \widetilde{I}_p^{-1}(x))|_{\Omega_\lambda})(1)
=\mu(\Omega_\lambda)^{1/q-1}\int_{\Omega_\lambda}(g|_{\Omega_\lambda} \widetilde{I}_p^{-1}(x)|_{\Omega_\lambda})(\omega)d\mu(\omega)
$$
$$
=\mu(\Omega_\lambda)^{1/q-1}\int_{\Omega_\lambda}(g \mu(\Omega)^{-1/p}x(\lambda)\chi_{\Omega_{\lambda}})(\omega)d\mu(\omega)
=x(\lambda)\mu(\Omega_\lambda)^{1/q-1/p-1}\int_{\Omega_\lambda} g(\omega)d\mu(\omega)
$$
Thus $\widetilde{M}_{\widetilde{g}}$ is a multiplication operator where $\widetilde{g}(\lambda)=\mu(\Omega_\lambda)^{1/q-1/p-1}\int_{\Omega_\lambda} g(\omega)d\mu(\omega)$.
\end{proof}

Since $\widetilde{I}_p$ and $\widetilde{I}_q$ are isometric isomorphisms then $M_g$ is topologically injective iff $\widetilde{M}_{\widetilde{g}}$ is topologically injective. 

\begin{proposition}\label{TopInjMultOpCharacOnPureAtomMeasSp} Let $(\Omega,\Sigma,\mu)$ be $\sigma$-finite purely atomic measure space, $1\leq p,q\leq +\infty$ and $g\in L_0(\Omega,\Sigma)$. Then the following are equivalent:

$i)$ $M_g\in\mathcal{B}(L_p(\Omega,\mu),L_q(\Omega,\mu))$ is topologically injective;

$ii)$ $|g|\geq c$ for some $c>0$ and if $p\neq q$ the space $(\Omega,\Sigma,\mu)$ consist of finitely many atoms.
\end{proposition}
\begin{proof}
$i)$$\implies$$ ii)$ Assume $M_g$ is topologically injective, then so does $\widetilde{M}_{\widetilde{g}}$, i.e. $\Vert\widetilde{M}_{\widetilde{g}}(x)\Vert_{\ell_q(\Lambda)}\geq c'\Vert x\Vert_{\ell_p(\Lambda)}$ for some $c'>0$ and all $x\in\ell_p(\Lambda)$. By $\{\Omega_\lambda:\lambda\in\Lambda\}$ we denote at most countable decomposition  of $\Omega$ into disjoint family atoms. We shall consider two cases.

1) Let $p\neq q$. Assume $\Lambda$ is countable. 

1.1) Consider subcase $p,q<+\infty$. Since $\Lambda$ is countable, then we get a  contradiction, because by Pitt's theorem [\cite{KalAlbTopicsBanSpTh}, proposition 2.1.6] there is no embedding of $\ell_p(\Lambda)$ into $\ell_q(\Lambda)$ for countable $\Lambda$ and $1\leq p,q< +\infty$, $p\neq q$. 

1.2) Consider subcase $1\leq p <+\infty$ and $q=+\infty$. Take any $F\in\mathcal{P}_0(\Lambda)$, then 
$$
\sup_{\lambda\in\Lambda}|\widetilde{g}(\lambda)|
\geq\max_{\lambda\in F}|\widetilde{g}(\lambda)|
=\left\Vert\widetilde{M}_{\widetilde{g}}\left(\sum_{\lambda\in F}\delta_\lambda\right)\right\Vert_{\ell_\infty(\Lambda)}
\geq c'\left\Vert\sum_{\lambda\in F}\delta_\lambda\right\Vert_{\ell_p(\Lambda)}
=c'\operatorname{Card}(F)^{1/p}
$$
Since $\Lambda$ is countable, then $\sup_{\lambda\in\Lambda}|\widetilde{g}(\lambda)|\geq c'\sup_{F\in\mathcal{P}_0(\Lambda)}\operatorname{Card}(F)^{1/p}=+\infty$. On the other hand, since $\widetilde{M}_{\widetilde{g}}$ is bounded we have 
$$\sup_{\lambda\in\Lambda}|\widetilde{g}(\lambda)|
=\sup_{\lambda\in\Lambda}\Vert\widetilde{M}_{\widetilde{g}}(\delta_\lambda)\Vert_{\ell_\infty(\Lambda)}
\leq\Vert\widetilde{M}_{\widetilde{g}}\Vert\Vert \delta_\lambda\Vert_{\ell_p(\Lambda)}
=\Vert\widetilde{M}_{\widetilde{g}}\Vert<+\infty
$$
Contradiction.

1.3) Consider subcase $1\leq q<+\infty$ and $p=+\infty$. Since $\Lambda$ is countable, then $\ell_\infty(\Lambda)$ is non separable and $\ell_q(\Lambda)$ is separable. As $\widetilde{M}_{\widetilde{g}}$ is topologically injective, then $\operatorname{Im}(\widetilde{M}_{\widetilde{g}})$ is a non separable subspace of $\ell_q(\Lambda)$. Contradiction.

In all subcases we got a contradiction. Hence $\Lambda$ is finite i.e. $(\Omega,\Sigma,\mu)$ consist of finitely many atoms. Obviously,
$g$ is completely determined by its values $k_\lambda\in\mathbb{C}$ on atoms $\{\Omega_\lambda:\lambda\in\Lambda\}$. By proposition \ref{MultpOpSurjInjDesc} the function $g$ is zero only on sets of measure zero, so $k_\lambda\neq 0$ for all $\lambda\in\Lambda$. Since $\Lambda$ is finite we conclude $|g|\geq c$ for $c=\min_{\lambda\in\Lambda}|k_\lambda|$. 


2) Let $p=q$. Fix $\lambda\in\Lambda$, then
$$
|\widetilde{g}(\lambda)|
=\Vert \widetilde{g} \delta_\lambda\Vert_{\ell_q(\Lambda)}
=\Vert \widetilde{M}_{\widetilde{g}}(\delta_\lambda)\Vert_{\ell_q(\Lambda)}
\geq c'\Vert \delta_\lambda\Vert_{\ell_p(\Lambda)}
=c'
$$
For $\mu$-almost all $\omega\in\Omega_\lambda$ we have
$$
|g(\omega)|
=\left|\mu(\Omega_\lambda)^{-1}\int_{\Omega_\lambda}g(\omega)d\mu(\omega)\right|
=\left|\mu(\Omega_\lambda)^{-1}\mu(\Omega_\lambda)^{1+1/p-1/p}\widetilde{g}(\lambda)\right|
=|\widetilde{g}(\lambda)|\geq c'
$$
Since $\lambda\in\Lambda$ is arbitrary and $\Omega=\bigcup_{\lambda\in\Lambda}\Omega_\lambda$, then $|g|\geq c'$.



$ii)$$\implies$$ i)$ Assume $|g|\geq c$ for $c>0$. Then from proposition \ref{EquivMultOp} we see that $|\widetilde{g}|\geq c$.

1) Let $p\neq q$. By assumption $(\Omega,\Sigma,\mu)$ consist of finitely many atoms, then $L_p(\Omega,\mu)$ is finite dimensional. From assumption on $g$ we see that it has no zero values, hence operator $M_g$ is topologically injective. 

2) Let $p=q$, then for all $x\in\ell_p(\Lambda)$ we have
$$
\Vert \widetilde{M}_{\widetilde{g}}(x)\Vert_{\ell_p(\Lambda)}=\Vert g x\Vert_{\ell_p(\Lambda)}\geq c\Vert x\Vert_{\ell_p(\Lambda)}
$$
so $\widetilde{M}_{\widetilde{g}}$ is topologically injective and so does $M_g$.
\end{proof}

\begin{proposition}\label{TopInjMultOpCharacOnNonAtomMeasSp} Let $(\Omega,\Sigma,\mu)$ be a non atomic measure space, $1\leq p,q\leq +\infty$ and $g\in L_0(\Omega,\Sigma)$. Then the following are equivalent:

$i)$ $M_g\in\mathcal{B}(L_p(\Omega,\mu),L_q(\Omega,\mu))$ is topologically injective;

$ii)$ $|g|\geq c$ for some $c>0$ and $p=q$.
\end{proposition}
\begin{proof}
$i)$$\implies$$ ii)$ Assume $M_g$ is topologically injective i.e. $\Vert M_g(f)\Vert_{L_q(\Omega,\mu)}\geq c\Vert f\Vert_{L_p(\Omega,\mu)}$ for some $c>0$ and all $f\in L_p(\Omega,\mu)$. We shall consider three cases.

1)  Let $p>q$. There exist $c'>0$ and $E\in\Sigma$ with $\mu(E)>0$ such that $|g|_E|\leq c'$, otherwise $M_g$ is not well defined. Since $(\Omega,\Sigma,\mu)$ is a non atomic measure space, then we have a sequence $\{E_n:n\in\mathbb{N}\}\subset\Sigma$ of subsets of $E$ such that $\mu(E_n)=2^{-n}$. Then since $p>q$ we get
$$
c
\leq\frac{\Vert M_g(\chi_{E_n})\Vert_{L_q(\Omega,\mu)}}{\Vert \chi_{E_n}\Vert_{L_p(\Omega,\mu)}}
\leq\frac{c'\Vert\chi_{E_n}\Vert_{L_q(\Omega,\mu)}}{\Vert \chi_{E_n}\Vert_{L_p(\Omega,\mu)}}
\leq c'\mu(E_n)^{1/q-1/p},
$$
$$
c
\leq\inf_{n\in\mathbb{N}}c'\mu(E_n)^{1/q-1/p}
=c'\inf_{n\in\mathbb{N}} 2^{n(1/p-1/q)}=0
$$
Contradiction, so in this case $M_g$ can not be topologically injective.

2) Let $p=q$. Fix $\epsilon > 0$. Assume there exist $E\in\Sigma$ with $\mu(E)>0$ and $|g|_{E}|<c-\epsilon$, then
$$
\Vert M_g(\chi_{E})\Vert_{L_p(\Omega,\mu)}
=\Vert g \chi_{E}\Vert_{L_p(\Omega,\mu)}
\leq (c-\epsilon) \Vert \chi_{E}\Vert_{L_p(\Omega,\mu)}
<c\Vert \chi_{E}\Vert_{L_p(\Omega,\mu)}
$$
Contradiction. Therefore $|g|_E|\geq c$ for any $E\in\Sigma$ with $\mu(E)>0$. Thus $|g|\geq c$.

3) Let $p<q$. Assume we have some $c'>0$ and $E\in\Sigma$ such that $\mu(E)>0$, $|g|_E|>c'$. Again we have a sequence  $\{E_n:n\in\mathbb{N}\}\subset\Sigma$ of subsets of $E$ such that $\mu(E_n)=2^{-n}$, because $(\Omega,\Sigma,\mu)$ is a non atomic measure space. Then from inequality $p<q$ we get
$$
\Vert M_g\Vert
\geq\frac{\Vert M_g(\chi_{E_n})\Vert_{L_q(\Omega,\mu)}}{\Vert \chi_{E_n}\Vert_{L_p(\Omega,\mu)}}
\geq\frac{c'\Vert\chi_{E_n}\Vert_{L_q(\Omega,\mu)}}{\Vert \chi_{E_n}\Vert_{L_p(\Omega,\mu)}}
\geq c'\mu(E_n)^{1/q-1/p}
$$
$$
\Vert M_g\Vert
\geq\sup_{n\in\mathbb{N}}c'\mu(E_n)^{1/q-1/p}
\geq c'\sup_{n\in\mathbb{N}}2^{n(1/p-1/q)}
=+\infty
$$
Contradiction, hence $g=0$. In this case by proposition \ref{MultpOpSurjInjDesc} operator $M_g$ is not topologically injective.

$ii)$$\implies$$ i)$ Conversely, assume $|g|\geq c$ for $c>0$ and $p=q$. Then for all $f\in L_p(\Omega,\mu)$ we have
$$
\Vert M_g(f)\Vert_{L_p(\Omega,\mu)}
=\Vert g f\Vert_{L_p(\Omega,\mu)}
\geq c\Vert f\Vert_{L_p(\Omega,\mu)}
$$
So $M_g$ is topologically injective.
\end{proof}

\begin{proposition}\label{TopInjMultOpCharacOnMeasSp} Let $(\Omega,\Sigma,\mu)$ be a $\sigma$-finite measure space, $1\leq p,q\leq +\infty$ and $g\in L_0(\Omega,\Sigma)$. Then the following are equivalent:

$i)$ $M_g\in\mathcal{B}(L_p(\Omega,\mu),L_q(\Omega,\mu))$ is topologically injective;

$ii)$ $M_g$ is a topological isomorphism;

$iii)$ $|g|\geq c$ for some $c>0$, if $p\neq q$ the space $(\Omega,\Sigma,\mu)$ consist of finitely many atoms.
\end{proposition}
\begin{proof} $i)\Longleftrightarrow iii)$ Consider decomposition
$\Omega=\Omega_a^{\mu}\cup\Omega_{na}^{\mu}$, where $(\Omega_{na}^{\mu},\Sigma|_{\Omega_{na}^{\mu}},\mu|_{\Omega_{na}^{\mu}})$ is a non atomic measure space and $(\Omega_a^{\mu},\Sigma|_{\Omega_a^{\mu}},\mu|_{\Omega_a^{\mu}})$ is a purely atomic measure space. By proposition \ref{MultOpDecompDecomp} operator $M_g$ is topologically injective iff so does $M_g^{\Omega_a^{\mu}}$ and $M_g^{\Omega_{na}^{\mu}}$. Propositions \ref{TopInjMultOpCharacOnPureAtomMeasSp}, \ref{TopInjMultOpCharacOnNonAtomMeasSp} give necessary and sufficient conditions for this to happen. 

$i)$$\implies$$ ii)$ Assume $M_g$ is topologically injective. If $p=q$ from considerations above it follows that $|g|\geq c$ for some $c>0$. Since operator $M_g$ is bounded, then from proposition \ref{MultpOpPropIfPeqqualsQ} we also have $c'\geq |g|$ for some $c'>0$. Now from the same proposition we conclude that $M_g$ is a topological isomorphism because $c'\geq|g|\geq c$. Assume $p\neq q$, then from previous paragraph the space $(\Omega,\Sigma,\mu)$ consist of finite amount of atoms and $g$ is non zero. Hence $\operatorname{dim}(L_p(\Omega,\Sigma,\mu))=\operatorname{dim}(\ell_p(\Lambda))=\operatorname{Card}(\Lambda)<+\infty$. Similarly, $\operatorname{dim}(L_q(\Omega,\Sigma,\mu))=\operatorname{Card}(\Lambda)<+\infty$. Since $g$ is non zero, then by proposition \ref{MultpOpSurjInjDesc} operator $M_g$ is injective. Thus $M_g$ is an injective operator between finite dimensional spaces of equal dimension. Hence it is a topological isomorphism.

$ii)$$\implies$$ i)$ Conversely, if $M_g$ is a topological isomorphism, then, clearly, it is topologically injective.
\end{proof}

\begin{proposition}\label{TopInjMultOpCharacBtwnTwoContMeasSp} Let $(\Omega,\Sigma,\mu)$ be a $\sigma$-finite measure space, $1\leq p,q\leq+\infty$. Assume $g,\rho\in L_0(\Omega,\Sigma)$ and $\rho$ is non negative. Then the following are equivalent:

$i)$ $M_g\in\mathcal{B}(L_p(\Omega,\mu),L_q(\Omega,\rho\mu))$ is topologically injective;

$ii)$ $M_g$ is a topological isomorphism;

$iii)$ $\rho$ is  positive, $|g \rho^{1/q}|\geq c$ for some $c>0$, if $p\neq q$ the space $(\Omega,\Sigma,\mu)$ consist of finitely many atoms.
\end{proposition}
\begin{proof} $i)$$\implies$$ iii)$ Consider set $E=\rho^{-1}(\{0\})$. Assume $\mu(E)>0$ then $\chi_E\neq 0$ in $L_p(\Omega,\mu)$. On the other hand $(\rho\mu)(E)=\int_E\rho(\omega)d\mu(\omega)=0$, so $\chi_E=0$ in $L_q(\Omega,\rho\mu)$ and $M_g(\chi_E)=g\chi_E=0$ in $L_q(\Omega,\rho \mu)$. Thus we see that $M_g$ is not injective and as the consequence it is not topologically injective. Contradiction, so $\mu(E)=0$ and $\rho$ is  positive. Hence we have an isometric isomorphism $\bar{I}_q:L_q(\Omega,\mu)\to L_q(\Omega,\rho\mu):f\mapsto \rho^{-1/q} f$. Obviously $M_{g\rho^{1/q}}=\bar{I}_q^{-1} M_g\in\mathcal{B}(L_p(\Omega,\mu),L_q(\Omega,\mu))$. Since $\bar{I}_q$ is an isometric isomorphism and $M_g$ is topologically injective, then $M_{g \rho^{1/q}}$ is topologically injective too. From proposition \ref{TopInjMultOpCharacOnMeasSp} we get that $|g\rho^{1/q}|\geq c$ for some $c>0$ and if $p\neq q$ the space is $(\Omega,\Sigma,\mu)$ consist of finite amount of atoms.

$iii)$$\implies$$ i)$ By proposition \ref{TopInjMultOpCharacOnMeasSp} operator $M_{g \rho^{1/q}}$ is topologically injective. Since $\rho$ is positive, then  
we have an isometric isomorphism $\bar{I}_q$. Then from equality $M_g=\bar{I}_q M_{g \rho^{1/q}}$ it follows that $M_g$ is also topologically injective.

$i)$$\implies$$ ii)$ As we proved above assumption implies that $M_{g \rho^{1/q}}$ is topologically injective and $\bar{I}_q$ is an isometric isomorphism. By proposition \ref{TopInjMultOpCharacOnMeasSp} $M_{g \rho^{1/q}}$ is a topological isomorphism. Since $M_g=\bar{I}_q M_{g \rho^{1/q}}$ and $\bar{I}_q$ is an isometric isomorphism, then $M_g$ is a topological isomorphism.

$ii)$$\implies$$ i)$ If $M_g$ is a topological isomorphism, then, obviously, it is topologically injective.
\end{proof}
\begin{proposition}\label{TopInjMultOpCharacBtwnTwoMeasSp} Let $(\Omega,\Sigma,\mu)$, $(\Omega,\Sigma,\nu)$ be two $\sigma$-finite measure spaces, $1\leq p,q\leq +\infty$ and $g\in L_0(\Omega,\Sigma)$. Then the following are equivalent:

$i)$ $M_g\in\mathcal{B}(L_p(\Omega,\mu), L_q(\Omega,\nu))$ is topologically injective;

$ii)$ $M_g^{\Omega_c^{\nu,\mu}}$ is a topological isomorphism;

$iii)$ $\rho_{\nu,\mu}|_{\Omega_c^{\nu,\mu}}$ is positive, $|g \rho_{\nu,\mu}^{1/q}|_{\Omega_c^{\nu,\mu}}|\geq c$ for some $c>0$, if $p\neq q$ the space $(\Omega,\Sigma,\mu)$ consist of finitely many atoms.
\end{proposition}
\begin{proof}
By proposition \ref{MultOpDecompDecomp} operator $M_g$ is topologically injective iff operators $M_g^{\Omega_c^{\nu,\mu}}:L_p(\Omega_c^{\nu,\mu},\mu|_{\Omega_c^{\nu,\mu}})\to L_q(\Omega_c^{\nu,\mu},\rho_{\nu,\mu} \mu|_{\Omega_c^{\nu,\mu}})$ and $M_g^{\Omega_s^{\nu,\mu}}:L_p(\Omega_s^{\nu,\mu},\mu|_{\Omega_s^{\nu,\mu}})\to L_q(\Omega_s^{\nu,\mu},\nu_s|_{\Omega_s^{\nu,\mu}})$ are topologically injective. By proposition \ref{MultOpCharacBtwnTwoSingMeasSp} operator $M_g^{\Omega_s^{\nu,\mu}}$ is zero. Since $\mu(\Omega_s^{\nu,\mu})=0$, then $L_p(\Omega_s^{\nu,\mu},\mu|_{\Omega_s^{\nu,\mu}})=\{0\}$. From these two facts we conclude that $M_g^{\Omega_s^{\nu,\mu}}$ is topologically injective. Thus topological injectivity of $M_g$ is equivalent to topological injectivity of  $M_g^{\Omega_c^{\nu,\mu}}$. It remains to apply proposition \ref{TopInjMultOpCharacBtwnTwoContMeasSp}.
\end{proof}

\begin{proposition}\label{TopInjMultOpDescBtwnTwoMeasSp} Let $(\Omega,\Sigma,\mu)$, $(\Omega,\Sigma,\nu)$ be two $\sigma$-finite measure spaces, $1\leq p,q\leq +\infty$ and $g\in L_0(\Omega,\Sigma)$. Then the following are equivalent:

$i)$ $M_g\in\mathcal{B}(L_p(\Omega,\mu),L_q(\Omega,\nu))$ is topologically injective;

$ii)$ $M_{\chi_{\Omega_c^{\nu,\mu}}/g}\in\mathcal{B}(L_q(\Omega,\nu), L_p(\Omega,\mu))$ is topologically surjective and a left inverse to $M_g$.
\end{proposition}
\begin{proof}
$i)$$\implies$$ ii)$ By proposition \ref{MultOpDecompDecomp}  $M_g^{\Omega_c^{\nu,\mu}}$ is topologically injective. By proposition \ref{TopInjMultOpCharacBtwnTwoContMeasSp} operator $M_g^{\Omega_c^{\nu,\mu}}$ is invertible and $(M_g^{\Omega_c^{\nu,\mu}})^{-1}=M_{1/g}^{\Omega_c^{\nu,\mu}}$. Then for all $h\in L_q(\Omega,\nu)$ we have
$$
\Vert M_{\chi_{\Omega_c^{\nu,\mu}}/g}(h)\Vert_{L_p(\Omega,\mu)}=
\Vert M_{1/g}(h)\chi_{\Omega_c^{\nu,\mu}}\Vert_{L_p(\Omega,\mu)}=
\Vert M_{1/g}^{\Omega_c^{\nu,\mu}}(h|_{\Omega_c^{\nu,\mu}})\Vert_{L_p(\Omega_c^{\nu,\mu},\mu|_{\Omega_c^{\nu,\mu}})}
$$
$$
\leq\Vert M_{1/g}^{\Omega_c^{\nu,\mu}}\Vert\Vert h|_{\Omega_c^{\nu,\mu}}\Vert_{L_q(\Omega_c^{\nu,\mu},\nu|_{\Omega_c^{\nu,\mu}})}
\leq\Vert M_{1/g}^{\Omega_c^{\nu,\mu}}\Vert\Vert h\Vert_{L_q(\Omega,\nu)}
$$ 
So $M_{\chi_{\Omega_c^{\nu,\mu}}/g}$ is bounded. Now note that for all $f\in L_p(\Omega,\mu)$ we have 
$$
M_{\chi_{\Omega_c^{\nu,\mu}}/g}(M_g(f))
=M_{\chi_{\Omega_c^{\nu,\mu}}/g}(g  f)
=(\chi_{\Omega_c^{\nu,\mu}}/g)  g  f
=f \chi_{\Omega_c^{\nu,\mu}}
$$
Since $\mu(\Omega\setminus\Omega_c^{\nu,\mu})=0$, then $\chi_{\Omega_c^{\nu,\mu}}=\chi_{\Omega}$, so $M_{\chi_{\Omega_c^{\nu,\mu}}/g}(M_g(f))=f \chi_{\Omega_c^{\nu,\mu}}=f \chi_{\Omega}=f$. This means that $M_{\chi_{\Omega_c^{\nu,\mu}}/g}$ is a left inverse multiplication operator to $M_g$. Take any $f\in L_p(\Omega,\mu)$, then for $h=M_g(f)$ we have $M_{\chi_{\Omega_c^{\nu,\mu}}/g}(h)=f$ and $\Vert h\Vert_{L_q(\Omega,\nu)}\leq\Vert M_g\Vert\Vert f\Vert_{L_p(\Omega,\mu)}$. Since $f$ is arbitrary, then $M_{\chi_{\Omega_c^{\nu,\mu}}/g}$ is topologically surjective.

Conversely, if $M_g$ has a left inverse operator $M_{\chi_{\Omega_c^{\nu,\mu}}/g}$ then for all $f\in L_p(\Omega,\mu)$ we have 
$$
\Vert M_g(f)\Vert_{L_q(\Omega,\nu)}
\geq\Vert M_{\chi_{\Omega_c^{\nu,\mu}}/g}\Vert^{-1}\Vert M_{\chi_{\Omega_c^{\nu,\mu}}/g}(M_g(f))\vert_{L_p(\Omega,\mu)}
\geq\Vert M_{\chi_{\Omega_c^{\nu,\mu}}/g}\Vert^{-1}\Vert f\Vert_{L_p(\Omega,\mu)}
$$
So $M_g$ is topologically injective.
\end{proof}


\begin{proposition}\label{IsomMultOpCharacOnMeasSp} Let $(\Omega,\Sigma,\mu)$ be a $\sigma$-finite measure space, $1\leq p,q\leq +\infty$ and $g\in L_0(\Omega,\Sigma)$. Then the following are equivalent:

$i)$ $M_g\in\mathcal{B}(L_p(\Omega,\mu),L_q(\Omega,\mu))$ is an isometry;

$ii)$ $|g|=\mu(\Omega)^{1/p-1/q}$, if $p\neq q$, then $(\Omega,\Sigma,\mu)$ consist of single atom.

\end{proposition}
\begin{proof} $i)$$\implies$$ ii)$ Let $p=q$. Assume there exist $E\in\Sigma$ with $\mu(E)>0$ such that $|g|_E|<1$, then
$$
\Vert M_g(\chi_E)\Vert_{L_p(\Omega,\mu)}
=\Vert g \chi_E\Vert_{L_p(\Omega,\mu)}
<\Vert\chi_E\Vert_{L_p(\Omega,\mu)}
=\Vert M_g(\chi_E)\Vert_{L_p(\Omega,\mu)}
$$
Contradiction, hence for all $E\in\Sigma$ with $\mu(E)>0$ we have $|g|_E|\geq 1$ i.e.  $|g|\geq 1$. Assume there exist $E\in\Sigma$ with $\mu(E)>0$ such that $|g|_E|>1$, then
$$
\Vert M_g(\chi_E)\Vert_{L_p(\Omega,\mu)}
=\Vert g \chi_E\Vert_{L_p(\Omega,\mu)}
>\Vert\chi_E\Vert_{L_p(\Omega,\mu)}
=\Vert M_g(\chi_E)\Vert_{L_p(\Omega,\mu)}
$$
Contradiction, hence for all $E\in\Sigma$ with $\mu(E)>0$ we have $|g|_E|\leq 1$ i.e.  $|g|\leq 1$. From both inequalities we get $|g|=1=\mu(\Omega)^{1/p-1/q}$. Let $p\neq q$, then since $M_g$ is an isometry it is topologically injective. By proposition \ref{TopInjMultOpCharacOnMeasSp} the space $(\Omega,\Sigma,\mu)$ consist of finitely many atoms. Assume there are at least two disjoint atoms, say $\Omega_1$ and $\Omega_2$. They are of finite measure, so we can consider respective normalized functions $h_k=\Vert\chi_{\Omega_k}\Vert_{L_p(\Omega,\mu)}^{-1}\chi_{\Omega_k}$ where $k\in\mathbb{N}_2$. Since these atoms are disjoint, then $h_1h_2=0$ and as the result $M_g(h_1)M_g(h_2)=0$. Note that for any $1\leq r\leq +\infty$ and all $f_1,f_2\in L_r(\Omega,\mu)$ such that $f_1f_2=0$ we have 
$$
\Vert f_1+f_2\Vert_{L_r(\Omega,\mu)}
=\left\Vert\left(\Vert f_\lambda\Vert_{L_r(\Omega,\mu)}:\lambda\in\mathbb{N}_2\right)\right\Vert_{\ell_r(\mathbb{N}_2)}
$$
Hence
$$
\Vert M_g(h_1+h_2)\Vert_{L_q(\Omega,\mu)}
=\Vert h_1+h_2\Vert_{L_p(\Omega,\mu)}
=\left\Vert\left( 1 :\lambda\in\mathbb{N}_2\right)\right\Vert_{\ell_p(\mathbb{N}_2)}
=2^{1/p}
$$
But on the other hand
$$
\Vert M_g(h_1+h_2)\Vert_{L_q(\Omega,\mu)}
=\Vert M_g(h_1)+M_g(h_2)\Vert_{L_q(\Omega,\mu)}
$$
$$
=\left\Vert\left(\Vert M_g(h_\lambda)\Vert_{L_q(\Omega,\mu)}:\lambda\in\mathbb{N}_2\right)\right\Vert_{\ell_q(\mathbb{N}_2)}
=\left\Vert\left(\Vert h_\lambda\Vert_{L_p(\Omega,\mu)}:\lambda\in\mathbb{N}_2\right)\right\Vert_{\ell_q(\mathbb{N}_2)}
=2^{1/q}
$$
Thus $2^{1/p}=2^{1/q}$. Contradiction, so $(\Omega,\Sigma,\mu)$ consist of single atom. In this case for all $f\in L_p(\Omega,\mu)$ we have
$$
\Vert M_g(f)\Vert_{L_q(\Omega,\mu)}
=\Vert J_q(M_g(f))\Vert_{\ell_q(\mathbb{N}_1)}
=\Vert J_q(g  f)\Vert_{\ell_q(\mathbb{N}_1)}
=\mu(\Omega)^{1/q-1}\left|\int_\Omega g(\omega) f(\omega)d\mu(\omega)\right|
$$
$$
\Vert f\Vert_{L_p(\Omega,\mu)}
=\Vert J_p(f)\Vert_{\ell_p(\mathbb{N}_1)}
=\mu(\Omega)^{1/p-1}\left|\int_\Omega f(\omega)d\mu(\omega)\right|
$$
By $c$ we denote the constant value of $g$, then
$$
\Vert M_g(f)\Vert_{L_q(\Omega,\mu)}
=\mu(\Omega)^{1/q-1}\left|\int_\Omega g(\omega) f(\omega)d\mu(\omega)\right|
=\mu(\Omega)^{1/q-1}|c|\left|\int_\Omega f(\omega)d\mu(\omega)\right|
$$
From this equality we conclude that in this case $M_g$ is an isometry if
$$
|g|=|c|=\mu(\Omega)^{1/p-1/q}
$$ 

$ii)$$\implies$$ i)$. Let $p=q$, then $|g|=1$. So for all $f\in L_p(\Omega,\mu)$ we have
$$
\Vert M_g(f)\Vert_{L_p(\Omega,\mu)}
=\Vert g  f\Vert_{L_p(\Omega,\mu)}
=\Vert |g|  f\Vert_{L_p(\Omega,\mu)}
=\Vert f\Vert_{L_p(\Omega,\mu)}
$$
hence $M_g$ is an isometry. Let $p\neq q$, then $(\Omega,\Sigma,\mu)$ consist of single atom and we conclude
$$
\Vert M_g(f)\Vert_{L_q(\Omega,\mu)}
=\mu(\Omega)^{1/q-1}\left|\int_\Omega g(\omega) f(\omega)d\mu(\omega)\right|
=\mu(\Omega)^{1/q-1}|c|\left|\int_\Omega f(\omega)d\mu(\omega)\right|
$$
$$
=\mu(\Omega)^{1/p-1}\left|\int_\Omega f(\omega)d\mu(\omega)\right|
=\Vert f\Vert_{L_p(\Omega,\mu)}
$$
Hence $M_g$ is isometric.
\end{proof}

\begin{proposition}\label{IsomMultOpCharacBtwnTwoContMeasSp} Let $(\Omega,\Sigma,\mu)$ be a $\sigma$-finite measure space and $1\leq p,q\leq +\infty$. Assume $g,\rho\in L_0(\Omega,\Sigma)$ and $\rho$ is non negative. Then the following are equivalent:

$i)$ $M_g\in\mathcal{B}(L_p(\Omega,\mu), L_q(\Omega,\rho \mu))$ is isometric;

$ii)$ $M_g$ is an isometric isomorphism;

$iii)$ $\rho$ is positive, $|g  \rho^{1/q}|=\mu(\Omega)^{1/p-1/q}$ and if $p\neq q$ the space $(\Omega,\Sigma,\mu)$ consist of single atom.
\end{proposition}
\begin{proof} $i)$$\implies$$ iii)$ Since $M_g$ is isometric, then it is topologically injective and by proposition \ref{TopInjMultOpCharacBtwnTwoMeasSp} we see that $\rho$ is positive. Hence we have an isometric isomorphism $\bar{I}_q:L_q(\Omega,\mu)\to L_q(\Omega,\rho \mu):f\mapsto \rho^{-1/q}  f$. Obviously $M_{g \rho^{1/q}}=\bar{I}_q^{-1} M_g\in\mathcal{B}(L_p(\Omega,\mu),L_q(\Omega,\mu))$. Since $\bar{I}_q$ is an isometric isomorphism and $M_g$ is isometric, then $M_{g  \rho^{1/q}}$ is isometric too. It remains to apply proposition \ref{IsomMultOpCharacOnMeasSp}.

$iii)$$\implies$$ i)$ By proposition \ref{IsomMultOpCharacOnMeasSp} operator $M_{g \rho^{1/q}}$ is isometric. Since $\rho$ is positive, then we have an isometric isomorphism $\bar{I}_q$. Then from equality $M_g=\bar{I}_q M_{g \rho^{1/q}}$ it follows that $M_g$ is also isometric.

$i)$$\implies$$ ii)$ Since $M_g$ is isometric, then it is topologically injective and by proposition \ref{TopInjMultOpCharacBtwnTwoContMeasSp} it is a topological isomorphism, which is isometric by assumption.

$ii)$$\implies$$ i)$ Since $M_g$ is an isometric isomorphism, trivially, it is isometric.
\end{proof}

\begin{proposition}\label{IsomMultOpCharacBtwnTwoMeasSp} Let $(\Omega,\Sigma,\mu)$, $(\Omega,\Sigma,\nu)$ be two $\sigma$-finite measure spaces, $1\leq p,q\leq +\infty$ and $g\in L_0(\Omega,\Sigma)$. Then the following are equivalent:

$i)$ $M_g$ is isometric;

$ii)$ $M_g^{\Omega_c^{\nu,\mu}}$ is isometric;

$iii)$ $\rho_{\nu,\mu}|_{\Omega_c^{\nu,\mu}}$ is positive, $|g  \rho_{\nu,\mu}^{1/q}|_{\Omega_c^{\nu,\mu}}|=\mu(\Omega_c^{\nu,\mu})^{1/p-1/q}$ and if  $p\neq q$ the space $(\Omega,\Sigma,\mu)$ consist of single atom.
\end{proposition}
\begin{proof}
$i)$$\implies$$ ii)$$\implies$$ iii)$ Since $M_g$ is isometric, then by proposition \ref{MultOpDecompDecomp} operator $M_g^{\Omega_c^{\nu,\mu}}$, is isometric. It remains to apply proposition \ref{IsomMultOpCharacBtwnTwoContMeasSp}.

$iii)$$\implies$$ i)$ By proposition \ref{IsomMultOpCharacBtwnTwoContMeasSp} operator $M_g^{\Omega_c^{\nu,\mu}}$ is isometric. Now take arbitrary $f\in L_p(\Omega,\mu)$. Since $\mu(\Omega\setminus\Omega_c^{\nu,\mu})=0$, then $\chi_{\Omega_c^{\nu,\mu}}=\chi_{\Omega}$ in $L_p(\Omega,\mu)$. As the result $f=f\chi_{\Omega}=f\chi_{\Omega_c^{\nu,\mu}}=f\chi_{\Omega_c^{\nu,\mu}}\chi_{\Omega_c^{\nu,\mu}}$ in $L_p(\Omega,\mu)$ and $M_g(f)=M_g(f\chi_{\Omega_c^{\nu,\mu}})\chi_{\Omega_c^{\nu,\mu}}$. Thus using that $M_g^{\Omega_c^{\nu,\mu}}$ is isometric we get
$$
\Vert M_g(f)\Vert_{L_q(\Omega,\nu)}
=\Vert M_g(f\chi_{\Omega_c^{\nu,\mu}})\chi_{\Omega_c^{\nu,\mu}}\Vert_{L_q(\Omega,\nu)}
=\Vert M_g(f\chi_{\Omega_c^{\nu,\mu}})\Vert_{L_q(\Omega_c^{\nu,\mu},\nu|_{\Omega_c^{\nu,\mu}})}
$$
$$
=\Vert M_g^{\Omega_c^{\nu,\mu}}(f|_{\Omega_c^{\nu,\mu}})\Vert_{L_q(\Omega_c^{\nu,\mu},\nu|_{\Omega_c^{\nu,\mu}})}
=\Vert f|_{\Omega_c^{\nu,\mu}}\Vert_{L_p(\Omega_c^{\nu,\mu},\mu|_{\Omega_c^{\nu,\mu}})}
$$
Since $\mu(\Omega\setminus\Omega_c^{\nu,\mu})=0$ we have $\Vert f|_{\Omega_c^{\nu,\mu}}\Vert_{L_p(\Omega_c^{\nu,\mu},\mu|_{\Omega_c^{\nu,\mu}})}=\Vert f\Vert_{L_p(\Omega,\mu)}$ so $\Vert M_g(f)\Vert_{L_q(\Omega,\nu)}=\Vert f\Vert_{L_p(\Omega,\mu)}$. Therefore $M_g$ is isometric.
\end{proof}

\begin{proposition}\label{IsomMultOpDescBtwnTwoMeasSp} Let $(\Omega,\Sigma,\mu)$, $(\Omega,\Sigma,\nu)$ be two $\sigma$-finite measure spaces, $1\leq p,q\leq +\infty$ and $g\in L_0(\Omega,\Sigma)$. Then the following are equivalent:

$i)$ $M_g\in\mathcal{B}(L_p(\Omega,\mu),L_q(\Omega,\nu))$ is isometric;

$ii)$ $M_{\chi_{\Omega_c^{\nu,\mu}}/g}\in\mathcal{B}(L_q(\Omega,\nu), L_p(\Omega,\mu))$ is strictly coisometric and a left inverse to $M_g$.
\end{proposition}
\begin{proof}
$i)$$\implies$$ ii)$ By proposition \ref{MultOpDecompDecomp} operator $M_g^{\Omega_c^{\nu,\mu}}$ is isometric and by proposition \ref{IsomMultOpCharacBtwnTwoContMeasSp} it is invertible with $(M_g^{\Omega_c^{\nu,\mu}})^{-1}=M_{1/g}^{\Omega_c^{\nu,\mu}}$. Since $M_g^{\Omega_c^{\nu,\mu}}$ is isometric then so does its inverse. Then for all $h\in L_q(\Omega,\nu)$ we have
$$
\Vert M_{\chi_{\Omega_c^{\nu,\mu}}/g}(h)\Vert_{L_p(\Omega,\mu)}=
\Vert M_{1/g}(h|_{\Omega_c^{\nu,\mu}})\Vert_{L_p(\Omega_c^{\nu,\mu},\mu|_{\Omega_c^{\nu,\mu}})}=
\Vert M_{1/g}^{\Omega_c^{\nu,\mu}}(h|_{\Omega_c^{\nu,\mu}})\Vert_{L_p(\Omega_c^{\nu,\mu},\mu|_{\Omega_c^{\nu,\mu}})}
$$
$$
=\Vert h|_{\Omega_c^{\nu,\mu}}\Vert_{L_q(\Omega_c^{\nu,\mu},\nu|_{\Omega_c^{\nu,\mu}})}
\leq \Vert h \Vert_{L_q(\Omega,\nu)}
$$
So $M_{\chi_{\Omega_c^{\nu,\mu}}/g}$ is contractive. Now note that for all $f\in L_p(\Omega,\mu)$ we have 
$$
M_{\chi_{\Omega_c^{\nu,\mu}}/g}(M_g(f))
=M_{\chi_{\Omega_c^{\nu,\mu}}/g}(g  f)
=(\chi_{\Omega_c^{\nu,\mu}}/g)  g  f
=f \chi_{\Omega_c^{\nu,\mu}}
$$
Since $\mu(\Omega\setminus\Omega_c^{\nu,\mu})=0$, then $\chi_{\Omega_c^{\nu,\mu}}=\chi_{\Omega}$, so $M_{\chi_{\Omega_c^{\nu,\mu}}/g}(M_g(f))=f \chi_{\Omega_c^{\nu,\mu}}=f \chi_{\Omega}=f$. This means that $M_{\chi_{\Omega_c^{\nu,\mu}}/g}$ is a left inverse multiplication operator to $M_g$. Take any $f\in L_p(\Omega,\mu)$, then for $h=M_g(f)$ we have $M_{\chi_{\Omega_c^{\nu,\mu}}/g}(h)=f$ and $\Vert h\Vert_{L_q(\Omega,\nu)}\leq\Vert f\Vert_{L_p(\Omega,\mu)}$ i.e.  $M_{\chi_{\Omega_c^{\nu,\mu}}}/g$ is strictly $1$-topologically surjective. Since $M_{\chi_{\Omega_c^{\nu,\mu}}/g}$ is also contractive, then it is strictly coisometric.

$ii)$$\implies$$ i)$ Take any $f\in L_p(\Omega,\mu)$, then there exist $h\in L_q(\Omega,\nu)$ such that $M_{\chi_{\Omega_c^{\nu,\mu}}/g}(h)=f$ and $\Vert h\Vert_{L_q(\Omega,\nu)}\leq \Vert f\Vert_{L_p(\Omega,\mu)}$. Hence
$$
\Vert M_g(f)\Vert_{L_q(\Omega,\nu)}
=\Vert M_g(M_{\chi_{\Omega_c^{\nu,\mu}}/g}(h))\Vert_{L_q(\Omega,\nu)}
=\Vert \chi_{\Omega_c^{\nu,\mu}}h\Vert_{L_q(\Omega,\nu)}
\leq\Vert h\Vert_{L_q(\Omega,\nu|)}
\leq\Vert f\Vert_{L_p(\Omega,\mu)}
$$
Since $M_{\chi_{\Omega_c^{\nu,\mu}}/g}$ is contractive and left inverse to $M_g$ then
$$
\Vert f\Vert_{L_p(\Omega,\mu)}
=\Vert M_{\chi_{\Omega_c^{\nu,\mu}}/g}(M_g(f))\Vert_{L_p(\Omega,\mu)}
\leq\Vert M_g(f)\Vert_{L_q(\Omega,\nu)}
$$
so $\Vert M_g(f)\Vert_{L_q(\Omega,\nu)}=\Vert f\Vert_{L_p(\Omega,\mu)}$. Since $f$ is arbitrary $M_g$ is isometric.
\end{proof}

Now we shall discuss topologically surjective and coisometric multiplication operators. Their description is easier to achieve and most of the proofs are similar (but not identical) to the case of topologically injective and isometric operators.

\begin{proposition}\label{TopSurMultOpCharacOnMeasSp} Let $(\Omega,\Sigma,\mu)$ be a $\sigma$-finite measure space, $1\leq p,q\leq +\infty$ and $g\in L_0(\Omega,\Sigma)$. Then the following are equivalent:

$i)$ $M_g\in\mathcal{B}(L_p(\Omega,\mu),L_q(\Omega,\mu))$ is topologically surjective;

$ii)$ $M_g$ is a topological isomorphism;

$iii)$ $|g|\geq c$ for some $c>0$, if $p\neq q$ the space $(\Omega,\Sigma,\mu)$ consist of finitely many atoms.
\end{proposition}
\begin{proof} $i)$$\implies$$ ii)$ Since $M_g$ is topologically surjective, then it is surjective and by proposition \ref{MultpOpSurjInjDesc} it is also injective. Thus $M_g$ is bijective. Since $L_p$ spaces are complete, from open mapping theorem we see that $M_g$ is a topological isomorphism. 

$ii)$$\implies$$ i)$ If $M_g$ is a topological isomorphism, obviously, it is topologically surjective.

$ii)\Longleftrightarrow iii)$ Follows from proposition \ref{TopInjMultOpCharacOnMeasSp}
\end{proof}
 
\begin{proposition}\label{TopSurMultOpCharacBtwnTwoContMeasSp} Let $(\Omega,\Sigma,\nu)$ be a $\sigma$-finite measure space, $1\leq p,q\leq +\infty$ and $g,\rho\in L_0(\Omega,\Sigma)$ and $\rho$ is non negative, then the following are equivalent:

$i)$ $M_g\in\mathcal{B}(L_p(\Omega,\rho \nu),L_q(\Omega,\nu))$ is topologically surjective;

$ii)$ $M_g$ is a topological isomorphism;

$iii)$ $\rho$ is positive, $|g  \rho^{-1/p}|\geq c$ for some $c>0$, if $p\neq q$ the space $(\Omega,\Sigma,\mu)$ consist of finitely many atoms.
\end{proposition}
\begin{proof} $i)$$\implies$$ iii)$ Consider set $E=\rho^{-1}(\{0\})$. Assume $\nu(E)>0$ then $\chi_E\neq 0$ in $L_q(\Omega,\nu)$. On the other hand $(\rho \nu)(E)=\int_E\rho(\omega)d\nu(\omega)=0$, so $\chi_E=0$ in $L_p(\Omega,\rho \nu)$. Then for all $f\in L_p(\Omega,\rho \nu)$ holds $M_g(f)\chi_E=M_g(f \chi_E)=M_g(0)=0$ in $L_q(\Omega,\nu)$. The last equality means that $\operatorname{Im}(M_g)\subset\{h\in L_q(\Omega,\nu): h|_E=0\}$. Since $\nu(E)\neq 0$ we see that $M_g$ is not surjective and as the consequence it is not topologically surjective. Contradiction, so $\nu(E)=0$ and $\rho$ is positive. Hence we have an isometric isomorphism $\bar{I}_p:L_p(\Omega,\nu)\to L_p(\Omega,\rho \nu):f\mapsto \rho^{-1/p}  f$. Obviously $M_{g \rho^{-1/p}}=M_g \bar{I}_p\in\mathcal{B}(L_p(\Omega,\nu),L_q(\Omega,\nu))$. Since $\bar{I}_p$ is an isometric isomorphism and $M_g$ is topologically surjective, then $M_{g  \rho^{-1/p}}$ is topologically surjective too. It remains to apply proposition \ref{TopSurMultOpCharacOnMeasSp}.

$iii)$$\implies$$ i)$ By proposition \ref{TopSurMultOpCharacOnMeasSp} operator $M_{g \rho^{-1/p}}$ is topologically surjective. Since $\rho$ is positive, then we have an isometric isomorphism $\bar{I}_p$. Then from equality $M_g= M_{g \rho^{-1/p}}\bar{I}_p^{-1}$ it follows that $M_g$ is also topologically surjective.

$i)$$\implies$$ ii)$ As we proved above the operator $M_{g \rho^{1/q}}$ is topologically injective and $\bar{I}_q$ is an isometric isomorphism. By proposition \ref{TopSurMultOpCharacOnMeasSp} $M_{g \rho^{1/q}}$ is a topological isomorphism. Since $M_g=\bar{I}_q M_{g \rho^{1/q}}$ we see that $M_g$ is also a topological isomorphism, as composition of such.

$ii)$$\implies$$ i)$. If $M_g$ is a topological isomorphism, obviously, it is topologically surjective.
\end{proof}

\begin{proposition}\label{TopSurMultOpCharacBtwnTwoMeasSp} Let $(\Omega,\Sigma,\mu)$, $(\Omega,\Sigma,\nu)$ be two $\sigma$-finite measure spaces, $1\leq p,q\leq +\infty$ and $g\in L_0(\Omega,\Sigma)$. Then the following are equivalent:

$i)$ $M_g\in\mathcal{B}(L_p(\Omega,\mu), L_q(\Omega,\nu))$ is topologically surjective;

$ii)$ $M_g^{\Omega_c^{\mu,\nu}}$ is topologically surjective;

$iii)$ $\rho_{\mu,\nu}|_{\Omega_c^{\mu,\nu}}$ is positive, $|g \rho_{\mu,\nu}^{-1/p}|_{\Omega_c^{\mu,\nu}}|\geq c$ for some $c>0$, if $p\neq q$ the space $(\Omega,\Sigma,\mu)$ consist of finitely many atoms.
\end{proposition}
\begin{proof}
By proposition \ref{MultOpDecompDecomp} operator $M_g$ is topologically surjective iff operators $M_g^{\Omega_c^{\mu,\nu}}:L_p(\Omega_c^{\mu,\nu},\rho_{\mu,\nu} \nu|_{\Omega_c^{\mu,\nu}})\to L_q(\Omega_c^{\mu,\nu},\nu|_{\Omega_c^{\mu,\nu}})$ and $M_g^{\Omega_s^{\mu,\nu}}:L_p(\Omega_s^{\mu,\nu},\mu_s|_{\Omega_s^{\mu,\nu}})\to L_q(\Omega_s^{\mu,\nu},\nu|_{\Omega_s^{\mu,\nu}})$ are topologically surjective. By proposition \ref{MultOpCharacBtwnTwoSingMeasSp} operator $M_g^{\Omega_s^{\mu,\nu}}$ is zero. Since $\nu(\Omega_s^{\mu,\nu})=0$, then $L_p(\Omega_s^{\mu,\nu},\nu|_{\Omega_s^{\mu,\nu}})=\{0\}$. From these two facts we conclude that $M_g^{\Omega_s^{\mu,\nu}}$ is topologically surjective. Thus topological surjectivity of $M_g$ is equivalent to topological surjectivity of  $M_g^{\Omega_c^{\mu,\nu}}$. It remains to apply proposition \ref{TopSurMultOpCharacBtwnTwoContMeasSp}.
\end{proof}

\begin{proposition}\label{TopSurMultOpDescBtwnTwoMeasSp} Let $(\Omega,\Sigma,\mu)$, $(\Omega,\Sigma,\nu)$ be two $\sigma$-finite measure spaces, $1\leq p,q\leq +\infty$ and $g\in L_0(\Omega,\Sigma)$. Then the following are equivalent:

$i)$ $M_g\in\mathcal{B}(L_p(\Omega,\mu),L_q(\Omega,\nu))$ is topologically surjective;

$ii)$ $M_{\chi_{\Omega_c^{\mu,\nu}}/g}\in\mathcal{B}(L_q(\Omega,\nu), L_p(\Omega,\mu))$ is topologically injective operator and a right inverse to $M_g$.
\end{proposition}
\begin{proof}
$i)$$\implies$$ ii)$ By proposition \ref{MultOpDecompDecomp} operator $M_g^{\Omega_c^{\mu,\nu}}$ is topologically surjective. By proposition \ref{TopSurMultOpCharacBtwnTwoContMeasSp} it is invertible and $(M_g^{\Omega_c^{\mu,\nu}})^{-1}=M_{1/g}^{\Omega_c^{\mu,\nu}}$. Then for all $h\in L_q(\Omega,\nu)$ we have
$$
\Vert M_{\chi_{\Omega_c^{\mu,\nu}}/g}(h)\Vert_{L_p(\Omega,\mu)}=
\Vert M_{1/g}(h|_{\Omega_c^{\mu,\nu}})\Vert_{L_p(\Omega_c^{\mu,\nu},\mu|_{\Omega_c^{\mu,\nu}})}=
\Vert M_{1/g}^{\Omega_c^{\mu,\nu}}(h|_{\Omega_c^{\mu,\nu}})\Vert_{L_p(\Omega_c^{\mu,\nu},\mu|_{\Omega_c^{\mu,\nu}})}
$$
$$
\leq\Vert M_{1/g}^{\Omega_c^{\mu,\nu}}\Vert\Vert h|_{\Omega_c^{\mu,\nu}}\Vert_{L_q(\Omega_c^{\mu,\nu},\nu|_{\Omega_c^{\mu,\nu}})}
\leq\Vert M_{1/g}^{\Omega_c^{\mu,\nu}}\Vert\Vert h\Vert_{L_q(\Omega,\nu)}
$$ 
So $M_{\chi_{\Omega_c^{\mu,\nu}}/g}$ is bounded. Now note that for all $h\in L_q(\Omega,\nu)$ we have 
$$
M_g(M_{\chi_{\Omega_c^{\mu,\nu}}/g}(h))
=M_g(\chi_{\Omega_c^{\mu,\nu}}/g  h)
=g (\chi_{\Omega_c^{\mu,\nu}}/g)   h
=h \chi_{\Omega_c^{\mu,\nu}}
$$
Since $\nu(\Omega\setminus\Omega_c^{\mu,\nu})=0$, then $\chi_{\Omega_c^{\mu,\nu}}=\chi_{\Omega}$, so $M_g(M_{\chi_{\Omega_c^{\mu,\nu}}/g}(h))=h \chi_{\Omega_c^{\mu,\nu}}=h \chi_{\Omega}=h$. This means that $M_{\chi_{\Omega_c^{\mu,\nu}}/g}$ is a right inverse multiplication operator to $M_g$. Take any $h\in L_q(\Omega,\nu)$, then
$$
\Vert M_{\chi_{\Omega_c^{\mu,\nu}}/g}(h)\Vert_{L_p(\Omega,\mu)}
\geq\Vert M_g\Vert\Vert M_g(M_{\chi_{\Omega_c^{\mu,\nu}}/g}(h))\Vert_{L_q(\Omega,\nu)}
\geq\Vert M_g\Vert\Vert h\Vert_{L_q(\Omega,\nu)}
$$
Since $h$ is arbitrary we get that $M_{\chi_{\Omega_c^{\mu,\nu}}/g}$ is topologically injective.

$ii)$$\implies$$ i)$ Take arbitrary $h\in L_q(\Omega,\nu)$ and consider $f=M_{\chi_{\Omega_c^{\mu,\nu}}/g}(h)$. Then $M_g(f)=M_g(M_{\chi_{\Omega_c^{\mu,\nu}}/g}(h))=h$ and $\Vert f\Vert_{L_p(\Omega,\mu)}\leq\Vert M_{\chi_{\Omega_c^{\mu,\nu}}/g}\Vert\Vert h\Vert_{L_q(\Omega,\nu)}$. Since $h$ is arbitrary we get that $M_g$ is topologically surjective.
\end{proof}


\begin{proposition}\label{CoisomMultOpCharacOnMeasSp} Let $(\Omega,\Sigma,\mu)$ be a $\sigma$-finite measure space, $1\leq p,q\leq +\infty$ and $g\in L_0(\Omega,\Sigma)$. Then the following are equivalent:

$i)$ $M_g\in\mathcal{B}(L_p(\Omega,\mu),L_q(\Omega,\mu))$ is coisometric;

$ii)$ $M_g$ is an isometric isomorphism;

$iii)$ $|g|=\mu(\Omega)^{1/q-1/p}$, if $p\neq q$ the space $(\Omega,\Sigma,\mu)$ consist of single atom.
\end{proposition}
\begin{proof} Since $M_g$ is coisometric it is topologically injective, so from proposition \ref{TopSurMultOpCharacOnMeasSp} we get that $M_g$ is in fact a topological isomorphism. As the consequence it is injective, but injective coisometric operator is an isometric isomorphisms. It remains to note that every isometric isomorphism is a strict coisometry. Thus we conclude that $M_g$ is coisometric iff it is strictly coisometric iff it is an isometric isomorphism. Now we apply proposition \ref{IsomMultOpCharacOnMeasSp}.
\end{proof}

\begin{proposition}\label{CoisomMultOpCharacBtwnTwoContMeasSp} Let $(\Omega,\Sigma,\nu)$ be a $\sigma$-finite measure space, $1\leq p,q\leq +\infty$. Assume $g,\rho\in L_0(\Omega,\Sigma)$ and $\rho$ is non negative. Then the following are equivalent:

$i)$ $M_g\in\mathcal{B}(L_p(\Omega,\rho \nu),L_q(\Omega,\nu))$ is coisometric;

$ii)$ $M_g$ is an isometric isomorphism;

$iii)$ $\rho$ is positive, $|g  \rho^{-1/p}|=\mu(\Omega)^{1/p-1/q}$, if $p\neq q$ the space $(\Omega,\Sigma,\mu)$ consist single atom.
\end{proposition}
\begin{proof} $i)$$\implies$$ ii)$ Assume $M_g$ is coisometric, then it is topologically surjective. By proposition \ref{TopSurMultOpCharacBtwnTwoContMeasSp} operator $M_g$ is a topological isomorphism, hence bijective. It remains to note that bijective coisometry is an isometric isomorphism.

$ii)$$\implies$$ i)$ If $M_g$ is an isometric isomorphism, of course, it is coisometry and even more a strict coisometry.

$ii)\Longleftrightarrow iii)$ Follows from proposition \ref{IsomMultOpCharacBtwnTwoContMeasSp}.
\end{proof}

\begin{proposition}\label{CoisomMultOpCharacBtwnTwoMeasSp} Let $(\Omega,\Sigma,\mu)$, $(\Omega,\Sigma,\nu)$ be two $\sigma$-finite measure spaces, $1\leq p,q\leq +\infty$ and $g\in L_0(\Omega,\Sigma)$. Then the following are equivalent: 

$i)$ $M_g\in\mathcal{B}(L_p(\Omega,\mu), L_q(\Omega,\nu))$ is coisometric;

$ii)$ $M_g^{\Omega_c^{\mu,\nu}}$ is an isometric isomorphism;

$iii)$ $\rho_{\mu,\nu}|_{\Omega_c^{\mu,\nu}}$ is positive, $|g \rho_{\mu,\nu}^{-1/p}|_{\Omega_c^{\mu,\nu}}|=\mu(\Omega_c^{\mu,\nu})^{1/p-1/q}$, if $p\neq q$ the space $(\Omega,\Sigma,\mu)$ consist of single atom.
\end{proposition}
\begin{proof} $i)$$\implies$$ ii)$ Since $M_g$ is coisometric, then from proposition \ref{MultOpDecompDecomp} we know that $M_g^{\Omega_c^{\mu,\nu}}$ is also coisometric. From proposition \ref{CoisomMultOpCharacBtwnTwoContMeasSp} we get that $M_g^{\Omega_c^{\mu,\nu}}$ is an isometric isomorphism. 

$ii)$$\implies$$ i)$ Take arbitrary $h\in L_q(\Omega,\nu)$, then there exists $f\in L_p(\Omega_c^{\mu,\nu},\mu|_{\Omega_c^{\mu,\nu}})$ such that $M_g^{\Omega_c^{\mu,\nu}}(f)=h|_{\Omega_c^{\mu,\nu}}$. By proposition \ref{MultOpCharacBtwnTwoSingMeasSp} operator $M_g^{\Omega_s^{\mu,\nu}}=0$, so
$$
M_g(\widetilde{f})
=\widetilde{M_g^{\Omega_c^{\mu,\nu}}(\widetilde{f}|_{\Omega_c^{\mu,\nu}})}+\widetilde{M_g^{\Omega_s^{\mu,\nu}}(\widetilde{f}|_{\Omega_s^{\mu,\nu}})}
=\widetilde{h|_{\Omega_c^{\mu,\nu}}}
$$
Since $\nu(\Omega_s^{\mu,\nu})=0$, then $\Vert h-\widetilde{h|_{\Omega_c^{\mu,\nu}}}\Vert_{L_q(\Omega,\nu)}=\Vert h\chi_{\Omega_s^{\mu,\nu}}\Vert_{L_q(\Omega,\nu)}=0$ and we conclude $h=\widetilde{h|_{\Omega_c^{\mu,\nu}}}$. So we found $\widetilde{f}\in L_p(\Omega,\mu)$ such that $M_g(\widetilde{f})=h$ and $\Vert \widetilde{f}\Vert_{L_p(\Omega,\mu)}=\Vert f\Vert_{L_p(\Omega_c^{\mu,\nu},\mu|_{\Omega_c^{\mu,\nu}})}=\Vert h|_{\Omega_c^{\mu,\nu}}\Vert_{L_q(\Omega_c^{\mu,\nu},\nu|_{\Omega_c^{\mu,\nu}})}\leq\Vert h\Vert_{L_q(\Omega,\nu)}$. Since $h$ is arbitrary, then $M_g$ is $1$-topologically surjective. For all $f\in L_p(\Omega,\mu)$ we have
$$
\Vert M_g(f)\Vert_{L_q(\Omega,\nu)}
=\Vert\widetilde{M_g^{\Omega_c^{\mu,\nu}}(f|_{\Omega_c^{\mu,\nu}})}+\widetilde{M_g^{\Omega_s^{\mu,\nu}}(f|_{\Omega_s^{\mu,\nu}})}\Vert_{L_q(\Omega,\nu)}
=\Vert\widetilde{M_g^{\Omega_c^{\mu,\nu}}(f|_{\Omega_c^{\mu,\nu}})}\Vert_{L_q(\Omega,\nu)}
$$
$$
=\Vert M_g^{\Omega_c^{\mu,\nu}}(f|_{\Omega_c^{\mu,\nu}})\Vert_{L_q(\Omega_c^{\mu,\nu},\nu|_{\Omega_c^{\mu,\nu}})}
=\Vert f|_{\Omega_c^{\mu,\nu}}\Vert_{L_p(\Omega_c^{\mu,\nu},\mu|_{\Omega_c^{\mu,\nu}})}
\leq\Vert f \Vert_{L_p(\Omega,\mu)}
$$
Since $f$ is arbitrary, then $M_g$ is contractive, but it is also $1$-topologically surjective. Thus $M_g$ is coisometric.

$ii)\Longleftrightarrow iii)$ Follows from proposition \ref{CoisomMultOpCharacBtwnTwoContMeasSp}.
\end{proof}

\begin{proposition}\label{CoisomMultOpDescBtwnTwoMeasSp} Let $(\Omega,\Sigma,\mu)$, $(\Omega,\Sigma,\nu)$ be two $\sigma$-finite measure spaces, $1\leq p,q\leq +\infty$ and $g\in L_0(\Omega,\Sigma)$. Then the following are equivalent: 

$i)$ $M_g\in\mathcal{B}(L_p(\Omega,\mu),L_q(\Omega,\nu))$ is coisometric;

$ii)$ $M_{\chi_{\Omega_c^{\mu,\nu}}/g}\in\mathcal{B}(L_q(\Omega,\nu), L_p(\Omega,\mu))$ is isometric and a right inverse to $M_g$.
\end{proposition}
\begin{proof}
$i)$$\implies$$ ii)$ By proposition \ref{MultOpDecompDecomp} operator $M_g^{\Omega_c^{\mu,\nu}}$ is coisometric and by proposition \ref{CoisomMultOpCharacBtwnTwoContMeasSp} it is isometric, invertible and $(M_g^{\Omega_c^{\mu,\nu}})^{-1}=M_{1/g}^{\Omega_c^{\mu,\nu}}$. Then for all $h\in L_q(\Omega,\nu)$ we have
$$
\Vert M_{\chi_{\Omega_c^{\mu,\nu}}/g}(h)\Vert_{L_p(\Omega,\mu)}=
\Vert M_{1/g}(h)\chi_{\Omega_c^{\mu,\nu}}\Vert_{L_p(\Omega,\mu)}=
\Vert M_{1/g}(h|_{\Omega_c^{\mu,\nu}})\Vert_{L_p(\Omega_c^{\mu,\nu},\mu|_{\Omega_c^{\mu,\nu}})}$$
$$
=
\Vert M_{1/g}^{\Omega_c^{\mu,\nu}}(h|_{\Omega_c^{\mu,\nu}})\Vert_{L_p(\Omega_c^{\mu,\nu},\mu|_{\Omega_c^{\mu,\nu}})}
=\Vert h|_{\Omega_c^{\mu,\nu}}\Vert_{L_q(\Omega_c^{\mu,\nu},\nu|_{\Omega_c^{\mu,\nu}})}
\leq\Vert h\Vert_{L_q(\Omega,\nu)}
$$ 
So $M_{\chi_{\Omega_c^{\mu,\nu}}/g}$ is contractive. Now note that for all $h\in L_q(\Omega,\nu)$ we have 
$$
M_g(M_{\chi_{\Omega_c^{\mu,\nu}}/g}(h))
=M_g(\chi_{\Omega_c^{\mu,\nu}}/g  h)
=g (\chi_{\Omega_c^{\mu,\nu}}/g)   h
=h \chi_{\Omega_c^{\mu,\nu}}
$$
Since $\nu(\Omega\setminus\Omega_c^{\mu,\nu})=0$, then $\chi_{\Omega_c^{\mu,\nu}}=\chi_{\Omega}$, so $M_g(M_{\chi_{\Omega_c^{\mu,\nu}}/g}(h))=h \chi_{\Omega_c^{\mu,\nu}}=h \chi_{\Omega}=h$. This means that $M_{\chi_{\Omega_c^{\mu,\nu}}/g}$ is a right inverse multiplication operator to $M_g$. Take any $h\in L_q(\Omega,\nu)$, then
$$
\Vert M_{\chi_{\Omega_c^{\mu,\nu}}/g}(h)\Vert_{L_p(\Omega,\mu)}
\geq\Vert M_g\Vert\Vert M_g(M_{\chi_{\Omega_c^{\mu,\nu}}/g}(h))\Vert_{L_q(\Omega,\nu)}
\geq\Vert h\Vert_{L_q(\Omega,\nu)}
$$
Since $h$ is arbitrary $M_{\chi_{\Omega_c^{\mu,\nu}}/g}$ is $1$-topologically injective, but it is contractive. Thus $M_{\chi_{\Omega_c^{\mu,\nu}}/g}$ is isometric.

$ii)$$\implies$$ i)$ Take arbitrary $h\in L_q(\Omega,\nu)$ and consider $f=M_{\chi_{\Omega_c^{\mu,\nu}}/g}(h)$. Then $M_g(f)=M_g(M_{\chi_{\Omega_c^{\mu,\nu}}/g}(h))=h$ and $\Vert f\Vert_{L_p(\Omega,\mu)}\leq\Vert h\Vert_{L_q(\Omega,\nu)}$. Since $h$ is arbitrary $M_g$ is strictly $1$-topologically surjective. Let $f\in L_p(\Omega,\mu)$. By assumption $M_{\chi_{\Omega_c^{\mu,\nu}}/g}$ is isometric, so
$$
\Vert M_g(f)\Vert_{L_q(\Omega,\nu)}
=\Vert M_{\chi_{\Omega_c^{\mu,\nu}}/g}(M_g(f))\Vert_{L_p(\Omega,\mu)}
=\Vert f\chi_{\Omega_c^{\mu,\nu}}\Vert_{L_p(\Omega,\mu)}
\leq\Vert f\Vert_{L_p(\Omega,\mu)}
$$
Since $f$ is arbitrary $M_g$ is contractive, but it is also strictly $1$-topologically surjective, hence strictly coisometric.
\end{proof}

Note that this proof shows that every coisometric multiplication operator is strictly coisometric.



%----------------------------------------------------------------------------------------
%	Homological triviality of the category B(Omega,Sigma)-modules L_p
%----------------------------------------------------------------------------------------

\subsection{Homological triviality of the category \texorpdfstring{$B(\Omega,\Sigma)$-modules $L_p$}{B(Omega)-modules Lp}}
\label{SubSectionHomologicalTrivialityOfTheCategoryBOmegaSigmaModulesLp}

We are ready to prove simultaneously funny and disappointing result.

\begin{proposition}\label{HomTrivlOfLpCat} Let $(\Omega,\Sigma)$ be a measurable space and $\mu$ be a $\sigma$-finite measure on $\Omega$. Then
$L_p(\Omega,\mu)$ is $\langle$~metrically / topologically~$\rangle$ projective, injective and flat in $\langle$~$B(\Omega,\Sigma)-\mathbf{mod(L)}_1$ / $B(\Omega,\Sigma)-\mathbf{mod(L)}$~$\rangle$.
\end{proposition}
\begin{proof} Denote $X:=L_p(\Omega,\mu)$ and $\langle$~$\mathbf{C}:=B(\Omega,\Sigma)-\mathbf{mod(L)}_1$ / $\mathbf{C}:=B(\Omega,\Sigma)-\mathbf{mod(L)}$~$\rangle$. 

Consider covariant functor $\langle$~$F_{proj}:=\operatorname{Hom}_{\mathbf{C}}(X,-):\mathbf{C}\to\mathbf{Ban}_1$ / $F_{proj}:=\operatorname{Hom}_{\mathbf{C}}(X,-):\mathbf{C}\to\mathbf{Ban}$~$\rangle$. By proposition $\langle$~\ref{CoisomMultOpDescBtwnTwoMeasSp} / \ref{TopSurMultOpCharacBtwnTwoMeasSp}~$\rangle$ any $\langle$~coisometric / topologically surjective~$\rangle$ morphism $\xi$ in $\mathbf{C}$ is a retraction, hence $F_{proj}(\xi)$ is a retraction in $\langle$~$\mathbf{Ban}_1$ / $\mathbf{Ban}$~$\rangle$, and therefore $\langle$~strictly coisometric / surjective~$\rangle$. Since $\xi$ is arbitrary, then $X$ is $\langle$~metrically / topologically~$\rangle$ projective.

Consider contravariant functor $\langle$~$F_{inj}=\operatorname{Hom}_{\mathbf{C}}(-,X):\mathbf{C}\to\mathbf{Ban}_1$ / $F_{inj}=\operatorname{Hom}_{\mathbf{C}}(-,X):\mathbf{C}\to\mathbf{Ban}$~$\rangle$. By proposition $\langle$~\ref{IsomMultOpDescBtwnTwoMeasSp} / \ref{TopInjMultOpDescBtwnTwoMeasSp}~$\rangle$ any $\langle$~isometric / topologically injective~$\rangle$ morphism $\xi$ in $\mathbf{C}$ is a coretraction, hence $F_{inj}(\xi)$ is a retraction in $\langle$~$\mathbf{Ban}_1$ / $\mathbf{Ban}$~$\rangle$, and therefore $\langle$~strictly coisometric / surjective~$\rangle$. Since $\xi$ is arbitrary, then $X$ is $\langle$~metrically / topologically~$\rangle$ injective.

Consider covariant functor $\langle$~$F_{flat}=-\projmodtens{B(\Omega,\Sigma)}X:\mathbf{C}\to\mathbf{Ban}_1$ / $F_{flat}=-\projmodtens{B(\Omega,\Sigma)}X:\mathbf{C}\to\mathbf{Ban}$~$\rangle$. Again, by proposition $\langle$~\ref{IsomMultOpDescBtwnTwoMeasSp} / \ref{TopInjMultOpDescBtwnTwoMeasSp}~$\rangle$ any $\langle$~isometric / topologically injective~$\rangle$ morphism $\xi$ in $\mathbf{C}$ is a coretraction, hence  $F_{flat}(\xi)$ is a coretraction in $\langle$~$\mathbf{Ban}_1$ / $\mathbf{Ban}$~$\rangle$, in particular it is $\langle$~isometric / topologically injective~$\rangle$. Since $\xi$ is arbitrary, then $X$ is $\langle$~metrically / topologically~$\rangle$ flat.
\end{proof}

%----------------------------------------------------------------------------------------
%	THESIS CONTENT - APPENDICES
%----------------------------------------------------------------------------------------

\addtocontents{toc}{\vspace{2em}} % Add a gap in the Contents, for aesthetics

\appendix % Cue to tell LaTeX that the following 'chapters' are Appendices

% Include the appendices of the thesis as separate files from the Appendices folder
% Uncomment the lines as you write the Appendices

%\input{./Appendices/AppendixA}


\addtocontents{toc}{\vspace{2em}} % Add a gap in the Contents, for aesthetics

%----------------------------------------------------------------------------------------
%	SYMBOLS
%----------------------------------------------------------------------------------------

%\clearpage % Start a new page

%\lhead{\emph{Symbols}} % Set the left side page header to "Symbols"

%\listofnomenclature{ll} % Include a list of Symbols (a three column table)
%{
%$a$ & distance \\
%$P$ & power  \\
% Symbol & Description \\
%}


%----------------------------------------------------------------------------------------
%	BIBLIOGRAPHY
%----------------------------------------------------------------------------------------

\backmatter

\label{Bibliography}

\lhead{\emph{Bibliography}} % Change the page header to say "Bibliography"

\bibliographystyle{unsrtnat} % Use the "unsrtnat" BibTeX style for formatting the Bibliography

\bibliography{Bibliography} % The references (bibliography) information are stored in the file named "Bibliography.bib"

\end{document}