%\title{Топологически плоские банаховы модули}
\documentclass[12pt]{article}
\usepackage[left=2cm,right=2cm,top=2cm,bottom=2cm,bindingoffset=0cm]{geometry}
\usepackage{amssymb}
\usepackage{amsmath}
\usepackage{amsthm}
\usepackage{enumerate}
\usepackage[T1,T2A]{fontenc}
\usepackage[utf8]{inputenc}
\usepackage[russian]{babel}
\usepackage[colorlinks=true, urlcolor=blue, linkcolor=blue, citecolor=blue, pdfborder={0 0 0}]{hyperref}

%----------------------------------------------------------------------------------------

% My definitions

\newtheorem{theorem}{Теорема}[section]
\newtheorem{lemma}[theorem]{Лемма}
\newtheorem{proposition}[theorem]{Предложение}
\newtheorem{remark}[theorem]{Замечание}
\newtheorem{corollary}[theorem]{Следствие}
\newtheorem{definition}[theorem]{Определение}
\newtheorem{example}[theorem]{Пример}

\newcommand{\projtens}{\mathbin{\widehat{\otimes}}}
\newcommand{\convol}{\ast}
\newcommand{\projmodtens}[1]{\mathbin{\widehat{\otimes}}_{#1}}
\newcommand{\isom}[1]{\mathop{\mathbin{\cong}}\limits_{#1}}
%----------------------------------------------------------------------------------------

\begin{document}

\begin{center}
\Large \textbf{Топологически плоские банаховы модули}\\[0.5cm]
\small {Н. Т. Немеш}\footnote{Работа выполнена при поддержке Российского фонда фундаментальных исследований (грант номер 15-01-08392).}\\[0.5cm]
\end{center}

\thispagestyle{empty}

\medskip
\textbf{Аннотация:} В работе дано несколько необходимых условий топологической плоскости банаховых модулей. Основной результат звучит следующим образом: банахов модуль над относительно аменабельной банаховой алгеброй топологически плоский как банахово пространство является топологически плоским как банахов модуль. В заключение, мы рассмотрим примеры топологически плоских модулей анализа.

\medskip
\textbf{Ключевые слова:} банахов модуль, топологическая плоскость, аменабельность, $\mathcal{L}_1^g$-пространство, $\mathcal{L}_\infty^g$-пространство.

\textbf{Abstract:} Several necessary conditions of topological flatness of Banach modules are given in this paper. The main result of the paper is as follows: a Banach module over relatively amenable Banach algebra which is topologically flat as Banach space is topologically flat as Banach module. Finally, we provide few examples of topologically flat modules among classical modules of analysis.

\medskip
\textbf{Keywords:} Banach module, topological flatness, amenability, $\mathcal{L}_1^g$-space, $\mathcal{L}_\infty^g$-space.

\bigskip

%----------------------------------------------------------------------------------------
%	Introduction
%----------------------------------------------------------------------------------------

\section{Введение}
\label{SectionIntroduction}

Аменабельность, инъективность и плоскость всегда были тесно связанными понятиями банаховой гомологии. Мы покажем, что относительная аменабельность играет ключевую роль в исследовании топологической версии банаховой гомологии. В некоторых случаях мы даже получим полное описание топологически плоских банаховых модулей как $\mathcal{L}_1^g$-пространств. Из работы Резерфорда \cite{StegRethNucOpL1LInfSp} известно, что эти пространства совпадают с классом топологически плоских банаховых пространств. 

В дальнейшем, в предложениях мы будем использовать сразу несколько фраз, последовательно перечисляя их и заключая в скобки таким образом: $\langle$~.../...~$\rangle$. Например: число $x$ называется $\langle$~положительным / неотрицательным~$\rangle$ если $\langle$~$x>0$ / $x\geq 0$~$\rangle$. Мы будем использовать символ $:=$ для обозначения равенства по определению.

Все банаховы пространства рассматриваются над полем комплексных чисел. Пусть $E$ --- банахово пространство. Через $B_E$ мы будем обозначать замкнутый единичный шар в $E$. Символ $\operatorname{cl}_E(S)$ будет обозначать замыкание множества $S$ в $E$. Если $F$ --- еще одно банахово пространство, то мы будем говорить, что линейный оператор $T:E\to F$ является $\langle$~изометрическим / $c$-топологически инъективным~$\rangle$, если $\langle$~$\Vert T(x)\Vert=\Vert x\Vert$ / $c\Vert T(x)\Vert\geq\Vert x\Vert$~$\rangle$ для всех $x\in E$. Аналогично, $T$ называется $\langle$~строго коизометрически / строго $c$-топологически сюръективным~$\rangle$, если $\langle$~$T(B_E)=B_F$ / $c T(B_E)\supset B_F$~$\rangle$. В некоторых ситуациях мы не будем упоминать константу $c$. Мы будем использовать символ $\bigoplus_p$ для обозначения $\ell_p$-суммы банаховых пространств, и $\projtens$ для обозначения проективного тензорного произведения банаховых пространств. В работе мы встретимся с так называемыми $\mathcal{L}_p^g$-пространствами, которые являются небольшой модификацией $\mathcal{L}_p$-пространств определенных Линденштрауссом и Пелчинским в их пионерской работе \cite{LinPelAbsSumOpInLpSpAndApp}. Мы будем называть банахово пространство $E$  $\mathcal{L}_{p,C}^g$-пространством, если для любого $\epsilon>0$ и любого конечномерного подпространства $F$ в $E$ существует конечномерное $\ell_p$-пространство $G$ и два ограниченных линейных оператора $S:F\to G$, $T:G\to E$ такие, что $TS|^F=1_F$ и $\Vert T\Vert\Vert S\Vert\leq C+\epsilon$. Если $E$ является $\mathcal{L}_{p,C}^g$-пространством для некоторого $C\geq 1$, мы для краткости будем говорить, что $E$ есть $\mathcal{L}_p^g$-пространство.

Далее, через $A$ мы будем обозначать необязательно унитальную банахову алгебру с сжимающим билинейным оператором умножения. Через $A_+$ мы будем обозначать стандартную унитизацию $A$ как банаховой алгебры. Символом $A_\times$ будет обозначать условную унитизацию, то есть $A_\times=A$, если $A$ унитальна и $A_\times=A_+$ иначе. Через $A_\#$ мы будем обозначать унитизацию операторной алгебры $A$. Мы будем говорить, что аппроксимативная единица $(e_\nu)_{\nu\in N}$ в $A$ является $c$-ограниченной, если нормы всех ее элементов ограничены сверху константой $c$. Если аппроксимативная единица $1$-ограничена, то она называется сжимающей. 

Мы будем рассматривать банаховы модули только с сжимающим билинейным оператором внешнего умножения, обозначаемым ``$\cdot$''. Будем называть банахов $A$-модуль $X$ $\langle$~аннуляторным / существенным~$\rangle$, если $\langle$~$A\cdot X=\{0\}$ / $X_{ess}:=\operatorname{cl}_X(\operatorname{span}(A\cdot X))=X$~$\rangle$. Непрерывный морфизм $A$-модулей будем называть $A$-морфизмом. $A$-морфизм $\xi$ назовем $\langle$~$c$-ретракцией / $c$-коретракцией / $c$-изоморфизмом~$\rangle$, если он обладает $\langle$~правым обратным / левым обратным / обратным~$\rangle$ $A$-морфизмом $\eta$ таким, что $\Vert\xi\Vert\Vert\eta\Vert\leq c$. 

Через $\mathbf{Ban}$ мы будем обозначать категорию банаховых пространств с ограниченными линейными операторами в роли морфизмов. Если же рассматривать только сжимающие операторы, то получающаяся категория обозначается $\mathbf{Ban}_1$. Символом $A-\mathbf{mod}$ будем обозначать категорию левых банаховых $A$-модулей с $A$-морфизмами. Через $A-\mathbf{mod}_1$ мы обозначим подкатегорию $A-\mathbf{mod}$ с теми же объектами, но лишь сжимающими морфизмами. Соответствующие категории правых модулей обозначаются $\mathbf{mod}-A$ и $\mathbf{mod}_1-A$. Заметим, что для $A=\{0\}$ категория $A-\mathbf{mod}$ естественно изоморфна $\mathbf{Ban}$. Мономорфизмы во всех вышеупомянутых категориях суть инъективные операторы, а эпиморфизмы суть морфизмы с плотным образом. Через $\projtens_A$ мы будем обозначать функтор проективного модульного тензорного произведения, а через $\operatorname{Hom}$ мы обозначим функтор морфизмов.

В этой работе мы обсудим три версии банаховой гомологии. Существенная особенность этих теорий состоит в том, что они работают с комплексами составленными из допустимых морфизмов. Выбирая различные классы допустимых морфизмов мы получим разные версии банаховой гомологии. Будем говорить, что мономорфизм $\xi$ является $\langle$~метрически / $c$-топологически / $c$-относительно~$\rangle$ допустимым, если он $\langle$~изометричен / $c$-топологически инъективен / обладает левым обратным оператором нормы не более $c$~$\rangle$. Теперь мы можем дать основные определения этой работы. 

\begin{definition} Банахов $A$-модуль $J$ называется \emph{$\langle$~метрически / $C$-топологически / $C$-относительно~$\rangle$ инъективным}, если функтор $\langle$~$\operatorname{Hom}_{\mathbf{mod}_1-A}(-,J)$ / $\operatorname{Hom}_{\mathbf{mod}-A}(-,J)$ / $\operatorname{Hom}_{\mathbf{mod}-A}(-,J)$~$\rangle$ отображает $\langle$~метрически / $c$-топологически / $c$-относительно~$\rangle$ допустимые мономорфизмы в $\langle$~строго коизометрические / строго $cC$-топологически сюръективные / строго $cC$-топологически сюръективные~$\rangle$ операторы.
\end{definition}

Мы будем говорить что модуль $\langle$~топологически / относительно~$\rangle$ инъективный если он $\langle$~$C$-топологически / $C$-относительно~$\rangle$ инъективный для некоторой константы $C>0$.

\begin{definition} Банахов $A$-модуль $F$ называется \emph{$\langle$~метрически / $C$-топологически / $C$-относительно~$\rangle$ плоским}, если функтор $-\projtens_A F$ отображает $\langle$~метрически / $c$-топологически / $c$-относительно~$\rangle$ допустимые мономорфизмы в $\langle$~изометрические / $cC$-топологически инъективные / $cC$-топологически инъективные~$\rangle$ операторы.
\end{definition}

Мы будем говорить что модуль $\langle$~топологически / относительно~$\rangle$ плоский если он $\langle$~$C$-топологически / $C$-относительно~$\rangle$ плоский для некоторой константы $C>0$.

Изначально, несколько другая форма этих определений была дана Гравеном для метрической теории \cite{GravInjProjBanMod}, Уайтом для топологической теории \cite{WhiteInjmoduAlg} и Хелемским для относительной теории \cite{HelemHomolDimNorModBanAlg}. Обзор основ этих теорий дан в \cite{NemGeomProjInjFlatBanMod}. Ниже мы перечислим некоторые нужные нам результаты этой статьи.

Данные три теории тесно связаны. Например, каждый метрически инъективный модуль топологически инъективен, и каждый топологически инъективный модуль относительно инъективен. Аналогичные включения верны для плоских модулей. Плоскость и инъективность взаимосвязаны благодаря следующему критерию: банахов модуль является $C$-плоским тогда и только тогда, когда его сопряженный модуль $C$-инъективный. Типичный пример $\langle$~метрически / $1$-топологически / $1$-относительно~$\rangle$ инъективного модуля --- это $\langle$~$\mathcal{B}(A_\times, \ell_\infty(\Lambda))$ / $\mathcal{B}(A_\times, \ell_\infty(\Lambda))$ / $\mathcal{B}(A_\times, E)$~$\rangle$ для $\langle$~некоторого множества $\Lambda$ / некоторого множества $\Lambda$ / некоторого банахового пространства $E$~$\rangle$. В частности, правый банахов $A$-модуль $A_\times^*$ метрически, топологически и относительно инъективен. Некоторые категорные операции сохраняют инъективность и плоскость. Например, 
\begin{enumerate}[i)]
\item $\bigoplus_\infty$-сумма $\langle$~метрически / $C$-топологически~$\rangle$ инъективных модулей является $\langle$~метрически / $C$-топологически~$\rangle$ инъективной;

\item $c$-ретракт $C$-топологически инъективного модуля $cC$-топологически инъективен. Аналогичные утверждения верны для плоских модулей;

\item $\bigoplus_1$-сумма $\langle$~метрически / $C$-топологически~$\rangle$ плоских модулей является $\langle$~метрически / $C$-топологически~$\rangle$ плоской.
\end{enumerate}

Как простое следствие мы получаем, что $1$-топологически $\langle$ инъективные / плоские $\rangle$ модули метрически $\langle$ инъективные / плоские $\rangle$. Следующие два предложения из \cite{NemGeomProjInjFlatBanMod}, приведены для удобства читателя.
\begin{proposition}\label{MetTopFlatAnnihModCharac} Пусть $F$ --- ненулевой аннуляторный $A$-модуль. Тогда следующие условия эквивалентны:
\begin{enumerate}[i)]
\item $F$ --- $\langle$~метрически / $C$-топологически~$\rangle$ плоский $A$-модуль;
\item $\langle$~$A=\{0\}$ / $A$ обладает правой $(C-1)$-ограниченной аппроксимативной единицей~$\rangle$ и $F$ является $\langle$~метрически / $C$-топологически~$\rangle$ плоским банаховым пространством, то есть $\langle$~$F\isom{\mathbf{Ban}_1}L_1(\Omega,\mu)$ для некоторого пространства с мерой $(\Omega, \Sigma, \mu)$ / $F$ является $\mathcal{L}_{1,C}^g$-пространством~$\rangle$.
\end{enumerate}
\end{proposition}

\begin{proposition}\label{TopProjInjFlatModOverMthscrL1SpCharac} Пусть $A$ --- банахова алгебра топологически изоморфная как банахово пространство некоторому $\mathcal{L}_1^g$-пространству. Тогда любой топологически $\langle$~инъективный / плоский~$\rangle$ $A$-модуль является $\langle$~$\mathcal{L}_\infty^g$-пространством / $\mathcal{L}_1^g$-пространством~$\rangle$.
\end{proposition}

%----------------------------------------------------------------------------------------
%	Main reults
%----------------------------------------------------------------------------------------

\section{Основные результаты}
\label{SectionMainResults}

Мы начнем изложение с технической леммы о структуре сопряженных банаховых модулей.

\begin{proposition}\label{DualBanModDecomp} Пусть $B$ --- унитальная банахова алгебра, $A$ --- ее подалгебра с двусторонней ограниченной аппроксимативной единицей $(e_\nu)_{\nu\in N}$ и пусть $X$ --- левый банахов $B$-модуль. Обозначим $c_1=\sup_{\nu\in N}\Vert e_\nu\Vert$, $c_2=\sup_{\nu\in N}\Vert e_B-e_\nu\Vert$ и $X_{ess}=\operatorname{cl}_X(\operatorname{span}(A\cdot X))$. Тогда 
\begin{enumerate}[i)]
\item $X^*$ $c_2(c_1+1)$-изоморфен как правый $A$-модуль модулю $X_{ess}^*\bigoplus_\infty (X/X_{ess})^*$;
\item $\langle$~$X_{ess}^*$ / $(X/X_{ess})^*$~$\rangle$ является $\langle$ $c_1$-ретрактом / $c_2$-ретрактом~$\rangle$ $A$-модуля $X^*$;
\item если $X$ является $\mathcal{L}_{1,C}^g$-пространством, то $\langle$~$X_{ess}$ / $X/X_{ess}$~$\rangle$ является $\langle$~$\mathcal{L}_{1,c_1C}^g$-пространством / $\mathcal{L}_{1,c_2C}^g$-пространством~$\rangle$.
\end{enumerate}
\end{proposition}
\begin{proof} $i)$ Рассмотрим естественное вложение $\rho:X_{ess}\to X:x\mapsto x$ и фактор-отображение $\pi:X\to X/X_{ess}:x\mapsto x+X_{ess}$. Пусть $\mathfrak{F}$ --- фильтр сечений на $N$ и пусть $\mathfrak{U}$ --- ультрафильтр, содержащий $\mathfrak{F}$. Для каждого $f\in X ^*$ и $x\in X $ мы имеем $|f(x-e_\nu\cdot x)|\leq\Vert f\Vert\Vert e_B - e_\nu\Vert\Vert x\Vert\leq c_2\Vert f\Vert\Vert x\Vert$, то есть $(f(x-e_\nu\cdot x))_{\nu\in N}$ --- ограниченная направленность комплексных чисел. Следовательно, корректно определен предел $\lim_{\mathfrak{U}}f(x-e_\nu\cdot x)$ вдоль ультрафильтра $\mathfrak{U}$. Поскольку $(e_\nu)_{\nu\in N}$ --- двусторонняя аппроксимативная единица в $A$, и $\mathfrak{U}$ содержит фильтр сечений, то для всех $x\in X_{ess}$ выполнено $\lim_{\mathfrak{U}}f(x-e_\nu\cdot x)=\lim_{\nu}f(x-e_\nu\cdot x)=0$. Значит, для каждого $f\in X ^*$ корректно определено отображение $\tau(f):X /X_{ess}\to \mathbb{C}:x+X_{ess}\mapsto \lim_{\mathfrak{U}} f(x-e_\nu\cdot x)$. Ясно, что это линейный функционал и из предыдущих неравенств видно, что его норма ограничена константой $c_2\Vert f\Vert$. Теперь легко проверить, что $\tau:X^*\to (X/ X_{ess})^*:f\mapsto \tau(f)$ является $A$-морфизмом с нормой не более $c_2$. Аналогично, можно показать, что $\sigma:X_{ess}^*\to X^*:h\mapsto(x\mapsto \lim_{\mathfrak{U}}h(e_\nu\cdot x))$ является $A$-морфизмом с нормой не более $c_1$. Несложно убедиться, что $\tau \pi^*=1_{(X/X_{ess})^*}$, $\rho^*\sigma=1_{X_{ess}^*}$ и  $\pi^*\tau+\sigma\rho^*=1_{X^*}$. Из этих равенств видно, что отображения
\[
\xi:X^*\to X_{ess}^*\bigoplus{}_\infty (X/X_{ess})^*:f\mapsto \rho^*(f)\oplus{}_\infty \tau(f),
\]
\[
\eta:X_{ess}^*\bigoplus{}_\infty (X/X_{ess})^*\to X^*:h\oplus{}_\infty g\mapsto \pi^*(h)+\sigma(g)
\]
суть изоморфизмы правых $A$-модулей причем $\Vert\xi \Vert\leq c_2$ и $\Vert \eta\Vert\leq c_1+1$. Следовательно, $X^*$ $c_2(c_1+1)$-изоморфен в $\mathbf{mod}-A$ модулю $X_{ess}^*\bigoplus_\infty (X/X_{ess})^*$.

$ii)$ Оба утверждения немедленно следуют из равенств $\rho^*\sigma=1_{X_{ess}^*}$, $\tau \pi^*=1_{(X/X_{ess})^*}$ и оценок $\Vert \rho^*\Vert\Vert \sigma\Vert\leq c_1$, $\Vert\tau\Vert\Vert \pi^*\Vert\leq c_2$.

$iii)$ Теперь рассмотрим случай, когда $X$ является $\mathcal{L}_{1,C}^g$-пространством. Тогда $X^*$ есть $\mathcal{L}_{\infty,C}^g$-пространство \cite[следствие 23.2.1(1)]{DefFloTensNorOpId}. Поскольку модуль $\langle$~$X_{ess}^*$ / $(X/X_{ess})^*$~$\rangle$ $\langle$~$c_1$-дополняем / $c_2$-дополняем~$\rangle$ в $X^*$, он является $\langle$~$\mathcal{L}_{\infty,c_1C}^g$-пространством / $\mathcal{L}_{\infty,c_2C}^g$-пространством~$\rangle$ по \cite[следствие 23.2.1(1)]{DefFloTensNorOpId}. Снова применяя \cite[следствие 23.2.1(1)]{DefFloTensNorOpId}, мы заключаем, что $\langle$~$X_{ess}$  / $X/X_{ess}$~$\rangle$ --- это $\langle$~$\mathcal{L}_{1,c_1C}^g$-пространство / $\mathcal{L}_{1,c_2C}^g$-пространство~$\rangle$.
\end{proof}

\begin{proposition}\label{TopFlatModCharac} Пусть $A$ --- банахова алгебра с двусторонней $c$-ограниченной аппроксимативной единицей и $F$ --- левый банахов $A$-модуль. Тогда
\begin{enumerate}[i)]
\item если $F$ --- $C$-топологически плоский $A$-модуль, то $F_{ess}$ --- $(1+c)C$-топологически плоский $A$-модуль и $F/F_{ess}$ является $\mathcal{L}_{1,(1+c)C}^g$-пространством;
\item если $F_{ess}$ --- $C_1$-топологически плоский $A$-модуль и $F/F_{ess}$ является $\mathcal{L}_{1,C_1}^g$-пространством, то $F$ --- $(1+c)^2\max(C_1, C_2)$-топологически плоский $A$-модуль;
\item $F$ является топологически плоским $A$-модулем тогда и только тогда, когда $F_{ess}$  --- топологически плоский $A$-модуль и $F/F_{ess}$ --- $\mathcal{L}_1^g$-пространство.
\end{enumerate}
\end{proposition}
\begin{proof} Рассмотрим $A$ как замкнутую подалгебру в унитальной банаховой алгебре $B:=A_+$. Тогда $F$ --- унитальный левый банахов $B$-модуль. Используя обозначения предложения \ref{DualBanModDecomp}, мы можем сказать, что $c_1=c$ и $c_2=1+c$, поэтому правые $A$-модули $F_{ess}^*$ и $(F/F_{ess})^*$ суть $(1+c)$-ретракты $F^*$.

$i)$ Из предположения следует, что $F^*$ $C$-топологически инъективен. Следовательно, его ретракты $F_{ess}^*$ и $F/F_{ess}^*$ будут $(1+c)C$-топологически инъективными, а модули $F_{ess}$ и $F/F_{ess}$ будут $(1+c)C$-топологически плоскими. Осталось заметить, что $F/F_{ess}$ --- аннуляторный $A$-модуль, и по предложению \ref{MetTopFlatAnnihModCharac} он является $\mathcal{L}_{1,(1+c)C}^g$-пространством.

$ii)$ Снова из предположения мы получаем, что правые $A$-модули $F_{ess}^*$ и $(F/F_{ess})^*$ соответственно $C_1$- и $C_2$-топологически инъективные. Следовательно, их $\bigoplus_\infty$-сумма будет $\max(C_1,C_2)$-топологически инъективной. По предложению, \ref{DualBanModDecomp} эта сумма $(1+c)^2$-изоморфна $F^*$ в $\mathbf{mod}-A$. Значит $F^*$ ---  $(1+c)^2\max(C_1, C_2)$-топологически инъективный $A$-модуль, откуда следует $(1+c)^2\max(C_1, C_2)$-топологическая плоскость модуля $F$.

$iii)$ Утверждение немедленно следует из пунктов $i)$ и $ii)$.
\end{proof}

Прежде чем перейти к доказательству главного утверждения работы,
стоит напомнить одно из многочисленных эквивалентных определений относительно аменабельной банаховой алгебры. Банахова алгебра $A$ называется относительно $c$-аменабельной, если существует так называемая аппроксимативная диагональ $(d_\nu)_{\nu\in N}\subset A\projtens A$ ограниченная по норме сверху константой $c$ со свойствами:
\[
\lim_\nu(a\cdot d_\nu-d_\nu\cdot a)=0,\qquad \lim_\nu a \Pi_A(d_\nu)=\lim_\nu\Pi_A(d_\nu)a=a,
\]
где $\Pi_A:A\projtens A\to A:a\projtens b\mapsto ab$. Банахова алгебра $A$ называется относительно аменабельной, если она относительно $c$-аменабельна для некоторого $c>0$.

\begin{proposition}\label{MetTopEssL1FlatModAoverAmenBanAlg} Пусть $A$ --- относительно $\langle$~$1$-аменабельная / $c$-аменабельная~$\rangle$ банахова алгебра и $F$ --- существенный банахов $A$-модуль являющийся $\langle$~$L_1$-пространством / $\mathcal{L}_{1,C}^g$-пространством~$\rangle$. Тогда $F$ --- $\langle$~метрически / $c^2C$-топологически~$\rangle$ плоский $A$-модуль.
\end{proposition}
\begin{proof} Рассмотрим морфизм $A$-модулей $\pi_F:A\projtens \ell_1(B_F)\to F:a\projtens \delta_x\mapsto a\cdot x$. Мы покажем, что его сопряженный морфизм является коретракцией. Пусть $(d_\nu)_{\nu\in N}$ аппроксимативная диагональ для $A$ с нормой не более $c$. Напомним, что $(\Pi_A(d_\nu))_{\nu\in N}$ --- двусторонняя $\langle$~сжимающая / $c$-ограниченная~$\rangle$ аппроксимативная единица в $A$. Так как $A$-модуль $F$ существенный, то $\lim_{\nu}\Pi_A(d_\nu)\cdot x=x$ для всех $x\in F$ \cite[предложение 0.3.15]{HelHomolBanTopAlg}. Как следствие, $c\pi_F(B_{A\projtens\ell_1(B_F)})$ плотно в $B_F$. Тогда для всех $f\in F^*$ мы имеем
\[
\Vert\pi_F^*(f)\Vert
=\sup\{|f(\pi_F(u))|:u\in B_{A\projtens\ell_1(B_F)}\}
=\sup\{|f(x)|:x\in \operatorname{cl}_F(\pi_F(B_{A\projtens\ell_1(B_F)}))\}
\]
\[
\geq\sup\{c^{-1}|f(x)|:x\in B_F\}=c^{-1}\Vert f\Vert.
\]
Последнее значит, что $\pi_F^*$ $c$-топологически инъективен. По предположению $F$ --- $\langle$~$L_1$-пространство / $\mathcal{L}_{1,C}^g$-пространство~$\rangle$, тогда из $\langle$~\cite[теорема 1]{GrothMetrProjFlatBanSp} / замечания после \cite[следствие 23.5(1)]{DefFloTensNorOpId}~$\rangle$ следует, что банахово пространство $F^*$ $\langle$~метрически / $C$-топологически~$\rangle$ инъективно. Так как оператор $\pi_F^*$ $\langle$~изометричен / $c$-топологически инъективен~$\rangle$, то существует линейный оператор $R:(A\projtens\ell_1(B_F))^*\to F^*$ нормы $\langle$~не более $1$ / не более $cC$~$\rangle$ такой, что $R\pi_F^*=1_{F^*}$.

Зафиксируем $h\in (A\projtens\ell_1(B_F))^*$ и $x\in F$. Рассмотрим билинейный функционал $M_{h,x}:A\times A\to\mathbb{C}:(a,b)\mapsto R(h\cdot a)(b\cdot x)$. Очевидно, $\Vert M_{h,x}\Vert\leq\Vert R\Vert\Vert h\Vert\Vert x\Vert$. По универсальному свойству проективного тензорного произведения существует ограниченный линейный функционал $m_{h,x}:A\projtens A\to\mathbb{C}:a\projtens b\mapsto R(h\cdot a)(b\cdot x)$. Заметим, что $m_{h,x}$ линеен по $h$ и $x$. Более того, для всех $u\in A\projtens A$, $a\in A$ и $f\in F^*$ выполнено $m_{\pi_F^*(f),x}(u)=f(\Pi_A(u)\cdot x)$, $m_{h\cdot a,x}(u)=m_{h,x}(a\cdot u)$, $m_{h,a\cdot x}(u)=m_{h,x}(u\cdot a)$. Это легко проверить на элементарных тензорах. Далее будет достаточно вспомнить, что линейная оболочка элементарных тензоров плотна в $A\projtens A$.

Пусть $\mathfrak{F}$ --- фильтр сечений на $N$, и $\mathfrak{U}$ --- ультрафильтр, содержащий $\mathfrak{F}$. Для всех $h\in (A\projtens\ell_1(B_F))^*$ и $x\in F$ мы имеем $|m_{h,x}(d_\nu)|\leq c\Vert R\Vert\Vert h\Vert\Vert x\Vert$, то есть $(m_{h,x}(d_\nu))_{\nu\in N}$ --- ограниченная направленность комплексных чисел. Следовательно, корректно определен предел $\lim_{\mathfrak{U}}m_{h,x}(d_\nu)$ вдоль ультрафильтра $\mathfrak{U}$. Рассмотрим линейный оператор $\tau:(A\projtens\ell_1(B_F))^*\to F^*:h\mapsto(x\mapsto\lim_{\mathfrak{U}}m_{h,x}(d_\nu))$. Из оценок на нормы для $m_{h,x}$ следует, что $\tau$ ограничен по норме $\Vert\tau\Vert\leq c\Vert R\Vert$. Для всех $a\in A$, $x\in F$ и $h\in (A\projtens\ell_1(B_F))^*$ мы имеем
\[
\tau(h\cdot a)(x)-(\tau(h)\cdot a)(x)
=\tau(h\cdot a)(x)-\tau(h)(a\cdot x)
\]
\[
=\lim_{\mathfrak{U}}m_{h\cdot a,x}(d_\nu)-\lim_{\mathfrak{U}}m_{h,a\cdot x}(d_\nu)
=\lim_{\mathfrak{U}}m_{h,x}(a\cdot d_\nu)-m_{h,x}(d_\nu\cdot a)
\]
\[
=m_{h,x}\left(\lim_{\mathfrak{U}}(a\cdot d_\nu-d_\nu\cdot a)\right)
=m_{h,x}\left(\lim_{\nu}(a\cdot d_\nu-d_\nu\cdot a)\right)
=m_{h,x}(0)
=0.
\]
Следовательно, $\tau$ --- морфизм правых $A$-модулей. Теперь для каждого $f\in F^*$ и $x\in F$ мы получаем
\[
(\tau(\pi_F^*)(f))(x)
=\lim_{\mathfrak{U}}m_{\pi_F^*(f),x}(d_\nu)
=\lim_{\mathfrak{U}}f(\Pi_A(d_\nu)\cdot x)
\]
\[
=\lim_{\nu}f(\Pi_A(d_\nu)\cdot x)
=f\left(\lim_{\nu}\Pi_A(d_\nu)\cdot x\right)
=f(x).
\]
То есть $\tau\pi_F^*=1_{F^*}$. Это значит, что $F^*$ --- $\langle$~$1$-ретракт / $c^2 C$-ретракт~$\rangle$ модуля $(A\projtens\ell_1(B_F))^*$
 в $\langle$~$\mathbf{mod}_1-A$ / $\mathbf{mod}-A$~$\rangle$. Этот $A$-модуль $\langle$~метрически / $1$-топологически~$\rangle$ инъективен, так как $(A_+\projtens\ell_1(B_F))^*\isom{\mathbf{mod}_1-A}\mathcal{B}(A_+,\ell_\infty(B_F))$, и как следствие $F^*$ тоже $\langle$~метрически / $c^2C$топологически~$\rangle$ инъективен. Последнее означает $\langle$~метрическую / $c^2 C$-топологическую~$\rangle$ плоскость модуля $F$.
\end{proof}

\begin{theorem}\label{TopL1FlatModAoverAmenBanAlg} Пусть $A$ --- относительно $c$-аменабельная банахова алгебра и $F$ --- левый банахов $A$-модуль являющийся $\mathcal{L}_{1, C}^g$-пространством. Тогда $F$ --- $(1+c)^2C\max(c^2,(1+c))$-топологически плоский $A$-модуль. Другими словами, банахов модуль над относительно аменабельной алгеброй топологически плоский как банахово пространство является топологически плоским как модуль.
\end{theorem}
\begin{proof} Поскольку алгебра $A$ $c$-аменабельна, она обладает двусторонней $c$-ограниченной аппроксимативной единицей. По предложению \ref{DualBanModDecomp} аннуляторный $A$-модуль $F/F_{ess}$ является $\mathcal{L}_{1,1+c}^g$-пространством. Из предложения \ref{MetTopEssL1FlatModAoverAmenBanAlg} мы знаем, что существенный $A$-модуль $F_{ess}$ $c^2 C$-топологически плоский. Теперь утверждение теоремы следует из предложения \ref{TopFlatModCharac}.
\end{proof}

Следует отметить, что в относительной банаховой гомологии любой левый банахов модуль над относительно аменабельной банаховой алгеброй является относительно плоским \cite[теорема 7.1.60]{HelBanLocConvAlg}. Топологическая теория  (не говоря уже о метрической) настолько ограничивает класс плоских модулей, что иногда удается получить их полное описание.

\begin{proposition}\label{TopFlatModAoverAmenL1BanAlgCharac} Пусть $A$ --- относительно аменабельная банахова алгебра являющаяся $\mathcal{L}_1^g$-пространством. Тогда для банахова $A$-модуля $F$ следующие условия эквивалентны:
\begin{enumerate}[i)]
\item $F$ топологически плоский $A$-модуль;
\item $F$ --- $\mathcal{L}_1^g$-пространство.
\end{enumerate}
\end{proposition}
\begin{proof} Эквивалентность следует из предложения \ref{TopProjInjFlatModOverMthscrL1SpCharac} и теоремы \ref{TopL1FlatModAoverAmenBanAlg}.
\end{proof}

%----------------------------------------------------------------------------------------
%	A few examples
%----------------------------------------------------------------------------------------

\section{Примеры}
\label{SectionAFewExamples}

Теперь мы дадим несколько примеров топологически плоских и неплоских модулей. 

Для начала рассмотрим сверточную алгебру $A=L_1(G)$ аменабельной локально компактной группы $G$. Эта алгебра относительно аменабельна \cite[предложение VII.1.86]{HelBanLocConvAlg}, и, очевидно, она является $\mathcal{L}_1^g$-пространством. По предложению \ref{TopFlatModAoverAmenL1BanAlgCharac} любой банахов $A$-модуль который является $\mathcal{L}_1^g$-пространством будет топологически плоским. Примеры включают конечномерные модули, дополняемые идеалы $L_1(G)$ и алгебру мер $M(G)$.

\begin{example} Для локально компактного пространства $S$ $C_0(S)$-модуль $M(S)$ метрически плоский.
\end{example}
\begin{proof}
Заметим, что алгебра $C_0(S)$ непрерывных функций, исчезающих на бесконечности, относительно аменабельна \cite[теорема 7.1.87]{HelBanLocConvAlg}. Более того, она относительно $1$-аменабельна, как любая аменабельная $C^*$-алгебра \cite[пример 2]{RundeAmenConstFour}. Напомним, что алгебра мер $M(S)$ является существенным $C_0(S)$-модулем изометрически изоморфным некоторому $L_1$-пространству (см. обсуждение после \cite[предложение 2.14]{DalLauSecondDualOfMeasAlg}). Осталось применить предложение \ref{MetTopEssL1FlatModAoverAmenBanAlg}.
\end{proof}

Может показаться, что топологическая плоскость бывает только если модуль или алгебра являются $\mathcal{L}_1^g$-пространством. Как показывает следующее предложение, это не так.

\begin{proposition}\label{MetTopFlatIdealsInUnitalAlg} Пусть $I$ --- левый идеал в банаховой алгебре $A_\times $ и $I$ обладает правой $\langle$~сжимающей / $c$-ограниченной~$\rangle$ аппроксимативной единицей. Тогда $I$ --- $\langle$~метрически / $c$-топологически~$\rangle$ плоский $A$-модуль.
\end{proposition}
\begin{proof} Пусть $\mathfrak{F}$ фильтр сечений на $N$ и пусть $\mathfrak{U}$ --- ультрафильтр, содержащий $\mathfrak{F}$. Несложно проверить, что отображение $\sigma:A_\times ^*\to I^*:f\mapsto (a\mapsto \lim_{\mathfrak{U}}f(ae_\nu))$ --- $A$-морфизм нормы $\langle$~не более $1$ / не более $c$~$\rangle$. Пусть $\rho:I\to A_\times$ --- естественное вложение, тогда для всех $f\in A_\times^*$ и $a\in I$ выполнено
\[
\rho^*(\sigma(f))(a)
=\sigma(f)(\rho(a))
=\sigma(f)(a)
=\lim_{\mathfrak{U}}f(a e_\nu)
=\lim_{\nu}f(a e_\nu)
=f(\lim_{\nu}a e_\nu)
=f(a),
\]
то есть $\sigma:I^*\to A_\times^*$ есть $\langle$~$1$-коретракция / $c$-коретракция~$\rangle$. Правый $A$-модуль $A_\times ^*$ $\langle$~метрически / $1$-топологически~$\rangle$ инъективен, следовательно его $\langle$~$1$-ретракт / $c$-ретракт~$\rangle$ $I^*$ будет $\langle$~метрически / $c$-топологически~$\rangle$ инъективен. Значит $A$-модуль $I$  $\langle$~метрически / $c$-топологически~$\rangle$ плоский.
\end{proof}

Вышеупомянутый результат верен и в относительной банаховой гомологии \cite[предложение 7.1.45]{HelBanLocConvAlg}, поэтому нам следует предъявить пример относительно плоского, но не топологически плоского идеала.

\begin{example} В алгебре $L_1(\mathbb{T})$ существует идеал изоморфный гильбертову пространству, который является относительно плоским, но не топологически плоским.
\end{example}
\begin{proof}
Обозначим $A=L_1(\mathbb{T})$. Известно, что $A$ обладаем трансляционно инвариантным бесконечномерным замкнутым подпространством $I$ изоморфным гильбертову пространству \cite[страница 52]{RosProjTransInvSbspLpG}. Из \cite[предложение 1.4.7]{KaniBanAlg} мы получаем, что $I$ --- двусторонний идеал $A$, как всякое трансляционно инвариантное подпространство в $A$. Из \cite[параграф 23.3]{DefFloTensNorOpId} следует, что этот идеал не может быть $\mathcal{L}_1^g$-пространством. Тогда по предложению \ref{TopFlatModAoverAmenL1BanAlgCharac} идеал $I$ не может быть топологически плоским. При этом он относительно плоский. Так как $\mathbb{T}$ --- компактная группа, то она аменабельна \cite[предложение 3.12.1]{PierAmenLCA}. Следовательно, алгебра $A$ относительно аменабельна \cite[предложение VII.1.86]{HelBanLocConvAlg}, поэтому все левые идеалы в $A$ относительно плоские \cite[предложение VII.1.60(I)]{HelBanLocConvAlg}. В частности, $I$ относительно плоский.
\end{proof}

Рассмотрим пример, где аменабельность не требуется для наличия топологической плоскости.

\begin{example} Для локально компактной группы $G$ $L_1(G)$-модуль $M(G)$ топологически плоский.
\end{example}
\begin{proof}
Так как модуль $M(G)$ есть $L_1$-пространство, то он тем более $\mathcal{L}_1^g$-пространство. Поскольку $L_1(G)$-модуль $M_s(G)$, состоящий из мер сингулярных по отношению к мере Хаара, $1$-дополняем в $M(G)$, то $M_s(G)$ тоже является $\mathcal{L}_1^g$-пространством. Заметим, что $M_s(G)$ --- аннуляторный $L_1(G)$-модуль, поэтому из предложения \ref{MetTopFlatAnnihModCharac} мы знаем, что $M_s(G)$ --- топологически плоский $L_1(G)$-модуль. С другой стороны $L_1(G)$-модуль $L_1(G)$ тоже топологически плоский по предложению \ref{MetTopFlatIdealsInUnitalAlg}. Так как $M(G)\isom{L_1(G)-\mathbf{mod}_1}L_1(G)\bigoplus_1 M_s(G)$, то $M(G)$ тоже является топологически плоским $L_1(G)$-модулем, как  $\bigoplus_1$-сумма плоских модулей.
\end{proof}

Большой источник примеров не топологически плоских модулей следует искать среди $C^*$-алгебр. Интуитивно ясно, что они ``далеки'' от $\mathcal{L}_1^g$-пространств и должно быть много контрпримеров. Мы можем их найти даже среди идеалов $C^*$-алгебр. Начнем с подготовительного предложения.

\begin{proposition}\label{CStarAlgIsL1IfFinDim} Пусть $A$ --- $C^*$-алгебра, тогда $A$ является $\langle$~$L_1$-пространством / $\mathcal{L}_1^g$-пространством~$\rangle$ тогда и только тогда когда $\langle$~$\operatorname{dim}(A)\leq 1$ / $A$ конечномерно~$\rangle$.
\end{proposition}
\begin{proof} Допустим, $A$ является $\mathcal{L}_1^g$-пространством, тогда $A^{**}$ дополняемо в некотором $L_1$-пространстве \cite[следствие 23.2.1(2)]{DefFloTensNorOpId}. Так как $A$ изометрически вкладывается в свое второе сопряженное пространство, мы может считать $A$ замкнутым подпространством некоторого $L_1$-пространства. Любое $L_1$-пространство слабо секвенциально полно \cite[следствие III.C.14]{WojBanSpForAnalysts}. Это свойство наследуется замкнутыми подпространствами, поэтому $A$ тоже слабо секвенциально полно. По предложению 2 из \cite{SakWeakCompOpOnOpAlg} каждая слабо секвенциально полная $C^*$-алгебра конечномерна, поэтому $A$ конечномерна. Обратно, если $A$ конечномерна, то она --- $\mathcal{L}_1^g$-пространство как любое конечномерное банахово пространство.

Допустим, $A$ является $L_1$-пространством и, тем более, $\mathcal{L}_1^g$-пространством. Как было отмечено выше, $A$ конечномерно, поэтому $A\isom{\mathbf{Ban}_1}\ell_1^n$ для $n=\operatorname{dim}(A)$. С другой стороны, $A$ --- конечномерная $C^*$-алгебра, и поэтому изометрически изоморфна $\bigoplus_\infty\{ \mathcal{B}(\ell_2^{n_k}):k\in\{1,\ldots,m\}\}$ для некоторых натуральных чисел $n_1,\ldots,n_m$ \cite[теорема III.1.1]{DavCSatrAlgByExmpl}. Допустим, $\operatorname{dim}(A)>1$, тогда $A$ содержит изометрическую копию $\ell_\infty^2$. Следовательно, существует изометрическое вложение $\ell_\infty^2$ в $\ell_1^n$. Это невозможно по теореме 1 из \cite{LyubIsomEmdbFinDimLp}. Значит, $\operatorname{dim}(A)\leq 1$. 
\end{proof}

\begin{proposition}\label{CStarAlgIsTopFlatOverItsIdeal} Пусть $I$ --- собственный двусторонний идеал $C^*$-алгебры $A$. Тогда следующие условия эквивалентны:
\begin{enumerate}[i)]
\item $A$ --- $\langle$~метрически / топологически~$\rangle$ плоский $I$-модуль;
\item $\langle$~$\operatorname{dim}(A)=1$, $I=\{0\}$ / $A/I$ конечномерно~$\rangle$.
\end{enumerate}
\end{proposition}
\begin{proof} Мы можем рассматривать $I$ как идеал в унитизации $A_\#$ алгебры $A$. Так как $I$ --- двусторонний идеал, то он обладает сжимающей аппроксимативной единицей $(e_\nu)_{\nu\in N}$ такой, что $0\leq e_\nu\leq e_{A_\#}$ \cite[предложение 4.7.79]{HelBanLocConvAlg}. Как следствие, $\sup_{\nu\in N}\Vert e_{A_\#}-e_\nu\Vert\leq 1$. Поскольку $I$ обладает аппроксимативной единицей $A_{ess}:=\operatorname{cl}_A(\operatorname{span}(IA))=I$. Так как $I$ --- двусторонний идеал, то $A/I$ является $C^*$-алгеброй \cite[теорема 4.7.81]{HelBanLocConvAlg}.

Допустим, $A$ --- метрически плоский $I$-модуль. Так как $\sup_{\nu\in N}\Vert e_{A_\#}-e_\nu\Vert\leq 1$, то из пункта $ii)$ предложения \ref{DualBanModDecomp} следует, что $(A/A_{ess})^*=(A/I)^*$ является ретрактом $A^*$ в $\mathbf{mod}_1-I$. Значит, $A/I$ --- метрически плоский $I$-модуль. Поскольку это еще и аннуляторный модуль, то из предложения \ref{MetTopFlatAnnihModCharac} мы получаем, что $I=\{0\}$ и $A/I$ есть $L_1$-пространство. Теперь по предложению \ref{CStarAlgIsL1IfFinDim} выполнено $\operatorname{dim}(A/I)\leq 1$. Так как $A$ содержит собственный идеал $I=\{0\}$, то $\operatorname{dim}(A)=1$. Обратно, если $I=\{0\}$ и $\operatorname{dim}(A)=1$, то мы имеем аннуляторный $I$-модуль $A$ который изометрически изоморфен $\ell_1^1$. По предложению \ref{MetTopFlatAnnihModCharac} он метрически плоский. 

По предложению \ref{TopFlatModCharac} $I$-модуль $A$ топологически плоский тогда и только тогда, когда $A_{ess}=I$ и $A/A_{ess}=A/I$ --- суть топологически плоские $I$-модули. По предложению \ref{MetTopFlatIdealsInUnitalAlg} идеал $I$ топологически плоский как $I$-модуль, поскольку $I$ обладает сжимающей аппроксимативной единицей. Из предложения \ref{MetTopFlatAnnihModCharac} следует, что аннуляторный $I$-модуль $A/I$ будет топологически плоским тогда и только тогда когда он будет $\mathcal{L}_1^g$-пространством. По предложению \ref{CStarAlgIsL1IfFinDim} это равносильно конечномерности $A/I$.
\end{proof}

Как следствие, мы получаем, что модуль $\mathcal{B}(H)$ ограниченных операторов на бесконечномерном сепарабельном гильбертовом пространстве $H$ над алгеброй $\mathcal{K}(H)$ компактных операторов не является топологически плоским. Отметим, что все же этот модуль относительно плоский, так как алгебра $\mathcal{K}(H)$ относительно аменабельна \cite[VII.1.89]{HelBanLocConvAlg}, а все модули над относительно аменабельной банаховой алгеброй относительно плоские \cite[VII.1.60(I)]{HelBanLocConvAlg}.


\begin{thebibliography}{999}

\bibitem{DalLauSecondDualOfMeasAlg}
\textit{H. G. Dales, A.T.-M. Lau, D. Strauss.} Second duals of measure algebras, Dissertationes Math. (Rozprawy Mat.) 481 (2012) 1--121.
%
\bibitem{DavCSatrAlgByExmpl}
\textit{K. R. Davidson.} $\it{C^*}$-algebras by example (American Mathematical Society, Vol. 6, 1996).
%
\bibitem{DefFloTensNorOpId}
\textit{A. Defant, K. Floret.} Tensor norms and operator ideals (Elsevier, Vol. 176, 1992).
%
\bibitem{GravInjProjBanMod}
\textit{A. W. M. Graven.} Injective and projective Banach modules, Indag. Math. (Proceedings) 82 (1979) 253--272.
%
\bibitem{GrothMetrProjFlatBanSp}
\textit{A. Grothendieck.} Une caract{\'e}risation vectorielle-m{\'e}trique des espaces $\it{L}_1$, Canad. J. Math 7 (1955) 552--561.
%
\bibitem{HelemHomolDimNorModBanAlg}
\textit{А. Я. Хелемский.} О гомологической размерности нормированных модулей над банаховыми алгебрами, Матем. сборник 81 (1970) 430-444.
%
\bibitem{HelHomolBanTopAlg}
\textit{А. Я. Хелемский.} Гомология в банаховых и топологических алгебрах (М.:изд-во МГУ, 1986).
%
\bibitem{HelBanLocConvAlg}
\textit{А. Я. Хелемский.} Банаховы и полинормированные алгебры: общая теория, представления, гомологии (М.:Наука, 1989).
%
\bibitem{KaniBanAlg}
\textit{E. Kaniuth.} A course in commutative Banach algebras (Springer, Vol. 246, 2009).
%
\bibitem{LinPelAbsSumOpInLpSpAndApp}
\textit{J. Lindenstrauss, A. Pelczynski.} Absolutely summing operators in $\mathcal{L}_p$-spaces and their applications, Studia Mathematica 29 (1968) 275--326.
%
\bibitem{LyubIsomEmdbFinDimLp}
\textit{Yu. I. Lyubich, O. A. Shatalova.} Isometric embeddings of finite-dimensional $\ell_p$-spaces over the quaternions, St. Petersburg Math. J. 16 (2005) 9--24.
%
\bibitem{NemGeomProjInjFlatBanMod}
\textit{Н.Т. Немеш.} Геометрия проективных, инъективных и плоских банаховых модулей, Фундамент. и прикл. матем. 21(3) (2016) 161--184.
%
\bibitem{PierAmenLCA}
\textit{J.-P. Pier.} Amenable locally compact groups (Wiley-Interscience, 1984).
%
\bibitem{RosProjTransInvSbspLpG}
\textit{H. P. Rosenthal.} Projections onto translation-invariant subspaces of $L_p(G)$, American Mathematical Society 63 (1966).
%
\bibitem{RundeAmenConstFour}
\textit{V. Runde.} The amenability constant of the Fourier algebra, Proc. Amer. Math. Soc. 134 (2006) 1473--1481.
%
\bibitem{SakWeakCompOpOnOpAlg}
\textit{S. Sakai.} Weakly compact operators on operator algebras, Pacific J. Math. 14 (1964) 659--664.
%
\bibitem{StegRethNucOpL1LInfSp}
\textit{C. P. Stegall, J. R. Retherford.} Fully nuclear and completely nuclear operators with applications to $\mathcal{L}_1$-and $\mathcal{L}_\infty$-spaces, Transactions of the American Mathematical Society 163 (1972) 457--492.
%
\bibitem{WhiteInjmoduAlg}
\textit{M.C. White.} Injective modules for uniform algebras, Proceedings of the London Mathematical Society 3 (1966) 155--184.
%
\bibitem{WojBanSpForAnalysts}
\textit{P. Wojtaszczyk.} Banach spaces for analysts (Cambridge University Press, Vol. 25, 1996)
\end{thebibliography}

Norbert Nemesh, Faculty of Mechanics and Mathematics, Moscow State University, Moscow 119991 Russia

\textit{E-mail address:} nemeshnorbert@yandex.ru


\end{document}