%\title{Геометрия проективных, инъективных и плоских банаховых модулей}

\documentclass[12pt]{article}
\usepackage[left=2cm,right=2cm,top=2cm,bottom=2cm,bindingoffset=0cm]{geometry}
\usepackage{amssymb,amsmath,amsthm}
\usepackage[T1,T2A]{fontenc}
\usepackage[utf8]{inputenc}
\usepackage[russian]{babel} 
\usepackage{mathrsfs}
\usepackage[matrix,arrow,curve]{xy}
\usepackage[colorlinks=true, urlcolor=blue, linkcolor=blue, citecolor=blue, pdfborder={0 0 0}]{hyperref}

% Мои определения
\newcommand{\projtens}{\mathbin{\widehat{\otimes}}}
\newcommand{\convol}{\ast}
\newcommand{\projmodtens}[1]{\mathbin{\widehat{\otimes}}_{#1}}
\newcommand{\isom}[1]{\mathop{\mathbin{\cong}}\limits_{#1}}

\newtheorem{theorem}{Теорема}[section]
\newtheorem{lemma}[theorem]{Лемма}
\newtheorem{proposition}[theorem]{Предложение}
\newtheorem{remark}[theorem]{Замечание}
\newtheorem{corollary}[theorem]{Следствие}
\newtheorem{definition}[theorem]{Определение}
\newtheorem{example}[theorem]{Пример}
\renewenvironment{proof}{\paragraph{Доказательство.}}{\hfill$\square$\medskip}

\begin{document}

\begin{flushleft}
\Large \textbf{Геометрия проективных, инъективных и плоских\\ банаховых модулей\footnote{Работа выполнена при поддержке Российского фонда фундаментальных исследований (грант номер 15-01-08392).}}\\[0.5cm]
\end{flushleft}
\begin{flushright}
\normalsize \textbf{Н. Т. Немеш}\\[0.5cm]
\end{flushright}
\begin{flushleft}
\small {УДК 517.986.22}\\[0.5cm]
\end{flushleft}

\thispagestyle{empty}

\textbf{Ключевые слова:} проективность, инъективность, плоскость, аннуляторный модуль, свойство Данфорда-Петтиса.
\medskip

\textbf{Аннотация:} В данной статье изложены общие результаты о метрически и топологически проективных, инъективных и плоских банаховых модулях. Доказаны теоремы указывающие на тесную связь метрической и топологической банаховой гомологии с банаховой геометрией. Например, в геометрических терминах дано описание проективных, инъективных и плоских аннуляторных модулей. Доказано, что для алгебр являющихся  $\mathscr{L}_1$- или $\mathscr{L}_\infty$-пространством ее гомологически тривиальные модули обладают свойством Данфорда-Петтиса.
\medskip

%----------------------------------------------------------------------------------------
%	Introduction
%----------------------------------------------------------------------------------------

\section{Введение}
\label{SectionIntroduction}

Понятия проективного, инъективного и плоского модуля являются тремя китами на которых покоится здание гомологической алгебры. Методы гомологической алгебры в функциональном анализе были внедрены и развиты Хелемским и его школой. Точнее, Хелемский рассматривал специальную версию относительной гомологии, связывающую воедино алгебру и топологию. Существовали и другие варианты гомологической алгебры в функциональном анализе, например метрическая и топологическая. Активно изучать их стали только сейчас. В данной статье мы докажем несколько теорем подтверждающих тесную связь метрической и топологической банаховой гомологии с геометрией банаховых пространств.

Несколько слов об обозначениях. Здесь и далее символ $A$ будет обозначать не обязательно унитальную банахову алгебру со сжимающим билинейным оператором умножения. Через $A_+$ мы будем обозначать стандартную унитизацию $A$ как банаховой алгебры. Символом $A_\times$ мы будем обозначать условную унитизацию, то есть $A_\times=A$ если $A$ унитальна и $A_\times=A_+$ в противном случае. Мы будем рассматривать только банаховы модули со сжимающим билинейным оператором внешнего умножения, обозначаемого точкой ``$\cdot$''. Наконец, непрерывные морфизмы $A$-модулей мы будем называть $A$-морфизмами. Через $\mathbf{Ban}$ мы будем обозначать категорию банаховых пространств с ограниченными операторами в роли морфизмов. Если рассматривать в роли морфизмов только сжимающие операторы, то мы получим еще одну категорию обозначаемую $\mathbf{Ban}_1$. Через $A-\mathbf{mod}$ мы обозначим категорию левых банаховых $A$-модулей с ограниченными $A$-морфизмами в роли морфизмов. Через $A-\mathbf{mod}_1$ мы обозначим подкатегорию $A-\mathbf{mod}$ c теми же объектами, но только лишь сжимающими морфизмами. В дальнейшем, в предложениях мы будем использовать сразу несколько фраз, последовательно перечисляя их и заключая в скобки таким образом: $\langle$~... / ...~$\rangle$. Например: число $x$ называется $\langle$~положительным / неотрицательным~$\rangle$ если $\langle$~$x>0$ / $x\geq 0$~$\rangle$.

Напомним пару определений и фактов из относительной банаховой гомологии. Будем говорить, что морфизм $\xi:X\to Y$ левых $A$-модулей $X$ и $Y$ есть относительно допустимый эпиморфизм, если он имеет правый обратный ограниченный линейный оператор. Левый $A$-модуль $P$ называется относительно проективным, если для любого относительно допустимого эпиморфизма $\xi:X\to Y$ и любого $A$-морфизма $\phi:P\to Y$ существует $A$-морфизм $\psi:P\to X$ такой, что диаграмма
$$
\xymatrix{
& {X} \ar@{->>}[d]^{\xi}\\
{P} \ar@{-->}[ur]^{\psi} \ar[r]^{\phi} &{Y}}
$$
коммутативна, то есть $\xi\psi=\phi$. Аналогично, будем говорить, что морфизм $\xi:Y\to X$ правых $A$-модулей $X$ и $Y$ есть относительно допустимый мономорфизм, если он имеет левый обратный ограниченный линейный оператор. Правый $A$-модуль $J$ называется относительно инъективным, если для любого относительно допустимого мономорфизма $\xi:Y\to X$ и любого $A$-морфизма $\phi:Y\to J$ существует $A$-морфизм $\psi:X\to J$ такой, что диаграмма
$$
\xymatrix{
& {X} \ar@{-->}[dl]_{\psi} \\
{J} &{Y} \ar[l]_{\phi} \ar@{^{(}->}[u]_{\xi}}
$$
коммутативна, то есть $\psi\xi=\phi$. 

Специальный класс относительно $\langle$~проективных / инъективных~$\rangle$ $A$-модулей --- это так называемые относительно $\langle$~свободные / косвободные~$\rangle$ модули. Они имеют вид $\langle$~$A_+\projtens E$ / $\mathcal{B}(A_+,E)$~$\rangle$ для некоторого банахова пространства $E$. Главное свойство таких модулей состоит в том, $A$-модуль относительно $\langle$~проективен / инъективен~$\rangle$ тогда и только тогда, когда он является ретрактом некоторого относительно $\langle$~свободного / косвободного~$\rangle$ $A$-модуля. 

И метрическая и топологическая и относительная банахова гомология могут быть изложены с общекатегорных позиций. В работе \cite{HelMetrFrQMod} Хелемским была построена теория оснащенных категорий, позволившая единообразно доказывать многие утверждения о проективных и инъективных банаховых модулях. Мы дадим определения и кратко перечислим некоторые результаты об оснащенных категориях. Через $\mathbf{Set}$ мы будем обозначать категорию множеств. Тот факт что объекты $X$ и $Y$ категории $\mathbf{C}$ изоморфны мы будем записывать как $X\isom{\mathbf{C}}Y$. Пусть $\mathbf{C}$ и $\mathbf{D}$ --- две фиксированные категории. Упорядоченная пара ($\mathbf{C}, \square:\mathbf{C}\to\mathbf{D}$), где $\square$ --- верный ковариантный функтор, называется оснащенной категорией. Морфизм $\xi$ в $\mathbf{C}$ называется $\square$-допустимым эпиморфизмом если $\square (\xi)$ --- ретракция в $\mathbf{D}$. Объект $P$ в $\mathbf{C}$ называется $\square$-проективным, если для каждого $\square$-допустимого эпиморфизма $\xi$ в $\mathbf{C}$ отображение $\operatorname{Hom}_{\mathbf{C}}(P,\xi)$ сюръективно. Объект $F$ в $\mathbf{C}$ называется $\square$-свободным с базой $M$ в  $\mathbf{D}$, если существует изоморфизм функторов $\operatorname{Hom}_{\mathbf{D}}(M,\square(-))\cong\operatorname{Hom}_{\mathbf{C}}(F,-)$. Оснащенная категория $(\mathbf{C},\square)$ называется свободолюбивой [\cite{HelMetrFrQMod}, определение 2.10], если каждый объект в $\mathbf{D}$ является базой некоторого $\square$-свободного объекта из $\mathbf{C}$. Резюме предложений 2.3, 2.11  и 2.12 из \cite{HelMetrFrQMod} выглядит следующим образом:

$i)$ любой ретракт $\square$-проективного объекта $\square$-проективен;

$ii)$ любой $\square$-допустимый эпиморфизм в $\square$-проективный объект есть ретракция;

$iii)$ любой $\square$-свободный объект $\square$-проективен;

$iv)$ если $(\mathbf{C},\square)$ --- свободолюбивая оснащенная категория, то любой объект $\square$-проективен тогда и только тогда, когда он есть ретракт $\square$-свободного объекта;

$v)$ копроизведение семейства $\square$-проективных объектов $\square$-проективно.

Символом $\mathbf{C}^{o}$ мы будем обозначать категорию противоположную к $\mathbf{C}$. Противоположной к оснащенной категории $(\mathbf{C}, \square)$ 
будем называть оснащенную категорию $(\mathbf{C}^{o},\square^{o}:\mathbf{C}^{o}\to\mathbf{D}^{o})$. Здесь, $\mathbf{C}^o$ и $\mathbf{D}^o$ обозначают противоположные категории, то есть категории в которых  те же объекты, но все стрелки направлены в противоположную сторону. Переходя к противоположной оснащенной категории, мы можем определить допустимые мономорфизмы, инъективность и косвободу. Морфизм $\xi$ называется $\square$-допустимым мономорфизмом, если он $\square^o$-допустимый эпиморфизм. Объект $J$ из $\mathbf{C}$ называется $\square$-инъективным, если он $\square^o$-проективен. Объект $F$ из $\mathbf{C}$ называется $\square$-косвободным, если он $\square^o$-свободный. Наконец, категория $(\mathbf{C}, \square)$ называется косвободолюбивой, если  противоположная категория $(\mathbf{C}^{o},\square^{o})$ свободолюбива. Таким образом, для инъективности и косвободы мы можем сформулировать результаты аналогичные тем, что были для проективности и свободы.

Теперь рассмотрим верный функтор $\square_{rel}:A-\mathbf{mod}\to\mathbf{Ban}$, который просто ``забывает'' модульную структуру. Легко видеть, что $(A-\mathbf{mod},\square_{rel})$ --- оснащенная категория, у которой $\square_{rel}$-допустимые $\langle$~эпиморфизмы / мономорфизмы~$\rangle$ в точности относительно допустимые $\langle$~эпиморфизмы / мономорфизмы~$\rangle$ и $\langle$~$\square_{rel}$-проективные / $\square_{rel}$-инъективные~$\rangle$ объекты в точности относительно $\langle$~проективные / инъективные~$\rangle$ $A$-модули. Более того, можно показать, что все $\langle$~$\square_{rel}$-свободные / $\square_{rel}$-косвободные~$\rangle$ объекты изоморфны в $A-\mathbf{mod}$ модулям вида $\langle$~$A_+\projtens E$ / $\mathcal{B}(A_+,E)$ ~$\rangle$ для некоторого банахова пространства $E$. Этот пример показывает, что относительная теория прекрасно вписывается в схему оснащенных категорий.

В этой работе мы применим такой же общекатегорный подход к метрической и топологической теории. В этих теориях накладываются значительно более слабые ограничения на допустимые морфизмы.

%----------------------------------------------------------------------------------------
%	Projectivity, injectivity and flatness
%----------------------------------------------------------------------------------------

\section{Проективность, инъективность и плоскость}
\label{SectionProjectivityInjectivityAndFlatness}

%----------------------------------------------------------------------------------------
%	Metric and topological projectivity
%----------------------------------------------------------------------------------------

\subsection{Метрическая и топологическая проективность}
\label{SubSectionMetricAndTopologicalProjectivity}

При изучении метрической и топологической проективности мы будем рассматривать два широких класса эпиморфизмов, а именно строго коизометрические и топологически сюръективные $A$-морфизмы. Через $\langle$~$B_E$ / $B_E^\circ$~$\rangle$ мы будем обозначать $\langle$~замкнутый / открытый~$\rangle$ единичный шар пространства $E$. Ограниченный линейный оператор $T:E\to F$ будем называть $\langle$~строго коизометрическим / топологически сюръективным~$\rangle$ если $\langle$~$B_F=T(B_E)$ / $B_F^\circ\subset cT(B_E^\circ)$ для некоторого $c>0$~$\rangle$. В дальнейшем $A$ обозначает необязательно унитальную банахову алгебру. 

\begin{definition}[\cite{HelMetrFrQMod}, определения 1.2, 1.4]\label{MetTopProjMod} $A$-модуль $P$ называется $\langle$~метрически / топологически~$\rangle$ проективным, если для любого $\langle$~строго коизометрического / топологически сюръективного~$\rangle$ $A$-морфизма $\xi:X\to Y$ и любого $A$-морфизма $\phi:P\to Y$ существует $A$-морфизм $\psi:P\to X$ такой, что $\langle$~$\xi\psi=\phi$ и $\Vert\psi\Vert=\Vert\phi\Vert$ / $\xi\psi=\phi$~$\rangle$.
\end{definition}

Теперь мы нацелены применить аппарат оснащенных категорий к этим типам проективности. В работах Хелемского \cite{HelMetrFrQMod} и Штейнера \cite{ShtTopFrClassicQuantMod} были построены два верных функтора: 
$$
\square_{met}:A-\mathbf{mod}_1\to\mathbf{Set},
\qquad
\square_{top}:A-\mathbf{mod}\to\mathbf{HNor}.
$$
Здесь $\mathbf{HNor}$ --- это категория так называемых полунормированных пространств, введенных Штейнером. Мы не будем подробно объяснять как действуют эти функторы. Нам достаточно их существования. В тех же статьях было доказано, что, во-первых, $A$-морфизм $\xi$ $\langle$~строго коизометричен / топологически сюръективен~$\rangle$ тогда и только тогда, когда он $\langle$~$\square_{met}$-допустимый / $\square_{top}$-допустимый~$\rangle$ эпиморфизм и, во-вторых, $A$-модуль $P$ является $\langle$~метрически / топологически~$\rangle$ проективным тогда и только тогда, когда он $\langle$~$\square_{met}$-проективен / $\square_{top}$-проективен~$\rangle$. Таким образом, мы немедленно получаем следующее предложение.

\begin{proposition}\label{RetrMetTopProjIsMetTopProj} Всякий ретракт $\langle$~метрически / топологически~$\rangle$ проективного модуля в $\langle$~$A-\mathbf{mod}_1$ / $A-\mathbf{mod}$ ~$\rangle$ снова $\langle$~метрически / топологически~$\rangle$ проективен.
\end{proposition}

Помимо этого в \cite{HelMetrFrQMod} и \cite{ShtTopFrClassicQuantMod} было доказано, что оснащенная категория $\langle$~$(A-\mathbf{mod}_1,\square_{met})$ / $(A-\mathbf{mod},\square_{top})$~$\rangle$ свободолюбива, и что $\langle$~$\square_{met}$-свободные / $\square_{top}$-свободные~$\rangle$ модули изоморфны в $\langle$~$A-\mathbf{mod}_1$ / $A-\mathbf{mod}$~$\rangle$ модулям вида $A_+\projtens \ell_1(\Lambda)$ для некоторого множества $\Lambda$. Более того, для любого $A$-модуля $X$ существует $\langle$~$\square_{met}$-допустимый / $\square_{top}$-допустимый~$\rangle$ эпиморфизм
$$
\pi_X^+:A_+\projtens \ell_1(B_X):a\projtens \delta_x\mapsto a\cdot x.
$$
Здесь, через $\delta_x$ мы обозначаем функцию из $\ell_1(B_X)$ равную $1$ в точке $x$ и $0$ в остальных точках. Как следствие общих результатов об оснащенных категориях мы получаем следующий критерий  $\langle$~метрической / топологической~$\rangle$ проективности банахова модуля.

\begin{proposition}\label{MetTopProjModViaCanonicMorph}
Модуль $P$ $\langle$~метрически / топологически~$\rangle$ проективен тогда и только тогда, когда  $\pi_P^+$ --- ретракция в $\langle$~$A-\mathbf{mod}_1$ / $A-\mathbf{mod}$~$\rangle$.
\end{proposition}

Так как $\langle$~$\square_{met}$-свободные / $\square_{top}$-свободные~$\rangle$ модули совпадают с точностью до изоморфизма в $A-\mathbf{mod}$, то из предложения \ref{RetrMetTopProjIsMetTopProj} мы видим, что любой метрически проективный $A$-модуль топологически проективен. Напомним, что каждый относительно проективный модуль есть ретракт в $A-\mathbf{mod}$ модуля вида $A_+\projtens E$ для некоторого банахова пространства $E$. Следовательно, каждый топологически проективный $A$-модуль будет относительно проективным. Мы резюмируем эти результаты в следующем предложении.

\begin{proposition}\label{MetProjIsTopProjAndTopProjIsRelProj} Каждый метрически проективный модуль топологически проективен, и каждый топологически проективный модуль относительно проективен.
\end{proposition}

Заметим, что категория банаховых пространств может рассматриваться как категория левых банаховых модулей над нулевой алгеброй. Как следствие, мы получаем определение $\langle$~метрически / топологически~$\rangle$ проективного банахова пространства. Все результаты полученные выше верны для этого типа проективности. Оба типа проективных банаховых пространств уже описаны. В \cite{KotheTopProjBanSp} Кёте доказал, что все топологически проективные банаховы пространства топологически изоморфны $\ell_1(\Lambda)$ для некоторого множества $\Lambda$. Используя результат Гротендика из \cite{GrothMetrProjFlatBanSp}, Хелемский показал, что метрически проективные банаховы пространства изометрически изоморфны $\ell_1(\Lambda)$ для некоторого множества $\Lambda$ [\cite{HelMetrFrQMod}, предложение 3.2].

Теперь перейдем к обсуждению модулей. Легко доказать по определению, 
что $A$-модуль $A_\times$ метрически и топологически проективен, но чаще для доказательства проективности модуля решают задачу ретракции для морфизма $\pi_P^+$. Как показывают следующие два предложения, решение последней задачи иногда можно свести к более простой. 

\begin{proposition}\label{NonDegenMetTopProjCharac} Пусть $P$ --- существенный $A$-модуль, то есть линейная оболочка $A\cdot P$ плотна в $P$. Тогда $P$ $\langle$~метрически / топологически~$\rangle$ проективен тогда и только тогда, когда отображение $\pi_P:A\projtens\ell_1(B_P):a\projtens\delta_x\mapsto a\cdot x$ есть ретракция в $\langle$~$A-\mathbf{mod}_1$ / $A-\mathbf{mod}$~$\rangle$.
\end{proposition}
\begin{proof} Доказательство такое же как и в [\cite{HelBanLocConvAlg}, предложение 7.1.14].
\end{proof}

\begin{proposition}\label{MetTopProjUnderChangeOfAlg} Пусть $I$ --- замкнутая подалгебра в $A$, и $P$ --- банахов $A$-модуль, существенный как $I$-модуль. Тогда:

$i)$ если $I$ --- левый идеал в $A$ и $P$ $\langle$~метрически / топологически~$\rangle$ проективен как $I$-модуль, то $P$ $\langle$~метрически / топологически~$\rangle$ проективен как $A$-модуль;

$ii)$ если $I$ ---  $\langle$~$1$-дополняемый / дополняемый~$\rangle$ правый идеал $A$ и $P$ $\langle$~метрически / топологически~$\rangle$ проективен как $A$-модуль, то $P$ $\langle$~метрически / топологически~$\rangle$ проективен как $I$-модуль.
\end{proposition}
\begin{proof} Доказательство аналогично рассуждениям из [\cite{RamsHomPropSemgroupAlg}, предложение 2.3.3].
\end{proof}



Приведем несколько конструкций сохраняющих проективность модулей. Здесь и далее через $\bigoplus_p\{ E_\lambda:\lambda\in\Lambda\}$ мы будем обозначать $\ell_p$-сумму банаховых пространств $(E_\lambda)_{\lambda\in\Lambda}$. При $p=0$ мы будем подразумевать $c_0$-суммы. Если все пространства $E_\lambda$ являются банаховыми $A$-модулями, то на их $\ell_p$-сумме можно задать структуру банахова $A$-модуля с помощью покоординатного умножения. Следует напомнить, что $\langle$~произвольное / лишь конечное~$\rangle$ семейство модулей обладает категорным копроизведением в $\langle$~$A-\mathbf{mod}_1$ / $A-\mathbf{mod}$~$\rangle$, которое на самом деле есть их $\ell_1$-сумма. В этом и состоит причина почему мы делаем дополнительное предположение во втором пункте следующего предложения.

\begin{proposition}\label{MetTopProjModCoprod} Пусть $(P_\lambda)_{\lambda\in\Lambda}$ --- семейство банаховых $A$-модулей. Тогда 

$i)$ $A$-модуль $\bigoplus_1\{P_\lambda:\lambda\in\Lambda\}$ метрически проективен тогда и только тогда, когда для всех $\lambda\in\Lambda$ банахов $A$-модуль $P_\lambda$ метрически проективен;

$ii)$ если для некоторого $C>1$ и всех $\lambda\in\Lambda$ морфизм $\pi_{P_\lambda}^+$ имеет правый обратный морфизм нормы не более $C$, то $A$-модуль $\bigoplus_1\{P_\lambda:\lambda\in\Lambda\}$ топологически проективен.
\end{proposition}
\begin{proof} Обозначим $P:=\bigoplus_1\{P_\lambda:\lambda\in\Lambda\}$.

$i)$ Если $P$ метрически проективен, то по предложению \ref{RetrMetTopProjIsMetTopProj} для каждого $\lambda\in\Lambda$ банахов $A$-модуль $P_\lambda$ метрически проективен как ретракт $P$ посредством естественной проекции $p_\lambda:P\to P_\lambda$. Обратно, если все модули $(P_\lambda)_{\lambda\in\Lambda}$ метрически проективны, то по общекатегорной схеме метрически проективно их категорное копроизведение $P$ в $A-\mathbf{mod}_1$.

$ii)$ Допустим $P_\lambda$ топологически проективен для каждого $\lambda\in\Lambda$. Из предположения следует, что 	$\bigoplus_1\{\pi_{P_\lambda}^+:\lambda\in\Lambda\}$ является ретракцией в $A-\mathbf{mod}$. Как следствие $\bigoplus_1\{P_\lambda:\lambda\in\Lambda\}$ есть ретракт 
$$
\bigoplus\nolimits_1\left\{A_+\projtens \ell_1(B_{P_\lambda}):\lambda\in\Lambda\right\}
\isom{A-\mathbf{mod}_1}
\bigoplus\nolimits_1\left\{\bigoplus\nolimits_1\{A_+:\lambda'\in B_{P_\lambda}\}:\lambda\in\Lambda\right\}
\isom{A-\mathbf{mod}_1}
\bigoplus\nolimits_1\{A_+:\lambda\in\Lambda_0\}
$$
в $A-\mathbf{mod}$, где $\Lambda_0=\bigcup_{\lambda\in\Lambda}B_{P_\lambda}$. Тогда, по предположению \ref{RetrMetTopProjIsMetTopProj} банахов $A$-модуль $P$ топологически проективен как ретракт топологически проективного $A$-модуля.
\end{proof}

\begin{corollary}\label{MetTopProjTensProdWithl1} Пусть $P$ --- банахов $A$-модуль и $\Lambda$ --- произвольное множество. Тогда $A$-модуль $P\projtens \ell_1(\Lambda)$ $\langle$~метрически / топологически~$\rangle$ проективен тогда и только тогда, когда $P$ $\langle$~метрически / топологически~$\rangle$ проективен.
\end{corollary}

Чтобы понять отличия метрической и топологической банаховой гомологии от относительной рассмотрим еще два примера касающиеся идеалов и циклических модулей. 

\begin{proposition}\label{GoodCommIdealMetTopProjIsUnital} Пусть $I$ --- идеал коммутативной банаховой алгебры $A$ и $I$ имеет $\langle$~сжимающую / ограниченную~$\rangle$ аппроксимативную единицу. Тогда $I$ $\langle$~метрически / топологически~$\rangle$ проективен как $A$-модуль тогда и только тогда, когда $I$ имеет $\langle$~единицу нормы $1$ / единицу~$\rangle$.
\end{proposition}
\begin{proof} См. [\cite{NemMetTopProjIdBanAlg}, теорема 1].
\end{proof}

Этот результат показывает, что метрически или топологически проективные идеалы с ограниченной аппроксимативной единицей должны иметь компактный спектр. В то же время существует множество примеров относительно проективных идеалов со ``всего лишь'' паракомпактным спектором [\cite{HelHomolBanTopAlg}, теорема 3.7].

Следующее предложение является очевидной модификацией описания алгебраически проективных циклических модулей. 

\begin{proposition}\label{MetTopProjCycModCharac} Пусть $I$ --- левый идеал в $A_\times $. Тогда следующие условия эквивалентны:

$i)$ $A$-модуль $A_\times /I$ $\langle$~метрически / топологически~$\rangle$ проективен $\langle$~и естественное фактор-отображение $\pi:A_\times \to A_\times /I$ является строгой коизометрией /~$\rangle$;

$ii)$ существует идемпотент $p\in I$ такой, что $I=A_\times  p$ $\langle$~и $\Vert e_{A_\times }-p\Vert= 1$ /~$\rangle$
\end{proposition}
\begin{proof} С использованием несколько иной терминологии этот факт доказан в [\cite{WhiteInjmoduAlg}, предложение 2.11].
\end{proof}

%----------------------------------------------------------------------------------------
%	Metric and topological injectivity
%----------------------------------------------------------------------------------------

\subsection{Метрическая и топологическая инъективность}
\label{SubSectionMetricAndTopologicalInjectivity}

Как легко догадаться, при изучении метрической и топологической инъективности мы будем использовать два широких класса мономорфизмов, а именно топологически инъективные и изометрические $A$-морфизмы. Напомним, что ограниченный линейный оператор $T:E\to F$ называется топологически инъективным, если для некоторого $c>0$ при всех $x\in E$ выполнено $c\Vert T(x)\Vert\geq \Vert x\Vert$. Далее, если не оговорено иначе, мы будем считать все модули правыми.

\begin{definition}[\cite{HelMetrFrQMod}, определение 4.3]\label{MetTopInjMod} $A$-модуль $J$ называется $\langle$~метрически / топологически~$\rangle$ инъективным, если для любого $\langle$~изометрического / топологически инъективного~$\rangle$ $A$-морфизма $\xi:Y\to X$ и любого $A$-морфизма $\phi:Y\to J$ существует $A$-морфизм $\psi:X\to J$ такой, что $\langle$~$\psi\xi=\phi$ и $\Vert\psi\Vert=\Vert\phi\Vert$ / $\psi\xi=\phi$~$\rangle$.
\end{definition}

В работах \cite{HelMetrFrQMod} и \cite{ShtTopFrClassicQuantMod} были построены верные функторы:
$$
\square_{met}^d:\mathbf{mod}_1-A\to\mathbf{Set},
\quad
\square_{top}^d:\mathbf{mod}-A\to\mathbf{HNor}.
$$
Было доказано, что, во-первых, $A$-морфизм $\xi$ $\langle$~изометричен / топологически инъективен~$\rangle$ тогда и только тогда, когда он $\langle$~$\square_{met}^d$-допустимый / $\square_{top}^d$-допустимый~$\rangle$ мономорфизм и, во-вторых, $A$-модуль $J$ $\langle$~метрически / топологически~$\rangle$ инъективен тогда и только тогда, когда он $\langle$~$\square_{met}^d$-инъективен / $\square_{top}^d$-инъективен~$\rangle$. Таким образом, мы немедленно получаем следующее утверждение.

\begin{proposition}\label{RetrMetTopInjIsMetTopInj} Всякий ретракт $\langle$~метрически / топологически~$\rangle$ инъективного модуля в $\langle$~$\mathbf{mod}_1-A$ / $\mathbf{mod}-A$~$\rangle$ снова $\langle$~метрически / топологически~$\rangle$ инъективен.
\end{proposition}

В \cite{HelMetrFrQMod} и \cite{ShtTopFrClassicQuantMod} также было доказано, что оснащенная категория $\langle$~$(\mathbf{mod}_1-A,\square_{met}^d)$ / $(\mathbf{mod}-A,\square_{top}^d)$~$\rangle$ косвободолюбива, и что $\langle$~$\square_{met}^d$-косвободные / $\square_{top}^d$-косвободные~$\rangle$ модули изоморфны в $\langle$~$\mathbf{mod}_1-A$ / $\mathbf{mod}-A$~$\rangle$ модулям вида $\mathcal{B}(A_+, \ell_\infty(\Lambda))$ для некоторого множества $\Lambda$. Более того, для любого $A$-модуля $X$ существует $\langle$~$\square_{met}^d$-допустимый / $\square_{top}^d$-допустимый~$\rangle$ мономорфизм
$$
\rho_X^+:X\to\mathcal{B}(A_+,\ell_\infty(B_{X^*})):x\mapsto(a\mapsto(f\mapsto f(x\cdot a))).
$$
Как следствие общих результатов об оснащенных категориях мы получаем следующее предложение.

\begin{proposition}\label{MetTopInjModViaCanonicMorph}
Модуль $J$ $\langle$~метрически / топологически~$\rangle$ инъективен тогда и только тогда, когда $\rho_J^+$ --- коретракция в $\langle$~$\mathbf{mod}_1-A$ / $\mathbf{mod}-A$~$\rangle$.
\end{proposition}

Как и для проективных модулей легко доказать следующее предложение. 

\begin{proposition}\label{MetInjIsTopInjAndTopInjIsRelInj} Каждый метрически инъективный модуль топологически инъективен, и каждый топологически инъективный модуль относительно инъективен.
\end{proposition}

Отождествим банаховы пространства с правыми банаховыми модулями над нулевой алгеброй, тогда мы получим определение $\langle$~метрически / топологически~$\rangle$ инъективного банахова пространства. Эквивалентное определение говорит, что банахово пространство $\langle$~метрически / топологически~$\rangle$ инъективно, если оно $\langle$~$1$-дополняемо / дополняемо ~$\rangle$ в любом объемлющем банаховом пространстве. На данный момент полностью описаны только метрически инъективные банаховы пространства --- эти пространства изометрически изоморфны $C(K)$-пространствам для некоторого экстремально несвязного компактного хаусдорфова пространства $K$ [\cite{LaceyIsomThOfClassicBanSp}, теорема 3.11.6]. Обычно такие топологические пространства называются стоуновыми. В частности, метрически инъективно всякое $L_\infty$-пространство. Самые последние достижения в изучении топологически инъективных банаховых пространств можно найти в [\cite{JohnLinHandbookGeomBanSp}, глава 40].

Теперь перейдем к обсуждению модулей. И снова простой факт: $A$-модуль $A_\times^*$ метрически и топологически инъективен. Его легко доказать по определению с использованием теоремы Хана-Банаха. По аналогии с проективными модулями, проверку инъективности модулей часто можно свести к рассмотрению чуть более простых задач ретракции. 

\begin{proposition}\label{NonDegenMetTopInjCharac}  Пусть $J$ --- верный $A$-модуль, то есть равенство $x\cdot A=\{0\}$ влечет $x=0$. Тогда $J$ $\langle$~метрически / топологически~$\rangle$ инъективен тогда и только тогда, когда отображение $\rho_J:J\to\mathcal{B}(A,\ell_\infty(B_{J^*})):x\mapsto(a\mapsto(f\mapsto f(x\cdot a)))$ есть коретракция в $\langle$~$\mathbf{mod}_1-A$ / $\mathbf{mod}-A$~$\rangle$.
\end{proposition} 
\begin{proof} Доказательство аналогично рассуждениям из [\cite{DalPolHomolPropGrAlg}, предложение 1.7].
\end{proof}

\begin{proposition}\label{MetTopInjUnderChangeOfAlg} Пусть $I$ --- замкнутая подалгебра в $A$, и $J$ --- правый банахов $A$-модуль верный как $I$-модуль. Тогда:

$i)$ если $I$ --- левый идеал в $A$ и $J$ $\langle$~метрически / топологически~$\rangle$ инъективный $I$-модуль, то $J$ $\langle$~метрически / топологически~$\rangle$ инъективен как $A$-модуль;

$ii)$ если $I$ --- $\langle$~$1$-дополняемый / дополняемый~$\rangle$ правый идеал $A$ и $J$ $\langle$~метрически / топологически~$\rangle$ инъективен как $A$-модуль, то $J$ $\langle$~метрически / топологически~$\rangle$ инъективен как $I$-модуль.
\end{proposition}
\begin{proof} Доказательство незначительно отличается от [\cite{RamsHomPropSemgroupAlg}, предложение 2.3.4].
\end{proof}

Теперь обсудим конструкции которые сохраняют метрическую и топологическую инъективность. Следует напомнить, что $\langle$~произвольное / лишь конечное~$\rangle$ семейство объектов в $\langle$~$\mathbf{mod}_1-A$ / $\mathbf{mod}-A$~$\rangle$ обладает категорным произведением, которое на самом деле есть их $\ell_\infty$-сумма. Именно поэтому мы делаем дополнительное предположение во втором пункте следующего предложения.

\begin{proposition}\label{MetTopInjModProd} Пусть $(J_\lambda)_{\lambda\in\Lambda}$ --- семейство банаховых $A$-модулей. Тогда:

$i)$ $A$-модуль $\bigoplus_\infty\{J_\lambda:\lambda\in\Lambda\}$ метрически инъективен тогда и только тогда, когда для всех $\lambda\in\Lambda$ банахов $A$-модуль $J_\lambda$ метрически инъективен;

$ii)$ если для некоторого $C>1$ и всех $\lambda\in\Lambda$  морфизм $\rho_{J_\lambda}^+$ имеет левый обратный морфизм нормы не более $C$, то $A$-модуль $\bigoplus_\infty\{J_\lambda:\lambda\in\Lambda\}$ топологически инъективен.
\end{proposition}
\begin{proof} Доказательство мало отличается от рассуждений предложения \ref{MetTopProjModCoprod}. Нужно лишь использовать другой изоморфизм: $\mathcal{B}(A_+,\ell_\infty(\Lambda))\isom{\mathbf{mod}_1-A}\bigoplus_\infty\{A_+^*:\lambda\in\Lambda\}$
\end{proof}

\begin{corollary}\label{MetTopInjlInftySum} Пусть $J$ --- банахов $A$-модуль и $\Lambda$ --- произвольное множество. Тогда $A$-модуль  $\bigoplus_\infty\{J:\lambda\in\Lambda\}$ $\langle$~метрически / топологически~$\rangle$ инъективен тогда и только тогда, когда $J$ $\langle$~метрически / топологически~$\rangle$ инъективен.
\end{corollary}

В отличие от проективности имеется еще один способ конструирования инъективных модулей.

\begin{proposition}\label{MapsFroml1toMetTopInj} Пусть $J$ --- банахов $A$-модуль и $\Lambda$ --- произвольное множество. Тогда $A$-модуль $\mathcal{B}(\ell_1(\Lambda),J)$ $\langle$~метрически / топологически~$\rangle$ инъективен тогда и только тогда, когда $J$ $\langle$~метрически / топологически~$\rangle$ инъективен.
\end{proposition}
\begin{proof} 
Допустим, $\mathcal{B}(\ell_1(\Lambda), J)$  $\langle$~метрически / топологически~$\rangle$ инъективен. Зафиксируем $\lambda\in\Lambda$ и рассмотрим сжимающие $A$-морфизмы $i_\lambda:J\to\mathcal{B}(\ell_1(\Lambda),J):x\mapsto(f\mapsto f(\lambda)x)$ и $p_\lambda:\mathcal{B}(\ell_1(\Lambda),J)\to J:T\mapsto T(\delta_\lambda)$. Очевидно, $p_\lambda i_\lambda=1_J$, то есть $J$ есть ретракт $\mathcal{B}(\ell_1(\Lambda),J)$ в $\langle$~$\mathbf{mod}_1-A$ / $\mathbf{mod}-A$~$\rangle$. Из предложения \ref{RetrMetTopInjIsMetTopInj} следует, что $A$-модуль $J$ $\langle$~метрически / топологически~$\rangle$ инъективен.

Обратно, поскольку $J$ $\langle$~метрически / топологически~$\rangle$ инъективен, то по предложению \ref{MetTopInjModViaCanonicMorph} морфизм $\rho_J^+$ является коретракцией в $\langle$~$\mathbf{mod}_1-A$ / $\mathbf{mod}-A$~$\rangle$. Применим функтор $\mathcal{B}(\ell_1(\Lambda),-)$ к этой коретракции, чтобы получить другую коретракцию $\mathcal{B}(\ell_1(\Lambda),\rho_J^+)$. Заметим, что 
$$
\mathcal{B}(\ell_1(\Lambda),\ell_\infty(B_{J^*}))\isom{\mathbf{Ban}_1}(\ell_1(\Lambda)\projtens \ell_1(B_{J^*}))^*\isom{\mathbf{Ban}_1}\ell_1(\Lambda\times B_{J^*})^*\isom{\mathbf{Ban}_1}\ell_\infty(\Lambda\times B_{J^*}),
$$ 
поэтому мы получаем изометрически изоморфизм банаховых модулей: 
$$
\mathcal{B}(\ell_1(\Lambda),\mathcal{B}(A_+,\ell_\infty(B_{J^*})))\isom{\mathbf{mod}_1-A}\mathcal{B}(A_+,\mathcal{B}(\ell_1(\Lambda),\ell_\infty(B_{J^*}))\isom{\mathbf{mod}_1-A}\mathcal{B}(A_+,\ell_\infty(\Lambda\times B_{J^*})).
$$ 
Значит $\mathcal{B}(\ell_1(\Lambda),J)$ --- ретракт $\mathcal{B}(A_+,\ell_\infty(\Lambda\times B_{J^*}))$ в $\langle$~$\mathbf{mod}_1-A$ / $\mathbf{mod}-A$~$\rangle$, то есть ретракт $\langle$~метрически / топологически~$\rangle$ инъективного $A$-модуля. По предложению \ref{RetrMetTopInjIsMetTopInj} $A$-модуль $\mathcal{B}(\ell_1(\Lambda), J)$ $\langle$~метрически / топологически~$\rangle$ инъективен.
\end{proof}

%----------------------------------------------------------------------------------------
%	Metric and topological flatness
%----------------------------------------------------------------------------------------

\subsection{Метрическая и топологическая плоскость}
\label{SubSectionMetricAndTopologicalFlatness}

Чтобы сохранить единый стиль обозначений мы будем называть метрически плоскими $A$-модули статьи \cite{HelMetrFlatNorMod}, где они были названы экстремально плоскими. Через $\projmodtens{A}$ мы будем обозначать проективное модульное тензорное произведение банаховых модулей. Тем же символом мы будем обозначать и соответствующий функтор.

\begin{definition}[\cite{HelMetrFlatNorMod}, I]\label{MetTopFlatMod} $A$-модуль $F$ называется $\langle$~метрически / топологически~$\rangle$ плоским, если для каждого $\langle$~изометрического / топологически инъективного~$\rangle$ $A$-морфизма $\xi:X\to Y$ правых $A$-модулей линейный оператор $\xi\projmodtens{A} 1_F:X\projmodtens{A} F\to Y\projmodtens{A} F$ $\langle$~изометричен / топологически инъективен~$\rangle$.
\end{definition}

Прежде чем переходить к примерам, нам потребуется дать определение $\mathscr{L}_1$-пространства. Пусть $E$ и $F$ --- два изоморфных банаховых пространства. Тогда расстояние Банаха-Мазура между ними определяется по формуле 
$$
d_{BM}(E,F):=\inf\{\Vert T\Vert\Vert T^{-1}\Vert: T \in \mathcal{B}(E,F) \mbox{ --- изоморфизм}\}.
$$ 
Пусть $\mathcal{F}$ --- некоторое семейство конечномерных банаховых пространств. Будем говорить, что банахово пространство $E$ имеет $\mathcal{F}$-локальную структуру, если для некоторого $C\geq 1$ и для каждого конечномерного подпространства $F$ в $E$ существует содержащее $F$ конечномерное подпространство $G$ в $E$ такое, что $d_{BM}(G,H)\leq C$ для некоторого $H$ из $\mathcal{F}$. Один из самых важных примеров такого типа --- это так называемые $\mathscr{L}_p$-пространства. Впервые они были определены в новаторской работе \cite{LinPelAbsSumOpInLpSpAndApp} и стали незаменимым инструментом в локальной теории банаховых пространств. Для заданного $1\leq p\leq +\infty$ мы будем говорить, что банахово пространство $E$ является $\mathscr{L}_p$-пространством, если оно имеет $\mathcal{F}$-локальную структуру для класса $\mathcal{F}$ конечномерных $\ell_p$-пространств. Наибольший интерес для нас будут представлять $\mathscr{L}_1$- и $\mathscr{L}_\infty$-пространства.

И снова, в качестве примера мы рассмотрим категорию банаховых пространств как категорию модулей над нулевой алгеброй. Из работы Гротендика \cite{GrothMetrProjFlatBanSp} следует, что любое метрически плоское банахово пространство есть $L_1$-пространство. Для топологически плоских банаховых пространств, в отличие от топологически инъективных, мы также имеем критерий [\cite{StegRethNucOpL1LInfSp}, теорема V.1]: банахово пространство топологически плоское тогда и только тогда, когда оно является $\mathscr{L}_1$-пространством.

Хорошо известно, что $A$-модуль $F$ относительно плоский тогда и только тогда, когда $F^*$ относительно инъективный [\cite{HelBanLocConvAlg}, теорема 7.1.42]. Следующее предложение есть очевидный аналог данного результата. Доказательство незначительно отличается от критерия относительной плоскости.

\begin{proposition}\label{MetTopFlatCharac} $A$-модуль $F$ $\langle$~метрически / топологически~$\rangle$ плоский тогда и только тогда, когда $F^*$ $\langle$~метрически / топологически~$\rangle$ инъективен.
\end{proposition}

Комбинируя предложение \ref{MetTopFlatCharac} с предложениями \ref{RetrMetTopInjIsMetTopInj} и \ref{MetInjIsTopInjAndTopInjIsRelInj}, мы получаем следующее.

\begin{proposition}\label{RetrMetTopFlatIsMetTopFlat} Всякий ретракт $\langle$~метрически / топологически~$\rangle$ плоского модуля в $\langle$~$A-\mathbf{mod}_1$ / $A-\mathbf{mod}$~$\rangle$ снова $\langle$~метрически / топологически~$\rangle$ плоский.
\end{proposition}

\begin{proposition}\label{MetFlatIsTopFlatAndTopFlatIsRelFlat} Каждый метрически плоский модуль топологический плоский, и каждый топологически плоский модуль относительно плоский.
\end{proposition}

Отметим, еще одно полезное следствие предложения \ref{MetTopFlatCharac}.

\begin{proposition}\label{MetTopFlatUnderChangeOfAlg} Пусть $I$ --- замкнутая подалгебра в $A$, и $F$ --- банахов $A$-модуль существенный как $I$-модуль. Тогда:

$i)$ если $I$ --- левый идеал в $A$ и $F$ $\langle$~метрически / топологически~$\rangle$ плоский $I$-модуль, то $F$ $\langle$~метрически / топологически~$\rangle$ плоский $A$-модуль;

$ii)$ если $I$ --- $\langle$~$1$-дополняемый / дополняемый~$\rangle$ правый идеал $A$ и $F$ есть $\langle$~метрически / топологически~$\rangle$ плоский $A$-модуль, то $F$ $\langle$~метрически / топологически~$\rangle$ плоский $I$-модуль.
\end{proposition}
\begin{proof} Заметим, что модуль, сопряженный к существенному модулю, будет верным. Теперь все результаты следуют из предложений \ref{MetTopFlatCharac} и \ref{MetTopInjUnderChangeOfAlg}.
\end{proof}

\begin{proposition}\label{DualMetTopProjIsMetrInj} Пусть $P$ --- $\langle$~метрически / топологически~$\rangle$ проективный $A$-модуль, и $\Lambda$ --- произвольное множество. Тогда $A$-модуль $\mathcal{B}(P,\ell_\infty(\Lambda))$ $\langle$~метрически / топологически~$\rangle$ инъективен как $A$-модуль. В частности, $P^*$ $\langle$~метрически / топологически~$\rangle$ инъективен как $A$-модуль.
\end{proposition}
\begin{proof} Из предложения \ref{MetTopProjModViaCanonicMorph} мы знаем, что $\pi_P^+$ --- ретракция в $\langle$~$A-\mathbf{mod}_1$ / $A-\mathbf{mod}$~$\rangle$. Тогда $A$-морфизм $\rho^+=\mathcal{B}(\pi_P^+,\ell_\infty(\Lambda))$ есть коретракция в $\langle$~$\mathbf{mod}_1-A$ / $\mathbf{mod}-A$~$\rangle$. Заметим, что $\mathcal{B}(A_+\projtens\ell_1(B_P),\ell_\infty(\Lambda))\isom{\mathbf{mod}_1-A}\mathcal{B}(A_+,\mathcal{B}(\ell_1(B_P),\ell_\infty(\Lambda)))\isom{\mathbf{mod}_1-A}\mathcal{B}(A_+,\ell_\infty(B_P\times\Lambda))$. Итак, мы показали, что существует коретракция из $\mathcal{B}(P,\ell_\infty(\Lambda))$ в $\langle$~метрически / топологически~$\rangle$ инъективный $A$-модуль. По предложению \ref{RetrMetTopInjIsMetTopInj} банахов $A$-модуль $\mathcal{B}(P,\ell_\infty(\Lambda))$ является $\langle$~метрически / топологически~$\rangle$ инъективным. Чтобы доказать последнее утверждение достаточно взять в качестве $\Lambda$ одноточечное множество.
\end{proof}

Как следствие предложений \ref{MetTopFlatCharac} и \ref{DualMetTopProjIsMetrInj}, мы получаем следующий важный факт.

\begin{proposition}\label{MetTopProjIsMetTopFlat} Каждый $\langle$~метрически / топологически~$\rangle$ проективный модуль является $\langle$~метрически / топологически~$\rangle$ плоским.
\end{proposition}

Позже мы убедимся, что $\langle$~метрическая / топологическая~$\rangle$ плоскость --- это более слабое свойство, чем $\langle$~метрическая / топологическая~$\rangle$ проективность.

По аналогии с проективностью, теперь легко показать, что копроизведения сохраняют метрическую, а иногда и топологическую плоскость модулей.

\begin{proposition}\label{MetTopFlatModCoProd} Пусть $(F_\lambda)_{\lambda\in\Lambda}$ --- семейство банаховых $A$-модулей. Тогда: 

$i)$ $A$-модуль $\bigoplus_1\{F_\lambda:\lambda\in\Lambda\}$ метрически плоский тогда и только тогда, когда для всех $\lambda\in\Lambda$ модуль $F_\lambda$ метрически плоский;

$ii)$ если для некоторого $C>1$ и всех $\lambda\in\Lambda$ морфизм $\rho_{F_\lambda^*}^+$ имеет левый обратный морфизм нормы не более $C$, то $A$-модуль $\bigoplus_1\{F_\lambda:\lambda\in\Lambda\}$ топологически плоский.
\end{proposition}
\begin{proof} По предложению \ref{MetTopFlatCharac} $A$-модуль $F$ $\langle$~метрически / топологически~$\rangle$ плоский тогда и только тогда, когда $F^*$ $\langle$~метрически / топологически~$\rangle$ инъективен. Осталось применить предложение \ref{MetTopInjModProd} с $J_\lambda=F_\lambda^*$ для всех $\lambda\in\Lambda$ и вспомнить, что 
$$
\left(\bigoplus\nolimits_1\{ F_\lambda:\lambda\in\Lambda\}\right)^*\isom{\mathbf{mod}_1-A}\bigoplus\nolimits_\infty\{ F_\lambda^*:\lambda\in\Lambda\}.
$$
\end{proof}

Теперь	 мы обсудим условия, при которых идеалы и циклические модули будут метрически и топологически плоскими. Доказательство, следующего предложения практически не отличается от своего ``относительного аналога'' в [\cite{HelBanLocConvAlg}, предложение 7.1.45].

\begin{proposition}\label{MetTopFlatIdealsInUnitalAlg} Пусть $I$ --- левый идеал в $A_\times $ и $I$ имеет правую $\langle$~сжимающую / ограниченную~$\rangle$ аппроксимативную единицу. Тогда $A$-модуль $I$ $\langle$~метрически / топологически~$\rangle$ плоский.
\end{proposition}

Теперь, кстати, мы можем дать пример метрически плоского модуля, который не является даже топологически проективным. Очевидно, $\ell_\infty(\mathbb{N})$-модуль $c_0(\mathbb{N})$ не унитален как идеал алгебры $\ell_\infty(\mathbb{N})$, но имеет сжимающую аппроксимативную единицу. По теореме \ref{GoodCommIdealMetTopProjIsUnital} этот модуль не является топологически проективным, но он метрически плоский по предложению \ref{MetTopFlatIdealsInUnitalAlg}.

Вторая часть следующего предложения есть небольшая переформулировка [\cite{WhiteInjmoduAlg}, предложение 4.11]. Случай топологической плоскости идеалов был изучен Хелемским в [\cite{HelHomolBanTopAlg}, теорема VI.1.20].

\begin{proposition}\label{MetTopFlatCycModCharac} Пусть $I$ --- левый собственный идеал в $A_\times $. Тогда следующие условия эквивалентны:

$i)$ $A$-модуль $A_\times /I$ $\langle$~метрически / топологически~$\rangle$ плоский;

$ii)$ $I$ имеет правую ограниченную аппроксимативную единицу $(e_\nu)_{\nu\in N}$ $\langle$~такую, что $\sup_{\nu\in N}\Vert e_{A_\times }-e_\nu\Vert\leq 1$ /~$\rangle$.
\end{proposition}

Следует сказать, что всякая операторная алгебра $A$ (не обязательно самосопряженная) обладающая сжимающей аппроксимативной единицей имеет сжимающую аппроксимативную единицу $(e_\nu)_{\nu\in N}$ со свойством $\sup_{\nu\in N}\Vert e_{A_\#}-e_\nu\Vert\leq 1$ и даже $\sup_{\nu\in N}\Vert e_{A_\#}-2e_\nu\Vert\leq 1$. Здесь $A_\#$ --- унитизация $A$ как операторной алгебры. Подробности можно найти в \cite{PosAndApproxIdinBanAlg}, \cite{BleContrAppIdInOpAlg}.

Сравним эти результаты о метрической и топологической плоскости циклических модулей с их относительными аналогами. Хелемский и Шейнберг показали [\cite{HelHomolBanTopAlg}, теорема VII.1.20], что циклический модуль будет относительно плоским если $I$ имеет правую ограниченную аппроксимативную единицу. В случае когда $I^\perp$ дополняемо в $A_\times^*$ верна и обратная импликация. В топологической теории это требование излишне, поэтому удается получить критерий. Метрическая плоскость циклических модулей слишком сильное свойство из-за специфических ограничений на норму аппроксимативной единицы. Как мы увидим в следующем параграфе, оно настолько ограничительное, что не позволяет построить ни одного ненулевого аннуляторного метрически проективного, инъективного или плоского модуля над ненулевой банаховой алгеброй.

%----------------------------------------------------------------------------------------
%	The impact of Banach geometry
%----------------------------------------------------------------------------------------

\section{Влияние банаховой геометрии}
\label{SectionTheImpactOfBanachGeometry}


%----------------------------------------------------------------------------------------
%	Homologically trivial annihilator modules
%----------------------------------------------------------------------------------------

\subsection{Гомологически тривиальные аннуляторные модули}
\label{SubSectionHomoligicallyTrivialAnnihilatorModules}

В этом параграфе мы сконцентрируем наше внимание на метрической и топологической проективности, инъективности и плоскости аннуляторных модулей, то есть модулей с нулевым внешним умножением. Если не оговорено иначе, все банаховы пространства в этом параграфе рассматриваются как аннуляторные модули. Отметим очевидный факт: всякий ограниченный линейный оператор между аннуляторными $A$-модулями является $A$-морфизмом.

\begin{proposition}\label{AnnihCModIsRetAnnihMod} Пусть $X$ --- ненулевой аннуляторный $A$-модуль. Тогда $\mathbb{C}$ есть ретракт $X$ в $A-\mathbf{mod}_1$.
\end{proposition}
\begin{proof} Рассмотрим произвольный вектор $x_0\in X$ нормы $1$. Используя теорему Хана-Банаха выберем функционал $f_0\in X^*$ так, чтобы $\Vert f_0\Vert=f_0(x_0)=1$. Рассмотрим линейные операторы $\pi:X\to \mathbb{C}:x\mapsto f_0(x)$, $\sigma:\mathbb{C}\to X:z\mapsto zx_0$. Легко проверить, что $\pi$ и $\sigma$ суть сжимающие $A$-морфизмы и, более того, $\pi\sigma=1_\mathbb{C}$. Другими словами, $\mathbb{C}$ есть ретракт $X$ в $A-\mathbf{mod}_1$.
\end{proof}

Пришло время вспомнить, что любая банахова алгебра $A$ может рассматриваться как собственный максимальный идеал в $A_+$, причем $\mathbb{C}\isom{A-\mathbf{mod}_1} A_+/A$. Если рассматривать $\mathbb{C}$ как правый аннуляторный $A$-модуль, то имеет место еще один изоморфизм  $\mathbb{C}\isom{\mathbf{mod}_1-A}(A_+/A)^*$. 

\begin{proposition}\label{MetTopProjModCCharac} Аннуляторный $A$-модуль $\mathbb{C}$ $\langle$~метрически / топологически~$\rangle$ проективен тогда и только тогда, когда $\langle$~$A=\{0\}$ / $A$ имеет правую единицу~$\rangle$.
\end{proposition}
\begin{proof} 
Достаточно исследовать $\langle$~метрическую / топологическую~$\rangle$ проективность модуля $A_+/A$. Естественное фактор-отображение $\pi:A_+\to A_+/A$ является строгой коизометрией, поэтому по предложению \ref{MetTopProjCycModCharac} $\langle$~метрическая / топологическая~$\rangle$ проективность $A_+/A$ эквивалентна существованию $p\in A$ такого, что $A=A_+p$ $\langle$~и $\Vert e_{A_+}-p\Vert=1$ /~$\rangle$. $\langle$~Осталось заметить, что $\Vert e_{A_+}-p\Vert=1$ тогда и только тогда, когда $p=0$, что эквивалентно $A=A_+p=\{0\}$ /~$\rangle$.
\end{proof}

\begin{proposition}\label{MetTopProjOfAnnihModCharac} Пусть $P$ --- ненулевой аннуляторный $A$-модуль. Тогда следующие условия эквивалентны:

$i)$ $P$ --- $\langle$~метрически / топологически~$\rangle$ проективный $A$-модуль;

$ii)$ $\langle$~$A=\{0\}$ / $A$ имеет правую единицу~$\rangle$ и $P$ --- $\langle$~метрически / топологически~$\rangle$ проективное банахово пространство, то есть $\langle$~$P\isom{\mathbf{Ban}_1}\ell_1(\Lambda)$ / $P\isom{\mathbf{Ban}}\ell_1(\Lambda)$~$\rangle$ для некоторого множества $\Lambda$.
\end{proposition}
\begin{proof} $i)$$\implies$$ ii)$ Из предложений \ref{RetrMetTopProjIsMetTopProj} и \ref{AnnihCModIsRetAnnihMod} следует, что $A$-модуль $\mathbb{C}$ $\langle$~метрически / топологически~$\rangle$ проективен как ретракт $\langle$~метрически / топологически~$\rangle$ проективного модуля $P$. Предложение \ref{MetTopProjModCCharac} дает, что $\langle$~$A=\{0\}$ / $A$ имеет правую единицу~$\rangle$.  По следствию \ref{MetTopProjTensProdWithl1} аннуляторный $A$-модуль $\mathbb{C}\projtens\ell_1(B_P)\isom{A-\mathbf{mod}_1}\ell_1(B_P)$ $\langle$~метрически / топологически~$\rangle$ проективен. Рассмотрим строгую коизометрию $\pi:\ell_1(B_P)\to P$ корректно определенную равенством $\pi(\delta_x)=x$. Так как $P$ $\langle$~метрически / топологически~$\rangle$ проективен, то $A$-морфизм $\pi$ имеет правый обратный морфизм $\sigma$ в $\langle$~$A-\mathbf{mod}_1$ / $A-\mathbf{mod}$~$\rangle$. Таким образом, $\sigma\pi$ есть $\langle$~сжимающий / ограниченный~$\rangle$ проектор из $\langle$~метрически / топологически~$\rangle$ проективного банахова пространства $\ell_1(B_P)$ на $P$, то есть $P$ --- $\langle$~метрически / топологически~$\rangle$ проективное банахово пространство. Теперь из $\langle$~[\cite{HelMetrFrQMod}, предложение 3.2] / результатов \cite{KotheTopProjBanSp}~$\rangle$ следует, что $P$ изоморфно $\ell_1(\Lambda)$ в $\langle$~$\mathbf{Ban}_1$ / $\mathbf{Ban}$~$\rangle$ для некоторого множества $\Lambda$. 

$ii)$$\implies$$ i)$ По предложению \ref{MetTopProjModCCharac} аннуляторный $A$-модуль $\mathbb{C}$ $\langle$~метрически / топологически~$\rangle$ проективен. По следствию \ref{MetTopProjTensProdWithl1} аннуляторный $A$-модуль $\mathbb{C}\projtens\ell_1(\Lambda)\isom{A-\mathbf{mod}_1}\ell_1(\Lambda)$ также $\langle$~метрически / топологически~$\rangle$ проективен.
\end{proof}

\begin{proposition}\label{MetTopInjModCCharac} Правый аннуляторный $A$-модуль $\mathbb{C}$ $\langle$~метрически / топологически~$\rangle$ инъективен тогда и только тогда, когда $\langle$~$A=\{0\}$ / $A$  имеет правую ограниченную аппроксимативную единицу~$\rangle$.
\end{proposition}
\begin{proof} Благодаря предложению \ref{MetTopFlatCharac} достаточно изучить $\langle$~метрическую / топологическую~$\rangle$ плоскость модуля $A_+/A$. По предложению \ref{MetTopFlatCycModCharac} это эквивалентно существованию правой ограниченной аппроксимативной единицы $(e_\nu)_{\nu\in N}$ в $A$ $\langle$~со свойством $\sup_{\nu\in N}\Vert e_{A_+}-e_\nu\Vert\leq 1$ /~$\rangle$. $\langle$~Осталось заметить, что $\Vert e_{A_+}-e_\nu\Vert\leq 1$ тогда и только тогда, когда $e_\nu=0$, что эквивалентно $A=\{0\}$ /~$\rangle$.
\end{proof}

\begin{proposition}\label{MetTopInjOfAnnihModCharac} Пусть $J$ --- ненулевой правый аннуляторный $A$-модуль. Тогда следующие условия эквивалентны:

$i)$ $J$ --- $\langle$~метрически / топологически~$\rangle$ инъективный $A$-модуль;

$ii)$ $\langle$~$A=\{0\}$ / $A$ имеет правую ограниченную аппроксимативную единицу~$\rangle$ и $J$ ---  $\langle$~метрически / топологически~$\rangle$ инъективное банахово пространство $\langle$~то есть $J\isom{\mathbf{Ban}_1}C(K)$ для некоторого для стоунова пространства $K$ /~$\rangle$.
\end{proposition}
\begin{proof} $i)$$\implies$$ ii)$  Из предложений \ref{RetrMetTopInjIsMetTopInj} и \ref{AnnihCModIsRetAnnihMod} мы получаем, что $A$-модуль $\mathbb{C}$ $\langle$~метрически / топологически~$\rangle$ инъективен как ретракт $\langle$~метрически / топологически~$\rangle$ инъективного модуля $J$. Предложение \ref{MetTopInjModCCharac} дает нам, что $\langle$~$A=\{0\}$ / $A$ имеет правую ограниченную аппроксимативную единицу~$\rangle$. Из предложения \ref{MapsFroml1toMetTopInj} следует, что аннуляторный $A$-модуль $\mathcal{B}(\ell_1(B_{J^*}),\mathbb{C})\isom{\mathbf{mod}_1-A}\ell_\infty(B_{J^*})$ $\langle$~метрически / топологически~$\rangle$ инъективен. Рассмотрим изометрию $\rho:J\to\ell_\infty(B_{J^*})$ корректно определенную равенством $\rho(x)(f)=f(x)$. Поскольку $J$ $\langle$~метрически / топологически~$\rangle$ инъективен, $\rho$ имеет левый обратный морфизм $\tau$ в $\langle$~$\mathbf{mod}_1-A$ / $\mathbf{mod}-A$~$\rangle$. Тогда $\rho\tau$ --- $\langle$~сжимающий / ограниченный~$\rangle$ проектор из $\langle$~метрически / топологически~$\rangle$ инъективного банахова пространства $\ell_\infty(B_{J^*})$ на $J$, поэтому $J$ также является $\langle$~метрически / топологически~$\rangle$ инъективным банаховым пространством. $\langle$~Из [\cite{LaceyIsomThOfClassicBanSp}, теорема 3.11.6] мы знаем, что $J$ изометрически изоморфно $C(K)$ для некоторого стоунова пространства $K$. /~$\rangle$ 

$ii)$$\implies$$ i)$ По предложению \ref{MetTopInjModCCharac} аннуляторный $A$-модуль $\mathbb{C}$ $\langle$~метрически / топологически~$\rangle$ инъективен. По предложению \ref{MapsFroml1toMetTopInj} аннуляторный $A$-модуль $\mathcal{B}(\ell_1(B_{J^*}),\mathbb{C})\isom{\mathbf{mod}_1-A}\ell_\infty(B_{J^*})$ также $\langle$~метрически / топологически~$\rangle$ инъективен. Так как $J$ --- $\langle$~метрически / топологически~$\rangle$ инъективное банахово пространство и существует изометрическое вложение $\rho:J\to \ell_\infty(B_{J^*})$, то $J$ является ретрактом $\ell_\infty(B_{J^*})$ в $\langle$~$\mathbf{Ban}_1$ / $\mathbf{Ban}$~$\rangle$. Напомним, что $J$ и $\ell_\infty(B_{J^*})$ аннуляторные модули, поэтому данная ретракция также является ретракцией в $\langle$~$\mathbf{mod}_1-A$ / $\mathbf{mod}-A$~$\rangle$. По предложению \ref{RetrMetTopInjIsMetTopInj} $A$-модуль $J$ $\langle$~метрически / топологически~$\rangle$ инъективен.
\end{proof}

\begin{proposition}\label{MetTopFlatAnnihModCharac} Пусть $F$ --- ненулевой аннуляторный $A$-модуль. Тогда следующие условия эквивалентны:

$i)$ $F$ --- $\langle$~метрически / топологически~$\rangle$ плоский $A$-модуль;

$ii)$ $\langle$~$A=\{0\}$ / $A$ имеет правую ограниченную аппроксимативную единицу~$\rangle$ и $F$ --- $\langle$~метрически / топологически~$\rangle$ плоское банахово пространство, то есть $F$ является $\langle$~$L_1$-пространством / $\mathscr{L}_1$-пространством~$\rangle$.
\end{proposition}
\begin{proof} Из $\langle$~[\cite{GrothMetrProjFlatBanSp}, теорема 1] / [\cite{StegRethNucOpL1LInfSp}, теорема VI.6]~$\rangle$ мы знаем, что банахово пространство $F^*$ $\langle$~метрически / топологически~$\rangle$ инъективно тогда и только тогда, когда $F$ есть $\langle$~$L_1$-пространство / $\mathscr{L}_1$-пространство~$\rangle$. Теперь эквивалентность следует из предложений \ref{MetTopInjOfAnnihModCharac} и \ref{MetTopFlatCharac}.
\end{proof}

Сравним эти результаты с их аналогами из относительной теории. Из $\langle$~[\cite{RamsHomPropSemgroupAlg}, предложение 2.1.7] / [\cite{RamsHomPropSemgroupAlg}, предложение 2.1.10]~$\rangle$ мы знаем, что аннуляторный модуль над банаховой алгеброй $A$  относительно $\langle$~проективный / плоский~$\rangle$ тогда и только тогда, когда $A$ имеет  $\langle$~правую единицу / правую ограниченную аппроксимативную единицу~$\rangle$. В метрической и топологической теории, в отличие от относительной, гомологическая тривиальность аннуляторных модулей налагает ограничения не только на алгебру, но и на геометрию самого модуля. Эти геометрические ограничения запрещают существование некоторых гомологически лучших банаховых алгебр. Одно из важных свойств относительно $\langle$~стягиваемых / аменабельных~$\rangle$ банаховых алгебр --- это $\langle$~проективность / плоскость~$\rangle$ всех (и в частности аннуляторных) левых банаховых модулей над ней. Резкое отличие метрической и топологической теории в том, что в них подобных алгебр не может быть.

\begin{proposition} Не существует банаховой алгебры $A$ такой, что все $A$-модули $\langle$~метрически / топологически~$\rangle$ плоские. Тем более, не существует банаховых алгебр таких, что все $A$-модули $\langle$~метрически / топологически~$\rangle$ проективны.
\end{proposition}
\begin{proof} Рассмотрим бесконечномерное $\mathscr{L}_\infty$-пространство $X$ (например $\ell_\infty(\mathbb{N})$) как аннуляторный $A$-модуль. Из [\cite{DefFloTensNorOpId}, следствие 23.3(4)] мы знаем, что пространство $X$ не является $\mathscr{L}_1$-пространством. Следовательно, по предложению \ref{MetTopFlatAnnihModCharac} модуль $X$ не является топологически плоским. По предложению \ref{MetFlatIsTopFlatAndTopFlatIsRelFlat} он также и не метрически плоский. Наконец, из предложения \ref{MetTopProjIsMetTopFlat} следует, что $X$ не является ни метрически, ни топологически проективным.
\end{proof}

%----------------------------------------------------------------------------------------
%	Homologically trivial modules over Banach algebras with specific geometry
%----------------------------------------------------------------------------------------

\subsection{Гомологически тривиальные модули над банаховыми алгебрами со специальной геометрией}
\label{SubSectionHomologicallyTrivialModulesOverBanachAlgebrasWithSpecificGeometry}

Цель данного параграфа --- убедить читателя в том, что гомологически тривиальные модули над некоторыми банаховыми алгебрами имеют с этими алгебрами схожие геометрические свойства. Результаты следующего предложения в случае метрической теории были получены Гравеном в \cite{GravInjProjBanMod}.

\begin{proposition}\label{TopProjInjFlatModOverL1Charac} Пусть $A$ --- банахова алгебра изометрически изоморфная как банахово некоторому $L_1$-пространству. Тогда:

$i)$ если $P$ --- $\langle$~метрически / топологически~$\rangle$ проективный $A$-модуль, то $P$ --- $\langle$~$L_1$-пространство / ретракт $L_1$-пространства~$\rangle$;

$ii)$ если $J$ --- $\langle$~метрически / топологически~$\rangle$ инъективный $A$-модуль, то $J$ --- $\langle$~$C(K)$-пространство для некоторого стоунова пространства $K$ / топологически инъективное банахово пространство~$\rangle$;

$iii)$ если $F$ --- $\langle$~метрически / топологически~$\rangle$ плоский $A$-модуль, то $F$ --- $\langle$~$L_1$-пространство / $\mathscr{L}_1$-пространство~$\rangle$.
\end{proposition}
\begin{proof} Через $(\Theta',\Sigma', \nu')$ обозначим пространство с мерой такое, что $A_+\isom{\mathbf{Ban}_1} L_1(\Theta',\nu')$.

$i)$ Так как $P$ $\langle$~метрически / топологически~$\rangle$ проективен как $A$-модуль, то по предложению \ref{MetTopProjModViaCanonicMorph} он является ретрактом $A_+\projtens \ell_1(B_P)$ в $\langle$~$A-\mathbf{mod}_1$ / $A-\mathbf{mod}$~$\rangle$. Пусть $\mu_c$ --- считающая мера на $B_P$, тогда по теореме Гротендика [\cite{HelLectAndExOnFuncAn}, теорема 2.7.5]
$$
A_+\projtens\ell_1(B_P)
\isom{\mathbf{Ban}_1}L_1(\Theta',\nu')\projtens L_1(B_P,\mu_c)
\isom{\mathbf{Ban}_1}L_1(\Theta'\times B_P,\nu'\times \mu_c).
$$
Следовательно, $P$ -- ретракт $L_1$-пространства в $\langle$~$\mathbf{Ban}_1$ / $\mathbf{Ban}$~$\rangle$. Осталось заметить, что любой ретракт $L_1$-пространства в $\mathbf{Ban}_1$ есть снова $L_1$-пространство [\cite{LaceyIsomThOfClassicBanSp}, теорема 6.17.3].

$ii)$ Так как $J$ $\langle$~метрически / топологически~$\rangle$ инъективный $A$-модуль, то по предложению \ref{MetTopInjModViaCanonicMorph} он является ретрактом $\mathcal{B}(A_+,\ell_\infty(B_{J^*}))$ в $\langle$~$\mathbf{mod}_1-A$ / $\mathbf{mod}-A$~$\rangle$. Пусть $\mu_c$ --- считающая мера на $B_{J^*}$, тогда
$$
\mathcal{B}(A_+,\ell_\infty(B_{J^*}))
\isom{\mathbf{Ban}_1}(A_+\projtens \ell_1(B_{J^*}))^*
\isom{\mathbf{Ban}_1}(L_1(\Theta',\nu')\projtens L_1(B_P,\mu_c))^*
$$
$$
\isom{\mathbf{Ban}_1}L_1(\Theta'\times B_P,\nu'\times \mu_c)^*
\isom{\mathbf{Ban}_1}L_\infty(\Theta'\times B_P,\nu'\times \mu_c).
$$
Следовательно, $J$ --- ретракт $L_\infty$-пространства в $\langle$~$\mathbf{Ban}_1$ / $\mathbf{Ban}$~$\rangle$. Поскольку $L_\infty$-пространства $\langle$~метрически / топологически~$\rangle$ инъективны, то таковы же и их ретракты $J$. Осталось напомнить, что каждое метрически инъективное банахово пространство суть $C(K)$-пространство для некоторого стоунова пространства $K$ [\cite{LaceyIsomThOfClassicBanSp}, теорема 3.11.6].

$iii)$ Из $\langle$~[\cite{GrothMetrProjFlatBanSp}, теорема 1] / [\cite{StegRethNucOpL1LInfSp}, теорема VI.6]~$\rangle$ мы знаем, что банахово пространство $F^*$ инъективно в $\langle$~$\mathbf{Ban}_1$ / $\mathbf{Ban}$~$\rangle$ тогда и только тогда, когда $F$ является $\langle$~$L_1$-пространством / $\mathscr{L}_1$-пространством~$\rangle$. Остается применить результаты пункта $ii)$ и предложение \ref{MetTopFlatCharac}.
\end{proof}

\begin{proposition}\label{TopProjInjFlatModOverMthscrL1SpCharac} Пусть $A$ --- банахова алгебра изоморфная, как банахово пространство, некоторому $\mathscr{L}_1$-пространству. Тогда любой топологически $\langle$~проективный / инъективный / плоский~$\rangle$ $A$-модуль является $\langle$~$\mathscr{L}_1$-пространством / $\mathscr{L}_\infty$-пространством / $\mathscr{L}_1$-пространством~$\rangle$.
\end{proposition}
\begin{proof} Если алгебра $A$ есть $\mathscr{L}_1$-пространство, то такова же и $A_+$. 

Пусть $P$ --- топологически проективный $A$-модуль. Тогда по предложению \ref{MetTopProjModViaCanonicMorph} он есть ретракт $A_+\projtens \ell_1(B_P)$ в $A-\mathbf{mod}$ и тем более в $\mathbf{Ban}$. Поскольку $\ell_1(B_P)$ есть $\mathscr{L}_1$-пространство, то таково же $A_+\projtens\ell_1(B_P)$ как проективное тензорное произведение $\mathscr{L}_1$-пространств [\cite{GonzDPPInTensProd}, предложение 1]. Следовательно, $P$ есть $\mathscr{L}_1$-пространство как ретракт $\mathscr{L}_1$-пространства [\cite{BourgNewClOfLpSp}, предложение 1.28].

Пусть $J$ --- топологически инъективный $A$-модуль. Тогда по предложению \ref{MetTopInjModViaCanonicMorph} он есть ретракт $\mathcal{B}(A_+,\ell_\infty(B_{J^*}))\isom{\mathbf{mod}_1-A}(A_+\projtens\ell_1(B_{J^*}))^*$ в $\mathbf{mod}-A$ и тем более в $\mathbf{Ban}$. Как мы показали выше, пространство $A_+\projtens\ell_1(B_{J^*})$ является $\mathscr{L}_1$-пространством, тогда его сопряженное $\mathcal{B}(A_+,\ell_\infty(B_{J^*}))$ есть $\mathscr{L}_\infty$-пространство [\cite{BourgNewClOfLpSp}, предложение 1.27]. Осталось заметить, что любой ретракт $\mathscr{L}_\infty$-пространства есть $\mathscr{L}_\infty$-пространство [\cite{BourgNewClOfLpSp}, предложение 1.28].

Наконец, пусть $F$ --- топологически плоский $A$-модуль, тогда $F^*$ топологически инъективен по предложению \ref{MetTopFlatCharac}. Из предыдущего абзаца следует, что $F^*$ --- это $\mathscr{L}_\infty$-пространство. По теореме VI.6 из \cite{StegRethNucOpL1LInfSp} пространство $F$ является $\mathscr{L}_1$-пространством.
\end{proof}

Перейдем к обсуждению свойства Данфорда-Петтиса для гомологически тривиальных банаховых модулей. Прежде напомним, что банахово пространство $E$ имеет свойство Данфорда-Петтиса, если любой слабо компактный оператор из $E$ в произвольное банахово пространство $F$ вполне непрерывен. Существует простое внутреннее описание этого свойства [\cite{KalAlbTopicsBanSpTh}, теорема 5.4.4]: банахово пространство $E$ обладает свойством Данфорда-Петтиса если $\lim_n f_n(x_n)=0$ для любых слабо сходящихся к $0$ последовательностей $(x_n)_{n\in\mathbb{N}}\subset E$ и $(f_n)_{n\in\mathbb{N}}\subset E^*$. Отсюда легко доказать, что если банахово пространство $E^*$ имеет свойство Данфорда-Петтиса, то его имеет и $E$. Любое $\mathscr{L}_1$-пространство и любое $\mathscr{L}_\infty$-пространство имеет свойство Данфорда-Петтиса  [\cite{BourgNewClOfLpSp}, предложение 1.30]. В частности, все $L_1$-пространства и $C(K)$-пространства имеют это свойство. Свойство Данфорда-Петтиса наследуется дополняемыми подпространствами [\cite{FabHabBanSpTh}, предложение 13.44]. 

Ключевым для нас будет результат Бургейна о банаховых пространствах со специальной локальной структурой. В [\cite{BourgOnTheDPP}, теорема 5] он доказал, что первое, второе и так далее сопряженное пространство банахова пространства с $E_p$-локальной структурой обладает свойством Данфорда-Петтиса. Здесь, $E_p$ обозначает класс всех $\ell_\infty$-сумм $p$ копий $p$-мерных $\ell_1$-пространств для некоторого натурального $p$. 

\begin{proposition}\label{C0SumOfL1SpHaveDPP} Пусть $\{L_1(\Omega_\lambda,\mu_\lambda):\lambda\in\Lambda\}$ --- семейство бесконечномерныx $L_1$-пространств. Тогда банахово пространство $\bigoplus_0\{L_1(\Omega_\lambda,\mu_\lambda):\lambda\in\Lambda\}$ имеет $(E_p,1+\epsilon)$-локальную структуру для всех $\epsilon>0$.
\end{proposition}
\begin{proof} Для каждого $\lambda\in\Lambda$ через $L_1^0(\Omega_\lambda,\mu_\lambda)$ обозначим плотное подпространство в $L_1(\Omega_\lambda,\mu_\lambda)$ натянутое на характеристические функции измеримых множеств из $\Sigma_\lambda$. Обозначим $E:=\bigoplus_0\{L_1(\Omega_\lambda,\mu_\lambda):\lambda\in\Lambda\}$, пусть $E_0:=\bigoplus_{00}\{L_1(\Omega_\lambda,\mu_\lambda):\lambda\in\Lambda\}$ --- не обязательно замкнутое подпространство в $E$, состоящее из векторов с конечным числом ненулевых координат.

Зафиксируем $\epsilon>0$ и конечномерное подпространство $F$ в $E$. Так как $F$ конечномерно, то существует ограниченный проектор $Q:E\to E$ на $F$. Выберем $\delta>0$ так, чтобы $\delta\Vert Q\Vert<1$ и $(1+\delta\Vert Q\Vert)(1-\delta\Vert Q\Vert)^{-1}<1+\epsilon$. Заметим, что $B_F$ компактно, потому что $F$ конечномерно. Следовательно, существует конечная $\delta/2$-сеть $(x_k)_{k\in\mathbb{N}_n}\subset E_0$ для $B_F$. Для каждого $k\in\mathbb{N}_n$ имеем $x_k=\bigoplus_0\{x_{k,\lambda}:\lambda\in\Lambda\}$, где $x_{k,\lambda}=\sum_{j=1}^{m_{k,\lambda}}d_{k,j,\lambda}\chi_{D_{j,k,\lambda}}$ для некоторых комплексных чисел $(d_{j,k,\lambda})_{j\in\mathbb{N}_{m_{k,\lambda}}}$ и измеримых множеств $(D_{j,k,\lambda})_{j\in\mathbb{N}_{m_{k,\lambda}}}$ конечной меры. Пусть $(C_{i,\lambda})_{i\in\mathbb{N}_{m_\lambda}}$ --- множество всех попарных пересечений элементов из $(D_{j,k,\lambda})_{j\in\mathbb{N}_{m_{k,\lambda}}}$ исключая множества меры $0$. Тогда $x_{k,\lambda}=\sum_{i=1}^{m_\lambda} c_{i,k,\lambda}\chi_{C_{i,\lambda}}$ для некоторых комплексных чисел $(c_{j,k,\lambda})_{j\in\mathbb{N}_{m_{\lambda}}}$. Обозначим $\Lambda_k=\{\lambda\in\lambda:x_{k,\lambda}\neq 0\}$. По определению пространства $E_0$ множество $\Lambda_k$ конечно для каждого $k\in\mathbb{N}_n$. Рассмотрим конечное множество $\Lambda_0=\bigcup_{k\in\mathbb{N}_n}\Lambda_k$. Так как пространство $L_1(\Omega_\lambda, \mu_\lambda)$ бесконечномерно, то мы можем добавить к семейству $\{\chi_{C_{i,\lambda}}:i\in\mathbb{N}_{m_\lambda}\}$ любое конечное число дизъюнктных множеств положительной конечной меры. Поэтому, далее считаем, что $m_\lambda=\operatorname{Card}(\Lambda_0)$ для всех $\lambda\in\Lambda_0$. Для каждого $\lambda\in\Lambda_0$ корректно определен проектор 
$$
P_\lambda:L_1(\Omega_\lambda,\mu_\lambda)\to L_1(\Omega_\lambda,\mu_\lambda):x_\lambda\mapsto \sum_{i=1}^{m_\lambda}\left( \mu(C_{i,\lambda})^{-1}\int_{C_{i,\lambda}}x_\lambda(\omega)d\mu_\lambda(\omega)\right)\chi_{C_{i,\lambda}}.
$$
Легко проверить, что $P_\lambda(\chi_{C_{i,\lambda}})=\chi_{C_{i,\lambda}}$ для всех $i\in\mathbb{N}_{m_\lambda}$. Следовательно, $P_\lambda(x_{k,\lambda})=x_{k,\lambda}$ для всех $k\in\mathbb{N}_n$. Так как множества $(C_{i,\lambda})_{i\in\mathbb{N}_{m_\lambda}}$ не пересекаются и имеют положительную меру, то $\operatorname{Im}(P_\lambda)\isom{\mathbf{Ban}_1}\ell_1(\mathbb{N}_{m_\lambda})$. Для $\lambda\in\Lambda\setminus\Lambda_0$ мы положим $P_\lambda=0$ и рассмотрим проектор $P:=\bigoplus_0\{P_\lambda:\lambda\in\Lambda\}$. По построению он сжимающий с образом $\operatorname{Im}(P)\isom{\mathbf{Ban}_1}\bigoplus_0\{\ell_1(\mathbb{N}_{m_\lambda}):\lambda\in\Lambda_0\}\in E_{p}$. Рассмотрим произвольный вектор $x\in B_F$, тогда существует номер $k\in\mathbb{N}_n$ такой, что $\Vert x-x_k\Vert\leq \delta/2$. Тогда $\Vert P(x)-x\Vert=\Vert P(x)-P(x_k)+x_k-x\Vert\leq\Vert P\Vert\Vert x-x_k\Vert+\Vert x_k-x\Vert\leq\delta$.

Построив проекторы $P$ и $Q$, рассмотрим оператор $I:=1_E+PQ-Q$. Очевидно, $\Vert 1_E-I\Vert=\Vert PQ-Q\Vert\leq \delta\Vert Q\Vert$. Следовательно $I$ --- изоморфизм по стандартному трюку с рядами фон Нойманна [\cite{KalAlbTopicsBanSpTh}, предложение A.2]. Более того, $I^{-1}=\sum_{p=0}^\infty(1_E-I)^p$, поэтому
$$
\Vert I^{-1}\Vert\leq\sum_{p=0}^\infty\Vert 1_E-I\Vert^p\leq\sum_{p=0}^\infty(\delta\Vert Q\Vert)^p=(1-\delta\Vert Q\Vert)^{-1},
\quad
\Vert I\Vert\leq\Vert 1_E\Vert+\Vert I-1_E\Vert\leq 1+\delta\Vert Q\Vert.
$$
Заметим, что $PI=P+P^2Q-PQ=P+PQ-PQ=P$, поэтому для всех $x\in F$ выполнено 
$$
I(x)=x+P(Q(x))-Q(x)=x+P(x)-x=P(x)=P(P(x))=P(I(x))
$$
и $x=(I^{-1}PI)(x)$. Последнее означает, что $F$ содержится в образе ограниченного проектора $R=I^{-1}PI$. Обозначим этот образ через $F_0$ и рассмотрим биограничение  $I_0=I|_{F_0}^{\operatorname{Im}(P)}$ изоморфизма $I$. Так как $\Vert I_0\Vert\Vert I_0^{-1}\Vert\leq\Vert I\Vert\Vert I^{-1}\Vert\leq(1+\delta\Vert Q\Vert)(1-\delta\Vert Q\Vert)^{-1}<1+\epsilon$, то $d_{BM}(F_0,\operatorname{Im}(P))<1+\epsilon$. Таким образом, мы показали, что для любого конечномерного подпространства в $E$ существует подпространство $F_0$ в $E$ содержащее $F$ такое, что $d_{BM}(F_0,U)<1+\epsilon$ для некоторого $U\in E_{p}$. Это значит, что $E$ имеет $(E_{p}, 1+\epsilon)$-локальную структуру.
\end{proof}

\begin{proposition}\label{ProdOfL1SpHaveDPP} Пусть $\{(\Omega_\lambda,\Sigma_\lambda,\mu_\lambda):\lambda\in\Lambda\}$ --- семейство пространств с мерой. Тогда банахово пространство $\bigoplus_\infty\{L_1(\Omega_\lambda,\mu_\lambda):\lambda\in\Lambda\}$ обладает свойством Данфорда-Петтиса.
\end{proposition}
\begin{proof} Сначала предположим, что пространства $L_1(\Omega_\lambda, \mu_\lambda)$ бесконечномерны для всех $\lambda\in\Lambda$. Из предложения \ref{C0SumOfL1SpHaveDPP} мы знаем, что банахово пространство $F:=\bigoplus_0\{L_1(\Omega_\lambda,\mu_\lambda):\lambda\in\Lambda\}$ имеет $E_{p}$-локальную структуру. Тогда по теореме 5 из \cite{BourgOnTheDPP} первое, второе и так далее сопряженное пространство пространства $F$ обладают свойством Данфорда-Петтиса. Как следствие, мы получаем, что $F^{**}=\left(\bigoplus_0\{L_1(\Omega_\lambda,\mu_\lambda):\lambda\in\Lambda\}\right)^{**}$ $\isom{\mathbf{Ban}_1}\bigoplus_\infty\{L_1(\Omega_\lambda,\mu_\lambda)^{**}:\lambda\in\Lambda\}$ имеет свойство Данфорда-Петтиса. Из [\cite{DefFloTensNorOpId}, предложение B10] мы знаем, что каждое $L_1$-пространство 1-дополняемо в своем втором сопряженном. Для каждого $\lambda\in\Lambda$ через $P_\lambda$ обозначим соответствующий проектор в $L_1(\Omega_\lambda,\mu_\lambda)^{**}$. Таким образом $\bigoplus_\infty\{L_1(\Omega_\lambda,\mu_\lambda):\lambda\in\Lambda\}$ 1-дополняемо в $F^{**}\isom{\mathbf{Ban}_1}\bigoplus_\infty\{L_1(\Omega_\lambda,\mu_\lambda)^{**}:\lambda\in\Lambda\}$ посредством проектора $\bigoplus_\infty \{P_\lambda:\lambda\in\Lambda\}$. Так как $F^{**}$ имеет свойство Данфорда-Петтиса, то из [\cite{FabHabBanSpTh}, предложение 13.44] следует, что это свойство имеет и дополняемое в $F^{**}$ подпространство $\bigoplus_\infty\{L_1(\Omega_\lambda,\mu_\lambda):\lambda\in\Lambda\}$.

Теперь рассмотрим общий случай. Любое $L_1$-пространство можно рассматривать как $1$-дополняемое подпространство некоторого бесконечномерного $L_1$-пространства. Как следствие, пространство $\bigoplus_\infty\{L_1(\Omega_\lambda,\mu_\lambda):\lambda\in\Lambda\}$ будет $1$-дополняемо в $\bigoplus_\infty$-сумме бесконечномерных $L_1$-пространств. Как было показано выше, такая сумма обладает свойством Данфорда-Петтиса, а значит,  Осталось вспомнить, что это свойство Данфорда-Петтиса наследуется дополняемыми подпространствами [\cite{FabHabBanSpTh}, предложение 13.44].
\end{proof}

\begin{proposition}\label{ProdOfDualsOfMthscrLInftySpHaveDPP} Пусть $E$ --- $\mathscr{L}_\infty$-пространство и $\Lambda$ --- произвольное множество. Тогда банахово пространство $\bigoplus_\infty\{E^*:\lambda\in\Lambda\}$ имеет свойство Данфорда-Петтиса.
\end{proposition}
\begin{proof} Поскольку $E$ --- это $\mathscr{L}_\infty$-пространство, то $E^*$ дополняемо в некотором $L_1$-пространстве [\cite{LinPelAbsSumOpInLpSpAndApp}, предложение 7.4]. То есть существует ограниченный линейный проектор $P:L_1(\Omega,\mu)\to L_1(\Omega,\mu)$ с образом  изоморфным в $\mathbf{Ban}$ пространству $E$. В этом случае $\bigoplus_\infty\{ E^*:\lambda\in\Lambda\}$ дополняемо в $\bigoplus_\infty\{ L_1(\Omega,\mu):\lambda\in\Lambda\}$ посредством проектора $\bigoplus_\infty\{P:\lambda\in\Lambda\}$. Пространство $\bigoplus_\infty\{ L_1(\Omega,\mu):\lambda\in\Lambda\}$ имеет свойство Данфорда-Петтиса по предложению \ref{ProdOfL1SpHaveDPP}. Тогда из [\cite{FabHabBanSpTh}, предложение 13.44] следует, что этим свойством обладает и его дополняемое подпространство $\bigoplus_\infty\{ E^*:\lambda\in\Lambda\}$.
\end{proof}

\begin{theorem}\label{TopProjInjFlatModOverMthscrL1OrLInftySpHaveDPP} Пусть $A$ --- банахова алгебра, являющаяся, как банахово пространство, $\mathscr{L}_1$- или $\mathscr{L}_\infty$-пространством. Тогда топологически проективные, инъективные и плоские $A$-модули имеют свойство Данфорда-Петтиса.
\end{theorem}
\begin{proof} Предположим, что $A$ является $\mathscr{L}_1$-пространством. Заметим, что $\mathscr{L}_1$- и $\mathscr{L}_\infty$-пространства имеют свойство Данфорда-Петтиса [\cite{BourgNewClOfLpSp}, предложение 1.30]. Теперь результат следует из предложения \ref{TopProjInjFlatModOverMthscrL1SpCharac}.

Предположим $A$ является $\mathscr{L}_\infty$-пространством, тогда такова же и $A_+$. Пусть $J$ --- топологически инъективный $A$-модуль, тогда по предложению \ref{MetTopInjModViaCanonicMorph} он ретракт
$$
\mathcal{B}(A_+,\ell_\infty(B_{J^*}))\isom{\mathbf{mod}_1-A}(A_+\projtens\ell_1(B_{J^*}))^*\isom{\mathbf{mod}_1-A}
\left(\bigoplus\nolimits_1\{ A_+:\lambda\in B_{J^*}\}\right)^*\isom{\mathbf{mod}_1-A}
\bigoplus\nolimits_\infty\{ A_+^*:\lambda\in B_{J^*}\}
$$ 
в $\mathbf{mod}-A$ и тем более в $\mathbf{Ban}$. По предложению \ref{ProdOfDualsOfMthscrLInftySpHaveDPP} последний модуль имеет свойство Данфорда-Петтиса. Так как $J$ его ретракт, то он тоже обладает этим свойством [\cite{FabHabBanSpTh}, предложение 13.44]. 

Если $F$ топологически плоский $A$-модуль, то $F^*$ топологически инъективен по предложению \ref{MetTopFlatCharac}. Из предыдущего абзаца мы знаем, что тогда $F^*$ имеет свойство Данфорда-Петтиса и, как следствие, этим свойством обладает сам модуль $F$.

Пусть $P$ --- топологически проективный $A$-модуль. По предположению \ref{MetTopProjIsMetTopFlat} он топологически плоский и тогда из предыдущего абзаца мы видим, что $P$ имеет свойство Данфорда-Петтиса.
\end{proof}

\begin{corollary}\label{NoInfDimRefMetTopProjInjFlatModOverMthscrL1OrLInfty} Пусть $A$ --- банахова алгебра, являющаяся, как банахово пространство, $\mathscr{L}_1$- или $\mathscr{L}_\infty$-пространством. Тогда не существует топологически проективного, инъективного или плоского бесконечномерного рефлексивного $A$-модуля. Тем более не существует метрически проективного, инъективного или плоского бесконечномерного рефлексивного $A$-модуля.
\end{corollary}
\begin{proof} Из теоремы \ref{TopProjInjFlatModOverMthscrL1OrLInftySpHaveDPP} мы знаем, что любой топологически инъективный $A$-модуль имеет свойство Данфорда-Петтиса. С другой стороны не существует бесконечномерного рефлексивного банахова пространства с этим свойством [\cite{FabHabBanSpTh}, примечание после предложения 13.42]. Итак, мы получили желаемый результат в контексте топологической инъективности. Так как пространство, сопряженное к рефлексивному снова рефлексивно, то из предложения \ref{MetTopFlatCharac} следует результат для топологической плоскости. Осталось вспомнить, что по предложению \ref{MetTopProjIsMetTopFlat} каждый топологически проективный модуль является топологически плоским. Чтобы доказать последнее утверждение вспомним, что по предложению $\langle$~\ref{MetProjIsTopProjAndTopProjIsRelProj} / \ref{MetInjIsTopInjAndTopInjIsRelInj} / \ref{MetFlatIsTopFlatAndTopFlatIsRelFlat}~$\rangle$ метрическая $\langle$~проективность / инъективность / плоскость~$\rangle$ влечет топологическую $\langle$~проективность / инъективность / плоскость~$\rangle$.
\end{proof}

Стоит сказать, что в относительной теории существуют примеры рефлексивных бесконечномерных относительно проективных, инъективных и плоских модулей над банаховыми алгебрами, являющимися $\mathscr{L}_1$- или $\mathscr{L}_\infty$-пространствами. Приведем два примера. Первый связан с сверточной алгеброй $L_1(G)$ локально компактной группы $G$ с мерой Хаара. Эта алгебра является $\mathscr{L}_1$-пространством. В [\cite{DalPolHomolPropGrAlg}, \S6] и \cite{RachInjModAndAmenGr} было доказано, что для для $1<p<+\infty$ банахов $L_1(G)$-модуль $L_p(G)$ является относительно $\langle$~проективным / инъективным / плоским~$\rangle$ тогда и только тогда, когда группа $G$  $\langle$~компактна / аменабельна / аменабельна~$\rangle$. Заметим, что любая компактная группа аменабельна [\cite{PierAmenLCA}, предложение 3.12.1], и поэтому для компактной группы $G$ модуль $L_p(G)$ будет относительно проективным инъективным и плоским для всех $1<p<+\infty$. Второй пример связан с алгеброй $c_0(\Lambda)$ для бесконечного множества $\Lambda$. Это $\mathscr{L}_\infty$-пространство. Алгебра $c_0(\Lambda)$ относительно бипроективна и аменабельна, поэтому $c_0(\Lambda)$-модули $\ell_p(\Lambda)$ для $1<p<\infty$ всегда являются относительно проективными, инъективными и плоскими. 

\bibliographystyle{unsrt}
\bibliography{bibliography.bib}
\end{document}