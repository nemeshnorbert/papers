% Chapter Template

\chapter{Общая теория} % Main chapter title

\label{ChapterGeneralTheory} % Change X to a consecutive number; for referencing this chapter elsewhere, use \ref{ChapterX}

%----------------------------------------------------------------------------------------
%	Projectivity, injectivity and flatness
%----------------------------------------------------------------------------------------

\section{Проективность, инъективность и плоскость}
\label{SectionProjectivityInjectivityAndFlatness}


%----------------------------------------------------------------------------------------
%	Metric and topological projectivity
%----------------------------------------------------------------------------------------

\subsection{Метрическая и топологическая проективность}
\label{SubSectionMetricAndTopologicalProjectivity}

В дальнейшем $A$ обозначает необязательно унитальную банахову алгебру. Мы сразу же приступим к формулировке, пожалуй, двух самых важных определений в этой работе.

\begin{definition}[\cite{HelMetrFrQMod}, определение 1.4]\label{MetProjMod} $A$-модуль $P$ называется метрически проективным, если для любого строго коизометрического $A$-морфизма $\xi:X\to Y$ и любого $A$-морфизма $\phi:P\to Y$ существует $A$-морфизм $\psi:P\to X$ такой, что $\xi\psi=\phi$ и $\Vert\psi\Vert=\Vert\phi\Vert$.
\end{definition}

\begin{definition}[\cite{HelMetrFrQMod}, определение 1.2]\label{TopProjMod} $A$-модуль $P$ называется топологически проективным, если для любого топологически сюръективного $A$-морфизма $\xi:X\to Y$ и любого $A$-морфизма $\phi:P\to Y$ существует $A$-морфизм $\psi:P\to X$ такой, что $\xi\psi=\phi$.
\end{definition}

Эквивалентные и более короткие определения звучат так: $A$-модуль $P$ называется $\langle$~метрически / топологически~$\rangle$ проективным, если функтор $\langle$~$\operatorname{Hom}_{A-\mathbf{mod}_1}(P,-):A-\mathbf{mod}_1\to\mathbf{Ban}_1$ / $\operatorname{Hom}_{A-\mathbf{mod}}(P,-):A-\mathbf{mod}\to\mathbf{Ban}$~$\rangle$ переводит $\langle$~строго коизометрические / топологически сюръективные~$\rangle$ $A$-морфизмы в $\langle$~строго коизометрические / сюръективные~$\rangle$ операторы.

Теперь мы нацелены применить аппарат оснащенных категорий к метрической и топологической проективности. Прежде чем это сделать, нам нужно описать категории так называемых полунормированных пространств определенных Штейнером в \cite{ShtTopFrClassicQuantMod}. Полулинейное пространство над полем $\mathbb{C}$ это множество $E$, элементы которого называются векторами, с бинарной операцией $\cdot:\mathbb{C}\times E\to E$, удовлетворяющей трем аксиомам:

$i)$ $\alpha\cdot(\beta\cdot x)=\alpha\beta\cdot x$ для всех $\alpha,\beta\in\mathbb{C}$ и $x\in E$; 

$ii)$ $1\cdot x=x$ для всех $x\in E$; 

$iii)$ существует нулевой вектор $0\in E$ такой, что $0\cdot x=0$ для всех $x\in E$. 

Отображение $T:E\to F$ между полулинейными пространствами называется полулинейным оператором, если $T(\alpha\cdot x)=\alpha\cdot T(x)$ для всех $\alpha\in\mathbb{C}$ и $x\in E$. Полунормированное пространство $E$ --- это полулинейное пространство вместе с функцией  $\Vert\cdot\Vert:E\to\mathbb{R}_+$ (называемой нормой) такой, что

$i)$ $\Vert x\Vert=0$ тогда и только тогда, когда $x=0$;

$ii)$ $\Vert\alpha\cdot x\Vert=|\alpha|\Vert x\Vert$ для всех $\alpha\in\mathbb{C}$ и $x\in E$. 

Полулинейный оператор $T:E\to F$ между полунормированными пространствами $E$ и $F$ называется ограниченным, если существует константа $C\geq 0$ такая, что $\Vert\phi(x)\Vert\leq C\Vert x\Vert$ для всех $x\in E$. Наименьшая такая константа называется нормой $\phi$ и обозначается как $\Vert \phi\Vert$. Наконец, мы определим категорию полунормированных пространств $\mathbf{HNor}$: ее объекты --- полунормированные пространства, ее морфизмы --- полулинейные операторы. На самом деле, полунормированные пространства и ограниченные полулинейные операторы это то, что останется от нормированных пространств и ограниченных линейных операторов, если убрать из их определений все упоминания операции сложения векторов. Вот типичный пример  полунормированного пространства. Для заданного непустого множества $\Lambda$ рассмотрим букет $\mathbb{C}^\Lambda:=\bigvee \{\mathbb{C}:\lambda\in\Lambda\}$ копий $\mathbb{C}$ с общим нулевым вектором. Умножение на скаляры и норма в $\mathbb{C}^\Lambda$ определяются очевидным образом. Также положим по определению $\mathbb{C}^\varnothing=\{0\}$. Любое полунормированное пространство изоморфно в $\mathbf{HNor}$ пространству $\mathbb{C}^\Lambda$ для некоторого множества $\Lambda$ [\cite{ShtTopFrClassicQuantMod}, предложение 1.1.9].

В \cite{HelMetrFrQMod} и \cite{ShtTopFrClassicQuantMod} были построены два верных функтора: 
$$
\square_{met}:A-\mathbf{mod}_1\to\mathbf{Set}:X\mapsto B_X,\phi\mapsto\phi|_{B_X}^{B_Y},
$$
$$
\square_{top}:A-\mathbf{mod}\to\mathbf{HNor}:X\mapsto X,\phi\mapsto\phi.
$$
Первый из них отправляет банахов $A$-модуль в его единичный шар, а всякий сжимающий $A$-морфизм в соответствующее биограничение. Второй функтор ``забывает'' о модульной и аддитивной структуре.
В тех же статьях было доказано, что, во-первых, $A$-морфизм $\xi$ $\langle$~строго коизометричен / топологически сюръективен~$\rangle$ тогда и только тогда, когда он $\langle$~$\square_{met}$-допустимый / $\square_{top}$-допустимый~$\rangle$ эпиморфизм и, во-вторых, $A$-модуль $P$ является $\langle$~метрически / топологически~$\rangle$ проективным тогда и только тогда, когда он $\langle$~$\square_{met}$-проективен / $\square_{top}$-проективен~$\rangle$. Таким образом, мы немедленно получаем следующее предложение.

\begin{proposition}\label{RetrMetTopProjIsMetTopProj} Всякий ретракт $\langle$~метрически / топологически~$\rangle$ проективного модуля в $\langle$~$A-\mathbf{mod}_1$ / $A-\mathbf{mod}$~$\rangle$ снова $\langle$~метрически / топологически~$\rangle$ проективен.
\end{proposition}

Также было доказано, что оснащенная категория $\langle$~$(A-\mathbf{mod}_1,\square_{met})$ / $(A-\mathbf{mod},\square_{top})$~$\rangle$ свободолюбива, и что $\langle$~$\square_{met}$-свободные / $\square_{top}$-свободные~$\rangle$ модули изоморфны в $\langle$~$A-\mathbf{mod}_1$ / $A-\mathbf{mod}$~$\rangle$ модулям вида $A_+\projtens \ell_1(\Lambda)$ для некоторого множества $\Lambda$. Более того, для любого $A$-модуля $X$ существует $\langle$~$\square_{met}$-допустимый / $\square_{top}$-допустимый~$\rangle$ эпиморфизм
$$
\pi_X^+:A_+\projtens \ell_1(B_X):a\projtens \delta_x\mapsto a\cdot x.
$$
Как следствие общих результатов об оснащенных категориях мы получаем следующее предложение.

\begin{proposition}\label{MetTopProjModViaCanonicMorph}
$A$-модуль $P$ $\langle$~метрически / топологически~$\rangle$ проективен тогда и только тогда, когда  $\pi_P^+$ --- ретракция в $\langle$~$A-\mathbf{mod}_1$ / $A-\mathbf{mod}$~$\rangle$.
\end{proposition}

Так как $\langle$~$\square_{met}$-свободные / $\square_{top}$-свободные~$\rangle$ модули совпадают с точностью до изоморфизма в $A-\mathbf{mod}$ и всякая ретракция в $A-\mathbf{mod}_1$ есть ретракция в $A-\mathbf{mod}$, то из предложения \ref{RetrMetTopProjIsMetTopProj} мы видим, что любой метрически проективный $A$-модуль топологически проективен. Напомним, что каждый относительно проективный модуль есть ретракт в $A-\mathbf{mod}$ модуля вида $A_+\projtens E$ для некоторого банахова пространства $E$. Следовательно, каждый топологически проективный $A$-модуль будет относительно проективным. Мы резюмируем эти результаты в следующем предложении.

\begin{proposition}\label{MetProjIsTopProjAndTopProjIsRelProj} Каждый метрически проективный модуль топологически проективен, и каждый топологически проективный модуль относительно проективен.
\end{proposition}

Количественный аналог определения топологической проективности был дан Уайтом.

\begin{definition}[\cite{WhiteInjmoduAlg}, определение 2.4]\label{CTopProjMod} $A$-модуль $P$ называется $C$-топологически проективным, если для любого строго $c$-топологически сюръективного $A$-морфизма $\xi:X\to Y$ и любого $A$-морфизма $\phi:P\to Y$ существует $A$-морфизм $\psi:P\to X$ такой, что $\xi\psi=\phi$ и $\Vert\psi\Vert\leq cC\Vert\phi\Vert$.
\end{definition}

Нам понадобятся следующие два факта об этом типе проективности.

\begin{proposition}[\cite{WhiteInjmoduAlg}, лемма 2.7]\label{RetrCTopProjIsCTopProj} Всякий $C_1$-ретракт $C_2$-топологически проективного модуля является $C_1C_2$-топологически проективным.
\end{proposition}

\begin{proposition}[\cite{WhiteInjmoduAlg}, предложение 2.10]\label{CTopProjModViaCanonicMorph} $A$-модуль $P$ является $C$-топологически проективным тогда и только тогда, когда $\pi_P^+$ --- $C$-ретракция в $A-\mathbf{mod}$.
\end{proposition}

Как следствие, банахов модуль топологически проективен тогда и только тогда, когда он $C$-топологически проективен для некоторого $C$. Далее мы будем использовать определения \ref{TopProjMod} и \ref{CTopProjMod} без ссылки на их эквивалентность.

Теперь перейдем к обсуждению примеров. Заметим, что категория банаховых пространств может рассматриваться как категория левых банаховых модулей над нулевой алгеброй. Как следствие, мы получаем определение $\langle$~метрически / топологически~$\rangle$ проективного банахова пространства. Все результаты полученные выше верны для этого типа проективности. Оба типа проективных банаховых пространств уже описаны. В \cite{KotheTopProjBanSp} Кёте доказал, что все топологически проективные банаховы пространства топологически изоморфны $\ell_1(\Lambda)$ для некоторого множества $\Lambda$. Используя результат Гротендика из \cite{GrothMetrProjFlatBanSp}, Хелемский показал, что метрически проективные банаховы пространства изометрически изоморфны $\ell_1(\Lambda)$ для некоторого индексного множества $\Lambda$ [\cite{HelMetrFrQMod}, предложение 3.2].

\begin{proposition}\label{UnitalAlgIsMetTopProj} $A$-модуль $A_\times$ метрически и топологически проективен.
\end{proposition} 
\begin{proof} Рассмотрим произвольный $A$-морфизм $\phi:A_\times\to Y$ и строго коизометрический $A$-морфизм $\xi:X\to Y$. Из определения строго коизометрического оператора следует, что существует $x_0\in X$ такой, что $\xi(x_0)=\phi(e_{A_\times})$ и $\Vert x_0\Vert=\Vert\phi(e_{A_\times})\Vert$. Рассмотрим $A$-морфизм $\psi:A_\times\to X:a\mapsto a\cdot x_0$. Очевидно, $\Vert\psi\Vert\leq\Vert x_0\Vert\leq\Vert\phi\Vert\Vert e_{A_\times}\Vert=\Vert\phi\Vert$. С другой стороны, $\xi\psi=\phi$, так что $\Vert\phi\Vert\leq\Vert\xi\Vert\Vert\psi\Vert=\Vert\psi\Vert$. Следовательно, $\Vert\phi\Vert=\Vert\psi\Vert$. Итак, мы доказали по определению, что $A_\times$ --- метрически проективный $A$-модуль. По предложению \ref{MetProjIsTopProjAndTopProjIsRelProj} он также топологически проективен.
\end{proof}

\begin{proposition}\label{NonDegenMetTopProjCharac} Пусть $P$ --- существенный $A$-модуль. Тогда $P$ $\langle$~метрически / $C$-топологически~$\rangle$ проективен тогда и только тогда, когда отображение $\pi_P:A\projtens\ell_1(B_P):a\projtens\delta_x\mapsto a\cdot x$ есть $\langle$~$1$-ретракция / $C$-ретракция~$\rangle$ в $A-\mathbf{mod}$.
\end{proposition} 
\begin{proof}
Если $P$ $\langle$~метрически / $C$-топологически~$\rangle$ проективен, то по предложению  $\langle$~\ref{MetTopProjModViaCanonicMorph} / \ref{CTopProjModViaCanonicMorph}~$\rangle$ морфизм $\pi_P^+$ имеет правый обратный морфизм $\sigma^+$ c нормой $\langle$~не более $1$ / не более $C$~$\rangle$. Тогда 
$$
\sigma^+(P)=\sigma^+(\operatorname{cl}_{A_+\projtens\ell_1(B_P)}(AP))\subset \operatorname{cl}_{A_+\projtens\ell_1(B_P)}(A\cdot\sigma(P))=
\operatorname{cl}_{A_+\projtens\ell_1(B_P)}(A\cdot(A_+\projtens\ell_1(B_P)))=A\projtens\ell_1(B_P).
$$
Поэтому корректно определено коограничение $\sigma:P\to A\projtens\ell_1(B_P)$, которое также есть $A$-морфизм c нормой $\langle$~не более $1$ / не более $C$~$\rangle$. Ясно, что $\pi_P\sigma=1_P$, поэтому $\pi_P$ является $\langle$~$1$-ретракцией / $C$-ретракцией~$\rangle$ в $A-\mathbf{mod}$.

Обратно, допустим $\pi_P$ имеет правый обратный морфизм $\sigma$ c нормой $\langle$~не более $1$ / не более $C$~$\rangle$. Тогда его копродолжение $\sigma^+$ также является правым обратным морфизмом к $\pi_P^+$ с той же нормой. Снова, по предложению $\langle$~\ref{MetTopProjModViaCanonicMorph} / \ref{CTopProjModViaCanonicMorph}~$\rangle$ модуль $P$ $\langle$~метрически / $C$-топологически~$\rangle$ проективен. 
\end{proof}

Следует напомнить, что $\langle$~произвольное / лишь конечное~$\rangle$ семейство объектов в $\langle$~$A-\mathbf{mod}_1$ / $A-\mathbf{mod}$~$\rangle$ обладает категорным копроизведением, которое на самом деле есть их $\bigoplus_1$-сумма. В этом и состоит причина почему мы делаем дополнительное предположение во втором пункте следующего предложения.

\begin{proposition}\label{MetTopProjModCoprod} Пусть $(P_\lambda)_{\lambda\in\Lambda}$ --- семейство банаховых $A$-модулей. Тогда 

$i)$ $A$-модуль $\bigoplus_1\{P_\lambda:\lambda\in\Lambda\}$ метрически проективен тогда и только тогда, когда для всех $\lambda\in\Lambda$ банахов $A$-модуль $P_\lambda$ метрически проективен;

$ii)$ $A$-модуль $\bigoplus_1\{P_\lambda:\lambda\in\Lambda\}$ $C$-топологически проективен тогда и только тогда, когда для всех $\lambda\in\Lambda$ банахов $A$-модуль $P_\lambda$ $C$-топологически проективен.
\end{proposition}
\begin{proof} Обозначим $P:=\bigoplus_1\{P_\lambda:\lambda\in\Lambda\}$.

$i)$ Доказательство аналогично доказательству из пункта $ii)$.

$ii)$ Допустим, что $P$ $C$-топологически проективен. Заметим, что для каждого $\lambda\in\Lambda$ модуль $P_\lambda$ является $1$-ретрактом $P$ посредством канонической проекции $p_\lambda:P\to P_\lambda$. По предложению \ref{RetrCTopProjIsCTopProj} модуль $P_\lambda$ $C$-топологически проективен.

Обратно, допустим, что для каждого $\lambda\in\Lambda$ модуль $P_\lambda$ $C$-топологически проективен. По предложению \ref{CTopProjModViaCanonicMorph} мы имеем семейство $C$-ретракций $\pi_\lambda:A_+\projtens\ell_1(S_\lambda)\to P_\lambda$. Следовательно, $\bigoplus_1\{\pi_{P_\lambda}^+:\lambda\in\Lambda\}$ является $C$-ретракцией в $A-\mathbf{mod}$. Значит, $P$ есть $C$-ретракт 
$$
\bigoplus\nolimits_1\left\{A_+\projtens \ell_1(S_\lambda):\lambda\in\Lambda\right\}
\isom{A-\mathbf{mod}_1}
\bigoplus\nolimits_1\left\{\bigoplus\nolimits_1\{A_+:s\in S_\lambda\}:\lambda\in\Lambda\right\}
\isom{A-\mathbf{mod}_1}
\bigoplus\nolimits_1\{A_+:s\in S\}
$$
в $A-\mathbf{mod}$, где $S=\bigsqcup_{\lambda\in\Lambda}S_\lambda$. Ясно, что последний модуль $1$-топологически проективен, поэтому из предложения \ref{RetrCTopProjIsCTopProj} следует, что $A$-модуль $P$ $C$-топологически проективен.
\end{proof}

\begin{corollary}\label{MetTopProjTensProdWithl1} Пусть $P$ --- банахов $A$-модуль и $\Lambda$ --- произвольное множество. Тогда $A$-модуль $P\projtens \ell_1(\Lambda)$ $\langle$~метрически / топологически~$\rangle$ проективен тогда и только тогда, когда $P$ $\langle$~метрически / топологически~$\rangle$ проективен.
\end{corollary}
\begin{proof} Заметим, что $P\projtens \ell_1(\Lambda)\isom{A-\mathbf{mod}_1}\bigoplus_1\{P:\lambda\in\Lambda\}$. Теперь достаточно применить предложение \ref{MetTopProjModCoprod} с $P_\lambda=P$ для всех $\lambda\in\Lambda$.
\end{proof}

%----------------------------------------------------------------------------------------
%	Metric and topological projectivity of ideals and cyclic modules
%----------------------------------------------------------------------------------------

\subsection{Метрическая и топологическая проективность идеалов и циклических модулей}
\label{SubSectionMetricAndTopologicalProjectivityOfIdealsAndCyclicModules}

Как мы вскоре увидим, идемпотенты играют основную роль в изучении метрической и топологической проективности. Поэтому нам следует напомнить одно из следствий теоремы Шилова об идемпотентах [\cite{KaniBanAlg}, параграф 3.5]: каждая полупростая коммутативная банахова алгебра с компактным спектром имеет единицу, но не обязательно нормы $1$. 

\begin{proposition}\label{UnIdeallIsMetTopProj}
Пусть $I$ --- левый идеал банаховой алгебры $A$. Тогда

$i)$ если $I=Ap$ для некоторого $\langle$~идемпотента $p\in I$ нормы $1$ / идемпотента $p\in I$~$\rangle$, то $I$ $\langle$~метрически / топологически~$\rangle$ проективен как $A$-модуль;

$ii)$ если $I$ --- коммутативная полупростая алгебра и $\operatorname{Spec}(I)$ компактен, то $I$ топологически проективен как $A$-модуль.
\end{proposition}
\begin{proof} 
$i)$ Очевидно, что $A$-модульные операторы $\pi:A_\times\to I:x\mapsto xp$ и $\sigma:I\to A_\times:x\mapsto x$ $\langle$~сжимающие / ограниченные~$\rangle$ и $\pi\sigma=1_I$. Тогда $I$ есть ретракт $A_\times$ в $\langle$~$A-\mathbf{mod}_1$ / $A-\mathbf{mod}$~$\rangle$. Теперь результат следует из предложений \ref{RetrMetTopProjIsMetTopProj} и \ref{UnitalAlgIsMetTopProj}.

$ii)$ По теореме Шилова об идемпотентах идеал $I$ унитален. Вообще говоря, норма его единицы не меньше $1$. Из пункта $i)$ следует, что идеал $I$ топологически проективен.
\end{proof}

Предположение полупростоты в \ref{UnIdeallIsMetTopProj} не обязательно. В [\cite{DalesIntroBanAlgOpHarmAnal}, упражнение 2.3.7] дан пример коммутативной унитальной банаховой алгебры $A$, которая не является полупростой. По предложению \ref{UnIdeallIsMetTopProj} она топологически проективна как $A$-модуль. Чтобы доказать главный результат этого параграфа нам нужны две подготовительные леммы.

\begin{lemma}\label{ImgOfAMorphFromBiIdToA} Пусть $I$ --- двусторонний идеал банаховой алгебры $A$, существенный как левый $I$-модуль и пусть задан $A$-морфизм $\phi:I\to A$. Тогда $\operatorname{Im}(\phi)\subset I$.
\end{lemma}
\begin{proof} Так как $I$ --- правый идеал, то $\phi(ab)=a\phi(b)\in I$ для всех $a,b\in I$. Поэтому $\phi(I\cdot I)\subset I$. Так как $I$ --- существенный левый $I$-модуль, то $I=\operatorname{cl}_A(\operatorname{span}(I\cdot I))$ и $\operatorname{Im}(\phi)\subset\operatorname{cl}_A(\operatorname{span}\phi(I\cdot I))=\operatorname{cl}_A(\operatorname{span}I)=I$.
\end{proof}

\begin{lemma}\label{GoodIdealMetTopProjIsUnital} Пусть $I$ --- левый идеал банаховой алгебры $A$. Допустим, выполнено одно из следующих условий:

$(*)$ $I$ имеет левую $\langle$~сжимающую / ограниченную~$\rangle$ аппроксимативную единицу, и для любого морфизма $\phi:I\to A$ левых $A$-модулей найдется морфизм $\psi:I\to I$ правых $I$-модулей со свойством $\phi(x)y=x\psi(y)$ для всех $x,y\in I$.

$(**)$ $I$ имеет правую $\langle$~сжимающую / ограниченную~$\rangle$ аппроксимативную единицу, и существует $\langle$~$C=1$ / $C\geq 1$~$\rangle$ такое, что для любого морфизма $\phi:I\to A$ левых $A$-модулей найдется морфизм $\psi:I\to I$ правых $I$-модулей со свойствами $\Vert\psi\Vert\leq C\Vert\phi\Vert$ и $\phi(x)y=x\psi(y)$ для всех $x,y\in I$.

Тогда следующие условия эквивалентны:

$i)$ $I$ $\langle$~метрически / топологически~$\rangle$ проективен как $A$-модуль;

$ii)$ $I$ обладает $\langle$~правой единицей нормы $1$ / правой единицей~$\rangle$.
\end{lemma} 
\begin{proof} $i)$$\implies$$ii)$ Если выполнено $(*)$ или $(**)$, то $I$ обладает односторонней аппроксимативной единицей. Следовательно, $I$ --- существенный левый $I$-модуль и тем более существенный $A$-модуль. По предложению \ref{NonDegenMetTopProjCharac}, существует правый обратный $A$-морфизм $\sigma:I\to A\projtens \ell_1(B_I)$ к $\pi_I$ в $\langle$~$A-\mathbf{mod}_1$ / $A-\mathbf{mod}$~$\rangle$. Для каждого $d\in B_I$ рассмотрим $A$-морфизм $p_d:A\projtens \ell_1(B_I)\to A:a\projtens \delta_x\mapsto \delta_x(d)a$ и $\sigma_d=p_d\sigma$. Тогда $\sigma(x)=\sum_{d\in B_I}\sigma_d(x)\projtens \delta_d$ для всех $x\in I$. Из отождествления $A\projtens\ell_1(B_I)\isom{\mathbf{Ban}_1}\bigoplus_1\{A:d\in B_I\}$ мы имеем $\Vert\sigma(x)\Vert=\sum_{d\in B_I} \Vert\sigma_d(x)\Vert$ для всех $x\in I$. Так как $\sigma$ --- правый обратный морфизм к $\pi_I$ то $x=\pi_I(\sigma(x))=\sum_{d\in B_I}\sigma_d(x)d$ для всех $x\in I$. 

Предположим, выполнено условие $(*)$. Тогда для каждого $d\in B_I$ существует морфизм правых $I$-модулей $\tau_d:I\to I$ такой, что $\sigma_d(x)d=x\tau_d(d)$ для всех $x\in I$.  Пусть $(e_\nu)_{\nu\in N}$ --- левая $\langle$~сжимающая / ограниченная~$\rangle$ аппроксимативная единица в $I$ ограниченная по норме константой $D$. Поскольку $\tau_d(d)\in I$ для всех $d\in B_I$, то для любого множества $S\in\mathcal{P}_0(B_I)$ выполнено
$$
\sum_{d\in S}\Vert \tau_d(d)\Vert
=\sum_{d\in S}\lim_{\nu}\Vert e_\nu \tau_d(d) \Vert
=\lim_{\nu}\sum_{d\in S}\Vert e_\nu \tau_d(d)\Vert
=\lim_{\nu}\sum_{d\in S}\Vert \sigma_d(e_\nu)d \Vert
$$
$$
\leq\liminf_{\nu}\sum_{d\in S}\Vert\sigma_d(e_\nu)\Vert\Vert d\Vert 
\leq\liminf_{\nu}\sum_{d\in S}\Vert\sigma_d(e_\nu)\Vert
\leq\liminf_{\nu}\sum_{d\in B_I}\Vert\sigma_d(e_\nu)\Vert
$$
$$
=\liminf_{\nu}\Vert\sigma(e_\nu)\Vert
\leq\Vert\sigma\Vert\liminf_{\nu}\Vert e_\nu\Vert
\leq D\Vert\sigma\Vert.
$$

Теперь предположим что, выполнено условие $(**)$. Из предположения, для каждого $d\in B_I$ существует морфизм правых $I$-модулей $\tau_d:I\to I$ такой, что $\sigma_d(x)d=x\tau_d(d)$ для всех $x\in I$ и $\Vert\tau_d\Vert\leq C\Vert\sigma_d\Vert$. Пусть $(e_\nu)_{\nu\in N}$ --- правая $\langle$~сжимающая / ограниченная~$\rangle$ аппроксимативная единица в $I$ ограниченная по норме некоторой константой $D$. Для всех $x\in I$ выполнено
$$
\Vert\sigma_d(x)\Vert
=\Vert\sigma_d(\lim_\nu x e_\nu)\Vert
=\lim_\nu\Vert x\sigma_d(e_\nu)\Vert
\leq\Vert x\Vert\liminf_\nu\Vert\sigma_d(e_\nu)\Vert,
$$
поэтому $\Vert\sigma_d\Vert\leq \liminf_\nu\Vert\sigma_d(e_\nu)\Vert$. Тогда для всех $S\in\mathcal{P}_0(B_I)$ выполнено
$$
\sum_{d\in S}\Vert \tau_d(d)\Vert
\leq \sum_{d\in S}\Vert \tau_d\Vert\Vert d\Vert
\leq C\sum_{d\in S}\Vert \sigma_d\Vert
\leq C\sum_{d\in S}\liminf_\nu \Vert \sigma_d(e_\nu)\Vert
\leq C\liminf_{\nu}\sum_{d\in S}\Vert \sigma_d(e_\nu) \Vert
$$
$$
\leq C\liminf_{\nu}\sum_{d\in B_I}\Vert \sigma_d(e_\nu) \Vert
=C\liminf_{\nu}\Vert\sigma(e_\nu)\Vert
\leq C\Vert\sigma\Vert\liminf_{\nu}\Vert e_\nu\Vert
\leq CD\Vert\sigma\Vert.
$$

Для обоих предположений $(*)$ и $(**)$ мы доказали, что число $\sum_{d\in S}\Vert \tau_d(d)\Vert$ ограничено $\langle$~единицей / некоторой константой~$\rangle$ для любого $S\in \mathcal{P}_0(B_I)$. Следовательно, существует $p=\sum_{d\in B_I}\tau_d(d)\in I$ со свойством $\langle$~$\Vert p\Vert\leq 1$ / $\Vert p\Vert< \infty$~$\rangle$. Более того, для всех $x\in I$ выполнено $x=\sum_{d\in B_I}\sigma_d(x)d=\sum_{d\in B_I}x\tau_d(d)=xp$, то есть $p$ --- правая единица в $I$. 

$ii)$$\implies$$i)$ Пусть $p\in I$  --- правая единица для $I$, тогда $I=Ap$. Теперь из предложения \ref{UnIdeallIsMetTopProj} мы получаем, что идеал $I$ $\langle$~метрически / топологически~$\rangle$ проективен как $A$-модуль.
\end{proof}

Условие $(*)$ предыдущей леммы будет использовано в следующей теореме. Что касается условия $(**)$, оно будет использовано значительно позже.

\begin{theorem}\label{GoodCommIdealMetTopProjIsUnital} Пусть $I$ --- идеал коммутативной банаховой алгебры $A$ и $I$ имеет $\langle$~сжимающую / ограниченную~$\rangle$ аппроксимативную единицу. Тогда $I$ $\langle$~метрически / топологически~$\rangle$ проективен как $A$-модуль тогда и только тогда, когда $I$ имеет $\langle$~единицу нормы $1$ / единицу~$\rangle$.
\end{theorem} 
\begin{proof} Поскольку $A$ коммутативна, то для любого $A$-морфизма $\phi:I\to A$ и любых $x,y\in I$ выполнено $\phi(x)y=x\phi(y)$. Так как $I$ имеет ограниченную аппроксимативную единицу и $I$ коммутативен, то мы можем применить лемму \ref{ImgOfAMorphFromBiIdToA}, чтобы заключить $\phi(y)\in I$. Теперь выполнено условие $(*)$ леммы \ref{GoodIdealMetTopProjIsUnital}, и мы получаем желаемую равносильность.
\end{proof}

В относительной теории нет аналогичного критерия проективности идеалов. Наиболее общий результат такого типа дает лишь необходимое условие: если идеал $I$ коммутативной банаховой алгебры $A$ относительно проективен как $A$-модуль, то $I$ имеет паракомпактный спектр. Этот результат получен Хелемским [\cite{HelHomolBanTopAlg}, теорема IV.3.6]. 

Отметим, что существование ограниченной аппроксимативной единицы не является необходимым условием для топологической проективности идеала коммутативной банаховой алгебры. Действительно, рассмотрим банахову алгебру  $A_0(\mathbb{D})$ --- идеал алгебры на диске состоящий из функций исчезающих в нуле. Комбинируя предложения 4.3.5 и 4.3.13 параграф $iii)$ из \cite{DalBanAlgAutCont} мы заключаем, что $A_0(\mathbb{D})$ не имеет ограниченных аппроксимативных единиц. С другой стороны, из [\cite{HelBanLocConvAlg}, пример IV.2.2] мы знаем, что $A_0(\mathbb{D})\isom{A_0(\mathbb{D})-\mathbf{mod}} A_0(\mathbb{D})_+$, поэтому согласно предложению \ref{UnitalAlgIsMetTopProj}, $A_0(\mathbb{D})$ --- топологически проективный $A_0(\mathbb{D})$-модуль.

Следующее предложение является очевидной модификацией описания алгебраически проективных циклических модулей. Оно схоже с [\cite{WhiteInjmoduAlg}, предложение 2.11].

\begin{proposition}\label{MetTopProjCycModCharac} Пусть $I$ --- левый идеал в $A_\times $. Тогда следующие условия эквивалентны:

$i)$ $A$-модуль $A_\times /I$ $\langle$~метрически / топологически~$\rangle$ проективен $\langle$~и естественное фактор-отображение $\pi:A_\times \to A_\times /I$ является строгой коизометрией /~$\rangle$;

$ii)$ существует идемпотент $p\in I$ такой, что $I=A_\times  p$ $\langle$~и $\Vert e_{A_\times }-p\Vert= 1$ /~$\rangle$
\end{proposition}
\begin{proof} $i)$$\implies$$ii)$ Поскольку отображение $\pi$ $\langle$~строго коизометрично / топологически сюръективно~$\rangle$ и модуль $A_\times /I$ $\langle$~метрически / топологически~$\rangle$ проективен, то $\pi$ имеет правый обратный морфизм $\sigma$, который $\langle$~изометричен / топологически инъективен~$\rangle$. Положим, $e_{A_\times }-p=(\sigma\pi)(e_{A_\times })$, тогда $(\sigma\pi)(a)=a(e_{A_\times }-p)$. По построению, $\pi\sigma=1_{A_\times }$, поэтому  
$$
e_{A_\times }-p=(\sigma\pi)(e_{A_\times })=(\sigma\pi)(\sigma\pi)(e_{A_\times })=(\sigma\pi)(e_{A_\times }-p)=(e_{A_\times }-p)(\sigma\pi)(e_{A_\times })=(e_{A_\times }-p)^2.
$$
Откуда следует, что $p^2=p$. Значит, $A_\times p=\operatorname{Ker}(\sigma\pi)$ так как $(\sigma\pi)(a)=a-ap$. Поскольку $\sigma$ инъективен, мы получаем $A_\times p=\operatorname{Ker}(\pi)=I$. Наконец, заметим, что $\Vert e_{A_\times }-p\Vert=\Vert(\sigma\pi)(e_{A_\times})\Vert\leq\Vert\sigma\Vert\Vert\pi\Vert\Vert e_{A_\times }\Vert=\Vert\sigma\Vert$.

$ii)$$\implies$$ i)$ Пусть $p^2=p$, рассмотрим левый идеал $I=A_\times p$ и морфизм $A$-модулей $\sigma:A_\times /I\to A_\times:a+I\mapsto a-ap$. Легко проверить, что $\pi\sigma=1_{A_\times/I}$ и $\Vert\sigma\Vert\leq\Vert e_{A_\times }-p\Vert$. Это означает, что $\pi:A_\times \to A_\times /I$ --- ретракция в $\langle$~$A-\mathbf{mod}_1$ / $A-\mathbf{mod}$~$\rangle$ и, в частности, $\langle$~строго коизометрический / топологически сюръективный~$\rangle$ оператор. Теперь из предложений \ref{UnitalAlgIsMetTopProj} и \ref{RetrMetTopProjIsMetTopProj} следует, что $A$-модуль $A_\times /I$ $\langle$~метрически / топологически~$\rangle$ проективен.
\end{proof} 

В отличие от топологической теории, в относительной теории нет полного описания относительно проективных циклических модулей. Есть частичные ответы при дополнительных предположениях. Например, если идеал $I$ дополняем в $A_\times$ как банахово пространство, то в относительной теории имеет место почти такой же критерий [\cite{HelBanLocConvAlg}, предложение 7.1.29]. Существуют другие описания относительно проективных циклических модулей при менее ограничительных требованиях на банахову геометрию. Например, Селиванов доказал, что если $I$ --- двусторонний идеал и либо $A/I$ имеет свойство аппроксимации либо все неприводимые $A$-модули имеют свойство аппроксимации, то $A/I$  относительно проективен тогда и только тогда, когда $A_\times\isom{A-\mathbf{mod}}I\bigoplus_1 I'$ для некоторого левого идеала $I'$ в $A$. Подробности можно найти в [\cite{HelHomolBanTopAlg}, глава IV, \S 4].



%----------------------------------------------------------------------------------------
%	Metric and topological injectivity
%----------------------------------------------------------------------------------------

\subsection{Метрическая и топологическая инъективность}
\label{SubSectionMetricAndTopologicalInjectivity}

В этом параграфе, если не оговорено иначе, мы будем считать все модули правыми.

\begin{definition}[\cite{HelMetrFrQMod}, определение 4.3]\label{MetInjMod} $A$-модуль $J$ называется метрически инъективным, если для любого изометрического $A$-морфизма $\xi:Y\to X$ и любого $A$-морфизма $\phi:Y\to J$ существует $A$-морфизм $\psi:X\to J$ такой, что $\psi\xi=\phi$  и $\Vert\psi\Vert=\Vert\phi\Vert$.
\end{definition}

\begin{definition}[\cite{HelMetrFrQMod}, определение 4.3]\label{TopInjMod} $A$-модуль $J$ называется топологически инъективным, если для любого топологически инъективного $A$-морфизма $\xi:Y\to X$ и любого $A$-морфизма $\phi:Y\to J$ существует $A$-морфизм $\psi:X\to J$ такой, что $\psi\xi=\phi$.
\end{definition}

Эквивалентные и более короткие определения звучат так: $A$-модуль $J$ называется $\langle$~метрически / топологически~$\rangle$ инъективным, если функтор $\langle$~$\operatorname{Hom}_{\mathbf{mod}_1-A}(-,J):\mathbf{mod}_1-A\to\mathbf{Ban}_1$ / $\operatorname{Hom}_{\mathbf{mod}-A}(-,J):\mathbf{mod}-A\to\mathbf{Ban}$~$\rangle$ переводит $\langle$~изометрические / топологически инъективные~$\rangle$ $A$-морфизмы в $\langle$~строго коизометрические / сюръективные~$\rangle$ операторы.

В \cite{HelMetrFrQMod} и \cite{ShtTopFrClassicQuantMod} были построены два верных функтора:
$$
\square_{met}^d:\mathbf{mod}_1-A\to\mathbf{Set}:X\mapsto B_{X^*},\phi\mapsto\phi^*|_{B_{Y^*}}^{B_{X^*}},
$$
$$
\square_{top}^d:\mathbf{mod}-A\to\mathbf{HNor}:X\mapsto X^*,\phi\mapsto\phi^*.
$$
Первый из них отправляет банахов $A$-модуль в единичный шар своего сопряженного пространства, а всякий сжимающий $A$-морфизм в соответствующее биограничение своего сопряженного. Второй функтор ``забывает'' о модульной и аддитивной структуре сопряженного пространства и сопряженного $A$-морфизма.
В тех же статьях было доказано, что, во-первых, $A$-морфизм $\xi$ $\langle$~изометричен / топологически инъективен~$\rangle$ тогда и только тогда, когда он $\langle$~$\square_{met}^d$-допустимый / $\square_{top}^d$-допустимый~$\rangle$ мономорфизм и, во-вторых, $A$-модуль $J$ $\langle$~метрически / топологически~$\rangle$ инъективен тогда и только тогда, когда он $\langle$~$\square_{met}^d$-инъективен / $\square_{top}^d$-инъективен~$\rangle$. Таким образом, мы немедленно получаем следующее утверждение.

\begin{proposition}\label{RetrMetTopInjIsMetTopInj} Всякий ретракт $\langle$~метрически / топологически~$\rangle$ инъективного модуля в $\langle$~$\mathbf{mod}_1-A$ / $\mathbf{mod}-A$~$\rangle$ снова $\langle$~метрически / топологически~$\rangle$ инъективен.
\end{proposition}

Также было доказано, что оснащенная категория $\langle$~$(\mathbf{mod}_1-A,\square_{met}^d)$ / $(\mathbf{mod}-A,\square_{top}^d)$~$\rangle$ косовободолюбива, и что $\langle$~$\square_{met}^d$-косвободные / $\square_{top}^d$-косвободные~$\rangle$ модули изоморфны в $\langle$~$\mathbf{mod}_1-A$ / $\mathbf{mod}-A$~$\rangle$ модулям вида $\mathcal{B}(A_+, \ell_\infty(\Lambda))$ для некоторого множества $\Lambda$. Более того, для любого $A$-модуля $X$ существует $\langle$~$\square_{met}^d$-допустимый / $\square_{top}^d$-допустимый~$\rangle$ мономорфизм
$$
\rho_X^+:X\to\mathcal{B}(A_+,\ell_\infty(B_{X^*})):x\mapsto(a\mapsto(f\mapsto f(x\cdot a))).
$$
Как следствие общих результатов об оснащенных категориях мы получаем следующее предложение.

\begin{proposition}\label{MetTopInjModViaCanonicMorph}
$A$-модуль $J$ $\langle$~метрически / топологически~$\rangle$ инъективен тогда и только тогда, когда $\rho_J^+$ --- коретракция в $\langle$~$\mathbf{mod}_1-A$ / $\mathbf{mod}-A$~$\rangle$.
\end{proposition}

Так как $\langle$~$\square_{met}^d$-косвободные / $\square_{top}^d$-косвободные~$\rangle$ модули совпадают с точностью до изоморфизма в $\mathbf{mod}-A$ и всякая ретракция в $\mathbf{mod}_1-A$ есть ретракция в $\mathbf{mod}-A$, то из предложения \ref{RetrMetTopInjIsMetTopInj} мы видим, что любой метрически инъективный $A$-модуль топологически инъективен. Напомним, что каждый относительно инъективный модуль есть ретракт в $\mathbf{mod}-A$ модуля вида $\mathcal{B}(A_+,E)$ для некоторого банахова пространства $E$. Следовательно, каждый топологически инъективный $A$-модуль будет относительно инъективным. Мы резюмируем эти результаты в следующем предложении.

\begin{proposition}\label{MetInjIsTopInjAndTopInjIsRelInj} Каждый метрически инъективный модуль топологически инъективен, и каждый топологически инъективный модуль относительно инъективен.
\end{proposition}

Количественный аналог определения топологической инъективности был дан Уайтом.

\begin{definition}[\cite{WhiteInjmoduAlg}, определение 3.4]\label{CTopInjMod} $A$-модуль $J$ называется $C$-топологически инъективным, если для любого $c$-топологически инъективного $A$-морфизма $\xi:Y\to X$ и любого $A$-морфизма $\phi:Y\to J$ существует $A$-морфизм $\psi:X\to J$ такой, что $\psi\xi=\phi$ и $\Vert\psi\Vert\leq cC\Vert\phi\Vert$.
\end{definition}

Нам понадобятся следующие два факта об этом типе инъективности.

\begin{proposition}[\cite{WhiteInjmoduAlg}, лемма 3.7]\label{RetrCTopInjIsCTopInj} Всякий $C_1$-ретракт $C_2$-топологически инъективного модуля является $C_1C_2$-топологически инъективным.
\end{proposition}

\begin{proposition}[\cite{WhiteInjmoduAlg}, предложение 3.10]\label{CTopInjModViaCanonicMorph} $A$-модуль $J$ является $C$-топологически инъективным тогда и только тогда, когда $\rho_J^+$ --- $C$-коретракция в $\mathbf{mod}-A$.
\end{proposition}

Как следствие, банахов модуль топологически инъективен тогда и только тогда, когда он $C$-топологически инъективен для некоторого $C$. Далее мы будем использовать определения \ref{TopInjMod} и \ref{CTopInjMod} без ссылки на их эквивалентность.

Теперь перейдем к обсуждению примеров. Заметим, что категория банаховых пространств может рассматриваться как категория правых банаховых модулей над нулевой алгеброй. Как следствие, мы получаем определение $\langle$~метрически / топологически~$\rangle$ инъективного банахова пространства. Все результаты, полученные выше, верны для этого типа инъективности. Эквивалентное определение говорит, что банахово пространство $\langle$~метрически / топологически~$\rangle$ инъективно, если оно $\langle$~$1$-дополняемо / дополняемо ~$\rangle$ в любом объемлющем банаховом пространстве. Стандартный пример метрически инъективного банахова пространства это $L_\infty$-пространство. На данный момент полностью описаны только метрически инъективные банаховы пространства --- эти пространства изометрически изоморфны $C(K)$-пространствам для некоторого экстремально несвязного компактного хаусдорфова пространства $K$ [\cite{LaceyIsomThOfClassicBanSp}, теорема 3.11.6]. Обычно такие топологические пространства называются стоуновыми. Самые последние достижения в изучении топологически инъективных банаховых пространств можно найти в [\cite{JohnLinHandbookGeomBanSp}, глава 40].

\begin{proposition}\label{DualOfUnitalAlgIsMetTopInj} $A$-модуль $A_\times^*$ метрически и топологически инъективен. 
\end{proposition}
\begin{proof} Рассмотрим произвольный $A$-морфизм $\phi:Y\to A_\times^*$ и изометрический $A$-морфизм $\xi:Y\to X$. Определим ограниченный линейный функционал $f:Y\to\mathbb{C}:y\mapsto \phi(y)(e_{A_\times})$. Так как $\xi$ --- изометрия, то по теореме Хана-Банаха мы может продолжить $f$ до некоторого ограниченного линейного функционала $g:X\to\mathbb{C}$ с той же самой нормой, что у $f$. Рассмотрим $A$-морфизм $\psi:X\to A_\times^*:x\mapsto (a\mapsto g(x\cdot a))$. Очевидно, $\Vert\psi\Vert\leq\Vert g\Vert=\Vert f\Vert\leq\Vert\phi\Vert$. С другой стороны $\psi\xi=\phi$, поэтому $\Vert\phi\Vert\leq\Vert\psi\Vert\Vert\xi\Vert=\Vert\psi\Vert$. Таким образом, $\Vert\phi\Vert=\Vert\psi\Vert$. Итак, мы доказали по определению, что $A_\times^*$ --- метрически инъективный $A$-модуль. По предложению \ref{MetInjIsTopInjAndTopInjIsRelInj} он также топологически инъективен.
\end{proof}

\begin{proposition}\label{NonDegenMetTopInjCharac}  Пусть $J$ --- верный $A$-модуль. Тогда $J$ $\langle$~метрически / $C$-топологически~$\rangle$ инъективен тогда и только тогда, когда отображение $\rho_J:J\to\mathcal{B}(A,\ell_\infty(B_{J^*})):x\mapsto(a\mapsto(f\mapsto f(x\cdot a)))$ есть $\langle$~$1$-коретракция / $C$-коретракция~$\rangle$ в $\mathbf{mod}-A$.
\end{proposition} 
\begin{proof}
Если $J$ $\langle$~метрически / $C$-топологически~$\rangle$ инъективен, то по предложению $\langle$~\ref{MetTopInjModViaCanonicMorph} / \ref{CTopInjModViaCanonicMorph}~$\rangle$ $A$-морфизм $\rho_J^+$ имеет правый обратный морфизм $\tau^+$ с нормой $\langle$~не более $1$ / не более $C$~$\rangle$. Допустим нам задан оператор $T\in \mathcal{B}(A_+,\ell_\infty(B_{J^*}))$ такой, что $T|_A=0$. Зафиксируем $a\in A$, тогда $T\cdot a=0$, и поэтому $\tau^+(T)\cdot a=\tau^+(T\cdot a)=0$. Поскольку $J$ --- верный модуль и $a\in A$ произвольно, то $\tau^+(T)=0$. Рассмотрим естественную проекцию $p:A_+\to A$ и определим $A$-морфизм $j=\mathcal{B}(p,\ell_\infty(B_{J^*}))$ и ограниченный линейный оператор $\tau=\tau^+ j$. Для любого $a\in A$ и $T\in\mathcal{B}(A,\ell_\infty(B_{J^*}))$ мы имеем $\tau(T\cdot a)-\tau(T)\cdot a=\tau^+(j(T\cdot a)-j(T)\cdot a)=0$, потому что $j(T\cdot a)-j(T)\cdot a|_A=0$. Значит $\tau$ --- $A$-морфизм. Заметим, что $\Vert\tau\Vert\leq\Vert\tau^+\Vert\Vert j\Vert\leq \Vert\tau^+\Vert$. Следовательно, $\tau$ имеет норму $\langle$~не более $1$ / не более $C$~$\rangle$. Очевидно, для всех $x\in J$ выполнено $\rho_J^+(x)-j(\rho_J(x))|_A=0$, поэтому $\tau^+(\rho_J^+(x)-j(\rho_J(x)))=0$. Как следствие, $\tau(\rho_J(x))=\tau^+(j(\rho_J(x)))=\tau^+(\rho_J^+(x))=x$ для всех $x\in J$. Так как $\tau\rho_J=1_J$, то $\rho_J$ --- $\langle$~$1$-коретракция / $C$-коретракция~$\rangle$ в $\mathbf{mod}-A$.

Обратно, допустим $\rho_J$ --- $\langle$~$1$-коретракция / $C$-коретракция~$\rangle$ в $\mathbf{mod}-A$, то есть $\rho_J$ имеет правый обратный морфизм $\tau$ с нормой $\langle$~не более $1$ / не более $C$~$\rangle$. Рассмотрим естественное вложение $i:A\to A_+$ и определим $A$-морфизм $q=\mathcal{B}(i,\ell_\infty(B_{J^*}))$. Очевидно, $\rho_J=q\rho_J^+$. Рассмотрим $A$-морфизм $\tau^+=\tau q$. Заметим, что $\Vert\tau^+\Vert\leq\Vert\tau\Vert\Vert q\Vert\leq \Vert\tau\Vert$. Следовательно, $\tau^+$ имеет норму $\langle$~не более $1$ / не более $C$~$\rangle$. Очевидно, $\tau^+\rho_J^+=\tau q\rho_J^+=\tau\rho_J=1_J$. Поэтому $\rho_J^+$ --- $\langle$~$1$-коретракция / $C$-коретракция~$\rangle$ в $\mathbf{mod}-A$ и тогда по предложению $\langle$~\ref{MetTopInjModViaCanonicMorph} / \ref{CTopInjModViaCanonicMorph}~$\rangle$ банахов $A$-модуль $J$ $\langle$~метрически / $C$-топологически~$\rangle$ инъективен.
\end{proof}

Следует напомнить, что $\langle$~произвольное / лишь конечное~$\rangle$ семейство объектов в $\langle$~$\mathbf{mod}_1-A$ / $\mathbf{mod}-A$~$\rangle$ обладает категорным произведением, которое на самом деле есть их $\bigoplus_\infty$-сумма. В этом и состоит причина почему мы делаем дополнительное предположение во втором пункте следующего предложения.

\begin{proposition}\label{MetTopInjModProd} Пусть $(J_\lambda)_{\lambda\in\Lambda}$ --- семейство банаховых $A$-модулей. Тогда 

$i)$ $A$-модуль $\bigoplus_\infty\{J_\lambda:\lambda\in\Lambda\}$ метрически инъективен тогда и только тогда, когда для всех $\lambda\in\Lambda$ банахов $A$-модуль $J_\lambda$ метрически инъективен;

$ii)$ $A$-модуль $\bigoplus_\infty\{J_\lambda:\lambda\in\Lambda\}$ $C$-топологически инъективен тогда и только тогда, когда для всех $\lambda\in\Lambda$ банахов $A$-модуль $J_\lambda$ $C$-топологически инъективен.
\end{proposition}
\begin{proof} Обозначим $J:=\bigoplus_\infty\{J_\lambda:\lambda\in\Lambda\}$.

$i)$ Доказательство аналогично доказательству из пункта $ii)$.

$ii)$ Допустим, что $J$ $C$-топологически инъективен. Заметим, что для каждого $\lambda\in\Lambda$ модуль $J_\lambda$ является $1$-ретрактом $J$ посредством канонической проекции $p_\lambda:J\to J_\lambda$. По предложению \ref{RetrCTopInjIsCTopInj} модуль $J_\lambda$ $C$-топологически инъективен.

Обратно, допустим, что для каждого $\lambda\in\Lambda$ модуль $J_\lambda$ $C$-топологически инъективен. По предложению \ref{CTopInjModViaCanonicMorph} мы имеем семейство $C$-коретракций $\rho_\lambda:J_\lambda\to\mathcal{B}(A_+,\ell_\infty(S_\lambda))$. Следовательно, $\bigoplus_\infty\{\rho_\lambda:\lambda\in\Lambda\}$ является $C$-коретракцией в $A-\mathbf{mod}$. Значит, $J$ есть $C$-ретракт 
$$
\bigoplus\nolimits_\infty\{\mathcal{B}(A_+,\ell_\infty(S_\lambda)):\lambda\in\Lambda\}
\isom{\mathbf{mod}_1-A}
\bigoplus\nolimits_\infty\left\{\bigoplus\nolimits_\infty\{ A_+^*:s\in S_\lambda\}:\lambda\in\Lambda\right\}
\isom{\mathbf{mod}_1-A}
$$
$$
\bigoplus\nolimits_\infty\{A_+^*:s\in S\}
\isom{\mathbf{mod}_1-A}
\mathcal{B}(A_+,\ell_\infty(S))
$$
в $\mathbf{mod}-A$, где $S=\bigsqcup_{\lambda\in\Lambda}S_\lambda$. Ясно, что последний модуль $1$-топологически инъективен, поэтому из предложения \ref{RetrCTopInjIsCTopInj} следует, что $A$-модуль $J$ $C$-топологически инъективен.
\end{proof}

\begin{corollary}\label{MetTopInjlInftySum} Пусть $J$ --- банахов $A$-модуль и $\Lambda$ --- произвольное множество. Тогда $A$-модуль  $\bigoplus_\infty\{J:\lambda\in\Lambda\}$ $\langle$~метрически / топологически~$\rangle$ инъективен тогда и только тогда, когда $J$ $\langle$~метрически / топологически~$\rangle$ инъективен.
\end{corollary}
\begin{proof} Для доказательства достаточно применить предложение \ref{MetTopInjModProd} c $J_\lambda=J$ для всех $\lambda\in\Lambda$.
\end{proof}

\begin{proposition}\label{MapsFroml1toMetTopInj} Пусть $J$ --- банахов $A$-модуль и $\Lambda$ --- произвольное множество. Тогда $A$-модуль $\mathcal{B}(\ell_1(\Lambda),J)$ $\langle$~метрически / топологически~$\rangle$ инъективен тогда и только тогда, когда $J$ $\langle$~метрически / топологически~$\rangle$ инъективен.
\end{proposition}
\begin{proof} 
Допустим, $A$-модуль $\mathcal{B}(\ell_1(\Lambda), J)$  $\langle$~метрически / топологически~$\rangle$ инъективен. Зафиксируем $\lambda\in\Lambda$ и рассмотрим сжимающие $A$-морфизмы $i_\lambda:J\to\mathcal{B}(\ell_1(\Lambda),J):x\mapsto(f\mapsto f(\lambda)x)$ и $p_\lambda:\mathcal{B}(\ell_1(\Lambda),J)\to J:T\mapsto T(\delta_\lambda)$. Очевидно, $p_\lambda i_\lambda=1_J$, то есть $J$ есть ретракт $\mathcal{B}(\ell_1(\Lambda),J)$ в $\langle$~$\mathbf{mod}_1-A$ / $\mathbf{mod}-A$~$\rangle$. Из предложения \ref{RetrMetTopInjIsMetTopInj} следует, что $A$-модуль $J$ $\langle$~метрически / топологически~$\rangle$ инъективен.

Обратно, поскольку $J$ $\langle$~метрически / топологически~$\rangle$ инъективен, то по предложению \ref{MetTopInjModViaCanonicMorph} морфизм $\rho_J^+$ является коретракцией в $\langle$~$\mathbf{mod}_1-A$ / $\mathbf{mod}-A$~$\rangle$. Применим функтор $\mathcal{B}(\ell_1(\Lambda),-)$ к этой коретракции, чтобы получить другую коретракцию $\mathcal{B}(\ell_1(\Lambda),\rho_J^+)$. Заметим, что 
$$
\mathcal{B}(\ell_1(\Lambda),\ell_\infty(B_{J^*}))\isom{\mathbf{Ban}_1}(\ell_1(\Lambda)\projtens \ell_1(B_{J^*}))^*\isom{\mathbf{Ban}_1}\ell_1(\Lambda\times B_{J^*})^*\isom{\mathbf{Ban}_1}\ell_\infty(\Lambda\times B_{J^*}),
$$ 
поэтому существует изометрический изоморфизм банаховых модулей:
$$
\mathcal{B}(\ell_1(\Lambda),\mathcal{B}(A_+,\ell_\infty(B_{J^*})))\isom{\mathbf{mod}_1-A}\mathcal{B}(A_+,\mathcal{B}(\ell_1(\Lambda),\ell_\infty(B_{J^*}))\isom{\mathbf{mod}_1-A}\mathcal{B}(A_+,\ell_\infty(\Lambda\times B_{J^*})).
$$ 
Значит $\mathcal{B}(\ell_1(\Lambda),J)$ --- ретракт $\mathcal{B}(A_+,\ell_\infty(\Lambda\times B_{J^*}))$ в $\langle$~$\mathbf{mod}_1-A$ / $\mathbf{mod}-A$~$\rangle$, то есть ретракт $\langle$~метрически / топологически~$\rangle$ инъективного $A$-модуля. По предложению \ref{RetrMetTopInjIsMetTopInj} $A$-модуль $\mathcal{B}(\ell_1(\Lambda), J)$ $\langle$~метрически / топологически~$\rangle$ инъективен.
\end{proof}

%----------------------------------------------------------------------------------------
%	Metric and topological flatness
%----------------------------------------------------------------------------------------

\subsection{Метрическая и топологическая плоскость}
\label{SubSectionMetricAndTopologicalFlatness}

Чтобы сохранить единый стиль обозначений мы будем называть метрически плоскими $A$-модули статьи \cite{HelMetrFlatNorMod}, где они были названы экстремально плоскими.

\begin{definition}[\cite{HelMetrFlatNorMod}, I]\label{MetFlatMod} $A$-модуль $F$ называется метрически плоским, если для каждого изометрического $A$-морфизма $\xi:X\to Y$ правых $A$-модулей оператор $\xi\projmodtens{A} 1_F:X\projmodtens{A} F\to Y\projmodtens{A} F$ изометричен.
\end{definition}

\begin{definition}[\cite{HelMetrFlatNorMod}, определение I]\label{TopFlatMod} $A$-модуль $F$ называется топологически плоским, если для каждого топологически инъективного $A$-морфизма $\xi:X\to Y$ правых $A$-модулей оператор $\xi\projmodtens{A} 1_F:X\projmodtens{A} F\to Y\projmodtens{A} F$ топологически инъективен.
\end{definition}

Эквивалентные и более короткие определения звучат так: $A$-модуль $J$ называется $\langle$~метрически / топологически~$\rangle$ плоским, если функтор $\langle$~$-\projmodtens{A} F:A-\mathbf{mod}_1\to\mathbf{Ban}_1$ / $-\projmodtens{A} F:A-\mathbf{mod}\to\mathbf{Ban}$~$\rangle$ переводит $\langle$~изометрические / топологически инъективные~$\rangle$ $A$-морфизмы в $\langle$~изометрические / топологически инъективные~$\rangle$ операторы.

Снова рассмотрим категорию банаховых пространств как категорию левых банаховых модулей над нулевой алгеброй, тогда
мы получим определения $\langle$~метрически / топологически~$\rangle$ плоского банахова пространства. Из работы Гротендика \cite{GrothMetrProjFlatBanSp} следует, что любое метрически плоское банахово пространство изометрически изоморфно $L_1(\Omega,\mu)$ для некоторого пространства с мерой $(\Omega,\Sigma,\mu)$. Для топологически плоских банаховых пространств, в отличие от топологически инъективных, мы также имеем критерий [\cite{StegRethNucOpL1LInfSp}, теорема V.1]: банахово пространство топологически плоское тогда и только тогда, когда оно является $\mathscr{L}_1$-пространством.

Хорошо известно, что $A$-модуль $F$ относительно плоский тогда и только тогда, когда $F^*$ относительно инъективный [\cite{HelBanLocConvAlg}, теорема 7.1.42]. Следующее предложение есть очевидный аналог данного результата.

\begin{proposition}\label{MetTopFlatCharac} $A$-модуль $F$ $\langle$~метрически / топологически~$\rangle$ плоский тогда и только тогда, когда $F^*$ $\langle$~метрически / топологически~$\rangle$ инъективен.
\end{proposition}
\begin{proof} Рассмотрим произвольный $\langle$~изометрический / топологически инъективный~$\rangle$ морфизм правых $A$-модулей, скажем, $\xi:X\to Y$. Оператор $\xi\projmodtens{A} 1_F$ $\langle$~изометричен / топологически инъективен~$\rangle$ тогда и только тогда, когда его сопряженный оператор $(\xi\projmodtens{A} 1_F)^*$ $\langle$~строго коизометричен / топологически сюръективен~$\rangle$  [\cite{HelLectAndExOnFuncAn}, упражнения 4.4.6, 4.4.7]. Так как операторы $(\xi\projmodtens{A} 1_F)^*$ и $\mathcal{B}_A(\xi,F^*)$ эквивалентны в $\mathbf{Ban}_1$ посредством универсального свойства модульного проективного тензорного произведения, то $\xi\projmodtens{A} 1_F$ $\langle$~изометричен / топологически инъективен~$\rangle$ тогда и только тогда, когда $\mathcal{B}_A(\xi,F^*)$ $\langle$~строго коизометричен / топологически сюръективен~$\rangle$. Поскольку морфизм $\xi$ произволен, мы видим, что $F$  $\langle$~метрически / топологически~$\rangle$ плоский тогда и только тогда, когда $F^*$  $\langle$~метрически / топологически~$\rangle$ инъективный.
\end{proof}

Комбинируя предложение \ref{MetTopFlatCharac} с предложениями \ref{RetrMetTopInjIsMetTopInj} и \ref{MetInjIsTopInjAndTopInjIsRelInj}, мы получаем следующие два факта.

\begin{proposition}\label{RetrMetTopFlatIsMetTopFlat} Всякий ретракт $\langle$~метрически / топологически~$\rangle$ плоского модуля в $\langle$~$A-\mathbf{mod}_1$ / $A-\mathbf{mod}$~$\rangle$ снова $\langle$~метрически / топологически~$\rangle$ плоский.
\end{proposition}

\begin{proposition}\label{MetFlatIsTopFlatAndTopFlatIsRelFlat} Каждый метрически плоский модуль топологический плоский, и каждый топологически плоский модуль относительно плоский.
\end{proposition}

Количественный аналог определения топологической плоскости был дан Уайтом. Его определение содержало ошибку, к счастью, не повлиявшую на основные результаты. Мы берем на себя ответственность исправить эту ошибку.

\begin{definition}[\cite{WhiteInjmoduAlg}, определение 4.8]\label{CTopFlatMod} $A$-модуль $F$ называется $C$-топологически плоским, если для каждого $c$-топологически инъективного $A$-морфизма $\xi:X\to Y$ правых $A$-модулей оператор $\xi\projmodtens{A} 1_F:X\projmodtens{A} F\to Y\projmodtens{A} F$ $cC$-топологически инъективен.
\end{definition}

Ключевым для нас будет следующий факт.

\begin{proposition}[\cite{WhiteInjmoduAlg}, лемма 4.10]\label{CTopFlatCharac} $A$-модуль $F$ является $C$-топологически плоским тогда и только тогда, когда $F^*$ $C$-топологически инъективен.
\end{proposition}

Как следствие, банахов модуль топологически плоский тогда и только тогда, когда он $C$-топологически плоский для некоторого $C$. Далее мы будем использовать определения \ref{TopFlatMod} и \ref{CTopFlatMod} без ссылки на их эквивалентность. Из предложений \ref{CTopFlatCharac} и \ref{RetrCTopInjIsCTopInj} мы получаем еще одно полезное предложение.

\begin{proposition}\label{RetrCTopFlatIsCTopFlat} Всякий $C_1$-ретракт $C_2$-топологически плоского модуля является $C_1C_2$-топологически плоским.
\end{proposition}

\begin{proposition}\label{DualMetTopProjIsMetrInj} Пусть $P$ --- $\langle$~метрически / топологически~$\rangle$ проективный $A$-модуль, и $\Lambda$ --- произвольное множество. Тогда $A$-модуль $\mathcal{B}(P,\ell_\infty(\Lambda))$ $\langle$~метрически / топологически~$\rangle$ инъективен как $A$-модуль. В частности, $P^*$ $\langle$~метрически / топологически~$\rangle$ инъективен как $A$-модуль.
\end{proposition}
\begin{proof} Из предложения \ref{MetTopProjModViaCanonicMorph} мы знаем, что $\pi_P^+$ --- ретракция в $\langle$~$A-\mathbf{mod}_1$ / $A-\mathbf{mod}$~$\rangle$. Тогда $A$-морфизм $\rho^+=\mathcal{B}(\pi_P^+,\ell_\infty(\Lambda))$ есть коретракция в $\langle$~$\mathbf{mod}_1-A$ / $\mathbf{mod}-A$~$\rangle$. Заметим, что $\mathcal{B}(A_+\projtens\ell_1(B_P),\ell_\infty(\Lambda))\isom{\mathbf{mod}_1-A}\mathcal{B}(A_+,\mathcal{B}(\ell_1(B_P),\ell_\infty(\Lambda)))\isom{\mathbf{mod}_1-A}\mathcal{B}(A_+,\ell_\infty(B_P\times\Lambda))$. Итак, мы показали, что существует коретракция из $\mathcal{B}(P,\ell_\infty(\Lambda))$ в $\langle$~метрически / топологически~$\rangle$ инъективный $A$-модуль. По предложению \ref{RetrMetTopInjIsMetTopInj} банахов $A$-модуль $\mathcal{B}(P,\ell_\infty(\Lambda))$ является $\langle$~метрически / топологически~$\rangle$ инъективным. Чтобы доказать последнее утверждение достаточно положить $\Lambda=\mathbb{N}_1$.
\end{proof}

Как следствие предложений \ref{MetTopFlatCharac} и \ref{DualMetTopProjIsMetrInj}, мы получаем следующее.

\begin{proposition}\label{MetTopProjIsMetTopFlat} Каждый $\langle$~метрически / топологически~$\rangle$ проективный модуль является $\langle$~метрически / топологически~$\rangle$ плоским.
\end{proposition}

Позже мы убедимся, что $\langle$~метрическая / топологическая~$\rangle$ плоскость --- это более слабое свойство, чем $\langle$~метрическая / топологическая~$\rangle$ проективность.

\begin{proposition}\label{MetTopFlatModCoProd} Пусть $(F_\lambda)_{\lambda\in\Lambda}$ --- семейство банаховых $A$-модулей. Тогда: 

$i)$ $A$-модуль $\bigoplus_1\{F_\lambda:\lambda\in\Lambda\}$ метрически плоский тогда и только тогда, когда для всех $\lambda\in\Lambda$ банахов $A$-модуль $F_\lambda$ метрически плоский;

$ii)$ $A$-модуль $\bigoplus_1\{F_\lambda:\lambda\in\Lambda\}$ $C$-топологически плоский тогда и только тогда, когда для всех $\lambda\in\Lambda$ банахов $A$-модуль $F_\lambda$ $C$-топологически плоский.
\end{proposition}
\begin{proof} По предложению  $\langle$~\ref{MetTopFlatCharac} / \ref{CTopFlatCharac}~$\rangle$ $A$-модуль $F$ $\langle$~метрически / $C$-топологически~$\rangle$ плоский тогда и только тогда, когда $F^*$ $\langle$~метрически / $C$-топологически~$\rangle$ инъективен. Осталось применить предложение \ref{MetTopInjModProd} с $J_\lambda=F_\lambda^*$ для всех $\lambda\in\Lambda$ и вспомнить, что $\left(\bigoplus_1\{ F_\lambda:\lambda\in\Lambda\}\right)^*\isom{\mathbf{mod}_1-A}\bigoplus_\infty\{ F_\lambda^*:\lambda\in\Lambda\}$.
\end{proof}

%----------------------------------------------------------------------------------------
%	Metric and topological flatness of ideals and cyclic modules
%----------------------------------------------------------------------------------------

\subsection{Метрическая и топологическая плоскость идеалов и циклических модулей}
\label{SubSectionMetricAndTopologicalFlatnessOfIdealsAndCyclicModules}

В этом параграфе мы обсудим условия, при которых идеалы и циклические модули будут метрически и топологически плоскими. Доказательства во многом схожи с подходами использованными при изучении относительной плоскости идеалов и циклических модулей.

\begin{proposition}\label{MetTopFlatIdealsInUnitalAlg} Пусть $I$ --- левый идеал в $A_\times $ и $I$ имеет правую $\langle$~сжимающую / ограниченную~$\rangle$ аппроксимативную единицу. Тогда $A$-модуль $I$ $\langle$~метрически / топологически~$\rangle$ плоский.
\end{proposition}
\begin{proof} Пусть $\xi:X\to Y$ --- $\langle$~изометрический / топологически инъективный~$\rangle$ морфизм правых $A$-модулей. Так как $I$ имеет правую $\langle$~сжимающую / ограниченную~$\rangle$ аппроксимативную единицу, то из [\cite{HelBanLocConvAlg}, предложение 6.3.24] следует, что линейные операторы $i_{X,I}:X\projmodtens{A} I\to \operatorname{cl}_X(XI):x\projmodtens{A} a\mapsto x\cdot a$, $i_{Y,I}:Y\projmodtens{A} I\to \operatorname{cl}_Y(YI):y\projmodtens{A} a\mapsto y\cdot a$ суть $\langle$~изометрические изоморфизмы / топологические изоморфизмы~$\rangle$ банаховых пространств. Очевидно, оператор $i_0=i_{Y,I}(\xi\projmodtens{A} 1_I)i_{X,I}^{-1}$, поточечно совпадает с $\xi$. Следовательно, $i_0$ --- $\langle$~изометрически / топологически инъективный~$\rangle$ оператор и таковым будет $\xi\projmodtens{A} 1_I$, потому что он изометрически эквивалентен $i_0$. Поскольку морфизм $\xi$ произволен, $A$-модуль $I$ $\langle$~метрически / топологически~$\rangle$ плоский. 
\end{proof}

Отметим, что такое же достаточное условие относительной плоскости идеалов есть и в относительной теории [\cite{HelBanLocConvAlg}, предложение 7.1.45]. Теперь мы можем дать пример метрически плоского модуля, который не является даже топологически проективным. Очевидно, $\ell_\infty(\mathbb{N})$-модуль $c_0(\mathbb{N})$ не унитален как идеал алгебры $\ell_\infty(\mathbb{N})$, но имеет сжимающую аппроксимативную единицу. По теореме \ref{GoodCommIdealMetTopProjIsUnital} этот модуль не является топологически проективным, но он метрически плоский по предложению \ref{MetTopFlatIdealsInUnitalAlg}.

``Метрическая'' часть следующего предложения есть небольшая модификация и восполнение пробелов в [\cite{WhiteInjmoduAlg}, предложение 4.11]. Случай топологической плоскости идеалов был изучен Хелемским в [\cite{HelHomolBanTopAlg}, теорема VI.1.20].

\begin{proposition}\label{MetTopFlatCycModCharac} Пусть $I$ --- левый собственный идеал в $A_\times $. Тогда следующие условия эквивалентны:

$i)$ $A$-модуль $A_\times /I$ $\langle$~метрически / топологически~$\rangle$ плоский;

$ii)$ $I$ имеет правую ограниченную аппроксимативную единицу $(e_\nu)_{\nu\in N}$ $\langle$~такую, что $\sup_{\nu\in N}\Vert e_{A_\times }-e_\nu\Vert\leq 1$ /~$\rangle$.
\end{proposition}
\begin{proof} $i)$$\implies$$ ii)$  Так как $A_\times /I$ $\langle$~метрически / топологически~$\rangle$ плоский, то по предложению \ref{MetTopFlatCharac} правый $A$-модуль $(A_\times /I)^*$ является $\langle$~метрически / топологически~$\rangle$ инъективным. Пусть $\pi:A_\times \to A_\times /I$ --- естественная проекция, тогда $\pi^*:(A_\times /I)^*\to A_\times ^*$ является изометрией. Поскольку $(A_\times /I)^*$ $\langle$~метрически / топологически~$\rangle$ инъективен, оператор $\pi^*$ --- коретракция, то есть существует $\langle$~строго коизометрический / топологически сюръективный~$\rangle$ $A$-морфизм $\tau:A_\times ^*\to (A_\times /I)^*$ такой, что $\tau\pi^*=1_{(A_\times /I)^*}$. Рассмотрим элемент $p\in A^{**}$ такой, что $\iota_{A_\times }(e_{A_\times })-p=\tau^*(\pi^{**}(\iota_{A_\times }(e_{A_\times })))$. Рассмотрим произвольный $f\in I^\perp$. Так как $I^\perp=\pi^*((A_\times /I)^*)$, то существует $g\in (A_\times /I)^*$ со свойством $f=\pi^*(g)$. Значит,
$$
(\iota_{A_\times }(e_{A_\times })-p)(f)
=\tau^*(\pi^{**}(\iota_{A_\times }(e_{A_\times })))(\pi^*(g))
=\pi^{**}(\iota_{A_\times }(e_{A_\times }))(\tau(\pi^*(g)))
$$
$$
=\pi^{**}(\iota_{A_\times }(e_{A_\times }))(g)
=\iota_{A_\times }(e_{A_\times })(\pi^*(g))
=\iota_{A_\times }(e_{A_\times })(f).
$$
Следовательно, $p(f)=0$ для всех $f\in I^\perp$, то есть $p\in I^{\perp\perp}$. Напомним, что $I^{\perp\perp}$ есть слабое${}^*$ замыкание $I$ в $A^{**}$, поэтому существует направленность $(e_\nu'')_{\nu\in N''}\subset I$ такая, что $(\iota_I(e_\nu''))_{\nu\in N''}$ сходится к $p$ в слабой${}^*$ топологии. Очевидно, $(\iota_{A_\times }(e_{A_\times }-e_\nu''))_{\nu\in N''}$ сходится к  $\iota_{A_\times }(e_{A_\times })-p$ в этой же топологии. Из [\cite{PosAndApproxIdinBanAlg}, лемма 1.1] следует, что существует направленность в выпуклой оболочке $\operatorname{conv}(\iota_{A_\times }(e_{A_\times }-e_\nu''))_{\nu\in N''}=\iota_{A_\times }(e_{A_\times })-\operatorname{conv}(\iota_{A_\times }(e_\nu''))_{\nu\in N''}$ которая слабо${}^*$ сходится к $\iota_{A_\times }(e_{A_\times })-p$ и ограничена по норме числом $\Vert \iota_{A_\times }(e_{A_\times })-p\Vert$. Обозначим эту направленность как $(\iota_{A_\times }(e_{A_\times })-\iota_{A_\times }(e_\nu'))_{\nu\in N'}$, тогда $(\iota_{A_\times }(e_\nu'))_{\nu\in N'}$ слабо${}^*$ сходится к $p$. Для любого $a\in I$ и $f\in I^*$ мы имеем
$$
\lim_{\nu}f(ae_\nu')
=\lim_{\nu}\iota_{A_\times }(e_\nu')(f\cdot a)
=p(f\cdot a)
=\iota_{A_\times }(e_{A_\times })(f\cdot a)-\tau^*(\pi^{**}(\iota_{A_\times }(e_{A_\times })))(f\cdot a)
$$
$$
=f(a)-\iota_{A_\times }(e_{A_\times })(\pi^*(\tau(f\cdot a)))
=f(a)-\pi^*(\tau(f)\cdot a)(e_{A_\times })
=f(a)-\tau(f)(\pi(a))
=f(a),
$$
поэтому $(e_\nu')_{\nu\in N'}$ --- правая слабая ограниченная аппроксимативная единица для $I$. Из [\cite{AppIdAndFactorInBanAlg}, предложение 33.2] мы получаем, что существует направленность $(e_\nu)_{\nu\in N}\subset\operatorname{conv}(e_\nu')_{\nu\in N'}$ которая является правой ограниченной аппроксимативной единицей для $I$. Для любого $\nu\in N$ мы имеем $e_{A_\times }-e_\nu\in\operatorname{conv}(e_{A_\times }-e_\nu')_{\nu\in N'}$, поэтому учитывая ограничение на нормы элементов направленности $(\iota_{A_\times }(e_{A_\times }-e_\nu'))_{\nu\in N'}$ мы получаем
$$
\sup_{\nu\in N}\Vert e_{A_\times }-e_\nu\Vert
\leq\Vert \iota_{A_\times }(e_{A_\times })-p\Vert
\leq\Vert\tau^*(\pi^{**}(\iota_{A_\times }(e_{A_\times })))\Vert
\leq\Vert\tau^*\Vert\Vert\pi^{**}\Vert\Vert\iota_{A_\times }(e_{A_\times })\Vert=\Vert\tau\Vert.
$$
Так как $\tau$ $\langle$~сжимающий / ограниченный~$\rangle$ морфизм, то мы получаем желаемую оценку. По построению, $(e_\nu)_{\nu\in N}$ --- правая ограниченная аппроксимативная единица для $I$.

$ii)$$\implies$$ i)$ Обозначим $C=\sup_{\nu\in N}\Vert e_{A_\times }-e_\nu\Vert$. Пусть $\mathfrak{F}$ --- фильтр сечений на $N$ и пусть $\mathfrak{U}$ --- ультрафильтр содержащий $\mathfrak{F}$. Для фиксированного $f\in A_\times ^*$ и $a\in A_\times $ выполнено $|f(a-a e_\nu)|=|f(a(e_{A_\times }-e_\nu))|\leq\Vert f\Vert\Vert a\Vert\Vert e_{A_\times }-e_\nu\Vert\leq C\Vert f\Vert\Vert a\Vert$, то есть $(f(a-ae_\nu))_{\nu\in N}$ --- ограниченная направленность комплексных чисел. Следовательно, корректно определен предел $\lim_{\mathfrak{U}}f(a-ae_\nu)$ по ультрафильтру $\mathfrak{U}$. Так как $(e_\nu)_{\nu\in N}$ --- правая аппроксимативная единица для $I$ и $\mathfrak{U}$ содержит фильтр сечений, то для всех $a\in I$ выполнено $\lim_{\mathfrak{U}}f(a-ae_\nu)=\lim_{\nu}f(a-ae_\nu)=0$. Таким образом, для каждого $f\in A_\times ^*$ корректно определено отображение $\tau(f):A_\times /I\to \mathbb{C}:a+I\mapsto \lim_{\mathfrak{U}} f(a-ae_\nu)$. Очевидно, это линейный функционал и из неравенств доказанных выше следует, что его норма не превосходит $C\Vert f\Vert$. Теперь легко проверить, что $\tau:A_\times ^*\to (A_\times /I)^*:f\mapsto \tau(f)$ есть $\langle$~сжимающий / ограниченный~$\rangle$ $A$-морфизм. Для всех $g\in(A_\times /I)^*$ и $a+I\in A_\times /I$ имеем
$$
\tau(\pi^*(g))(a+I)
=\lim_{\mathfrak{U}}\pi^*(g)(a-ae_\nu)
=\lim_{\mathfrak{U}} g(\pi(a-ae_\nu))
=\lim_{\mathfrak{U}} g(a+I)
=g(a+I),
$$
то есть $\tau:A_\times ^*\to (A_\times /I)^*$ --- ретракция. Правый $A$-модуль $A_\times ^*$ $\langle$~метрически / топологически~$\rangle$ инъективен по предложению \ref{DualOfUnitalAlgIsMetTopInj}, поэтому его ретракт $(A_\times /I)^*$ также $\langle$~метрически / топологически~$\rangle$ инъективен. Теперь предложение \ref{MetTopFlatCharac} гарантирует $\langle$~метрическую / топологическую~$\rangle$ плоскость $A$-модуля $A_\times /I$.
\end{proof}

Следует сказать, что всякая операторная алгебра $A$ (не обязательно самосопряженная) обладающая сжимающей аппроксимативной единицей имеет сжимающую аппроксимативную единицу $(e_\nu)_{\nu\in N}$ со свойством $\sup_{\nu\in N}\Vert e_{A_\#}-e_\nu\Vert\leq 1$ и даже $\sup_{\nu\in N}\Vert e_{A_\#}-2e_\nu\Vert\leq 1$. Здесь $A_\#$ --- унитизация $A$ как операторной алгебры. Подробности можно найти в \cite{PosAndApproxIdinBanAlg}, \cite{BleContrAppIdInOpAlg}.

Снова мы попробуем сравнить наши результаты о метрической и топологической плоскости циклических модулей с их относительными аналогами. Хелемский и Шейнберг показали [\cite{HelHomolBanTopAlg}, теорема VII.1.20], что циклический модуль будет относительно плоским если $I$ имеет правую ограниченную аппроксимативную единицу. В случае когда $I^\perp$ дополняемо в $A_\times^*$ верна и обратная импликация. В топологической теории это требование излишне, поэтому удается получить критерий. Метрическая плоскость циклических модулей слишком сильное свойство из-за специфических ограничений на норму аппроксимативной единицы. Как мы увидим в следующем параграфе, оно настолько ограничительное, что не позволяет построить ни одного ненулевого аннуляторного метрически проективного, инъективного или плоского модуля над ненулевой банаховой алгеброй.

%----------------------------------------------------------------------------------------
%	The impact of Banach geometry
%----------------------------------------------------------------------------------------

\section{Влияние банаховой геометрии}
\label{SectionTheImpactOfBanachGeometry}


%----------------------------------------------------------------------------------------
%	Homologically trivial annihilator modules
%----------------------------------------------------------------------------------------

\subsection{Гомологически тривиальные аннуляторные модули}
\label{SubSectionHomoligicallyTrivialAnnihilatorModules}

В этом параграфе мы сконцентрируем наше внимание на метрической и топологической проективности, инъективности и плоскости аннуляторных модулей. Если не оговорено иначе, все банаховы пространства в этом параграфе рассматриваются как аннуляторные модули. Отметим очевидный факт: всякий ограниченный линейный оператор между аннуляторными $A$-модулями является $A$-морфизмом.

\begin{proposition}\label{AnnihCModIsRetAnnihMod} Пусть $X$ --- ненулевой аннуляторный $A$-модуль. Тогда $\mathbb{C}$ есть ретракт $X$ в $A-\mathbf{mod}_1$.
\end{proposition}
\begin{proof} Рассмотрим произвольный вектор $x_0\in X$ нормы $1$. Используя теорему Хана-Банаха выберем функционал $f_0\in X^*$ так, чтобы $\Vert f_0\Vert=f_0(x_0)=1$. Рассмотрим линейные операторы $\pi:X\to \mathbb{C}:x\mapsto f_0(x)$, $\sigma:\mathbb{C}\to X:z\mapsto zx_0$. Легко проверить, что $\pi$ и $\sigma$ суть сжимающие $A$-морфизмы и, более того, $\pi\sigma=1_\mathbb{C}$. Другими словами, $\mathbb{C}$ есть ретракт $X$ в $A-\mathbf{mod}_1$.
\end{proof}

Пришло время вспомнить, что любая банахова алгебра $A$ может рассматриваться как собственный максимальный идеал в $A_+$, причем $\mathbb{C}\isom{A-\mathbf{mod}_1} A_+/A$. Если рассматривать $\mathbb{C}$ как правый аннуляторный $A$-модуль, то имеет место еще один изоморфизм  $\mathbb{C}\isom{\mathbf{mod}_1-A}(A_+/A)^*$. 

\begin{proposition}\label{MetTopProjModCCharac} Аннуляторный $A$-модуль $\mathbb{C}$ $\langle$~метрически / топологически~$\rangle$ проективен тогда и только тогда, когда $\langle$~$A=\{0\}$ / $A$ имеет правую единицу~$\rangle$.
\end{proposition}
\begin{proof} 
Достаточно исследовать $\langle$~метрическую / топологическую~$\rangle$ проективность модуля $A_+/A$. Естественное фактор-отображение $\pi:A_+\to A_+/A$ является строгой коизометрией, поэтому по предложению \ref{MetTopProjCycModCharac} $\langle$~метрическая / топологическая~$\rangle$ проективность $A_+/A$ эквивалентна существованию $p\in A$ такого, что $A=A_+p$ $\langle$~и $\Vert e_{A_+}-p\Vert=1$ /~$\rangle$. $\langle$~Осталось заметить, что $\Vert e_{A_+}-p\Vert=1$ тогда и только тогда, когда $p=0$, что эквивалентно $A=A_+p=\{0\}$ /~$\rangle$.
\end{proof}

\begin{proposition}\label{MetTopProjOfAnnihModCharac} Пусть $P$ --- ненулевой аннуляторный $A$-модуль. Тогда следующие условия эквивалентны:

$i)$ $P$ --- $\langle$~метрически / топологически~$\rangle$ проективный $A$-модуль;

$ii)$ $\langle$~$A=\{0\}$ / $A$ имеет правую единицу~$\rangle$ и $P$ --- $\langle$~метрически / топологически~$\rangle$ проективное банахово пространство, то есть $\langle$~$P\isom{\mathbf{Ban}_1}\ell_1(\Lambda)$ / $P\isom{\mathbf{Ban}}\ell_1(\Lambda)$~$\rangle$ для некоторого множества $\Lambda$.
\end{proposition}
\begin{proof} $i)$$\implies$$ ii)$ Из предложений \ref{RetrMetTopProjIsMetTopProj} и \ref{AnnihCModIsRetAnnihMod} следует, что $A$-модуль $\mathbb{C}$ $\langle$~метрически / топологически~$\rangle$ проективен как ретракт $\langle$~метрически / топологически~$\rangle$ проективного модуля $P$. Предложение \ref{MetTopProjModCCharac} дает, что $\langle$~$A=\{0\}$ / $A$ имеет правую единицу~$\rangle$.  По следствию \ref{MetTopProjTensProdWithl1} аннуляторный $A$-модуль $\mathbb{C}\projtens\ell_1(B_P)\isom{A-\mathbf{mod}_1}\ell_1(B_P)$ $\langle$~метрически / топологически~$\rangle$ проективен. Рассмотрим строгую коизометрию $\pi:\ell_1(B_P)\to P$ корректно определенную равенством $\pi(\delta_x)=x$. Поскольку $P$ и $\ell_1(B_P)$ --- аннуляторные модули, то $\pi$ --- $A$-морфизм. Так как $P$ $\langle$~метрически / топологически~$\rangle$ проективен, то $A$-морфизм $\pi$ имеет правый обратный морфизм $\sigma$ в $\langle$~$A-\mathbf{mod}_1$ / $A-\mathbf{mod}$~$\rangle$. Таким образом, $\sigma\pi$ есть $\langle$~сжимающий / ограниченный~$\rangle$ проектор из $\langle$~метрически / топологически~$\rangle$ проективного банахова пространства $\ell_1(B_P)$ на $P$, то есть $P$ --- $\langle$~метрически / топологически~$\rangle$ проективное банахово пространство. Теперь из $\langle$~[\cite{HelMetrFrQMod}, предложение 3.2] / результатов \cite{KotheTopProjBanSp}~$\rangle$ следует, что пространство $P$ $\langle$~метрически / топологически~$\rangle$ изоморфно $\ell_1(\Lambda)$ для некоторого множества $\Lambda$. 

$ii)$$\implies$$ i)$ По предложению \ref{MetTopProjModCCharac} аннуляторный $A$-модуль $\mathbb{C}$ $\langle$~метрически / топологически~$\rangle$ проективен. По следствию \ref{MetTopProjTensProdWithl1} аннуляторный $A$-модуль $\mathbb{C}\projtens\ell_1(\Lambda)\isom{A-\mathbf{mod}_1}\ell_1(\Lambda)$ также $\langle$~метрически / топологически~$\rangle$ проективен.
\end{proof}

\begin{proposition}\label{MetTopInjModCCharac} Правый аннуляторный $A$-модуль $\mathbb{C}$ $\langle$~метрически / топологически~$\rangle$ инъективен тогда и только тогда, когда $\langle$~$A=\{0\}$ / $A$  имеет правую ограниченную аппроксимативную единицу~$\rangle$.
\end{proposition}
\begin{proof} Благодаря предложению \ref{MetTopFlatCharac} достаточно изучить $\langle$~метрическую / топологическую~$\rangle$ плоскость модуля $A_+/A$. По предложению \ref{MetTopFlatCycModCharac} это эквивалентно существованию правой ограниченной аппроксимативной единицы $(e_\nu)_{\nu\in N}$ в $A$ $\langle$~со свойством $\sup_{\nu\in N}\Vert e_{A_+}-e_\nu\Vert\leq 1$ /~$\rangle$. $\langle$~Осталось заметить, что $\Vert e_{A_+}-e_\nu\Vert\leq 1$ тогда и только тогда, когда $e_\nu=0$, что эквивалентно $A=\{0\}$ /~$\rangle$.
\end{proof}

\begin{proposition}\label{MetTopInjOfAnnihModCharac} Пусть $J$ --- ненулевой правый аннуляторный $A$-модуль. Тогда следующие условия эквивалентны:

$i)$ $J$ --- $\langle$~метрически / топологически~$\rangle$ инъективный $A$-модуль;

$ii)$ $\langle$~$A=\{0\}$ / $A$ имеет правую ограниченную аппроксимативную единицу~$\rangle$ и $J$ ---  $\langle$~метрически / топологически~$\rangle$ инъективное банахово пространство $\langle$~то есть $J\isom{\mathbf{Ban}_1}C(K)$ для некоторого для стоунова пространства $K$ /~$\rangle$.
\end{proposition}
\begin{proof} $i)$$\implies$$ ii)$  Из предложений \ref{RetrMetTopInjIsMetTopInj} и \ref{AnnihCModIsRetAnnihMod} мы получаем, что $A$-модуль $\mathbb{C}$ $\langle$~метрически / топологически~$\rangle$ инъективен как ретракт $\langle$~метрически / топологически~$\rangle$ инъективного модуля $J$. Предложение \ref{MetTopInjModCCharac} дает нам, что $\langle$~$A=\{0\}$ / $A$ имеет правую ограниченную аппроксимативную единицу~$\rangle$. По предложению \ref{MapsFroml1toMetTopInj} аннуляторный $A$-модуль $\mathcal{B}(\ell_1(B_{J^*}),\mathbb{C})\isom{\mathbf{mod}_1-A}\ell_\infty(B_{J^*})$ $\langle$~метрически / топологически~$\rangle$ инъективен. Рассмотрим изометрию $\rho:J\to\ell_\infty(B_{J^*})$ корректно определенную равенством $\rho(x)(f)=f(x)$. Так как $J$ и $\ell_\infty(B_{J^*})$ --- аннуляторные модули, то $\rho$ является $A$-морфизмом. Поскольку $J$ $\langle$~метрически / топологически~$\rangle$ инъективен, $\rho$ имеет левый обратный морфизм $\tau$ в $\langle$~$\mathbf{mod}_1-A$ / $\mathbf{mod}-A$~$\rangle$. Тогда $\rho\tau$ --- $\langle$~сжимающий / ограниченный~$\rangle$ проектор из $\langle$~метрически / топологически~$\rangle$ инъективного банахова пространства $\ell_\infty(B_{J^*})$ на $J$, поэтому $J$ также является $\langle$~метрически / топологически~$\rangle$ инъективным банаховым пространством. $\langle$~Из [\cite{LaceyIsomThOfClassicBanSp}, теорема 3.11.6] мы знаем, что $J$ изометрически изоморфно $C(K)$ для некоторого стоунова пространства $K$. /~$\rangle$ 

$ii)$$\implies$$ i)$ По предложению \ref{MetTopInjModCCharac} аннуляторный $A$-модуль $\mathbb{C}$ $\langle$~метрически / топологически~$\rangle$ инъективен. По предложению \ref{MapsFroml1toMetTopInj} аннуляторный $A$-модуль $\mathcal{B}(\ell_1(B_{J^*}),\mathbb{C})\isom{\mathbf{mod}_1-A}\ell_\infty(B_{J^*})$ также $\langle$~метрически / топологически~$\rangle$ инъективен. Так как $J$ --- $\langle$~метрически / топологически~$\rangle$ инъективное банахово пространство и существует изометрическое вложение $\rho:J\to \ell_\infty(B_{J^*})$, то $J$ является $\langle$~$1$-ретрактом / ретрактом~$\rangle$ пространства $\ell_\infty(B_{J^*})$. Напомним, что $J$ и $\ell_\infty(B_{J^*})$ аннуляторные модули, поэтому данная ретракция также является ретракцией в $\langle$~$\mathbf{mod}_1-A$ / $\mathbf{mod}-A$~$\rangle$. По предложению \ref{RetrMetTopInjIsMetTopInj} $A$-модуль $J$ $\langle$~метрически / топологически~$\rangle$ инъективен.
\end{proof}

\begin{proposition}\label{MetTopFlatAnnihModCharac} Пусть $F$ --- ненулевой аннуляторный $A$-модуль. Тогда следующие условия эквивалентны:

$i)$ $F$ --- $\langle$~метрически / топологически~$\rangle$ плоский $A$-модуль;

$ii)$ $\langle$~$A=\{0\}$ / $A$ имеет правую ограниченную аппроксимативную единицу~$\rangle$ и $F$ --- $\langle$~метрически / топологически~$\rangle$ плоское банахово пространство, то есть $\langle$~$F\isom{\mathbf{Ban}_1}L_1(\Omega,\mu)$ для некоторого пространства с мерой $(\Omega, \Sigma, \mu)$ / $F$ есть $\mathscr{L}_1$-пространство~$\rangle$.
\end{proposition}
\begin{proof} Из $\langle$~[\cite{GrothMetrProjFlatBanSp}, теорема 1] / [\cite{StegRethNucOpL1LInfSp}, теорема VI.6]~$\rangle$ мы знаем, что банахово пространство $F^*$ $\langle$~метрически / топологически~$\rangle$ инъективно тогда и только тогда, когда $\langle$~$F\isom{\mathbf{Ban}_1}L_1(\Omega,\mu)$ для некоторого пространства с мерой $(\Omega, \Sigma, \mu)$ / $F$ есть $\mathscr{L}_1$-пространство~$\rangle$. Теперь эквивалентность следует из предложений \ref{MetTopInjOfAnnihModCharac} и \ref{MetTopFlatCharac}.
\end{proof}

Следует сравнить эти результаты с аналогичными в относительной теории. Из $\langle$~[\cite{RamsHomPropSemgroupAlg}, предложение 2.1.7] / [\cite{RamsHomPropSemgroupAlg}, предложение 2.1.10]~$\rangle$ мы знаем, что аннуляторный модуль над банаховой алгеброй $A$  относительно $\langle$~проективный / плоский~$\rangle$ тогда и только тогда, когда $A$ имеет  $\langle$~правую единицу / правую ограниченную аппроксимативную единицу~$\rangle$. В метрической и топологической теории, в отличие от относительной, гомологическая тривиальность аннуляторных модулей налагает ограничения не только на алгебру, но и на геометрию самого модуля. Эти геометрические ограничения запрещают существование некоторых гомологически лучших банаховых алгебр. Одно из важных свойств относительно $\langle$~стягиваемых / аменабельных~$\rangle$ банаховых алгебр --- это $\langle$~проективность / плоскость~$\rangle$ всех (и в частности аннуляторных) левых банаховых модулей над ней. Резкое отличие метрической и топологической теории в том, что в них подобных алгебр не может быть.

\begin{proposition} Не существует банаховой алгебры $A$ такой, что все $A$-модули $\langle$~метрически / топологически~$\rangle$ плоские. Тем более, не существует банаховых алгебр таких, что все $A$-модули $\langle$~метрически / топологически~$\rangle$ проективны.
\end{proposition}
\begin{proof} Рассмотрим бесконечномерное $\mathscr{L}_\infty$-пространство $X$ (например $\ell_\infty(\mathbb{N})$) как аннуляторный $A$-модуль. Из [\cite{DefFloTensNorOpId}, параграф 23.3] мы знаем, что $X$ не является $\mathscr{L}_1$-пространством. Следовательно, по предложению \ref{MetTopFlatAnnihModCharac} модуль $X$ не является топологически плоским. По предложению \ref{MetFlatIsTopFlatAndTopFlatIsRelFlat} он также и не метрически плоский. Наконец, из предложения \ref{MetTopProjIsMetTopFlat} следует, что $X$ не является ни метрически, ни топологически проективным.
\end{proof}

%----------------------------------------------------------------------------------------
%	Homologically trivial modules over Banach algebras with specific geometry
%----------------------------------------------------------------------------------------

\subsection{Гомологически тривиальные модули над банаховыми алгебрами со специальной геометрией}
\label{SubSectionHomologicallyTrivialModulesOverBanachAlgebrasWithSpecificGeometry}

Цель данного параграфа --- убедиться в том, что гомологически тривиальные модули над некоторыми банаховыми алгебрами имеют с этими алгебрами схожие геометрические свойства.

Снова напомним, что под пространством с мерой мы понимаем строго локализуемое пространство с мерой. В следующем предложении мы несколько раз встретимся с произведением пространств с мерой. Произведение двух пространств с мерой $(\Omega_1,\Sigma_1,\mu_1)$ и $(\Omega_2,\Sigma_2,\mu_2)$ мы будем обозначать $\mu_1\times \mu_2$. Произведение двух строго локализуемых пространств с мерой есть  строго локализуемое пространство с мерой [\cite{FremMeasTh}, предложение 251N]. 

Результаты следующего предложения в случае метрической теории были получены Гравеном в \cite{GravInjProjBanMod}.

\begin{proposition}\label{TopProjInjFlatModOverL1Charac} Пусть $A$ --- банахова алгебра изометрически изоморфная, как банахово пространство, пространству $L_1(\Theta,\nu)$ для некоторого пространства с мерой $(\Theta,\Sigma,\nu)$. Тогда:

$i)$ если $P$ --- $\langle$~метрически / топологически~$\rangle$ проективный $A$-модуль, то $P$ --- $\langle$~$L_1$-пространство / ретракт $L_1$-пространства~$\rangle$;

$ii)$ если $J$ --- $\langle$~метрически / топологически~$\rangle$ инъективный $A$-модуль, то $J$ --- $\langle$~$C(K)$-пространство для некоторого стоунова пространства $K$ / топологически инъективное банахово пространство~$\rangle$;

$iii)$ если $F$ --- $\langle$~метрически / топологически~$\rangle$ плоский $A$-модуль, то $F$ --- $\langle$~$L_1$-пространство / $\mathscr{L}_1$-пространство~$\rangle$.
\end{proposition}
\begin{proof} Через $(\Theta',\Sigma', \nu')$ обозначим пространство с мерой $(\Theta,\Sigma, \nu)$ с одним добавленным атомом, тогда $A_+\isom{\mathbf{Ban}_1} L_1(\Theta',\nu')$.

$i)$ Так как $P$ $\langle$~метрически / топологически~$\rangle$ проективен как $A$-модуль, то по предложению \ref{MetTopProjModViaCanonicMorph} он является ретрактом $A_+\projtens \ell_1(B_P)$ в $\langle$~$A-\mathbf{mod}_1$ / $A-\mathbf{mod}$~$\rangle$. Пусть $\mu_c$ --- считающая мера на $B_P$, тогда по теореме Гротендика [\cite{HelLectAndExOnFuncAn}, теорема 2.7.5]
$$
A_+\projtens\ell_1(B_P)
\isom{\mathbf{Ban}_1}L_1(\Theta',\nu')\projtens L_1(B_P,\mu_c)
\isom{\mathbf{Ban}_1}L_1(\Theta'\times B_P,\nu'\times \mu_c).
$$
Следовательно, $P$ -- $\langle$~$1$-ретракт / ретракт~$\rangle$ $L_1$-пространства. Осталось заметить, что любой $1$-ретракт $L_1$-пространства есть снова $L_1$-пространство [\cite{LaceyIsomThOfClassicBanSp}, теорема 6.17.3].

$ii)$ Так как $J$ $\langle$~метрически / топологически~$\rangle$ инъективный $A$-модуль, то по предложению \ref{MetTopInjModViaCanonicMorph} он является ретрактом $\mathcal{B}(A_+,\ell_\infty(B_{J^*}))$ в $\langle$~$\mathbf{mod}_1-A$ / $\mathbf{mod}-A$~$\rangle$. Пусть $\mu_c$ --- считающая мера на $B_{J^*}$, тогда по теореме Гротендика [\cite{HelLectAndExOnFuncAn}, теорема 2.7.5]
$$
\mathcal{B}(A_+,\ell_\infty(B_{J^*}))
\isom{\mathbf{Ban}_1}(A_+\projtens \ell_1(B_{J^*}))^*
\isom{\mathbf{Ban}_1}(L_1(\Theta',\nu')\projtens L_1(B_P,\mu_c))^*
$$
$$
\isom{\mathbf{Ban}_1}L_1(\Theta'\times B_P,\nu'\times \mu_c)^*
\isom{\mathbf{Ban}_1}L_\infty(\Theta'\times B_P,\nu'\times \mu_c).
$$
Следовательно, $J$ --- $\langle$~$1$-ретракт / ретракт~$\rangle$ $L_\infty$-пространства. Поскольку $L_\infty$-пространства $\langle$~метрически / топологически~$\rangle$ инъективны, то таковы же и их ретракты $J$. Осталось напомнить, что каждое метрически инъективное банахово пространство суть $C(K)$-пространство для некоторого стоунова пространства $K$ [\cite{LaceyIsomThOfClassicBanSp}, теорема 3.11.6].

$iii)$ Из $\langle$~[\cite{GrothMetrProjFlatBanSp}, теорема 1] / [\cite{StegRethNucOpL1LInfSp}, теорема VI.6]~$\rangle$ мы знаем, что банахово пространство $F^*$ $\langle$~метрически / топологически~$\rangle$ инъективно тогда и только тогда, когда $F$ является $\langle$~$L_1$-пространством / $\mathscr{L}_1$-пространством~$\rangle$. Остается применить результаты пункта $ii)$ и предложение \ref{MetTopFlatCharac}.
\end{proof}

\begin{proposition}\label{TopProjInjFlatModOverMthscrL1SpCharac} Пусть $A$ --- банахова алгебра изоморфная, как банахово пространство, некоторому $\mathscr{L}_1$-пространству. Тогда любой топологически $\langle$~проективный / инъективный / плоский~$\rangle$ $A$-модуль является $\langle$~$\mathscr{L}_1$-пространством / $\mathscr{L}_\infty$-пространством / $\mathscr{L}_1$-пространством~$\rangle$.
\end{proposition}
\begin{proof} Если алгебра $A$ есть $\mathscr{L}_1$-пространство, то такова же и $A_+$. 

Пусть $P$ --- топологически проективный $A$-модуль. Тогда по предложению \ref{MetTopProjModViaCanonicMorph} он есть ретракт $A_+\projtens \ell_1(B_P)$ в $A-\mathbf{mod}$ и тем более в $\mathbf{Ban}$. Поскольку $\ell_1(B_P)$ есть $\mathscr{L}_1$-пространство, то таково же $A_+\projtens\ell_1(B_P)$ как проективное тензорное произведение $\mathscr{L}_1$-пространств [\cite{GonzDPPInTensProd}, предложение 1]. Следовательно, $P$ есть $\mathscr{L}_1$-пространство как ретракт $\mathscr{L}_1$-пространства [\cite{BourgNewClOfLpSp}, предложение 1.28].

Пусть $J$ --- топологически инъективный $A$-модуль. Тогда по предложению \ref{MetTopInjModViaCanonicMorph} он есть ретракт $\mathcal{B}(A_+,\ell_\infty(B_{J^*}))\isom{\mathbf{mod}_1-A}(A_+\projtens\ell_1(B_{J^*}))^*$ в $\mathbf{mod}-A$ и тем более в $\mathbf{Ban}$. Как мы показали выше, пространство $A_+\projtens\ell_1(B_{J^*})$ является $\mathscr{L}_1$-пространством, тогда его сопряженное $\mathcal{B}(A_+,\ell_\infty(B_{J^*}))$ есть $\mathscr{L}_\infty$-пространство [\cite{BourgNewClOfLpSp}, предложение 1.27]. Осталось заметить, что любой ретракт $\mathscr{L}_\infty$-пространства есть $\mathscr{L}_\infty$-пространство [\cite{BourgNewClOfLpSp}, предложение 1.28].

Наконец, пусть $F$ --- топологически плоский $A$-модуль, тогда $F^*$ топологически инъективен по предложению \ref{MetTopFlatCharac}. Из предыдущего абзаца следует, что $F^*$ --- это $\mathscr{L}_\infty$-пространство. По теореме VI.6 из \cite{StegRethNucOpL1LInfSp} пространство $F$ является $\mathscr{L}_1$-пространством.
\end{proof}

Перейдем к обсуждению свойства Данфорда-Петтиса для гомологически тривиальных банаховых модулей. Прежде напомним определение и перечислим основные факты о свойстве Данфорда-Петтиса. Ограниченный линейный оператор $T:E\to F$ называется слабо компактным, если он отправляет единичный шар пространства $E$ в относительно слабо компактное подмножество в $F$. Ограниченный линейный оператор называется вполне непрерывным, если образ любого слабо компактного подмножества в $E$ компактен в нормовой топологии $F$. Говорят, что банахово пространство $E$ имеет свойство Данфорда-Петтиса, если любой слабо компактный оператор из $E$ в произвольное банахово пространство $F$ вполне непрерывен. Существует простое внутреннее описание этого свойства [\cite{KalAlbTopicsBanSpTh}, теорема 5.4.4]: банахово пространство $E$ обладает свойством Данфорда-Петтиса если $\lim_n f_n(x_n)=0$ для любых слабо сходящихся к $0$ последовательностей $(x_n)_{n\in\mathbb{N}}\subset E$ и $(f_n)_{n\in\mathbb{N}}\subset E^*$. Теперь легко доказать, что если банахово пространство $E^*$ имеет свойство Данфорда-Петтиса, то его имеет и $E$. В своей новаторской работе \cite{GrothApllFaiblCompSpCK} Гротендик показал, что все $L_1$-пространства и $C(K)$-пространства имеют это свойство. Более того любое $\mathscr{L}_1$-пространство и любое $\mathscr{L}_\infty$-пространство имеет свойство Данфорда-Петтиса  [\cite{BourgNewClOfLpSp}, предложение 1.30]. Поскольку единичный шар рефлексивного банахова пространства слабо компактен [\cite{MeggIntroBanSpTh}, теорема 2.8.2], то у таких пространств со свойством Данфорда-Петтиса единичный шар компактен в нормовой топологии, и следовательно, такие пространства конечномерны. Свойство Данфорда-Петтиса наследуется дополняемыми подпространствами [\cite{FabHabBanSpTh}, предложение 13.44]. 

Ключевым для нас будет результат Бургейна о банаховых пространствах со специальной локальной структурой. В [\cite{BourgOnTheDPP}, теорема 5] он доказал, что первое, второе и так далее сопряженное пространство банахова пространства с $E_p$-локальной структурой обладает свойством Данфорда-Петтиса. Здесь, $E_p$ обозначает класс всех $\bigoplus_\infty$-сумм $p$ копий $p$-мерных $\ell_1$-пространств для некоторого натурального $p$. 

\begin{proposition}\label{C0SumOfL1SpHaveDPP} Пусть $\{L_1(\Omega_\lambda,\mu_\lambda):\lambda\in\Lambda\}$ --- семейство бесконечномерныx $L_1$-пространств. Тогда банахово пространство $\bigoplus_0\{L_1(\Omega_\lambda,\mu_\lambda):\lambda\in\Lambda\}$ имеет $(E_p,1+\epsilon)$-локальную структуру для всех $\epsilon>0$.
\end{proposition}
\begin{proof} Для каждого $\lambda\in\Lambda$ через $L_1^0(\Omega_\lambda,\mu_\lambda)$ обозначим плотное подпространство в $L_1(\Omega_\lambda,\mu_\lambda)$ натянутое на характеристические функции измеримых множеств из $\Sigma_\lambda$. Обозначим $E:=\bigoplus_0\{L_1(\Omega_\lambda,\mu_\lambda):\lambda\in\Lambda\}$, пусть $E_0:=\bigoplus_{00}\{L_1(\Omega_\lambda,\mu_\lambda):\lambda\in\Lambda\}$ --- не обязательно замкнутое подпространство в $E$, состоящее из векторов с конечным числом ненулевых координат.

Зафиксируем $\epsilon>0$ и конечномерное подпространство $F$ в $E$. Так как $F$ конечномерно, то существует ограниченный проектор $Q:E\to E$ на $F$. Выберем $\delta>0$ так, чтобы $\delta\Vert Q\Vert<1$ и $(1+\delta\Vert Q\Vert)(1-\delta\Vert Q\Vert)^{-1}<1+\epsilon$. Заметим, что $B_F$ компактно, потому что $F$ конечномерно. Следовательно, существует конечная $\delta/2$-сеть $(x_k)_{k\in\mathbb{N}_n}\subset E_0$ для $B_F$. Для каждого $k\in\mathbb{N}_n$ имеем $x_k=\bigoplus_0\{x_{k,\lambda}:\lambda\in\Lambda\}$, где $x_{k,\lambda}=\sum_{j=1}^{m_{k,\lambda}}d_{k,j,\lambda}\chi_{D_{j,k,\lambda}}$ для некоторых комплексных чисел $(d_{j,k,\lambda})_{j\in\mathbb{N}_{m_{k,\lambda}}}$ и измеримых множеств $(D_{j,k,\lambda})_{j\in\mathbb{N}_{m_{k,\lambda}}}$ конечной меры. Пусть $(C_{i,\lambda})_{i\in\mathbb{N}_{m_\lambda}}$ --- множество всех попарных пересечений элементов из $(D_{j,k,\lambda})_{j\in\mathbb{N}_{m_{k,\lambda}}}$ исключая множества меры $0$. Тогда $x_{k,\lambda}=\sum_{i=1}^{m_\lambda} c_{i,k,\lambda}\chi_{C_{i,\lambda}}$ для некоторых комплексных чисел $(c_{j,k,\lambda})_{j\in\mathbb{N}_{m_{\lambda}}}$. Обозначим $\Lambda_k=\{\lambda\in\lambda:x_{k,\lambda}\neq 0\}$. По определению пространства $E_0$ множество $\Lambda_k$ конечно для каждого $k\in\mathbb{N}_n$. Рассмотрим конечное множество $\Lambda_0=\bigcup_{k\in\mathbb{N}_n}\Lambda_k$. Так как пространство $L_1(\Omega_\lambda, \mu_\lambda)$ бесконечномерно, то мы можем добавить к семейству $\{\chi_{C_{i,\lambda}}:i\in\mathbb{N}_{m_\lambda}\}$ любое конечное число дизъюнктных множеств положительной конечной меры. Поэтому, далее считаем, что $m_\lambda=\operatorname{Card}(\Lambda_0)$ для всех $\lambda\in\Lambda_0$. Для каждого $\lambda\in\Lambda_0$ корректно определен проектор 
$$
P_\lambda:L_1(\Omega_\lambda,\mu_\lambda)\to L_1(\Omega_\lambda,\mu_\lambda):x_\lambda\mapsto \sum_{i=1}^{m_\lambda}\left( \mu(C_{i,\lambda})^{-1}\int_{C_{i,\lambda}}x_\lambda(\omega)d\mu_\lambda(\omega)\right)\chi_{C_{i,\lambda}}.
$$
Легко проверить, что $P_\lambda(\chi_{C_{i,\lambda}})=\chi_{C_{i,\lambda}}$ для всех $i\in\mathbb{N}_{m_\lambda}$. Следовательно, $P_\lambda(x_{k,\lambda})=x_{k,\lambda}$ для всех $k\in\mathbb{N}_n$. Так как множества $(C_{i,\lambda})_{i\in\mathbb{N}_{m_\lambda}}$ не пересекаются и имеют положительную меру, то $\operatorname{Im}(P_\lambda)\isom{\mathbf{Ban}_1}\ell_1(\mathbb{N}_{m_\lambda})$. Для $\lambda\in\Lambda\setminus\Lambda_0$ мы положим $P_\lambda=0$ и рассмотрим проектор $P:=\bigoplus_0\{P_\lambda:\lambda\in\Lambda\}$. По построению он сжимающий с образом $\operatorname{Im}(P)\isom{\mathbf{Ban}_1}\bigoplus_0\{\ell_1(\mathbb{N}_{m_\lambda}):\lambda\in\Lambda_0\}\in E_{p}$. Рассмотрим произвольный вектор $x\in B_F$, тогда существует номер $k\in\mathbb{N}_n$ такой, что $\Vert x-x_k\Vert\leq \delta/2$. Тогда $\Vert P(x)-x\Vert=\Vert P(x)-P(x_k)+x_k-x\Vert\leq\Vert P\Vert\Vert x-x_k\Vert+\Vert x_k-x\Vert\leq\delta$.

Построив проекторы $P$ и $Q$, рассмотрим оператор $I:=1_E+PQ-Q$. Очевидно, $\Vert 1_E-I\Vert=\Vert PQ-Q\Vert\leq \delta\Vert Q\Vert$. Следовательно $I$ --- изоморфизм по стандартному трюку с рядами фон Нойманна [\cite{KalAlbTopicsBanSpTh}, предложение A.2]. Более того, $I^{-1}=\sum_{p=0}^\infty(1_E-I)^p$, поэтому
$$
\Vert I^{-1}\Vert\leq\sum_{p=0}^\infty\Vert 1_E-I\Vert^p\leq\sum_{p=0}^\infty(\delta\Vert Q\Vert)^p=(1-\delta\Vert Q\Vert)^{-1},
\quad
\Vert I\Vert\leq\Vert 1_E\Vert+\Vert I-1_E\Vert\leq 1+\delta\Vert Q\Vert.
$$
Заметим, что $PI=P+P^2Q-PQ=P+PQ-PQ=P$, поэтому для всех $x\in F$ выполнено 
$$
I(x)=x+P(Q(x))-Q(x)=x+P(x)-x=P(x)=P(P(x))=P(I(x))
$$
и $x=(I^{-1}PI)(x)$. Последнее означает, что $F$ содержится в образе ограниченного проектора $R=I^{-1}PI$. Обозначим этот образ через $F_0$ и рассмотрим биограничение  $I_0=I|_{F_0}^{\operatorname{Im}(P)}$ изоморфизма $I$. Так как $\Vert I_0\Vert\Vert I_0^{-1}\Vert\leq\Vert I\Vert\Vert I^{-1}\Vert\leq(1+\delta\Vert Q\Vert)(1-\delta\Vert Q\Vert)^{-1}<1+\epsilon$, то $d_{BM}(F_0,\operatorname{Im}(P))<1+\epsilon$. Таким образом, мы показали, что для любого конечномерного подпространства в $E$ существует подпространство $F_0$ в $E$ содержащее $F$ такое, что $d_{BM}(F_0,U)<1+\epsilon$ для некоторого $U\in E_{p}$. Это значит, что $E$ имеет $(E_{p}, 1+\epsilon)$-локальную структуру.
\end{proof}

\begin{proposition}\label{ProdOfL1SpHaveDPP} Пусть $\{(\Omega_\lambda,\Sigma_\lambda,\mu_\lambda):\lambda\in\Lambda\}$ --- семейство пространств с мерой. Тогда банахово пространство $\bigoplus_\infty\{L_1(\Omega_\lambda,\mu_\lambda):\lambda\in\Lambda\}$ обладает свойством Данфорда-Петтиса.
\end{proposition}
\begin{proof} Сначала предположим, что пространства $L_1(\Omega_\lambda, \mu_\lambda)$ бесконечномерны для всех $\lambda\in\Lambda$. Из предложения \ref{C0SumOfL1SpHaveDPP} мы знаем, что банахово пространство $F:=\bigoplus_0\{L_1(\Omega_\lambda,\mu_\lambda):\lambda\in\Lambda\}$ имеет $E_{p}$-локальную структуру. Тогда по теореме 5 из \cite{BourgOnTheDPP} первое, второе и так далее сопряженное пространство пространства $F$ обладают свойством Данфорда-Петтиса. Как следствие, мы получаем, что $F^{**}=\left(\bigoplus_0\{L_1(\Omega_\lambda,\mu_\lambda):\lambda\in\Lambda\}\right)^{**}$ $\isom{\mathbf{Ban}_1}\bigoplus_\infty\{L_1(\Omega_\lambda,\mu_\lambda)^{**}:\lambda\in\Lambda\}$ имеет свойство Данфорда-Петтиса. Из [\cite{DefFloTensNorOpId}, предложение B10] мы знаем, что каждое $L_1$-пространство 1-дополняемо в своем втором сопряженном. Для каждого $\lambda\in\Lambda$ через $P_\lambda$ обозначим соответствующий проектор в $L_1(\Omega_\lambda,\mu_\lambda)^{**}$. Таким образом $\bigoplus_\infty\{L_1(\Omega_\lambda,\mu_\lambda):\lambda\in\Lambda\}$ 1-дополняемо в $F^{**}\isom{\mathbf{Ban}_1}\bigoplus_\infty\{L_1(\Omega_\lambda,\mu_\lambda)^{**}:\lambda\in\Lambda\}$ посредством проектора $\bigoplus_\infty \{P_\lambda:\lambda\in\Lambda\}$. Так как $F^{**}$ имеет свойство Данфорда-Петтиса, то из [\cite{FabHabBanSpTh}, предложение 13.44] следует, что это свойство имеет и дополняемое в $F^{**}$ подпространство $\bigoplus_\infty\{L_1(\Omega_\lambda,\mu_\lambda):\lambda\in\Lambda\}$.

Теперь рассмотрим общий случай. Любое $L_1$-пространство можно рассматривать как $1$-дополняемое подпространство некоторого бесконечномерного $L_1$-пространства. Как следствие, пространство $\bigoplus_\infty\{L_1(\Omega_\lambda,\mu_\lambda):\lambda\in\Lambda\}$ будет $1$-дополняемо в $\bigoplus_\infty$-сумме бесконечномерных $L_1$-пространств. Как было показано выше, такая сумма обладает свойством Данфорда-Петтиса, а значит,  Осталось вспомнить, что это свойство Данфорда-Петтиса наследуется дополняемыми подпространствами [\cite{FabHabBanSpTh}, предложение 13.44].
\end{proof}

\begin{proposition}\label{ProdOfDualsOfMthscrLInftySpHaveDPP} Пусть $E$ --- $\mathscr{L}_\infty$-пространство и $\Lambda$ --- произвольное множество. Тогда банахово пространство $\bigoplus_\infty\{E^*:\lambda\in\Lambda\}$ имеет свойство Данфорда-Петтиса.
\end{proposition}
\begin{proof} Поскольку $E$ --- это $\mathscr{L}_\infty$-пространство, то $E^*$ дополняемо в некотором $L_1$-пространстве [\cite{LinPelAbsSumOpInLpSpAndApp}, предложение 7.4]. То есть существует ограниченный линейный проектор $P:L_1(\Omega,\mu)\to L_1(\Omega,\mu)$ с образом топологически изоморфным пространству $E^*$. В этом случае $\bigoplus_\infty\{ E^*:\lambda\in\Lambda\}$ дополняемо в $\bigoplus_\infty\{ L_1(\Omega,\mu):\lambda\in\Lambda\}$ посредством проектора $\bigoplus_\infty\{P:\lambda\in\Lambda\}$. Пространство $\bigoplus_\infty\{ L_1(\Omega,\mu):\lambda\in\Lambda\}$ имеет свойство Данфорда-Петтиса по предложению \ref{ProdOfL1SpHaveDPP}. Тогда из [\cite{FabHabBanSpTh}, предложение 13.44] следует, что этим свойством обладает и его дополняемое подпространство $\bigoplus_\infty\{ E^*:\lambda\in\Lambda\}$.
\end{proof}

\begin{theorem}\label{TopProjInjFlatModOverMthscrL1OrLInftySpHaveDPP} Пусть $A$ --- банахова алгебра, являющаяся, как банахово пространство, $\mathscr{L}_1$- или $\mathscr{L}_\infty$-пространством. Тогда топологически проективные, инъективные и плоские $A$-модули имеют свойство Данфорда-Петтиса.
\end{theorem}
\begin{proof} Предположим $A$ является $\mathscr{L}_1$-пространством. Заметим, что $\mathscr{L}_1$- и $\mathscr{L}_\infty$-пространства имеют свойство Данфорда-Петтиса [\cite{BourgNewClOfLpSp}, предложение 1.30]. Теперь результат следует из предложения \ref{TopProjInjFlatModOverMthscrL1SpCharac}.

Предположим $A$ является $\mathscr{L}_\infty$-пространством, тогда такова же и $A_+$. Пусть $J$ --- топологически инъективный $A$-модуль, тогда по предложению \ref{MetTopInjModViaCanonicMorph} он ретракт
$$
\mathcal{B}(A_+,\ell_\infty(B_{J^*}))\isom{\mathbf{mod}_1-A}(A_+\projtens\ell_1(B_{J^*}))^*\isom{\mathbf{mod}_1-A}
\left(\bigoplus\nolimits_1\{ A_+:\lambda\in B_{J^*}\}\right)^*\isom{\mathbf{mod}_1-A}
\bigoplus\nolimits_\infty\{ A_+^*:\lambda\in B_{J^*}\}
$$ 
в $\mathbf{mod}-A$ и тем более в $\mathbf{Ban}$. По предложению \ref{ProdOfDualsOfMthscrLInftySpHaveDPP} последний модуль имеет свойство Данфорда-Петтиса. Так как $J$ его ретракт, то он тоже обладает этим свойством [\cite{FabHabBanSpTh}, предложение 13.44]. 

Если $F$ топологически плоский $A$-модуль, то $F^*$ топологически инъективен по предложению \ref{MetTopFlatCharac}. Из предыдущего абзаца мы знаем, что тогда $F^*$ имеет свойство Данфорда-Петтиса и, как следствие, этим свойством обладает сам модуль $F$.

Пусть $P$ --- топологически проективный $A$-модуль. По предположению \ref{MetTopProjIsMetTopFlat} он топологически плоский и тогда из предыдущего абзаца мы видим, что $P$ имеет свойство Данфорда-Петтиса.
\end{proof}

\begin{corollary}\label{NoInfDimRefMetTopProjInjFlatModOverMthscrL1OrLInfty} Пусть $A$ --- банахова алгебра, являющаяся, как банахово пространство, $\mathscr{L}_1$- или $\mathscr{L}_\infty$-пространством. Тогда не существует топологически проективного, инъективного или плоского бесконечномерного рефлексивного $A$-модуля. Тем более не существует метрически проективного, инъективного или плоского бесконечномерного рефлексивного $A$-модуля.
\end{corollary}
\begin{proof} Из теоремы \ref{TopProjInjFlatModOverMthscrL1OrLInftySpHaveDPP} мы знаем, что любой топологически инъективный $A$-модуль имеет свойство Данфорда-Петтиса. С другой стороны не существует бесконечномерного рефлексивного банахова пространства с этим свойством. Итак, мы получили желаемый результат в контексте топологической инъективности. Так как пространство, сопряженное к рефлексивному снова рефлексивно, то из предложения \ref{MetTopFlatCharac} следует результат для топологической плоскости. Осталось вспомнить что по предложению \ref{MetTopProjIsMetTopFlat} каждый топологически проективный модуль является топологически плоским. Чтобы доказать последнее утверждение вспомним, что по предложению $\langle$~\ref{MetProjIsTopProjAndTopProjIsRelProj} / \ref{MetInjIsTopInjAndTopInjIsRelInj} / \ref{MetFlatIsTopFlatAndTopFlatIsRelFlat}~$\rangle$ метрическая $\langle$~проективность / инъективность / плоскость~$\rangle$ влечет топологическую $\langle$~проективность / инъективность / плоскость~$\rangle$.
\end{proof}

Стоит сказать, что в относительной теории существуют примеры рефлексивных бесконечномерных относительно проективных, инъективных и плоских модулей над банаховыми алгебрами, являющимися $\mathscr{L}_1$- или $\mathscr{L}_\infty$-пространствами. Приведем два примера. Первый связан с сверточной алгеброй $L_1(G)$ локально компактной группы $G$ с мерой Хаара. Эта алгебра --- $\mathscr{L}_1$-пространство. В [\cite{DalPolHomolPropGrAlg}, \S6] и \cite{RachInjModAndAmenGr} было доказано, что для для $1<p<+\infty$ банахов $L_1(G)$-модуль $L_p(G)$ является относительно $\langle$~проективным / инъективным / плоским~$\rangle$ тогда и только тогда, когда группа $G$  $\langle$~компактна / аменабельна / аменабельна~$\rangle$. Заметим, что любая компактная группа аменабельна [\cite{PierAmenLCA}, предложение 3.12.1], и поэтому для компактной группы $G$ модуль $L_p(G)$ будет относительно проективным инъективным и плоским для всех $1<p<+\infty$. Второй пример будет про алгебры $c_0(\Lambda)$ и $\ell_\infty(\Lambda)$ для бесконечного множества $\Lambda$. Это $\mathscr{L}_\infty$-пространства. Как мы покажем в предложении \ref{c0AndlInftyModsRelTh}, над этими алгебрами модули $\ell_p(\Lambda)$ для $1<p<\infty$ всегда являются относительно проективными, инъективными и плоскими. 

Чтобы обсудить еще одно банахово-геометрическое свойство, нам понадобятся долгие приготовления, а именно, определение банаховой решетки и безусловного базиса Шаудера. 

Действительное пространство Риса $E$ --- это линейное пространство над полем $\mathbb{R}$ со структурой частично упорядоченного множества, такой, что  $x\leq y$ влечет $x+z\leq y+z$ для всех $x,y,z\in E$ и $ax\geq 0$ для всех $x\geq 0$, $a\in\mathbb{R}_+$. Частично упорядоченное множество называется решеткой, если любые два элемента ${x,y}$ имеют точную верхнюю грань $x\vee y$ и точную нижнюю грань $x\wedge y$. Действительная векторная решетка --- это действительное пространство Риса, которое, как частично упорядоченное множество, является решеткой. Если $E$ --- действительная векторная решетка, то для каждого вектора $x\in E$ мы определяем его модуль по формуле $|x|:=x\vee(-x)$. Комплексная векторная решетка $E$ --- это линейное пространство над полем $\mathbb{C}$, такое, что существует действительное линейное подпространство $\operatorname{Re}(E)$, являющееся действительной векторной решеткой, причем:

$i)$ для каждого $x\in E$ существуют единственные $\operatorname{Re}(x),\operatorname{Im}(x)\in \operatorname{Re}(E)$ такие, что $x=\operatorname{Re}(x)+i\operatorname{Im}(x)$;

$ii)$ для каждого $x\in E$ определено абсолютное значение $|x|:=\sup\{\operatorname{Re}(e^{i\theta}x):\theta\in\mathbb{R}\}$.

Банахова решетка --- это банахово пространство со структурой комплексной векторной решетки такой, что $|x|\leq |y|$ влечет $\Vert x\Vert\leq \Vert y\Vert$. Классический пример банаховой решетки $E$ --- это $L_p$-пространство или $C(K)$-пространство. В обоих случаях $\operatorname{Re}(E)$ состоит из действительнозначных функций в $E$. Если $\{E_\lambda:\lambda\in\Lambda\}$ --- семейство банаховых решеток, то для любого $1\leq p\leq +\infty$ или $p=0$ их $\bigoplus_p$-сумма есть банахова решетка, причем для $x,y\in\bigoplus_p\{ E_\lambda:\lambda\in\Lambda\}$ выполнено $x\leq y$ если $x_\lambda\leq y_\lambda$ для всех $\lambda\in\Lambda$. Сопряженное пространство $E^*$  банаховой решетки $E$ есть снова банахова решетка, причем для $f,g\in  E^*$ выполнено $f\leq g$ если $f(x)\leq g(x)$ для всех $x\geq 0$. Хорошее введение в теорию банаховых решеток можно найти в [\cite{LaceyIsomThOfClassicBanSp}, параграф 1].

Свойство быть банаховой решеткой накладывает определенные ограничения на геометрию банахова пространства \cite{SherOrderInOpAlg}, \cite{KadOrderPropOfBoundSAOps}. Чтобы объяснить этот эффект, нам понадобится определение безусловного базиса Шаудера. Пусть $E$ --- банахово пространство. Набор функционалов $(f_\lambda)_{\lambda\in\Lambda}$ в $E^*$ называется биортогональной системой для векторов $(x_\lambda)_{\lambda\in\Lambda}$ из $E$, если $f_\lambda(x_{\lambda'})=1$ при $\lambda=\lambda'$, иначе $0$. Набор векторов $(x_\lambda)_{\lambda\in\Lambda}$ в $E$ называется безусловным базисом Шаудера, если в $E^*$ существует биортогональная система $(f_\lambda)_{\lambda\in\Lambda}$ для $(x_\lambda)_{\lambda\in\Lambda}$ такая, что
ряд $\sum_{\lambda\in\Lambda} f_\lambda(x)x_\lambda$ безусловно сходится к $x$ для любого $x\in E$. Все $\ell_p$-пространства при $1\leq p<+\infty$ имеют безусловный базис Шаудера, например, это $(\delta_\lambda)_{\lambda\in\Lambda}$. Классический пример пространства без безусловного базиса Шаудера --- это $C([0,1])$. Более того, это пространство не может быть даже подпространством банахова пространства с безусловным базисом Шаудера [\cite{KalAlbTopicsBanSpTh}, предложение 3.5.4]. Любой безусловный базис Шаудера $(x_\lambda)_{\lambda\in\Lambda}$ в $E$ удовлетворяет следующему свойству [\cite{DiestAbsSumOps}, предложение 1.6]: существует константа $\kappa\geq 1$ такая, что
$$
\left\Vert \sum_{\lambda\in\Lambda}t_\lambda f_\lambda(x)x_\lambda\right\Vert
\leq
\kappa\left\Vert \sum_{\lambda\in\Lambda}f_\lambda(x)x_\lambda\right\Vert
$$
для всех $x\in E$ и $t\in\ell_\infty(\Lambda)$. Наименьшая из таких констант $\kappa$ по всем безусловным базисам Шаудера пространства $E$ обозначается через $\kappa(E)$. Аналогичная характеристика может быть определена и для банаховых пространств без безусловного базиса Шаудера. Локальная безусловная константа $\kappa_u(E)$ банахова пространства $E$ определяется как инфимум всех чисел $C$ со следующим свойством: для каждого конечномерного подпространства $F$ в $E$ существует банахово пространство $G$ с безусловным базисом Шаудера и ограниченные линейные операторы $S:F\to G$, $T:G\to E$ такие, что $TS|^{F}=1_F$ и $\Vert T\Vert\Vert S\Vert\kappa(G)\leq C$. Говорят, что банахово пространство $E$ имеет локально безусловную структуру если $\kappa_u(E)$ конечно. Изначально это свойство было определено на английском и называлось the \textbf{l}ocal \textbf{u}nconditional \textbf{st}ructure property. Для него использовали аббревиатуру l.u.st, и мы поступим так же. Очевидно, любое банахово пространство $E$ с безусловным базисом Шаудера имеет свойство l.u.st, причем $\kappa_u(E)=\kappa(E)$. В частности, все конечномерные банаховы пространства имеют свойство l.u.st. Хотя произвольная банахова решетка $E$ может и не обладать безусловным базисом Шаудера, она все равно имеет свойство l.u.st, и при этом $\kappa_u(E)=1$  [\cite{DiestAbsSumOps}, теорема 17.1]. Непосредственно из определения следует, что свойство l.u.st наследуется дополняемыми подпространствами. Точнее, если $F$ --- $C$-дополняемое подпространство в $E$, то $\kappa_u(F)\leq C\kappa_u(E)$. Следовательно, все дополняемые подпространства банаховых решеток обладают свойством l.u.st. Это необходимое условие не так уж далеко от критерия [\cite{DiestAbsSumOps}, теорема 17.5]: банахово пространство $E$ имеет свойство l.u.st тогда и только тогда, когда $E^{**}$ топологически изоморфно, как банахово пространство, дополняемому подпространству некоторой банаховой решетки. Как следствие, этого критерия мы получаем, что банахово пространство $E$ имеет свойство l.u.st тогда и только тогда, когда его имеет $E^*$ [\cite{DiestAbsSumOps}, следствие 17.6].

\begin{proposition} Пусть $A$ --- банахова алгебра, обладающая свойством l.u.st. Тогда всякий топологически проективный, инъективный и плоский $A$-модуль тоже обладает этим свойством.
\end{proposition}
\begin{proof} Если $J$ --- топологически инъективный $A$-модуль, то по предложению \ref{MetTopInjModViaCanonicMorph} он является ретрактом $\mathcal{B}(A_+,\ell_\infty(B_{J^*}))\isom{\mathbf{mod}_1-A}\bigoplus_\infty\{ A_+^*:\lambda\in B_{J^*}\}$ в $\mathbf{mod}-A$ и тем более в $\mathbf{Ban}$. Если $A$ обладает свойством l.u.st, то $A^{**}$ дополняемо в некоторой банаховой решетке $E$ [\cite{DiestAbsSumOps}, теорема 17.5]. Как следствие $A_+^{***}$ дополняемо в банаховой решетке $F:=\left(E\bigoplus_1\mathbb{C}\right)^*$ посредством некоторого ограниченного проектора $P:F\to F$. Следовательно, $\bigoplus_\infty\{A_+^{***}:\lambda\in B_{J^*}\}$ дополняемо в банаховой решетке $\bigoplus_\infty\{F:\lambda\in B_{J^*}\}$ посредством ограниченного проектора $\bigoplus_\infty\{ P:\lambda\in B_{J^*}\}$. Любая банахова решетка имеет свойство l.u.st [\cite{DiestAbsSumOps}, теорема 17.1]. Это свойство наследуется дополняемыми подпространствами, поэтому $\bigoplus_\infty\{A_+^{***}:\lambda\in B_{J^*}\}$ тоже имеет свойство l.u.st. Заметим, что $A_+^*$ $1$-дополняемо в $A_+^{***}$ посредством проектора Диксмье $Q$, значит $\bigoplus_\infty\{A_+^*:\lambda\in B_{J^*}\}$ $1$-дополняемо в $\bigoplus_\infty\{A_+^{***}:\lambda\in B_{J^*}\}$ посредством проектора $\bigoplus_\infty\{Q:\lambda\in B_{J^*}\}$. Так как последнее пространство имеет свойство l.u.st, то им обладает и ретракт $\bigoplus_\infty\{A_+^*:\lambda\in B_{J^*}\}$. Наконец, $J$ --- это ретракт $\bigoplus_\infty\{A_+^*:\lambda\in B_{J^*}\}$, поэтому он тоже обладает этим свойством.

Если $F$ --- топологически плоский $A$-модуль, то $F^*$ топологически инъективен по предложению \ref{MetTopFlatCharac}. Из рассуждений предыдущего абзаца следует, что $F^*$ имеет свойство l.u.st. Из [\cite{DiestAbsSumOps}, следствие 17.6] мы заключаем, что этим свойством обладает и сам модуль $F$.

Наконец, если $P$ --- топологически проективный $A$-модуль, то он топологически плоский по предложению \ref{MetTopProjIsMetTopFlat}. Из  предыдущего абзаца мы знаем, что в этом случае $P$ имеет свойство l.u.st.
\end{proof}

%----------------------------------------------------------------------------------------
%	Further properties of projective injective and flat modules
%----------------------------------------------------------------------------------------

\section{Дальнейшие свойства проективных, инъективных и плоских модулей}
\label{SectionFurtherPropertiesOfProjectiveInjectiveAndFlatModules}

%----------------------------------------------------------------------------------------
%	Homological triviality of modules under change of algebra
%----------------------------------------------------------------------------------------

\subsection{Гомологическая тривиальность модулей при замене алгебры}
\label{SubSectionHomologicalTrivialityOfModulesUnderChangeOfAlgebra}

В дальнейшем при исследовании метрически и топологически гомологически тривиальных модулей над различными алгебрами анализа нам очень пригодятся следующие предложения. Они суть метрически и топологические версии предложений 2.3.2, 2.3.3 и 2.3.4 из \cite{RamsHomPropSemgroupAlg}.

\begin{proposition}\label{MorphCoincide} Пусть $X$ и $Y$ --- $\langle$~левые / правые~$\rangle$ банаховы $A$-модули. Допустим, что выполнено одно из условий:

$i)$ $I$ --- $\langle$~левый / правый~$\rangle$ идеал в $A$, и $X$ --- существенный $I$-модуль;

$ii)$ $I$ --- $\langle$~правый / левый~$\rangle$ идеал в $A$, и $Y$ --- верный $I$-модуль.

Тогда $\langle$~${}_A\mathcal{B}(X,Y)={}_I\mathcal{B}(X,Y)$ / $\mathcal{B}_A(X,Y)=\mathcal{B}_I(X,Y)$~$\rangle$.
\end{proposition}
\begin{proof} Мы рассмотрим случай только левых модулей, так как для правых модулей доказательства аналогичны. Возьмем произвольный морфизм $\phi\in {}_I\mathcal{B}(X,Y)$.

$i)$ Рассмотрим $x\in I\cdot X$, тогда $x=a'\cdot x'$ для некоторых $a'\in I$, $x'\in X$. Для любого $a\in A$ выполнено $\phi(a\cdot x)=\phi(aa'\cdot x')=aa'\cdot\phi(x')=a\cdot\phi(a'\cdot x')=a\cdot\phi(x)$. Следовательно, $\phi(a\cdot x)=a\cdot\phi(x)$ для всех $a\in A$ и $x\in \operatorname{cl}_X(IX)=X$. Значит, $\phi\in {}_A\mathcal{B}(X,Y)$.

$ii)$ Для любых $a\in I$, $a'\in A$ и $x\in X$ выполнено $a\cdot(\phi(a'\cdot x)-a'\cdot\phi(x))=\phi(aa'\cdot x)-aa'\cdot\phi(x)=0$. Так как $Y$ --- верный $I$-модуль, то $\phi(a'\cdot x)=a'\cdot \phi(x)$ для всех $x\in X$, $a'\in A$. Значит, $\phi\in{}_A\mathcal{B}(X,Y)$.

В обоих случаях мы доказали, что $\phi\in{}_A\mathcal{B}(X,Y)$ для любого $\phi\in{}_I\mathcal{B}(X,Y)$, следовательно, ${}_I\mathcal{B}(X,Y)\subset {}_A\mathcal{B}(X,Y)$. Обратное включение очевидно.
\end{proof}

\begin{proposition}\label{MetTopProjUnderChangeOfAlg} Пусть $I$ --- замкнутая подалгебра в $A$, и $P$ --- банахов $A$-модуль, существенный как $I$-модуль. Тогда:

$i)$ если $I$ --- левый идеал в $A$ и $P$ $\langle$~метрически / $C$-топологически~$\rangle$ проективен как $I$-модуль, то $P$ $\langle$~метрически / $C$-топологически~$\rangle$ проективен как $A$-модуль;

$ii)$ если $I$ --- $\langle$~$1$-дополняемый / $c$-дополняемый~$\rangle$ правый идеал $A$ и $P$ $\langle$~метрически / $C$-топологически~$\rangle$ проективен как $A$-модуль, то $P$ $\langle$~метрически / $cC$-топологически~$\rangle$ проективен как $I$-модуль.
\end{proposition}
\begin{proof} Через $\widetilde{\pi}_P: I\projtens \ell_1(B_P)\to P$ и $\pi_P:A\projtens \ell_1(B_P)\to P$ мы обозначим стандартные эпиморфизмы.

$i)$ По предложению \ref{NonDegenMetTopProjCharac} морфизм $\widetilde{\pi}_P$ имеет в $\langle$~$I-\mathbf{mod}_1$ / $I-\mathbf{mod}$~$\rangle$ правый обратный морфизм  $\widetilde{\sigma}$ нормы $\langle$~не более $1$ / не более $C$~$\rangle$. Пусть $i:I\to A$ --- естественное вложение, тогда рассмотрим $\langle$~сжимающий / ограниченный~$\rangle$ $I$-морфизм $\sigma=(i\projtens 1_{\ell_1(B_P)})\widetilde{\sigma}$. По предложению \ref{MorphCoincide} оператор $\sigma$ является $A$-морфизмом. Очевидно, $\sigma$ имеет норму $\langle$~не более $1$ / не более $C$~$\rangle$. Для $\pi_P:A\projtens \ell_1(B_P)\to P$ выполнено $\pi_P(i\projtens 1_{\ell_1(B_P)})=\widetilde{\pi}_P$, следовательно, $\pi_P\sigma=\pi_P(i\projtens 1_{\ell_1(B_P)})\widetilde{\sigma}=\widetilde{\pi}_P\widetilde{\sigma}=1_P$. Таким образом, $\pi_P$ есть $\langle$~$1$-ретракция / $C$-ретракция~$\rangle$ в $\langle$~$A-\mathbf{mod}_1$ / $A-\mathbf{mod}$~$\rangle$. По предложению \ref{NonDegenMetTopProjCharac} банахов $A$-модуль $P$ $\langle$~метрически / $C$-топологически~$\rangle$ проективен.

$ii)$ Поскольку $P$ существенный $I$-модуль, он тем более будет существенным $A$-модулем. По предложению \ref{NonDegenMetTopProjCharac} морфизм $\pi_P$ имеет правый обратный морфизм $\sigma$ в $\langle$~$A-\mathbf{mod}_1$ / $A-\mathbf{mod}$~$\rangle$ нормы $\langle$~не более $1$ / не более $C$~$\rangle$. Ясно,что $\sigma$ является правым обратным морфизмом в $\pi_P$ и в $\langle$~$I-\mathbf{mod}_1$ / $I-\mathbf{mod}$~$\rangle$. Через $i:I\to A$ мы обозначим естественное вложение, а через $r:A\to I$ $\langle$~сжимающий / ограниченный~$\rangle$ левый обратный оператор. По предположению $\Vert r\Vert\leq c$. Рассмотрим $\langle$~сжимающий / ограниченный~$\rangle$ линейный оператор $\widetilde{\sigma}=(r\projtens 1_{\ell_1(B_P)})\sigma$. Очевидно, его норма $\langle$~не превосходит $1$ / не превосходит $cC$~$\rangle$. Так как $I$ --- правый идеал в $A$ и $P$ является существенным $I$-модулем, то $\sigma(P)=\sigma(\operatorname{cl}_P(IP))=\operatorname{cl}_{A\projtens \ell_1(B_P)}(I\cdot (A\projtens \ell_1(B_P)))=I\projtens \ell_1(B_P)$, поэтому $\sigma=(ir\projtens 1_{\ell_1(B_P)})\sigma$. Более того, так как $\sigma(P)\subset I\projtens\ell_1(B_P)$ и $r|_I=1_I$, то $\sigma$ является $I$-морфизмом. Очевидно, $\pi_P(i\projtens 1_{\ell_1(B_P)})=\widetilde{\pi}_P$, поэтому $\widetilde{\pi}_P\widetilde{\sigma}=\pi_P(i\projtens 1_{\ell_1(B_P)})(r\projtens 1_{\ell_1(B_P)})\sigma=\pi_P(ir\projtens 1_{\ell_1(B_P)})\sigma=\pi_P\sigma=1_P$. Таким образом, $\widetilde{\pi}_P$ --- $\langle$~$1$-ретракция / $cC$-ретракция~$\rangle$ в $\langle$~$I-\mathbf{mod}_1$ / $I-\mathbf{mod}$~$\rangle$, поэтому из предложения \ref{NonDegenMetTopProjCharac} следует, что $I$-модуль $P$ $\langle$~метрически / $cC$-топологически~$\rangle$ проективен.
\end{proof}

\begin{proposition}\label{MetTopInjUnderChangeOfAlg} Пусть $I$ --- замкнутая подалгебра в $A$, и $J$ --- правый банахов $A$-модуль верный как $I$-модуль. Тогда:

$i)$ если $I$ --- левый идеал в $A$ и $J$ $\langle$~метрически / $C$-топологически~$\rangle$ инъективный $I$-модуль, то $J$ $\langle$~метрически / $C$-топологически~$\rangle$ инъективен как $A$-модуль;

$ii)$ если $I$ --- $\langle$~$1$-дополняемый / $c$-дополняемый~$\rangle$ правый идеал $A$ и $J$ $\langle$~метрически / $C$-топологически~$\rangle$ инъективен как $A$-модуль, то $J$ $\langle$~метрически / $cC$-топологически~$\rangle$ инъективен как $I$-модуль.
\end{proposition}
\begin{proof} Через $\widetilde{\rho}_J:J\to\mathcal{B}(I,\ell_\infty(B_{J^*}))$ и $\rho_J:J\to\mathcal{B}(A,\ell_\infty(B_{J^*}))$ мы обозначим стандартные мономорфизмы.

$i)$ По предложению \ref{NonDegenMetTopInjCharac} морфизм $\widetilde{\rho}_J: J\to\mathcal{B}(I,\ell_\infty(B_{J^*}))$ имеет левый обратный морфизм в $\langle$~$\mathbf{mod}_1-I$ / $\mathbf{mod}-I$~$\rangle$, скажем, $\widetilde{\tau}$ нормы $\langle$~не более $1$ / не более $C$~$\rangle$. Пусть $i:I\to A$ --- естественное вложение, тогда рассмотрим $I$-морфизм $q=\mathcal{B}(i,\ell_\infty(B_{J^*}))$. Очевидно $\widetilde{\rho}_J=q\rho_J$. Рассмотрим $I$-морфизм $\tau =\widetilde{\tau} q$. По предложению \ref{MorphCoincide} он также является $A$-морфизмом. Заметим, что $\Vert\tau \Vert\leq\Vert\widetilde{\tau}\Vert\Vert q\Vert\leq\Vert\widetilde{\tau}\Vert$, поэтому и $\tau$ имеет норму $\langle$~не более $1$ / не более $C$~$\rangle$. Ясно, что $\tau \rho_J=\widetilde{\tau} q\rho_J=\widetilde{\tau}\widetilde{\rho}_J=1_J$. Таким образом, $\rho_J$ есть $\langle$~$1$-коретракция / $C$-коретракция~$\rangle$ в $\langle$~$\mathbf{mod}_1-A$ / $\mathbf{mod}-A$~$\rangle$, поэтому из предложения \ref{NonDegenMetTopInjCharac} следует, что $A$-модуль $J$ $\langle$~метрически / $C$-топологически~$\rangle$ инъективен.

$ii)$ Поскольку $J$ $\langle$~метрически / $C$-топологически~$\rangle$ инъективен как $A$-модуль, то из предложения \ref{NonDegenMetTopInjCharac} морфизм $\rho_J$ имеет левый обратный морфизм в $\langle$~$\mathbf{mod}_1-A$ / $\mathbf{mod}-A$~$\rangle$, скажем, $\tau $ нормы $\langle$~не более $1$ / не более $C$~$\rangle$. Допустим, нам дан оператор $T\in \mathcal{B}(A,\ell_\infty(B_{J^*}))$ такой, что $T|_I=0$. Зафиксируем $a\in I$, тогда $T\cdot a=0$, и поэтому $\tau (T)\cdot a=\tau (T\cdot a)=0$. Так как $J$ является верным $I$-модулем и $a\in I$ произволен, то $\tau (T)=0$. Через $r:A\to I$  мы обозначим оператор левый обратный к $i$. Он существует по предположению и его норма не превосходит $c$. Определим ограниченные линейные операторы $j=\mathcal{B}(r,\ell_\infty(B_{J^*}))$ и $\widetilde{\tau}=\tau  j$. Для любого $a\in I$ и $T\in\mathcal{B}(I,\ell_\infty(B_{J^*}))$ выполнено $\widetilde{\tau}(T\cdot a)-\widetilde{\tau}(T)\cdot a=\tau (j(T\cdot a)-j(T)\cdot a)=0$ потому, что $j(T\cdot a)-j(T)\cdot a|_I=0$. Следовательно, $\widetilde{\tau}$ есть $I$-морфизм. Заметим, что $\Vert\widetilde{\tau}\Vert\leq\Vert\tau \Vert\Vert j\Vert$, поэтому $\widetilde{\tau}$ имеет норму $\langle$~не более $1$ / не более $cC$~$\rangle$. Очевидно, для всех $x\in J$ выполнено $\rho_J(x)-j(\widetilde{\rho}_J(x))|_I=0$, поэтому $\tau (\rho_J(x)-j(\widetilde{\rho}_J(x)))=0$. Как следствие, $\widetilde{\tau}(\widetilde{\rho}_J(x))=\tau (j(\widetilde{\rho}_J(x)))=\tau (\rho_J(x))=x$ для всех $x\in J$. Так как $\widetilde{\tau}\widetilde{\rho}_J=1_J$, то $\widetilde{\rho}_J$ --- $\langle$~$1$-коретракция / $cC$-коретракция~$\rangle$ в $\langle$~$\mathbf{mod}_1-I$ / $\mathbf{mod}-I$~$\rangle$, поэтому из предложения \ref{NonDegenMetTopInjCharac} следует, что $I$-модуль $J$ $\langle$~метрически / $cC$-топологически~$\rangle$ инъективен.
\end{proof}

\begin{proposition}\label{MetTopFlatUnderChangeOfAlg} Пусть $I$ --- замкнутая подалгебра в $A$, и $F$ --- банахов $A$-модуль существенный как $I$-модуль. Тогда:

$i)$ если $I$ --- левый идеал в $A$ и $F$ $\langle$~метрически / $C$-топологически~$\rangle$ плоский $I$-модуль, то $F$ $\langle$~метрически / $C$-топологически~$\rangle$ плоский $A$-модуль;

$ii)$ если $I$ --- $\langle$~$1$-дополняемый / $c$-дополняемый~$\rangle$ правый идеал $A$ и $F$ есть $\langle$~метрически / $C$-топологически~$\rangle$ плоский $A$-модуль, то $F$ $\langle$~метрически / $cC$-топологически~$\rangle$ плоский $I$-модуль.
\end{proposition}
\begin{proof} Заметим, что модуль, сопряженный к существенному модулю, будет верным. Теперь все результаты следуют из предложений \ref{MetTopFlatCharac} и \ref{MetTopInjUnderChangeOfAlg}.
\end{proof}

\begin{proposition}\label{MetTopProjInjFlatUnderSumOfAlg} Пусть  $(A_\lambda)_{\lambda\in\Lambda}$ --- семейство банаховых алгебр и для каждого $\lambda\in\Lambda$ пусть $X_\lambda$ ---  $\langle$~существенный / верный / существенный~$\rangle$ банахов $A_\lambda$-модуль. Обозначим $A=\bigoplus_p\{A_\lambda:\lambda\in\Lambda\}$, где $1\leq p\leq +\infty$ или $p=0$. Пусть $X$ обозначает $\langle$~$\bigoplus_1\{X_\lambda:\lambda\in\Lambda\}$ / $\bigoplus_\infty\{X_\lambda:\lambda\in\Lambda\}$ / $\bigoplus_1\{X_\lambda:\lambda\in\Lambda\}$~$\rangle$. Тогда:

$i)$ $X$ метрически $\langle$~проективный / инъективный / плоский~$\rangle$ $A$-модуль тогда и только тогда, когда для всех $\lambda\in\Lambda$ банахов $A_\lambda$-модуль $X_\lambda$ метрически $\langle$~проективный / инъективный / плоский~$\rangle$;

$ii)$ $X$ $C$-топологически $\langle$~проективный / инъективный / плоский~$\rangle$ $A$-модуль тогда и только тогда, когда для всех $\lambda\in\Lambda$ банахов $A_\lambda$-модуль $X_\lambda$ $C$-топологически $\langle$~проективный / инъективный / плоский~$\rangle$.
\end{proposition}
\begin{proof} Заметим, что для каждого $\lambda\in\Lambda$ естественное вложение $i_\lambda:A_\lambda\to A$ позволяет рассматривать $A_\lambda$ как $1$-дополняемый  двусторонний идеал в $A$.

$i)$ Доказательство дословно повторяет рассуждения из пункта $ii)$.

$ii)$ Допустим $X_\lambda$ $C$-топологически $\langle$~проективный / инъективный / плоский~$\rangle$ банахов $A_\lambda$-модуль для всех $\lambda\in\Lambda$, тогда из пункта $i)$ предложения $\langle$~\ref{MetTopProjUnderChangeOfAlg} / \ref{MetTopInjUnderChangeOfAlg} / \ref{MetTopFlatUnderChangeOfAlg}~$\rangle$ этот модуль $C$-топологически $\langle$~проективный / инъективный / плоский~$\rangle$ как $A$-модуль. Осталось применить предложение $\langle$~\ref{MetTopProjModCoprod} / \ref{MetTopInjModProd} / \ref{MetTopFlatModCoProd}~$\rangle$. 

Обратно, допустим, что $X$ $C$-топологически $\langle$~проективный / инъективный / плоский~$\rangle$ $A$-модуль. Зафиксируем произвольный $\lambda\in\Lambda$. Очевидно, мы можем рассматривать $X_\lambda$ как $A$-модуль и более того $X_\lambda$, очевидно, является $1$-ретрактом $X$ в $\langle$~$A-\mathbf{mod}_1$ / $\mathbf{mod}_1-A$ / $A-\mathbf{mod}_1$~$\rangle$. По предложению $\langle$~\ref{RetrCTopProjIsCTopProj} / \ref{RetrCTopInjIsCTopInj} / \ref{RetrCTopFlatIsCTopFlat}~$\rangle$ модуль $X_\lambda$ $C$-топологически $\langle$~проективный / инъективный / плоский~$\rangle$ как $A$-модуль. Осталось применить пункт  $ii)$ предложения $\langle$~\ref{MetTopProjUnderChangeOfAlg} / \ref{MetTopInjUnderChangeOfAlg} / \ref{MetTopFlatUnderChangeOfAlg}~$\rangle$.
\end{proof} 

%----------------------------------------------------------------------------------------
%	Flat modules and injective ideals
%----------------------------------------------------------------------------------------

\subsection{Плоские модули и инъективные идеалы}
\label{SubSectionFlatModulesAndInjectiveIdeals}

Используя результаты предыдущих параграфов, мы докажем еще несколько полезных фактов о метрической и топологической инъективности и плоскости банаховых модулей.

\begin{proposition}\label{DualBanModDecomp} Пусть $B$ --- унитальная банахова алгебра, $A$ --- ее подалгебра с двусторонней ограниченной аппроксимативной единицей $(e_\nu)_{\nu\in N}$ и пусть $X$ --- унитальный левый $B$-модуль. Тогда:

$i)$ $X^*\isom{\mathbf{mod}-A}X_{ess}^*\bigoplus_\infty (X/X_{ess})^*$, где $X_{ess}:=\operatorname{cl}_X(AX)$;

$ii)$ $X_{ess}^*$ есть $C_1$-дополняемый $A$-подмодуль в $X^*$ для $C_1=\sup_{\nu\in N}\Vert e_\nu\Vert$;

$iii)$ $(X/X_{ess})^*$ есть $C_2$-дополняемый $A$-подмодуль в $X^*$ для $C_2=\sup_{\nu\in N}\Vert e_B - e_\nu\Vert$;

$iv)$ если $X$ является $\mathscr{L}_1$-пространством, то $X_{ess}$ и $X/X_{ess}$ тоже $\mathscr{L}_1$-пространства.
\end{proposition}
\begin{proof} Рассмотрим естественное вложение $\rho:X_{ess}\to X:x\mapsto x$ и фактор-отображение $\pi:X\to X/X_{ess}:x\mapsto x+X_{ess}$. Пусть $\mathfrak{F}$ --- фильтр сечений на $N$ и пусть $\mathfrak{U}$ ультрафильтр содержащий $\mathfrak{F}$. Для заданного $f\in X ^*$ и $x\in X $ выполнено $|f(x-e_\nu\cdot x)|\leq\Vert f\Vert\Vert e_B - e_\nu\Vert\Vert x\Vert\leq C_2\Vert f\Vert\Vert x\Vert$, то есть $(f(x-e_\nu\cdot x))_{\nu\in N}$ --- ограниченная направленность комплексных чисел. Следовательно, корректно определен предел $\lim_{\mathfrak{U}}f(x-e_\nu\cdot x)$ вдоль ультрафильтра $\mathfrak{U}$. Поскольку $(e_\nu)_{\nu\in N}$ есть двусторонняя аппроксимативная единица для $A$ и $\mathfrak{U}$ содержит фильтр сечений, то для всех $x\in X_{ess}$ выполнено $\lim_{\mathfrak{U}}f(x-e_\nu\cdot x)=\lim_{\nu}f(x-e_\nu\cdot x)=0$. Следовательно, для каждого $f\in X ^*$ мы имеем корректно определенное отображение $\tau(f):X /X_{ess}\to \mathbb{C}:x+X_{ess}\mapsto \lim_{\mathfrak{U}} f(x-e_\nu\cdot x)$. Очевидно, это линейный функционал и из неравенств выше мы видим, что его норма ограничена сверху константой $C_2\Vert f\Vert$. Теперь легко проверить, что $\tau:X^*\to (X/ X_{ess})^*:f\mapsto \tau(f)$ есть $A$-морфизм нормы не более $C_2$. Аналогично можно показать, что $\sigma:X_{ess}^*\to X^*:h\mapsto(x\mapsto \lim_{\mathfrak{U}}h(e_\nu\cdot x))$ есть $A$-морфизм с нормой не превосходящей $C_1$. Для любых $f\in X^*$, $g\in (X/X_{ess})^*$, $h\in X_{ess}^*$ и $x\in X$, $y\in X_{ess}$ мы имеем
$$
\sigma(h)(y)
=\lim_{\mathfrak{U}}h(e_\nu\cdot y)
=\lim_{\nu}h(e_\nu\cdot y)
=h(y),
\qquad
(\rho^*\sigma)(h)(y)
=\sigma(h)(\rho(y))
\sigma(h)(y)
=h(y),
$$
$$
(\tau\pi^*)(g)(x+X_{ess})
=\lim_{\mathfrak{U}}\pi^*(g)(x-e_\nu\cdot x)
=\lim_{\mathfrak{U}}g(\pi(x-e_\nu\cdot x))
=\lim_{\mathfrak{U}}g(x+X_{ess})
=g(x+X_{ess}),
$$
$$
(\tau\sigma)(h)(x+X_{ess})
=\lim_{\mathfrak{U}}\sigma(h)(x-e_\nu\cdot x)
=\lim_{\mathfrak{U}}(\sigma(h)(x)-h(e_\nu\cdot x))
=\sigma(h)(x)-\lim_{\mathfrak{U}}h(e_\nu\cdot x)=0,
$$
$$
(\pi^*\tau + \sigma\rho^*)(f)(x)
=\tau(f)(x+X_{ess})+\lim_{\mathfrak{U}}\rho^*(f)(e_\nu\cdot x)
=\lim_{\mathfrak{U}}f(x - e_\nu\cdot x)+\lim_{\mathfrak{U}}f(e_\nu\cdot x)
=f(x).
$$
Следовательно, $\tau \pi^*=1_{(X/X_{ess})^*}$, $\rho^*\sigma=1_{X_{ess}^*}$, $\rho^*\pi^*=0$, $\tau\sigma=0$ и $\pi^*\tau+\sigma\rho^*=1_{X^*}$. Это эквивалентно тому, что $X^*\isom{\mathbf{mod}-A}X_{ess}^*\bigoplus_\infty (X/X_{ess})^*$.

Теперь рассмотрим $A$-морфизмы $P_1=\sigma\rho^*$, $P_2=\pi^*\tau$. Их равенств выше следует, что $P_1^2=P_1$, $P_2^2=P_2$ и $\operatorname{Im}(P_1)\isom{\mathbf{mod}-A}X_{ess}^*$, $\operatorname{Im}(P_2)\isom{\mathbf{mod}-A} (X/X_{ess})^*$. Нормы этих проекторов легко оценить: $\Vert P_1\Vert\leq\Vert \sigma\Vert\Vert \rho^*\Vert=C_1$ и $\Vert P_2\Vert\leq \Vert \pi^*\Vert\Vert\tau\Vert\leq C_2$.

Теперь рассмотрим случай, когда $X$ является $\mathscr{L}_1$-пространством. Тогда $X^*$ есть $\mathscr{L}_\infty$-пространство [\cite{BourgNewClOfLpSp}, предложение 1.27]. Так как $X_{ess}^*$ и $(X/X_{ess})^*$ дополняемы в $X^*$, то они тоже являются $\mathscr{L}_\infty$-пространствами [\cite{BourgNewClOfLpSp}, предложение 1.28]. Снова из [\cite{BourgNewClOfLpSp}, предложение 1.27] мы получаем, что $X_{ess}$ и $X/X_{ess}$ являются $\mathscr{L}_1$-пространствами.
\end{proof}

Следующее предложение является аналогом [\cite{RamsHomPropSemgroupAlg}, предложение 2.1.11] для топологической теории.

\begin{proposition}\label{TopFlatModCharac} Пусть $A$ --- банахова алгебра с двусторонней ограниченной аппроксимативной единицей, и пусть $F$ --- левый $A$-модуль. Тогда следующие условия эквивалентны:

$i)$ $F$ --- топологически плоский $A$-модуль;

$ii)$  $F_{ess}$ --- топологически плоский $A$-модуль и $F/F_{ess}$ является $\mathscr{L}_1$-пространством.
\end{proposition}
\begin{proof} Будем рассматривать $A$ как подалгебру унитальной банаховой алгебры $B:=A_+$. Тогда $F$ --- унитальный левый $B$-модуль. Из предложения \ref{DualBanModDecomp} мы имеем $F^*\isom{\mathbf{mod}-A}F_{ess}^*\bigoplus_\infty (F/F_{ess})^*$. Далее предложения \ref{MetTopFlatCharac} и \ref{MetTopInjModProd} дают, что $A$-модуль $F$ топологически плоский тогда и только тогда, когда таковы $F_{ess}$ и $F/F_{ess}$. Осталось заметить, что по предложению \ref{MetTopFlatAnnihModCharac} аннуляторный $A$-модуль $F/F_{ess}$ является топологически плоским тогда и только тогда, когда он является $\mathscr{L}_1$-пространством.
\end{proof}

Теперь нам необходимо более детально ознакомится с понятием аменабельной банаховой алгебры.
Рассмотрим морфизм $A$-бимодулей $\Pi_A:A\projtens A\to A:a\projtens b\mapsto ab$. Банахова алгебра $A$ называется относительно $C$-аменабельной если $\Pi_{A_+}^*$ является $C$-коретракцией $A$-бимодулей. Будем говорить, что банахова алгебра $A$ относительно аменабельна, если она относительно $C$-аменабельна для некоторого $C\geq 1$.  С небольшими изменениями в [\cite{HelBanLocConvAlg}, предложение 7.1.72] можно показать, что $A$ относительно $C$-аменабельна тогда и только тогда, когда существует направленность $(d_\nu)_{\nu\in N}\subset A\projtens A$ по норме не превосходящая $C$, причем для всех $a\in A$ выполнено $\lim_\nu(a\cdot d_\nu-d_\nu\cdot a)=0$ и $\lim_\nu a\Pi_A(d_\nu)=a$. Такая направленность называется аппроксимативной диагональю. С гомологической точки зрения, основное преимущество относительно аменабельной алгебры в том, что любой левый и правый банахов модуль над ней является относительно плоским [\cite{HelBanLocConvAlg}, теорема 7.1.60].

\begin{proposition}\label{MetTopEssL1FlatModAoverAmenBanAlg} Пусть $A$ ---относительно $\langle$~$1$-аменабельная / $c$-аменабельная~$\rangle$ банахова алгебра и $F$ --- существенный банахов $A$-модуль, являющийся, как банахово пространство, $\langle$~$L_1$-пространством / $\mathscr{L}_{1,C}$-пространством~$\rangle$. Тогда $F$ --- $\langle$~метрически / $c^2C$-топологически~$\rangle$ плоский $A$-модуль.
\end{proposition}
\begin{proof} 
Мы можем считать, что $A$ относительно $c$-аменабельна с $\langle$~$c=1$ / $c\geq 1$~$\rangle$. Пусть $(d_\nu)_{\nu\in N}$ --- аппроксимативная диагональ для $A$ по норме не превосходящая $c$. Напомним, что $(\Pi_A(d_\nu))_{\nu\in N}$ есть двусторонняя $\langle$~сжимающая / ограниченная~$\rangle$ аппроксимативная единица в $A$. Так как $F$ --- существенный $A$-модуль, то $\lim_{\nu}\Pi_A(d_\nu)\cdot x=x$ для всех $x\in F$ [\cite{HelHomolBanTopAlg}, предложение 0.3.15]. Как следствие, множество $c\pi_F(B_{A\projtens\ell_1(B_F)})$ плотно в $B_F$. Тогда для любого $f\in F^*$ выполнено
$$
\Vert\pi_F^*(f)\Vert
=\sup\{|f(\pi_F(u))|:u\in B_{A\projtens\ell_1(B_F)}\}
=\sup\{|f(x)|:x\in \operatorname{cl}_F(\pi_F(B_{A\projtens\ell_1(B_F)}))\}
$$
$$
\geq\sup\{c^{-1}|f(x)|:x\in B_F\}=c^{-1}\Vert f\Vert
$$
Это означает, что $\pi_F^*$ --- $c$-топологически инъективный оператор. По предположению $F$ является $\langle$~$L_1$-пространством / $\mathscr{L}_{1,C}$-пространством~$\rangle$, тогда из $\langle$~[\cite{GrothMetrProjFlatBanSp}, теорема 1] / [\cite{StegRethNucOpL1LInfSp}, теорема VI.6]~$\rangle$ следует, что банахово пространство $F^*$ будет $\langle$~метрически / $C$-топологически~$\rangle$ инъективным. Так как оператор $\pi_F^*$ $\langle$~изометричен / $c$-топологически инъективен~$\rangle$, то существует линейный оператор $R:(A\projtens\ell_1(B_F))^*\to F^*$ нормы $\langle$~не более $1$ / не более $cC$~$\rangle$ такой, что $R\pi_F^*=1_{F^*}$.

Пусть $h\in (A\projtens\ell_1(B_F))^*$ и $x\in F$. Рассмотрим билинейный функционал $M_{h,x}:A\times A\to\mathbb{C}:(a,b)\mapsto R(h\cdot a)(b\cdot x)$. Очевидно, $\Vert M_{h,x}\Vert\leq\Vert R\Vert\Vert h\Vert\Vert x\Vert$. По свойству универсальности проективного тензорного произведения мы получаем ограниченный линейный функционал $m_{h,x}:A\projtens A\to\mathbb{C}:a\projtens b\mapsto R(h\cdot a)(b\cdot x)$. Отметим, что $m_{h,x}$ линеен по $h$ и $x$. Более того, для любых $u\in A\projtens A$, $a\in A$ и $f\in F^*$ выполнено $m_{\pi_F^*(f),x}(u)=f(\Pi_A(u)\cdot x)$, $m_{h\cdot a,x}(u)=m_{h,x}(a\cdot u)$, $m_{h,a\cdot x}(u)=m_{h,x}(u\cdot a)$. Это легко проверить на элементарных тензорах. Далее остается заметить, что их линейная оболочка плотна в $A\projtens A$.

Пусть $\mathfrak{F}$ --- фильтр сечений на $N$ и пусть $\mathfrak{U}$ --- ультрафильтр содержащий $\mathfrak{F}$. Для всех $h\in (A\projtens\ell_1(B_F))^*$ и $x\in F$ мы имеем $|m_{h,x}(d_\nu)|\leq c\Vert R\Vert\Vert h\Vert\Vert x\Vert$, то есть $(m_{h,x}(d_\nu))_{\nu\in N}$ --- ограниченная направленность комплексных чисел. Следовательно корректно определен предел $\lim_{\mathfrak{U}}m_{h,x}(d_\nu)$ вдоль ультрафильтра $\mathfrak{U}$. Рассмотрим линейный оператор $\tau:(A\projtens\ell_1(B_F))^*\to F^*:h\mapsto(x\mapsto\lim_{\mathfrak{U}}m_{h,x}(d_\nu))$. Из оценок на норму $m_{h,x}$ следует, что $\tau$ --- ограниченный линейный оператор, причем $\Vert\tau\Vert\leq c\Vert R\Vert$. Для всех $a\in A$, $x\in F$ и $h\in (A\projtens\ell_1(B_F))^*$ выполнено
$$
\tau(h\cdot a)(x)-(\tau(h)\cdot a)(x)
=\tau(h\cdot a)(x)-\tau(h)(a\cdot x)
=\lim_{\mathfrak{U}}m_{h\cdot a,x}(d_\nu)-\lim_{\mathfrak{U}}m_{h,a\cdot x}(d_\nu).
$$
$$
=\lim_{\mathfrak{U}}m_{h,x}(a\cdot d_\nu)-m_{h,x}(d_\nu\cdot a)
=m_{h,x}\left(\lim_{\mathfrak{U}}(a\cdot d_\nu-d_\nu\cdot a)\right)
$$
$$
=m_{h,x}\left(\lim_{\nu}(a\cdot d_\nu-d_\nu\cdot a)\right)
=m_{h,x}(0)
=0.
$$
Следовательно, $\tau$ --- морфизм правых $A$-модулей. Далее, для всех $f\in F^*$ и $x\in F$ мы имеем
$$
(\tau(\pi_F^*)(f))(x)
=\lim_{\mathfrak{U}}m_{\pi_F^*(f),x}(d_\nu)
=\lim_{\mathfrak{U}}f(\Pi_A(d_\nu)\cdot x)
=\lim_{\nu}f(\Pi_A(d_\nu)\cdot x)
$$
$$
=f\left(\lim_{\nu}\Pi_A(d_\nu)\cdot x\right)
=f(x).
$$
Поэтому $\tau\pi_F^*=1_{F^*}$. Это значит, что $F^*$ --- $\langle$~$1$-ретракт / $c^2 C$-ретракт~$\rangle$ модуля $(A\projtens\ell_1(B_F))^*$
 в $\langle$~$\mathbf{mod}_1-A$ / $\mathbf{mod}-A$~$\rangle$. Последний модуль $\langle$~метрически / $1$-топологически~$\rangle$ инъективен, потому что $(A_+\projtens\ell_1(B_F))^*\isom{\mathbf{mod}_1-A}\mathcal{B}(A_+,\ell_\infty(B_F))$. По предложению $\langle$~\ref{RetrMetTopInjIsMetTopInj} / \ref{RetrCTopInjIsCTopInj}~$\rangle$ модуль $F^*$ также будет $\langle$~метрически / $c^2 C$-топологически~$\rangle$ инъективным.
\end{proof}

\begin{theorem}\label{TopL1FlatModAoverAmenBanAlg} Пусть $A$ --- относительно аменабельная банахова алгебра и $F$ --- левый банахов $A$-модуль, являющийся, как банахово пространство, $\mathscr{L}_1$-пространством. Тогда $F$ --- топологически плоский $A$-модуль.
\end{theorem}
\begin{proof} Так как $A$ аменабельна, то она обладает двусторонней ограниченной аппроксимативной единицей. По предложению \ref{DualBanModDecomp} существенный $A$-модуль $F_{ess}$ и аннуляторный $A$-модуль $F/F_{ess}$ являются $\mathscr{L}_1$-пространствами. Тогда из предложений \ref{MetTopEssL1FlatModAoverAmenBanAlg}, \ref{MetTopFlatAnnihModCharac} мы получаем, что $F_{ess}$ и $F/F_{ess}$ --- топологически плоские $A$-модули. Снова из предложению \ref{DualBanModDecomp} мы имеем $F^*\isom{\mathbf{mod}-A}F_{ess}^*\bigoplus_\infty (F/F_{ess})^*$. Учитывая вышесказанное, из предложений \ref{MetTopFlatCharac} и \ref{MetTopInjModProd} мы заключаем, что $F$ --- топологически плоский $A$-модуль.
\end{proof}

В топологической банаховой гомологии, в отличие от относительной, для некоторых алгебр можно дать полное описание плоских модулей.

\begin{corollary}\label{TopFlatModAoverAmenL1BanAlgCharac} Пусть $A$ --- относительно аменабельная банахова алгебра, являющаяся, как банахово пространство, $\mathscr{L}_1$-пространством. Тогда для банахова $A$-модуля $F$ следующие условия эквивалентны:

$i)$ $F$ --- топологически плоский $A$-модуль; 

$ii)$ $F$ является $\mathscr{L}_1$-пространством.
\end{corollary}
\begin{proof} Эквивалентность следует из предложения \ref{TopProjInjFlatModOverMthscrL1SpCharac} и теоремы \ref{TopL1FlatModAoverAmenBanAlg}.
\end{proof}

Теперь мы можем дать пример относительно плоского, но не топологически плоского идеала в банаховой алгебре. Рассмотрим $A=L_1(\mathbb{T})$. Известно, что $A$ имеет трансляционно инвариантное бесконечномерное замкнутое подпространство $I$ изоморфное гильбертову пространству [\cite{RosProjTransInvSbspLpG}, стр. 52]. Так как $I$ трансляционно инвариантно, то из [\cite{KaniBanAlg}, предложение 1.4.7] мы знаем, что $I$ является двусторонним идеалом в $A$. Тогда из [\cite{DefFloTensNorOpId}, следствие 23.3(4)] этот идеал не может быть $\mathscr{L}_1$-пространством. Тогда по следствию \ref{TopFlatModAoverAmenL1BanAlgCharac} идеал $I$ не является топологически плоским $A$-модулем. Мы утверждаем, что он все же относительно плоский. Так как $\mathbb{T}$ --- компактная группа, то она аменабельна [\cite{PierAmenLCA}, предложение 3.12.1]. Тогда $A$ --- относительно аменабельная банахова алгебра [\cite{HelBanLocConvAlg}, предложение VII.1.86], поэтому все ее левые идеалы, в частности $I$, являются относительно плоскими $A$-модулями [\cite{HelBanLocConvAlg}, предложение VII.1.60(I)].

Перейдем к обсуждению инъективных идеалов банаховых алгебр. Такие идеалы встретятся нам при изучении метрической и топологической инъективности $C^*$-алгебр.

\begin{proposition}\label{MetTopInjOfId} Пусть $I$ --- правый идеал банаховой алгебры $A$. Допустим, $I$ $\langle$~метрически / топологически~$\rangle$ инъективен как $A$-модуль. Тогда $I$ имеет $\langle$~левую единицу нормы $1$ / левую единицу~$\rangle$ и является ретрактом $A$ в $\langle$~$\mathbf{mod}_1-A$ / $\mathbf{mod}-A$~$\rangle$.
\end{proposition}
\begin{proof} Рассмотрим изометрическое вложение $\rho^+:I\to A_+$. Ясно, что это $A$-морфизм. Поскольку $I$ $\langle$~метрически / топологически~$\rangle$ инъективен, то $\rho^+$ имеет $\langle$~сжимающий / ограниченный~$\rangle$ левый обратный $A$-морфизм $\tau^+:A_+\to I$. Теперь для всех $x\in I$ мы имеем $x=\tau^+(\rho^+(x))=\tau^+(e_{A_+}\rho^+(x))=\tau^+(e_{A_+})\rho^+(x)=\tau^+(e_{A_+})x$. Другими словами $p=\tau^+(e_{A_+})\in I$ есть левая единица для $I$. Очевидно, $\Vert p\Vert\leq\Vert\tau^+\Vert\Vert e_{A_+}\Vert\leq\Vert\tau^+\Vert$, поэтому $\langle$~$\Vert p\Vert\leq 1$ / $\Vert p\Vert<\infty$~$\rangle$. Рассмотрим $A$-модульные операторы $
\rho:I\to A:x\mapsto x$ и $\tau:A\to I:x\mapsto p x$. Очевидно, они $\langle$~сжимающие / ограниченные~$\rangle$ морфизмы правых $A$-модулей и $\tau\rho=1_I$. Значит, $I$ есть ретракт $A$ в $\langle$~$\mathbf{mod}_1-A$ / $\mathbf{mod}-A$~$\rangle$.
\end{proof}

\begin{proposition}\label{ReduceInjIdToInjAlg} Пусть $I$ --- двусторонний идеал банаховой алгебры $A$, являющийся правым верным $I$-модулем. Тогда $I$ $\langle$~метрически / топологически~$\rangle$ инъективен как $A$-модуль тогда и только тогда, когда он $\langle$~метрически / топологически~$\rangle$ инъективен как $I$-модуль.
\end{proposition}
\begin{proof} Допустим, $I$ $\langle$~метрически / топологически~$\rangle$ инъективен как $A$-модуль, тогда по предложению \ref{MetTopInjOfId} он является ретрактом $A$ в $\langle$~$\mathbf{mod}_1-A$ / $\mathbf{mod}-A$~$\rangle$. Из пункта $ii)$ предложения \ref{MetTopInjUnderChangeOfAlg} мы получаем, что $I$-модуль $I$ $\langle$~метрически / топологически~$\rangle$ инъективен. 

Обратная импликация непосредственно следует из пункта $i)$ предложения \ref{MetTopInjUnderChangeOfAlg}.
\end{proof}
