% Chapter Template

\chapter{Предварительные сведения} % Main chapter title

\label{ChapterPreliminaries} % Change X to a consecutive number; for referencing this chapter elsewhere, use \ref{ChapterX}

В дальнейшем, в предложениях мы будем использовать сразу несколько фраз, последовательно перечисляя их и заключая в скобки таким образом: $\langle$~.../...~$\rangle$. Например: число $x$ называется $\langle$~положительным / неотрицательным~$\rangle$ если $\langle$~$x>0$ / $x\geq 0$~$\rangle$. Иногда некоторые части могут не  содержать фраз. Мы будем использовать символ $:=$ для обозначения равенства по определению.

Мы будем использовать следующие стандартные обозначения для множеств чисел: $\mathbb{C}$ --- множество комплексных чисел, $\mathbb{R}$ --- множество действительных чисел, $\mathbb{Z}$ --- множество целых чисел, $\mathbb{N}$ --- множество натуральных чисел, $\mathbb{N}_n$ --- множество первых $n$ натуральных чисел, $\mathbb{R}_+$ --- множество неотрицательных действительных чисел, $\mathbb{T}$ --- множество комплексных чисел с модулем равным $1$, наконец, $\mathbb{D}$ --- множество комплексных чисел с модулем меньшим $1$. Для $z\in\mathbb{C}$ символ $\overline{z}$ обозначает число, комплексно-сопряженное к $z$.

Для заданного отображения $f:M\to M'$ и подмножества $\langle$~$N\subset M$ / $N'\subset M'$ такого, что $\operatorname{Im}(f)\subset N'$~$\rangle$ через $\langle$~$f|_N$ / $f|^{N'}$~$\rangle$ мы обозначаем $\langle$~ограничение $f$ на $N$ / коограничение $f$ на $N'$~$\rangle$, то есть $\langle$~$f|_N:N\to M':x\mapsto f(x)$ / $f|^{N'}:M\to N':x\mapsto f(x)$~$\rangle$. Индикаторная функция подмножества $N$ обозначается как $\chi_{N}$, то есть $\chi_N(x)=1$ для $x\in N$ и $\chi_N(x)=0$ для $x\in M\setminus N$. Также мы будем активно использовать обозначение $\delta_x=\chi_{\{x\}}$, где $x\in M$. Через $\mathcal{P}(M)$ мы будем обозначать множество всех подмножеств $M$, а через $\mathcal{P}_0(M)$ --- множество всех конечных подмножеств $M$. Символ $M^N$, как обычно, будет обозначать множество всех отображений из $N$ в $M$. Мы будем писать $\operatorname{Card}(M)$ для обозначения мощности множества $M$. Через $\aleph_0$ мы будем обозначать мощность $\mathbb{N}$.


%----------------------------------------------------------------------------------------
%	Banach spaces
%----------------------------------------------------------------------------------------

\section{Банаховы пространства}
\label{SectionBanachSpaces}

Мы будем предполагать известными основные факты и конструкции функционального анализа, которые можно найти, например, в \cite{HelLectAndExOnFuncAn} или \cite{ConwACoursInFuncAn}. В этой работе мы также будем активно использовать результаты по геометрии банаховых пространств. Краткое введение можно найти в \cite{CarothShortCourseBanSp}, \cite{KalAlbTopicsBanSpTh} или \cite{FabHabBanSpTh}. Все банаховы пространства будут рассматриваться над полем комплексных чисел, если не оговорено иначе.

Пусть $E$ --- банахово пространство. Через $\langle$~$B_E$ / $B_E^\circ$~$\rangle$ мы будем обозначать $\langle$~замкнутый / открытый~$\rangle$ единичный шар пространства $E$ с центром в нуле. Если $S$  --- подмножество $E$, то через $\operatorname{cl}_E(S)$ мы будем обозначать замыкание $S$ в $E$. Если $F$ --- замкнутое подпространство в $E$, то $E/F$ будет обозначать факторпространство $E$ по $F$. Через $E^{cc}$ мы обозначим комплексно-сопряженное банахово пространство, то есть банахово пространство с тем же множеством векторов, что и у $E$, тем же сложением, но новым умножением на комплексно-сопряженные скаляры: $\alpha \overline{x}:=\overline{\overline{\alpha}x}$ для $\alpha\in\mathbb{C}$ и $x\in E$. Отметим, что элементы $E^{cc}$ мы обозначаем через $\overline{x}$. Очевидно, $(E^{cc})^{cc}=E$.

Теперь зафиксируем два банаховых пространства $E$ и $F$. Отображение $T:E\to F$ называется сопряженно-линейным, если соответствующее отображение $T:E^{cc}\to F$ линейно. Линейный оператор $T:E\to F$ называется:

$i)$ сжимающим, если его норма не превосходит $1$;

$ii)$ компактным, если $T(B_E)$ относительно компактно в $F$;

$iii)$ ядерным, если он может быть представлен как ряд одномерных операторов, абсолютно сходящийся по операторной норме.

Через $\langle$~$\mathcal{B}(E,F)$ / $\mathcal{K}(E,F)$ / $\mathcal{N}(E,F)$~$\rangle$ мы обозначим банахово пространство $\langle$~ограниченных / компактных / ядерных~$\rangle$ линейных операторов, действующих из $E$ в $F$. Если $F=E$, мы будем использовать обозначение $\langle$~$\mathcal{B}(E)$ / $\mathcal{K}(E)$ / $\mathcal{N}(E)$~$\rangle$ для этого пространства. Нормы в $\mathcal{B}(E,F)$ и $\mathcal{K}(E,F)$ суть операторные нормы. Норма ядерного оператора $T$ определяется равенством
$$
\Vert T\Vert:=\inf\left\{\sum_{n=1}^\infty\Vert S_n\Vert: T=\sum_{n=1}^\infty S_n,\quad (S_n)_{n\in\mathbb{N}} - \mbox{ одномерные операторы }\right\}.
$$

Пусть $E$, $F$ и $G$ --- три банаховых пространства. Тогда билинейный оператор $\Phi:E\times F\to G$ называется ограниченным, если его норма $\Vert \Phi\Vert:=\sup\{\Vert \Phi(x,y)\Vert:x\in B_E, y\in B_F\}$ конечна. Банахово пространство всех ограниченных билинейных операторов на $E\times F$ со значениями в $G$ будем обозначать через $\mathcal{B}(E\times F,G)$.

Ограниченный линейный оператор $T:E\to F$ будем называть:

$i)$ топологически инъективным, если он осуществляет гомеоморфизм на свой образ;

$ii)$ топологически сюръективным, если является открытым отображением;

$iii)$ коизометрическим, если $T(B_E^\circ)=B_F^\circ$;

$iv)$ строго коизометрическим, если $T(B_E)=B_F$. 

$v)$ $c$-топологически инъективным, если $\Vert x\Vert\leq c\Vert  T(x)\Vert$ для всех $x\in E$;

$vi)$ $c$-топологически сюръективным, если $cT(B_E^\circ)\supset B_F^\circ$;

$vii)$ строго $c$-топологически сюръективным, если $cT(B_E)\supset B_F$; 

Отметим, что $T$ топологически $\langle$~инъективен / сюръективен~$\rangle$ тогда и только тогда, когда он $c$-топологически $\langle$~инъективен / сюръективен~$\rangle$ для некоторой константы $c>0$. Очевидно, что $\langle$~коизометрические / строго коизометрические~$\rangle$ операторы это в точности сжимающие $\langle$~$1$-топологически сюръективные / строго $1$-топологически сюръективные~$\rangle$ операторы.

Два банаховых пространства $E$ и $F$ называются $\langle$~изометрически изоморфными / топологически изоморфными~$\rangle$, если существует ограниченный линейный оператор $T:E\to F$, который является $\langle$~изометрическим и сюръективным / топологически инъективным и топологически сюръективным~$\rangle$. Тот факт, что $E$ и $F$ являются $\langle$~изометрически изоморфными / топологически изоморфными~$\rangle$, мы будем записывать как $\langle$~$E\isom{\mathbf{Ban}_1}F$ / $E\isom{\mathbf{Ban}}F$~$\rangle$.

Еще один важный класс операторов --- это класс ограниченных проекторов. Ограниченный линейный оператор  $P:E\to E$ называется проектором, если $P^2=P$. Если $F=P(E)$, то мы будем говорить, что $P$ --- это проектор $E$ на $F$, и $F$ дополняемо в $E$. Если $\Vert P\Vert\leq C$, то мы будем говорить, что $F$ $C$-дополняемо в $E$. Другое эквивалентное определение говорит, что замкнутое подпространство $F$ в $E$ дополняемо, если существует замкнутое подпространство $G$ в $E$ такое, что $E\isom{\mathbf{Ban}}F\bigoplus G$. Все конечномерные пространства дополняемы, но не обязательно $1$-дополняемы. Пример $1$-дополняемого подпространства таков: для заданного банахова пространства $E$, пространство $E^*$ $1$-дополняемо в $E^{***}$ посредством проектора Диксмье $P=\iota_{E^*}(\iota_E)^*$, где $\iota_E$ обозначает естественное вложение $E$ в свое второе сопряженное $E^{**}$. Самый известный пример недополняемого пространства был построен Филлипсом. Он доказал, что банахово пространство $c_0(\mathbb{N})$ недополняемо в $\mathbb{\ell_\infty}(\mathbb{N})$ \cite{PhilOnLinTransfrm}.

Теперь рассмотрим алгебраическое тензорное произведение $E\otimes F$ банаховых пространств $E$ и $F$. Это линейное пространство можно наделить различными нормами, но самая важная среди них --- это проективная норма. Для заданного тензора $u\in E\otimes F$ мы определяем его проективную норму как
$$
\Vert u\Vert:=\inf\left\{\sum_{i=1}^n \Vert x_i\Vert\Vert y_i\Vert: u=\sum_{i=1}^n x_i\otimes y_i, (x_i)_{i\in\mathbb{N}_n}\subset E, (y_i)_{i\in\mathbb{N}_n}\subset F\right\}.
$$
Это действительно норма, но, вообще говоря, неполная. Символом $E\projtens F$ мы будем обозначать пополнение $E\otimes F$ по проективной норме. Получающееся пополнение мы будем называть проективным тензорным произведением банаховых пространств $E$ и $F$. Пусть $T:E_1\to E_2$ и $S:F_1\to F_2$ --- два  ограниченных линейных оператора между банаховыми пространствами. Тогда существует единственный ограниченный линейный оператор $T\projtens S:E_1\projtens F_1\to E_2\projtens F_2$ такой, что $(T\projtens S)(x\projtens y)=T(x)\projtens S(y)$ для всех $x\in E_1$ и $y\in F_1$. При этом $\Vert T\projtens S\Vert=\Vert T\Vert\Vert S\Vert$. Главное свойство проективного тензорного произведения, которое делает его таким важным, --- это свойство универсальности: для любых банаховых пространств $E$, $F$ и $G$ существует  изометрический изоморфизм:
$$
\mathcal{B}(E\projtens F,G)\isom{\mathbf{Ban}_1}\mathcal{B}(E\times F,G).
$$
Другими словами, проективное тензорное произведение линеаризует билинейные операторы. Также мы имеем следующие два  изометрических изоморфизма:
$$
\mathcal{B}(E\projtens F,G)
\isom{\mathbf{Ban}_1}\mathcal{B}(E,\mathcal{B}(F,G))
\isom{\mathbf{Ban}_1}\mathcal{B}(F,\mathcal{B}(E,G)).
$$
На алгебраическом тензорном произведении банаховых пространств существует много норм. Их детальное рассмотрение можно найти в \cite{DiestMetTheoryOfTensProd}.

Теперь перейдем к рассмотрению различных банаховых пространств функций.

Важный источник примеров банаховых пространств --- это $L_p$-пространства, также известные как пространства Лебега. Подробное обсуждение свойств $L_p$-пространств можно найти в  \cite{CarothShortCourseBanSp}. Прежде чем давать определения отметим, что мы будем рассматривать класс пространств с мерой более широкий, чем можно было ожидать. Пространство с мерой $(\Omega,\Sigma,\mu)$ называется строго локализуемым, если существует семейство измеримых множеств $(E_\lambda)_{\lambda\in\Lambda}$ конечной меры такое, что: 

$i)$ объединение $(E_\lambda)_{\lambda\in\Lambda}$ есть все пространство $\Omega$; 

$ii)$ множество $E$ измеримо тогда и только тогда, когда $E\cap E_\lambda$ измеримо для всех $\lambda\in\Lambda$; 

$iii)$ для любого измеримого множества $E$ выполнено $\mu(E)=\sum_{\lambda\in\Lambda}\mu(E\cap E_\lambda)$. 

\noindent
Класс строго локализуемых пространств с мерой огромен, он содержит все $\sigma$-конечные пространства с мерой, их произвольные объединения, меры Хаара локально компактных групп, считающие меры и многое другое. В дальнейшем мы будем рассматривать только строго локализуемые пространства с мерой и будем их просто называть пространствами с мерой. Итак, пусть $(\Omega,\Sigma,\mu)$ --- пространство с мерой. Множество $E$ называется пренебрежимым [\cite{FremMeasTh}, определение 112D] если оно содержится в множестве меры $0$. Будем говорить, что некоторое свойство выполняется почти всюду на $\Omega$, если оно нарушается только на пренебрежимом множестве. Функции на измеримом пространстве считаются эквивалентными если они равны почти всюду. Для $1\leq p <\infty$, как обычно, через $L_p(\Omega,\mu)$ будем обозначать банахово пространство классов эквивалентности функций $f:\Omega\to\mathbb{C}$ таких, что $|f|^p$ интегрируема по Лебегу по мере $\mu$. Норма такой функции определяется как 
$$
\Vert f\Vert_{L_p(\Omega,\mu)}:=\left(\int\limits_{\Omega}|f(\omega)|^pd\mu(\omega)\right)^{1/p}.
$$ 
Через $L_\infty(\Omega,\mu)$ мы будем обозначать банахово пространство классов эквивалентности ограниченных измеримых функций с нормой 
$$
\Vert f\Vert_{L_\infty(\Omega,\mu)}:=\inf\left\{\sup_{\omega\in\Omega\setminus N}|f(\omega)|:N\subset\Omega - \mbox{пренебрежимо}\right\}.
$$ 
Мы позволим себе вольность речи и будем говорить о функциях в $L_p(\Omega,\mu)$, а не о классах эквивалентности. Все равенства и неравенства, касающиеся функций из  $L_p$-пространств, будем считать выполненными почти всюду. Хорошо известно, что $L_p(\Omega,\mu)^*\isom{\mathbf{Ban}_1}L_{p^*}(\Omega,\mu)$ для $1\leq p<+\infty$ [\cite{FremMeasTh}, теоремы 243G, 244K]. Еще один хорошо известный факт  --- это рефлексивность $L_p$-пространств для $1<p<+\infty$. Здесь мы использовали стандартное обозначение $p^*=+\infty$ для $p=1$ и $p^*=p/(p-1)$ для $1<p<+\infty$.

Самые известный класс банаховых пространств --- это пространства непрерывных функций. Пусть $L$ --- локально компактное хаусдорфово пространство. Будем говорить, что функция $f:L\to\mathbb{C}$ исчезает на бесконечности, если для любого $\epsilon>0$ существует компакт $K\subset L$ такой, что $|f(t)|\leq\epsilon$ для всех $t\in L\setminus K$. Линейное пространство непрерывных на $L$ функций исчезающих на бесконечности обозначается $C_0(L)$. Наделенное $\sup$-нормой, $C_0(L)$ становится банаховым пространством. Любое множество $\Lambda$ с дискретной топологией является локально компактным пространством и, следуя стандартному обозначению, мы будем писать $c_0(\Lambda)$ вместо $C_0(\Lambda)$. Если $K$ --- хаусдорфов компакт, то все функции на $K$ исчезают на бесконечности, поэтому мы используем обозначение $C(K)$ для $C_0(K)$, чтобы подчеркнуть, что $K$ компактно. Некоторые банаховы пространства являются $C(K)$-пространствами в ``завуалированном'' виде. Например, для заданного пространства с мерой $(\Omega,\Sigma,\mu)$, пространство ограниченных измеримых функций $B(\Omega,\Sigma)$ с $\sup$-нормой или $L_\infty(\Omega,\mu)$ являются $C(K)$-пространствами для некоторого компактного хаусдорфова пространства $K$ [\cite{KalAlbTopicsBanSpTh}, замечание 4.2.8]. Если $L$ --- локально компактное хаусдорфово пространство, то через $M(L)$ обозначим банахово пространство комплексных конечных борелевских регулярных мер на $L$. Норма меры $\mu\in M(L)$ определяется равенством $\Vert\mu\Vert=|\mu|(L)$, где $|\mu|$ --- вариация меры $\mu$. По теореме Риса-Маркова-Какутани [\cite{ConwACoursInFuncAn}, параграф C.18] мы имеем $C_0(L)^*\isom{\mathbf{Ban}_1}M(L)$. На самом деле, $M(L)$ есть $L_1$-пространство [\cite{DalLauSecondDualOfMeasAlg}, обсуждение после предложения 2.14]. 

Следует упомянуть еще один важный подкласс $L_p$-пространств. Для заданного индексного множества $\Lambda$ и считающей меры $\mu_c:\mathcal{P}(\Lambda)\to[0,+\infty]$ соответствующее $L_p$-пространство обозначается как $\ell_p(\Lambda)$. Для этого типа пространств с мерой имеется еще один важный для нас изоморфизм: $c_0(\Lambda)^*\isom{\mathbf{Ban}_1}\ell_1(\Lambda)$. Для удобства мы положим по определению, что $c_0(\varnothing)=\ell_p(\varnothing)=\{0\}$ для $1\leq p\leq+\infty$. Примеры таких $L_p$-пространств дают мотивировку для следующей конструкции.

Пусть $\{E_\lambda:\lambda\in\Lambda\}$ --- произвольное семейство банаховых пространств. Для каждого $x\in \prod_{\lambda\in\Lambda} E_\lambda$ положим
$\Vert x\Vert_p=\Vert(\Vert x_\lambda\Vert)_{\lambda\in\Lambda}\Vert_{\ell_p(\Lambda)}$ для $1\leq p\leq +\infty$ и $\Vert x\Vert_0=\Vert(\Vert x_\lambda\Vert)_{\lambda\in\Lambda}\Vert_{c_0(\Lambda)}$. Тогда банахово пространство $\left\{x\in \prod_{\lambda\in\Lambda} E_\lambda: \Vert x\Vert_p<+\infty\right\}$ с нормой $\Vert\cdot\Vert_p$ мы будем обозначать как $\bigoplus_p\{E_\lambda:\lambda\in\Lambda\}$. Будем называть такие пространства $\bigoplus_p$-суммами банаховых пространств $\{E_\lambda:\lambda\in\Lambda\}$. Почти тавтологично утверждение, что $\ell_p(\Lambda)$ есть $\bigoplus_p$-сумма семейства $\{\mathbb{C}:\lambda\in\Lambda\}$. Важное свойство $\bigoplus_p$-сумм состоит в их связи с двойственностью:
$$
\left(\bigoplus\nolimits_p\{E_\lambda:\lambda\in\Lambda\}\right)^*\isom{\mathbf{Ban}_1}
\bigoplus\nolimits_{p^*}\{E_\lambda^*:\lambda\in\Lambda\}
$$
для $1\leq p<+\infty$ и 
$$
\left(\bigoplus\nolimits_0\{E_\lambda:\lambda\in\Lambda\}\right)^*\isom{\mathbf{Ban}_1}
\bigoplus\nolimits_1\{E_\lambda^*:\lambda\in\Lambda\}.
$$
Если $\{T_\lambda\in\mathcal{B}(E_\lambda, F_\lambda):\lambda\in\Lambda\}$ --- семейство ограниченных линейных операторов, то для любого $1\leq p\leq+\infty$ или $p=0$ корректно определен ограниченный линейный оператор
$$
T:\bigoplus\nolimits_p\{E_\lambda:\lambda\in\Lambda\}\to \bigoplus\nolimits_p\{ F_\lambda:\lambda\in\Lambda\}:x\mapsto \bigoplus\nolimits_p\{ T_\lambda(x_\lambda):\lambda\in\Lambda\},
$$
который мы будем обозначать как $\bigoplus_p\{T_\lambda:\lambda\in\Lambda\}$. Его норма равна $\sup_{\lambda\in\Lambda}\Vert T_\lambda\Vert$.

Теперь нам потребуется несколько определений из локальной теории банаховых пространств. Пусть $E$ и $F$ --- два топологически изоморфных банаховых пространства. Тогда расстояние Банаха-Мазура между ними определяется по формуле $d_{BM}(E,F):=\inf\{\Vert T\Vert\Vert T^{-1}\Vert: T \in \mathcal{B}(E,F) \mbox{ --- топологический изоморфизм}\}$. Если $E$ и $F$ не топологически изоморфны, то расстояние Банаха-Мазура между ними по определению равно бесконечности. Пусть $\mathcal{F}$ --- некоторое семейство конечномерных банаховых пространств. Говорят, что банахово пространство $E$ имеет $(\mathcal{F}, C)$-локальную структуру, если для каждого конечномерного подпространства $F$ в $E$ существует содержащее $F$ конечномерное подпространство $G$ в $E$ такое, что $d_{BM}(G,H)\leq C$ для некоторого $H$ из $\mathcal{F}$. Будем говорить, что $E$ имеет $\mathcal{F}$-локальную структуру, если оно имеет $(\mathcal{F},C)$-локальную структуру для некоторого $C\geq 1$. Один из самых важных примеров такого типа --- это так называемые $\mathscr{L}_p$-пространства. Впервые они были определены в новаторской работе \cite{LinPelAbsSumOpInLpSpAndApp} и стали незаменимым инструментом в локальной теории банаховых пространств. Для заданного $1\leq p\leq +\infty$ мы будем говорить, что банахово пространство $E$ является $\mathscr{L}_{p,C}$-пространством, если оно имеет $(\mathcal{F},C)$-локальную структуру для класса $\mathcal{F}$ конечномерных $\ell_p$-пространств. Если банахово пространство $E$ --- $\mathscr{L}_{p,C}$-пространство для некоторого $C\geq 1$, то говорят, что $E$ --- это $\mathscr{L}_p$-пространство. Ясно, что любое конечномерное банахово пространство является $\mathscr{L}_p$-пространством для всех $1\leq p\leq +\infty$. Любое $L_p$-пространство является $\mathscr{L}_p$-пространством  [\cite{DiestAbsSumOps}, теорема 3.2], но обратное неверно. В основном нас будут интересовать $\mathscr{L}_1$- и $\mathscr{L}_\infty$-пространства. Любое дополняемое подпространство $\langle$~$\mathscr{L}_1$-пространства / $\mathscr{L}_\infty$-пространства~$\rangle$ снова является $\langle$~$\mathscr{L}_1$-пространством / $\mathscr{L}_\infty$-пространством~$\rangle$ [\cite{BourgNewClOfLpSp}, предложение 1.28]. Сопряженное к $\langle$~$\mathscr{L}_1$-пространству / $\mathscr{L}_\infty$-пространству~$\rangle$ есть $\langle$~$\mathscr{L}_\infty$-пространство / $\mathscr{L}_1$-пространство~$\rangle$ [\cite{BourgNewClOfLpSp}, предложение 1.27]. Все $C(K)$-пространства являются $\mathscr{L}_\infty$-пространствами [\cite{DiestAbsSumOps}, теорема 3.2]. Отметим, что для заданного локально компактного хаусдорфова пространства $L$ банахово пространство $C_0(L)$ дополняемо в $C(\alpha L)$, где $\alpha L$ --- александровская компактификация $L$. Следовательно, $C_0(L)$-пространства тоже являются $\mathscr{L}_\infty$-пространствами. 

Впоследствии, подражая банаховым геометрам, мы будем говорить, что банахово пространство $E$ содержит $\langle$~изометрическую копию / копию~$\rangle$ банахова пространства $F$, если $F$ $\langle$~изометрически изоморфно / топологически изоморфно~$\rangle$ некоторому замкнутому подпространству в $E$.

%----------------------------------------------------------------------------------------
%	Banach algebras and their modules
%----------------------------------------------------------------------------------------

\section{Банаховы алгебры и их модули}
\label{SectionBanachAlgebrasAndTheirModules}

Банахова алгебра $A$ --- это банахово пространство со структурой ассоциативной алгебры над полем $\mathbb{C}$ и сжимающим билинейным оператором умножения $\cdot:A\times A\to A:(a,b)\mapsto ab$. Типичный пример коммутативной банаховой алгебры --- это алгебра непрерывных функций на компактном хаусдорфовом пространстве с поточечным умножением. Типичный пример некоммутативной алгебры --- алгебра ограниченных линейных операторов на гильбертовом пространстве с композицией в качестве умножения. Оба примера принадлежат к важному классу $C^*$-алгебр, которые мы обсудим позже. Под $\langle$~левым / правым / двусторонним~$\rangle$ идеалом $I$ банаховой алгебры $A$ мы всегда будем подразумевать замкнутую подалгебру $A$ такую, что $\langle$~$ax$ / $xa$ / $ax$ и $xa$~$\rangle$ принадлежит $I$ для всех $a\in A$ и $x\in I$.

Будем говорить, что элемент $p$ банаховой алгебры $A$ является $\langle$~левой / правой~$\rangle$  единицей в $A$, если $\langle$~$pa=a$ / $ap=a$~$\rangle$ для всех $a\in A$. Элемент который является левой и правой единицей называется единицей и обозначается $e_A$. Мы не предполагаем, что наши банаховы алгебры унитальны, то есть обладают единицей. Даже если банахова алгебра $A$ унитальна, мы не предполагаем, что ее единица имеет норму 1. Мы используем обозначение $A_+=A\bigoplus_1\mathbb{C}$ для стандартной унитизации банаховых алгебр. Умножение в $A_+$ определяется формулой $(a\oplus_1 z)(b\oplus_1 w)=(ab+wa+zb)\oplus_1 zw$, для $a,b\in A$ и $z,w\in\mathbb{C}$. Очевидно, $(0,1)$ есть единица в $A_+$. Через $A_\times$ мы будем обозначать условную унитизацию $A$, то есть $A_\times=A$, если $A$ имеет единицу нормы $1$ и $A_\times=A_+$ иначе. Даже в отсутствие единицы, в банаховой алгебре могут быть ее суррогаты называемые аппроксимативными единицами. Будем говорить, что направленность $(e_\nu)_{\nu\in N}$ в $A$ есть $\langle$~левая / правая / двусторонняя~$\rangle$ аппроксимативная единица, если $\langle$~$\lim_\nu e_\nu a=a$ / $\lim_\nu ae_\nu=a$ / $\lim_\nu e_\nu a=\lim_\nu ae_\nu=a$~$\rangle$ для всех $a\in A$. В этих определениях предполагается сходимость по норме. Если мы будем подразумевать сходимость в слабой топологии, то получим определение левой, правой и двусторонней слабой аппроксимативной единицы. Будем говорить, что аппроксимативная единица $(e_\nu)_{\nu\in N}$ $\langle$~ограниченная / сжимающая~$\rangle$, если величина $\sup_\nu\Vert e_\nu\Vert$ $\langle$~конечна / не превосходит $1$~$\rangle$. Иногда нам будет нужен следующий простой факт: если $A$ --- банахова алгебра с $\langle$~левой / правой~$\rangle$ единицей $p$ и $\langle$~правой / левой~$\rangle$ аппроксимативной единицей $(e_\nu)_{\nu\in N}$, то $A$ унитальна с единицей $p$ нормы $\lim_\nu\Vert e_\nu\Vert$. 

Для унитальной банаховой алгебры $A$ спектр $\operatorname{sp}_A(a)$ элемента $a\in A$ есть множество комплексных чисел $z$ таких, что $a-ze_A$ необратим в $A$. В банаховой алгебре спектр любого элемента не пуст и компактен в $\mathbb{C}$ [\cite{HelBanLocConvAlg}, следствие 2.1.16].

Характер банаховой алгебры $A$ --- это ненулевой гомоморфизм алгебр $\varkappa:A\to\mathbb{C}$. Все характеры непрерывны и содержатся в единичном шаре $A^*$ [\cite{HelBanLocConvAlg}, теорема 1.2.6]. Следовательно, мы можем рассматривать множество характеров с индуцированной слабой$^*$ топологией. Получающееся топологическое пространство хаусдорфово и локально компактно. Оно называется спектром банаховой алгебры $A$ и обозначается $\operatorname{Spec}(A)$. Если $A$ унитальна, то ее спектр компактен [\cite{HelBanLocConvAlg}, теорема 1.2.50]. Для заданной банаховой алгебры $A$ с непустым спектром мы можем построить сжимающий гомоморфизм $\Gamma_A:A\to C_0(\operatorname{Spec}(A)):a\mapsto(\varkappa\mapsto \varkappa(a))$, называемый преобразованием Гельфанда алгебры $A$ [\cite{HelBanLocConvAlg}, теорема 4.2.11]. Ядро этого гомоморфизма называется радикалом Джекобсона и обозначается $\operatorname{Rad}(A)$. Для банаховой алгебры $A$ с пустым спектром мы полагаем по определению $\operatorname{Rad}(A)=A$. Если $\operatorname{Rad}(A)=\{0\}$, то $A$ называется полупростой. По теореме Шилова об идемпотентах [\cite{KaniBanAlg}, глава 3.5] любая полупростая банахова алгебра с компактным спектром унитальна.

Большинство стандартных конструкций для банаховых пространств имеют свои аналоги для банаховых алгебр.
Например, $\bigoplus_p$-суммы банаховых алгебр с покоординатным умножением являются банаховыми алгебрами, фактор банаховой алгебры по двустороннему идеалу есть банахова алгебра. Даже проективное тензорное произведение двух банаховых алгебр является банаховой алгеброй, если определить умножение на элементарных тензорах так же, как и в чистой алгебре.

Перейдем к обсуждению наиболее важного класса банаховых алгебр. Пусть $A$ --- ассоциативная алгебра над полем $\mathbb{C}$, тогда сопряженно-линейный оператор ${}^*:A\to A$ называется инволюцией, если $(ab)^*=b^*a^*$ и $a^{**}=a$ для всех $a,b\in A$. Алгебры с инволюцией называются ${}^*$-алгебрами. Гомоморфизмы ${}^*$-алгебр, сохраняющие инволюцию, называются ${}^*$-гомоморфизмами. Банахова алгебра с изометрической инволюцией называется ${}^*$-банаховой алгеброй. Пример такой банаховой алгебры --- это алгебра ограниченных линейных оператров на гильбертовом пространстве с операцией взятия гильбертова сопряженного оператора в роли инволюции. Будем говорить, что ${}^*$-банахова алгебра $A$ есть $C^*$-алгебра, если для всех $a\in A$ выполнено равенство $\Vert a^*a\Vert=\Vert a\Vert^2$. Главное достоинство $C^*$-алгебр --- это их известные теоремы представления, доказанные Гельфандом и Наймарком. Первая теорема представления [\cite{HelBanLocConvAlg}, теорема 4.7.13] утверждает, что любая коммутативная $C^*$-алгебра $A$ изометрически изоморфна как ${}^*$-алгебра алгебре $C_0(\operatorname{Spec}(A))$. Вторая теорема [\cite{HelBanLocConvAlg}, теорема 4.7.57] дает описание произвольных $C^*$-алгебр как замкнутых ${}^*$-банаховых подалгебр в $\mathcal{B}(H)$ для некоторого гильбертова пространства $H$. Таких представлений может быть много, но норма (если она существует), которая превращает ${}^*$-алгебру в $C^*$-алгебру, всегда единственна. Если ${}^*$-подалгебра в $\mathcal{B}(H)$ слабо${}^*$ замкнута, то она называется алгеброй фон Нойманна. Если $C^*$-алгебра изоморфна как ${}^*$-алгебра алгебре фон Нойманна, то она называется $W^*$-алгеброй. По известной теореме Сакаи [\cite{BlackadarOpAlg}, теорема III.2.4.2] каждая $C^*$-алгебра, являющаяся сопряженным банаховым пространством, есть $W^*$-алгебра. Стоит отметить, что $W^*$-алгебра может быть представлена как не слабо${}^*$ замкнутая ${}^*$-подалгебра в $\mathcal{B}(H)$ для некоторого гильбертова пространства $H$. 

Многие стандартные конструкции переносятся с банаховых алгебр на $C^*$-алгебры, но не все. Например,  $\bigoplus_\infty$-сумма $C^*$-алгебр есть $C^*$-алгебра. Фактор $C^*$-алгебры по замкнутому двустороннему идеалу есть $C^*$-алгебра. В то же время проективное тензорное произведение $C^*$-алгебр редко имеет структуру $C^*$-алгебры, но существует много других способов наделить алгебраическое тензорное произведение таких алгебр структурой $C^*$-алгебры. 

Теперь перечислим несколько фактов о единицах и аппроксимативных единицах $C^*$-алгебр и их идеалов. Всякий замкнутый двусторонний идеал $C^*$-алгебры имеет двустороннюю сжимающую положительную аппроксимативную единицу [\cite{HelBanLocConvAlg}, теорема 4.7.79], и любой левый идеал  имеет правую сжимающую положительную аппроксимативную единицу. В некоторых случаях нам не будет достаточно даже аппроксимативной единицы, и для этого случая существует процедура наделения $C^*$-алгебры единицей, сохраняющая структуру $C^*$-алгебры [\cite{HelBanLocConvAlg}, предложение 4.7.6]. Этот тип унитизации мы будем обозначать как $A_\#$. До конца абзаца мы будем предполагать, что $A$ --- унитальная $C^*$-алгебра. Элемент $a\in A$ называется проектором (или ортогональным проектором), если $a=a^*=a^2$; самосопряженным, если $a=a^*$; положительным, если $a=b^*b$ для некоторого $b\in A$; унитарным, если $a^*a=aa^*=e_A$. Множество $A_{pos}$ всех положительных элементов в $A$ есть замкнутый конус в $A$. Если элемент $a\in A$ $\langle$~самосопряженный / положительный~$\rangle$, то $\langle$~$\operatorname{sp}_A(a)\subset[-\Vert a\Vert, \Vert a\Vert]$ / $\operatorname{sp}_A(a)\subset[0,\Vert a\Vert]$~$\rangle$. Для самосопряженного элемента $a\in A$, всегда существует единственный ${}^*$-гомоморфизм $\operatorname{Cont}_a:C(\operatorname{sp}_A(a))\to A$ такой, что $\operatorname{Cont}_a(f)=a$, где $f:\operatorname{sp}_A(a)\to\mathbb{C}:t\mapsto t$. Он называется непрерывным функциональным исчислением [\cite{HelBanLocConvAlg}, теорема 4.7.24]. Проще говоря, он позволяет брать непрерывные функции от самосопряженных элементов $C^*$-алгебр, поэтому мы используем стандартное обозначение $f(a)$ вместо $\operatorname{Cont}_a(f)$. Другой тесно связанный результат называется теоремой об отображении спектра и позволяет вычислять спектр элементов полученных с помощью непрерывного функционального исчисления: $\operatorname{sp}_A(f(a))=f(\operatorname{sp}_A(a))$. 

Наконец, мы переходим к обсуждению более общих объектов --- банаховых модулей. Пусть $A$ --- банахова алгебра. Будем говорить, что $X$ --- $\langle$~левый / правый~$\rangle$ банахов $A$-модуль, если $X$ --- это банахово пространство с билинейным оператором $\langle$~$\cdot:A\times X\to X$ / $\cdot: X\times A\to X$~$\rangle$ нормы не более $1$ (называемым внешним умножением) таким, что $\langle$~$a\cdot(b\cdot x)=ab\cdot x$ / $(x\cdot a)\cdot b=x\cdot ab$~$\rangle$ для всех $a,b\in A$ и $x\in X$. Любое банахово пространство $E$ можно наделить структурой $\langle$~левого / правого~$\rangle$ банахова $A$-модуля, положив по определению $\langle$~$a\cdot x=0$ / $x\cdot a=0$~$\rangle$ для всех $a\in A$ и $x\in E$. Любую банахову алгебру $A$ можно рассматривать как левый и правый банахов $A$-модуль --- внешнее умножение совпадает с умножением в алгебре. Конечно, есть и более содержательные примеры, которые мы встретим позже. Обычно мы будем обсуждать левые модули, потому что для их правых ``собратьев'' большинство определений и результатов аналогичны. Левый банахов модуль $X$ над унитальной банаховой алгеброй $A$ называется унитальным, если $e_A\cdot x=x$ для всех $x\in X$. Для заданного левого банахова $A$-модуля $X$ и подмножеств $S\subset A$, $M\subset X$ мы определим произведения $S\cdot M=\{a\cdot x:a\in S, x\in M\}$, $SM=\operatorname{span} (S\cdot M)$ и аннуляторы $S^{\perp M}=\{a\in S:a\cdot M=\{0\}\}$, ${}^{\perp S}M=\{x\in M: S\cdot x=\{0\}\}$. Существенная и аннуляторная часть $X$ определяются как $X_{ess}=\operatorname{cl}_X(A X)$ и $X_{ann}={}^{\perp A}X$, соответственно. Модуль $X$ называется $\langle$~верным / аннуляторным / существенным~$\rangle$, если $\langle$~$A^{\perp X}=\{0\}$ / $X=X_{ann}$ / $X=X_{ess}$~$\rangle$. Простое применение теоремы Хана-Банаха показывает, что $X$ --- существенный $A$-модуль тогда и только тогда, когда $X^*$ --- верный $A$-модуль.

Пусть $X$ и $Y$ --- два $\langle$~левых / правых~$\rangle$ банаховых $A$-модуля. Линейный оператор $\phi:X\to Y$ называется морфизмом $\langle$~левых / правых~$\rangle$ $A$-модулей, если $\langle$~$\phi(a\cdot x)=a\cdot \phi(x)$ / $\phi(x\cdot a)=\phi(x)\cdot a$~$\rangle$ для всех $a\in A$ и $x\in X$. Ограниченный морфизм $A$-модулей называется $A$-морфизмом. Множество $A$-морфизмов между $\langle$~левыми / правыми~$\rangle$ $A$-модулями $X$ и $Y$ обозначается как $\langle$~${}_A\mathcal{B}(X,Y)$ / $\mathcal{B}_A(X,Y)$~$\rangle$. Если $X$ и $Y$ являются $\langle$~левыми / правыми~$\rangle$ аннуляторными $A$-модулями, то $\langle$~${}_A\mathcal{B}(X,Y)=\mathcal{B}(X,Y)$ / $\mathcal{B}_A(X,Y)=\mathcal{B}(X,Y)$~$\rangle$.

Произвольный $A$-морфизм $\xi:X\to Y$ будем называть $\langle$~$C$-ретракцией / $C$-коретракцией~$\rangle$ если существует $A$-морфизм $\eta:Y\to X$ такой, что $\langle$~$\xi\eta=1_Y$ / $\eta\xi=1_X$~$\rangle$ и $\Vert\xi\Vert\Vert\eta\Vert\leq C$. Непосредственно из определения следует, что композиция $\langle$~$C_1$- и $C_2$-ретракции / $C_1$- и $C_2$-коретракции~$\rangle$ есть $\langle$~$C_1C_2$-ретракция / $C_1C_2$-коретракция~$\rangle$. Очевидно, $A$-морфизм сопряженный к $\langle$~$C$-ретракции / $C$-коретракции~$\rangle$ является $\langle$~$C$-коретракцией / $C$-ретракцией~$\rangle$.

Упомянем несколько конструкций над банаховыми модулями, которые в дальнейшем нам пригодятся. Любой левый банахов $A$-модуль можно рассматривать как унитальный банахов $A_+$-модуль, просто положив по определению $(a\oplus_1 z)\cdot x=a\cdot x+zx$ для всех $a\in A$, $x\in X$ и $z\in\mathbb{C}$. Линейное подпространство $ Y$ левого банахова $A$-модуля $X$ называется левым $A$-подмодулем в $X$ если $A\cdot Y\subset Y$. Например, любой левый идеал $I$ банаховой алгебры $A$ есть левый $A$-подмодуль в $A$. Если $Y$ --- замкнутый левый $A$-подмодуль в $X$, то банахово пространство $X/Y$ имеет структуру левого $A$-модуля с внешним умножением $a\cdot(x+Y)=a\cdot x+Y$ для всех $a\in A$ и $x+Y\in X/Y$. Такой объект называется фактор $A$-модулем. Фактормодули вида $A/I$, где $I$ --- левый идеал в $A$, называются циклическими. Мотивировку такого названия можно найти в [\cite{HelBanLocConvAlg}, предложение 6.2.2]. Очевидно, $X/X_{ess}$ --- аннуляторный $A$-модуль. Если $\{X_\lambda:\lambda\in\Lambda\}$ --- семейство левых банаховых $A$-модулей и $1\leq p\leq +\infty$ или $p=0$, то их $\bigoplus_p$-сумма есть левый банахов $A$-модуль с внешним умножением определенным равенством $a\cdot x=\bigoplus_p\{ a\cdot x_\lambda:\lambda\in\Lambda\}$, где $a\in A$, $x\in\bigoplus_p\{ X_\lambda:\lambda\in\Lambda\}$. Если $X$ --- левый банахов $A$-модуль и $E$ --- банахово пространство, то $\langle$~$\mathcal{B}(X,E)$ / $\mathcal{B}(E,X)$~$\rangle$ есть $\langle$~правый / левый~$\rangle$ банахов $A$-модуль с внешним умножением определенным равенством $\langle$~$(T\cdot a)(x)=T(a\cdot x)$ для всех $a\in A$, $x\in X$ и $T\in\mathcal{B}(X, E)$ / $(a\cdot T)(x)=a\cdot T(x)$ для всех $a\in A$, $x\in E$ и $T\in\mathcal{B}(E, X)$~$\rangle$. В частности, $X^*$  --- правый банахов $A$-модуль.

Проективное тензорное произведение банаховых пространств имеет свою модульную версию, которая называется проективным модульным тензорным произведением. Допустим $X$ --- правый, а $Y$ --- левый банахов $A$-модуль. Их проективное модульное тензорное произведение  $X\projmodtens{A}Y$ определяется как факторпространство $X\projtens Y / N$, где $N=\operatorname{cl}_{X\projtens Y}(\operatorname{span}\{x\cdot a\projtens y-x\projtens a\cdot y:x\in X,y\in Y,a\in A\})$. Пусть $\phi\in\mathcal{B}_A(X_1,X_2)$ и $\psi\in{}_A\mathcal{B}(Y_1,Y_2)$ где $X_1$, $X_2$ --- правые банаховы $A$-модули, а $Y_1$, $Y_2$ --- левые банаховы $A$-модули. Тогда существует единственный ограниченный линейный оператор $\phi\projmodtens{A} \psi:X_1\projmodtens{A} Y_1\to X_2\projmodtens{A} Y_2$ такой, что $(\phi\projmodtens{A} \psi)(x\projmodtens{A} y)=\phi(x)\projmodtens{A} \psi(y)$ для всех $x\in X_1$ и $y\in Y_1$. При этом $\Vert \phi\projmodtens{A} \psi\Vert\leq\Vert \phi\Vert\Vert \psi\Vert$. Проективное модульное тензорное произведение имеет свое свойство универсальности: для любого правого банахова $A$-модуля $X$, любого левого банахова $A$-модуля $Y$ и любого банахова пространства $E$ существует изометрический изоморфизм:
$$
\mathcal{B}(X\projmodtens{A}Y,E)\isom{\mathbf{Ban}_1}\mathcal{B}_{bal}(X\times Y, E),
$$
где $\mathcal{B}_{bal}(X\times Y, E)$ обозначает банахово пространство билинейных сбалансированных операторов $\Phi:X\times Y\to E$, то есть билинейных операторов со свойством $\Phi(x\cdot a,y)=\Phi(x,a\cdot y)$ для всех $x\in X$, $y\in Y$ и $a\in A$. Более того, имеются два изометрических изоморфизма:
$$
\mathcal{B}(X\projmodtens{A}Y,E)
\isom{\mathbf{Ban}_1}
{}_A\mathcal{B}(Y,\mathcal{B}(X,E))
\isom{\mathbf{Ban}_1}
\mathcal{B}_A(X,\mathcal{B}(Y,E)).
$$
Детальное обсуждение банаховых алгебр и банаховых модулей можно найти в \cite{HelBanLocConvAlg}, \cite{HelHomolBanTopAlg} или \cite{DalBanAlgAutCont}.

%----------------------------------------------------------------------------------------
%	Relative homology and rigged categories
%----------------------------------------------------------------------------------------

\section{Относительная гомология и оснащенные категории}
\label{SectionRelativeHomologyAndRiggedCategories}

Теперь нам нужно напомнить некоторые факты из теории категорий и договориться об обозначениях. Мы будем считать известными такие понятия теории категорий как категория, функтор, морфизм. Краткое  введение в теорию категорий можно найти в [\cite{HelLectAndExOnFuncAn}, глава 0] или [\cite{KashivShapCatsAndSheavs}, глава 1].

Для заданной категории $\mathbf{C}$ через $\operatorname{Ob}(\mathbf{C})$ мы будем обозначать класс ее объектов. Символ $\mathbf{C}^o$ обозначает противоположную категорию. Для объектов $X$ и $Y$ категории $\mathbf{C}$ через $\operatorname{Hom}_{\mathbf{C}}(X, Y)$ мы будем обозначать множество морфизмов из $X$ в $Y$. Часто мы будем писать $\phi:X\to Y$ вместо $\phi\in\operatorname{Hom}_{\mathbf{C}}(X,Y)$. Морфизм $\phi:X\to Y$ называется $\langle$~ретракцией / коретракцией~$\rangle$, если он имеет $\langle$~правый / левый~$\rangle$ обратный морфизм. Морфизм $\phi$ называется изоморфизмом, если он одновременно ретракция и коретракция. Наличие изоморфизма между $X$ и $Y$ мы будем записывать так: $X\isom{\mathbf{C}} Y$. Будем говорить, что морфизмы $\phi:X_1\to Y_1$ и $\psi:X_2\to Y_2$ эквивалентны в $\mathbf{C}$ если существуют изоморфизмы $\alpha:X_1\to X_2$ и $\beta:Y_1\to Y_2$ такие, что $\beta\phi=\psi\alpha$.

Приведем несколько примеров категорий важных для нас. Первая из них, конечно, категория всех множеств и всех отображений между ними. Мы обозначим ее $\mathbf{Set}$. Через $\mathbf{Ban}$ мы обозначим категорию банаховых пространств и ограниченных линейных операторов в роли морфизмов, а через $\mathbf{Ban}_1$ мы обозначим категорию банаховых пространств с сжимающими линейными операторами в роли морфизмов. Как следствие, $\operatorname{Hom}_{\mathbf{Ban}}(E,F)$ есть ничто иное как $\mathcal{B}(E,F)$. Символом $\langle A-\mathbf{mod}$ / $\mathbf{mod}-A\rangle$ мы будем обозначать категорию $\langle$~левых / правых~$\rangle$ $A$-модулей с непрерывными $A$-модульными операторами в роли морфизмов. Через $\langle$~~ $A-\mathbf{mod}_1$ / $\mathbf{mod}_1-A$~$\rangle$ мы будем обозначать подкатегорию $\langle$~$A-\mathbf{mod}$ / $\mathbf{mod}-A$~$\rangle$ с теми же объектами и лишь сжимающими морфизмами. Таким образом, $\langle$~ $\operatorname{Hom}_{A-\mathbf{mod}}(X,Y)={}_A\mathcal{B}(X,Y)$ /  $\operatorname{Hom}_{\mathbf{mod}-A}(X,Y)=\mathcal{B}_A(X,Y)$~$\rangle$.

Перейдем к обсуждению функторов. Два важных примера функторов, которые есть в каждой категории --- это функторы морфизмов. Для заданного объекта $X\in\operatorname{Ob}(\mathbf{C})$ можно определить ковариантный и контравариантный функторы
$$
\operatorname{Hom}_{\mathbf{C}}(X,-):\mathbf{C}\to\mathbf{Set}:Y\mapsto \operatorname{Hom}_{\mathbf{C}}(X,Y), \phi\mapsto(\psi\mapsto \phi\circ\psi),
$$
$$
\operatorname{Hom}_{\mathbf{C}}(-,X):\mathbf{C}\to\mathbf{Set}:Y\mapsto \operatorname{Hom}_{\mathbf{C}}(Y,X), \phi\mapsto(\psi\mapsto \psi\circ\phi).
$$

Пусть $E$ --- банахово пространство. Нам часто будут встречаться следующие функторы:
$$
\mathcal{B}(-,E):\mathbf{Ban}\to\mathbf{Ban},
\qquad\qquad
\mathcal{B}(E,-):\mathbf{Ban}\to\mathbf{Ban},
$$
$$
-\projtens E:\mathbf{Ban}\to\mathbf{Ban},
\qquad\qquad
E\projtens -:\mathbf{Ban}\to\mathbf{Ban}.
$$

В случае категорий модулей это будут функторы:
$$
\mathcal{B}(-,E):A-\mathbf{mod}\to \mathbf{mod}-A,
\qquad\qquad
\mathcal{B}(E,-):\mathbf{mod}-A\to \mathbf{mod}-A,
$$
$$
-\projmodtens{A} Y:\mathbf{mod}-A\to\mathbf{Ban},
\qquad\qquad
X\projmodtens{A} -:A-\mathbf{mod}\to\mathbf{Ban},
$$
где $E$ --- банахово пространство, $X$ --- правый $A$-модуль и $Y$ --- левый $A$-модуль. Все вышеупомянутые функторы имеют очевидные аналоги для категорий $\mathbf{Ban}_1$, $A-\mathbf{mod}_1$, $\mathbf{mod}_1-A$. Наконец, отметим, что хорошо известный функтор двойственности ${}^*$ есть ничто иное, как $\mathcal{B}(-,\mathbb{C})$.

Два ковариантных функтора $F:\mathbf{C}\to\mathbf{D}$ и $G:\mathbf{C}\to\mathbf{D}$ называются изоморфными, если существует класс изоморфизмов $\{\eta_X:X\in\operatorname{Ob}(\mathbf{C})\}$ в $\mathbf{D}$ таких, что $G(f)\circ\eta_X=\eta_Y\circ F(f)$ для всех $f:X\to Y$. В этом случае мы будем просто писать $F\cong G$. Пусть $F:\mathbf{C}\to\mathbf{D}$ --- $\langle$~ковариантный / контравариантный~$\rangle$ функтор, тогда $F$ называется представимым объектом $X$, если $\langle$~$F\cong\operatorname{Hom}_{\mathbf{C}}(X,-)$ / $F\cong\operatorname{Hom}_{\mathbf{C}}(-,X)$~$\rangle$. Если функтор представим, то его представляющий объект единственный с точностью до изоморфизма в $\mathbf{C}$.

Важнейшую роль для нас будут играть конструкции категорного произведения и копроизведения. Объект $X$ называется $\langle$~произведением / копроизведением~$\rangle$ семейства объектов $\{X_\lambda:\lambda\in\Lambda\}$, если функтор $\langle$~$\prod_{\lambda\in\Lambda}\operatorname{Hom}_{\mathbf{C}}(-,X_{\lambda}):\mathbf{C}\to\mathbf{Set}$ / $\prod_{\lambda\in\Lambda}\operatorname{Hom}_{\mathbf{C}}(X_{\lambda},-):\mathbf{C}\to\mathbf{Set}$~$\rangle$ представим объектом $X$. Как следствие, $\langle$~произведение / копроизведение~$\rangle$, если оно существует, единственно с точностью до изоморфизма. В таких категориях функционального анализа как $\mathbf{Ban}_1$, $A-\mathbf{mod}_1$ или $\mathbf{mod}_1-A$ любое семейство объектов имеет $\langle$~произведение / копроизведение~$\rangle$, и оно совпадает с $\langle$~$\bigoplus_1$-суммой / $\bigoplus_\infty$-суммой~$\rangle$ этого семейства. Аналогичное утверждение верно для категорий $\mathbf{Ban}$, $A-\mathbf{mod}$ и $\mathbf{mod}-A$, если ограничиться конечными семействами объектов [\cite{HelLectAndExOnFuncAn}, глава 2, параграф 5].

Далее мы кратко обсудим основы банаховой гомологии, созданной и активно изучаемой Хелемским и его школой. Зафиксируем произвольную банахову алгебру $A$. Будем говорить, что морфизм $\xi:X\to Y$ левых $A$-модулей $X$ и $Y$ есть относительно допустимый эпиморфизм, если он имеет правый обратный ограниченный линейный оператор. Левый $A$-модуль $P$ называется относительно проективным, если для любого относительно допустимого эпиморфизма $\xi:X\to Y$ и любого $A$-морфизма $\phi:P\to Y$ существует $A$-морфизм $\psi:P\to X$ такой, что диаграмма
$$
\xymatrix{
& {X} \ar[d]^{\xi}\\
{P} \ar@{-->}[ur]^{\psi} \ar[r]^{\phi} &{Y}}
$$
коммутативна. Такой $A$-морфизм $\psi$ называется подъемом морфизма $\phi$ и, вообще говоря, он не единственный. Аналогично, будем говорить, что морфизм $\xi:Y\to X$ правых $A$-модулей $X$ и $Y$ есть относительно допустимый мономорфизм, если он имеет левый обратный ограниченный линейный оператор. Правый $A$-модуль $J$ называется относительно инъективным, если для любого относительно допустимого мономорфизма $\xi:Y\to X$ и любого $A$-морфизма $\phi:Y\to J$ существует $A$-морфизм $\psi:X\to J$ такой, что диаграмма
$$
\xymatrix{
& {X} \ar@{-->}[dl]_{\psi} \\
{J} &{Y} \ar[l]_{\phi} \ar[u]_{\xi}}
$$
коммутативна. Такой $A$-морфизм $\psi$ называется продолжением $\phi$ и, вообще говоря, он не единственный.

Причина рассмотрения относительно допустимых морфизмов в этих определениях состоит в попытке отделить банахово-геометрические и алгебраические причины, по которым $A$-модуль может не быть относительно проективным или инъективным. Прямая проверка показывает, что ретракт относительно $\langle$~проективного / инъективного~$\rangle$ $A$-модуля снова относительно $\langle$~проективен / инъективен~$\rangle$. Очевидно, любой относительно допустимый $\langle$~эпиморфизм на относительно проективный $A$-модуль / мономорфизм из относительно инъективного $A$-модуля~$\rangle$ есть $\langle$~ретракция / коретракция~$\rangle$.

Специальный класс относительно $\langle$~проективных / инъективных~$\rangle$ $A$-модулей --- это так называемые относительно $\langle$~свободные / косвободные~$\rangle$ модули. Они имеют вид $\langle$~$A_+\projtens E$ / $\mathcal{B}(A_+,E)$~$\rangle$ для некоторого банахова пространства $E$. Главное свойство таких модулей состоит в том, что для любого $A$-модуля $X$ существует относительно $\langle$~свободный / косвободный~$\rangle$ $A$-модуль $F$ и относительно допустимый $\langle$~эпиморфизм $\xi:F\to X$ / мономорфизм $\xi:X\to F$~$\rangle$. Если $X$ относительно $\langle$~проективен / инъективен~$\rangle$, то мы немедленно заключаем, что морфизм $\xi$ есть $\langle$~ретракция / коретракция~$\rangle$. Таким образом, $A$-модуль относительно $\langle$~проективен / инъективен~$\rangle$ тогда и только тогда, когда он является ретрактом относительно $\langle$~свободного / косвободного~$\rangle$ $A$-модуля. 

Утверждения предыдущего абзаца имеют свои аналоги для многих других типов проективности и инъективности в других категориях математики \cite{SemadeniProjInjDual}. Более того, легко заметить, что проективность и инъективность, в некотором смысле, двойственны друг другу. Все эти наблюдения наводят на мысль, что существует общекатегорный подход к изучению свойств гомологически тривиальных объектов. Такой подход был предложен Хелемским в \cite{HelMetrFrQMod}. Как мы увидим, этот подход описывает относительную проективность и инъективность, а вышеупомянутые свойства являются простыми следствиями общих результатов. 

Пусть $\mathbf{C}$ и $\mathbf{D}$ --- две фиксированные категории. Упорядоченная пара ($\mathbf{C}, \square:\mathbf{C}\to\mathbf{D}$), где $\square$ --- верный ковариантный функтор, называется оснащенной категорией. Морфизм $\xi$ в $\mathbf{C}$ называется $\square$-допустимым эпиморфизмом если $\square (\xi)$ --- ретракция в $\mathbf{D}$. Объект $P$ в $\mathbf{C}$ называется $\square$-проективным, если для каждого $\square$-допустимого эпиморфизма $\xi$ в $\mathbf{C}$ отображение $\operatorname{Hom}_{\mathbf{C}}(P,\xi)$ сюръективно. Объект $F$ в $\mathbf{C}$ называется $\square$-свободным с базой $M$ в  $\mathbf{D}$, если существует изоморфизм функторов $\operatorname{Hom}_{\mathbf{D}}(M,\square(-))\cong\operatorname{Hom}_{\mathbf{C}}(F,-)$. Оснащенная категория $(\mathbf{C},\square)$ называется свободолюбивой [\cite{HelMetrFrQMod}, определение 2.10], если каждый объект в $\mathbf{D}$ является базой некоторого $\square$-свободного объекта из $\mathbf{C}$. Резюме предложений 2.3, 2.11  и 2.12 из \cite{HelMetrFrQMod} выглядит следующим образом:

$i)$ любой ретракт $\square$-проективного объекта $\square$-проективен;

$ii)$ любой $\square$-допустимый эпиморфизм в $\square$-проективный объект есть ретракция;

$iii)$ любой $\square$-свободный объект $\square$-проективен;

$iv)$ если $(\mathbf{C},\square)$ --- свободолюбивая оснащенная категория, то любой объект $\square$-проективен тогда и только тогда, когда он есть ретракт $\square$-свободного объекта;

$v)$ копроизведение семейства $\square$-проективных объектов $\square$-проективно.

Противоположной к оснащенной категории $(\mathbf{C}, \square)$ 
будем называть оснащенную категорию $(\mathbf{C}^{o},\square^{o}:\mathbf{C}^{o}\to\mathbf{D}^{o})$. 
Тогда, переходя к противоположной категории, мы можем определить допустимые мономорфизмы, инъективность и косвободу. Морфизм $\xi$ называется $\square$-допустимым мономорфизмом, если он $\square^o$-допустимый эпиморфизм. Объект $J$ из $\mathbf{C}$ называется $\square$-инъективным, если он $\square^o$-проективен. Наконец, объект $F$ из $\mathbf{C}$ называется $\square$-косвободным, если он $\square^o$-свободный. Следовательно, для инъективности и косвободы мы можем сформулировать результаты аналогичные тем, что были для проективности и свободы.

Теперь рассмотрим верный функтор $\square_{rel}:A-\mathbf{mod}\to\mathbf{Ban}$, который просто ``забывает'' модульную структуру. Легко видеть, что $(A-\mathbf{mod},\square_{rel})$ --- оснащенная категория, у которой $\square_{rel}$-допустимые $\langle$~эпиморфизмы / мономорфизмы~$\rangle$ в точности относительно допустимые $\langle$~эпиморфизмы / мономорфизмы~$\rangle$ и $\langle$~$\square_{rel}$-проективные / $\square_{rel}$-инъективные~$\rangle$ объекты в точности относительно $\langle$~проективные / инъективные~$\rangle$ $A$-модули. Более того, можно показать, что все $\langle$~$\square_{rel}$-свободные / $\square_{rel}$-косвободные~$\rangle$ объекты изоморфны в $A-\mathbf{mod}$ модулям вида $\langle$~$A_+\projtens E$ / $\mathcal{B}(A_+,E)$ ~$\rangle$ для некоторого банахова пространства $E$. Этот пример показывает, что относительная теория прекрасно вписывается в схему оснащенных категорий.

В этой работе мы применим данную схему к метрической и топологической теории гомологически тривиальных модулей. В этих теориях накладываются значительно более слабые ограничения на допустимые морфизмы. Поговорка ``много хочешь --- мало получишь'' отлично поясняет, что произойдет в следующих главах.
