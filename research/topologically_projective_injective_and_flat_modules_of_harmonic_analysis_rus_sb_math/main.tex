\documentclass{article}
\usepackage[T2A]{fontenc}
\usepackage[utf8]{inputenc}
\usepackage[english,russian]{babel}
\usepackage[tbtags]{amsmath}
\usepackage{amsfonts,amssymb,mathrsfs,amscd}
%для подключения графики используются стандартные команды, но кроме файла *.eps
%необходимо наличие в текущей директории соответствующего файла *.pdf
%-------------------------------------------------
\usepackage[hyper]{msb-a}
\JournalName{МАТЕМАТИЧЕСКИЙ СБОРНИК}
%\JournalName{}
%Пустой аргумент приводит к исчезновению всех атрибутов журнала "Математический
%сборник", файл можно представить в любой другой журнал
%-------------------------------------------------
\numberwithin{equation}{section}
%-------------------------------------------------
\theoremstyle{plain}
\newtheorem{theorem}{Теорема}
\newtheorem{lemma}{Лемма}[section]
\newtheorem{propos}{Предложение}
%-------------------------------------------------
\theoremstyle{definition}
\newtheorem{definition}{Определение}
\newtheorem{proof}{Доказательство}\def\theproof{}
\newtheorem{remark}{Замечание}
%-------------------------------------------------
% Мои определения
\newcommand{\projtens}{\mathbin{\widehat{\otimes}}}
\newcommand{\convol}{\ast}
\newcommand{\projmodtens}[1]{\mathbin{\widehat{\otimes}}_{#1}}
\newcommand{\isom}{\mathop{\mathbin{\cong}}}
\begin{document}
%-------------------------------------------------
\title{Топологически проективные, инъективные и плоские модули гармонического анализа}
\author[N.\,T.~Nemesh]{Н.\,Т.~Немеш}
\address{Московский государственный университет им.~М.\,В.~Ломоносова}
\email{nemeshnorbert@yandex.ru}

\date{31.10.2019}
\udk{517.968.22}

\maketitle

\begin{fulltext}

\begin{abstract} В работе изучаются гомологически тривиальные модули 
гармонического анализа на локально компактной группе $G$. Для $L_1(G)$- и 
$M(G)$-модулей $C_0(G)$, $L_p(G)$ и $M(G)$ даны критерии метрической и 
топологической проективности, инъективности и плоскости. В большинстве случаев 
модули обладающие этими свойствами должны быть конечномерными.

Библиография: 18 названий.
\end{abstract}

\begin{keywords} 
Банахов модуль, проективность, инъективность, плоскость, гармонический анализ.
\end{keywords}

\markright{Гомологически тривиальные модули гармонического анализа}

\footnotetext[0]{Работа выполнена при поддержке РФФИ (грант No. 19-01-00447).}

%----------------------------------------------------------------------------------------
%   Introduction	
%----------------------------------------------------------------------------------------

\section{Введение}\label{SectionIntroduction}

История банаховой гомологии начинается еще в 50-х годах прошлого века. 
Один из основных вопросов этой науки: является ли данный банахов модуль 
гомологически тривиальным, то есть проективным, инъективным или плоским? 
В качестве примера успешного ответа на этот вопрос можно привести работы 
Дейлса, Полякова, Рахера и Рамсдена 
\cite{DalPolHomolPropGrAlg, RamsHomPropSemgroupAlg, RachInjModAndAmenGr}, 
где они дали критерии гомологической тривиальности для классических модулей 
гармонического анализа. Следует сказать, что все эти результаты были получены 
для относительной банаховой гомологии. В этой статье мы ответим на те же 
самые вопросы, но для двух менее изученных версий банаховой гомологии --- 
метрической и топологической. Метрическая банахова гомология была впервые 
рассматрена в работе Гравена \cite{GravInjProjBanMod}, где он применяет передовые, 
на тот момент, гомологические и банахово-геометрические методы для изучения 
модулей гармонического анализа. Понятия топологической банаховой гомологии 
были определены в работе Уайта \cite{WhiteInjmoduAlg}. На первый взгляд, 
эта теория кажется намного менее ограничительной чем метрическая, но, как 
мы скоро увидим, это совсем не так.

%----------------------------------------------------------------------------------------
%	Preliminaries on Banach homoloogy
%----------------------------------------------------------------------------------------


\section{Предварительные сведения по банаховой гомологии}\label{SectionPreliminariesOnBanachHomology}

В дальнейшем, в предложениях мы будем использовать сразу несколько вариантов, 
последовательно перечисляя их и заключая в скобки таким образом: 
$\langle$~... / ...~$\rangle$. Например, число $x$ называется 
$\langle$~положительным / неотрицательным~$\rangle$ если 
$\langle$~$x>0$ / $x\geq 0$~$\rangle$.

Если не оговорено иначе, все банаховы пространства рассматриваются над полем 
комплексных чисел. Пусть $E$ --- банахово пространство, тогда через $B_E$ мы будем 
обозначать замкнутый единичный шар в $E$. Если $F$ --- еще одно банахово 
пространство, то мы будем говорить, что линейный оператор $T:E\to F$ является 
\emph{$\langle$~изометрическим / $c$-топологически инъективным~$\rangle$} если 
$\langle$~$\Vert T(x)\Vert=\Vert x\Vert$ / $c\Vert T(x)\Vert\geq\Vert x\Vert$~$\rangle$ 
для всех $x\in E$. Аналогично, $T$ называется \emph{$\langle$~строго коизометрическим / 
строго $c$-топологически сюръективным~$\rangle$} если $\langle$~$T(B_E)=B_F$ / 
$c T(B_E)\supset B_F$~$\rangle$. В некоторых случаях, мы будем опускать константу $c$. 
Для обозначения $\ell_p$-суммы банаховых пространств мы будем использовать 
символ $\bigoplus_p$, и $\projtens$ для проективного тензорного произведения. 

Далее, через $A$ мы будем обозначать произвольную банахову алгебру. Символом $A_+$ мы 
обозначим стандартную унитализацию $A$. Мы будем рассматривать банаховы модули только 
с сжимающим билинейным оператором внешнего умножения. Банахов $A$-модуль $X$ будем 
называть $\langle$~существенным / верным / аннуляторным~$\rangle$ если 
$\langle$~линейная оболочка множетсва $A\cdot X$ плотна в $X$ / $a\cdot X=\{0\}$ 
влечет $a=0$ / $A\cdot X=\{0\}$~$\rangle$. Всякий 
ограниченный линейный оператор являющийся морфизмом $A$-модулей мы будем называть 
$A$-морфизмом. Символ $A-\mathbf{mod}$ будет обозначать категорию левых банаховых 
$A$-модулей с $A$-морфизмами в качестве стрелок. Через $A-\mathbf{mod}_1$ мы обозначим 
подкатегорию $A-\mathbf{mod}$ с теми же объектами, но лишь сжимающими $A$-морфизмами. 
Аналогичные категории для правых $A$-модулей будем обозначать через $\mathbf{mod}-A$ 
и $\mathbf{mod}_1-A$, соответственно. Символом $\isom$ мы будем обозначать изоморфизм 
объектов в категории. Через $\projtens_A$ мы обозначим функтор модульного тензорного 
произведения, а стандартный функтор морфизмов через $\operatorname{Hom}$. Теперь мы 
можем дать наши основные определения.

\begin{definition} Левый банахов $A$-модуль $P$ называется
\emph{$\langle$~метрически / $C$-топологически / $C$-относительно~$\rangle$ проективным} 
если функтор морфизмов $\langle$~$\operatorname{Hom}_{A-\mathbf{mod}_1}(P,-)$ / $\operatorname{Hom}_{A-\mathbf{mod}}(P,-)$ / 
$\operatorname{Hom}_{A-\mathbf{mod}}(P,-)$~$\rangle$ переводит 
$\langle$~строго коизометрические морфизмы / строго $c$-топологически сюръективные 
морфизмы / морфизмы с правым обратным оператором нормы не более $c$~$\rangle$ в 
$\langle$~строго коизометрические / строго $c C$-топологически сюръективные / 
строго $c C$-топологически сюръективные~$\rangle$ операторы.
\end{definition}

\begin{definition} Правый банахов $A$-модуль $J$ называется \emph{$\langle$~метрически / 
$C$-топологически / $C$-относительно~$\rangle$ инъективным} если функтор морфизмов $\langle$~$\operatorname{Hom}_{\mathbf{mod}_1-A}(-,J)$ / 
$\operatorname{Hom}_{\mathbf{mod}-A}(-,J)$ / 
$\operatorname{Hom}_{\mathbf{mod}-A}(-,J)$~$\rangle$ переводит $\langle$~строго 
изометрические морфизмы / $c$-топологически инъективные морфизмы / морфизмы с 
левым обратным оператором нормы не более $c$~$\rangle$ в 
$\langle$~строго коизометрические / строго $c C$-топологически сюръективные / 
строго $c C$-топологически сюръективные~$\rangle$ операторы.
\end{definition}

\begin{definition} Левый банахов $A$-модуль $F$ называется \emph{$\langle$~метрически / 
$C$-топологически / $C$-относительно~$\rangle$ плоским} если функтор $-\projtens_A F$ 
переводит $\langle$~изометрические морфизмы / $c$-топологически инъективные морфизмы / 
морфизмы с левым обратным оператором нормы не более $c$~$\rangle$ в 
$\langle$~изометрические / $cC$-топологически инъективные / $cC$-топологически 
инъективные~$\rangle$ операторы.
\end{definition}

Для краткости мы будем называть банахов модуль \emph{$\langle$~топологически / 
относительно~$\rangle$} проективным, инъективным или плоским если он 
$\langle$~$C$-топологически / $C$-относительно~$\rangle$ проективный, инъективный 
или плоский для некоторого $C>0$.

Эти определения были изложены, в несколько иной форме, Гравеном для метрической 
теории \cite{GravInjProjBanMod},  Уайтом для топологической теории 
\cite{WhiteInjmoduAlg} и Хелемским для относительной \cite{HelemHomolDimNorModBanAlg}. 
В работе Уайта топологически проективные инъективные и плоские модули назывались 
соответственно строго проективными, инъективными и плоскими. Следует отметить, 
что понятие строго плоского плоского и строго инъективного модуля еще раньше 
были даны Хелемским в \cite[параграф~VII.1]{HelBanLocConvAlg}. Основы метрической, 
топологической и относительной теории можно найти в \cite{NemGeomProjInjFlatBanMod}. 
Мы будем активно использовать результаты этой статьи.

%----------------------------------------------------------------------------------------
%	Preliminaries on harmonic analysis
%----------------------------------------------------------------------------------------

\section{Предварительные сведения по гармоническому анализу}
\label{SectionPreliminariesOnHarmonicAnalysis} 

Пусть $G$ --- локально компактная группа с единицей $e_G$. Левая мера Хаара на $G$ 
будет обозначаться через $m_G$, а символ $\Delta_G$ будет использоваться для 
модулярной функции группы $G$. Для $\langle$~бесконечной дискретной / 
компактной~$\rangle$ группы $G$ мы будем нормировать меру $m_G$ так чтобы она 
была $\langle$~считающей / вероятностной~$\rangle$ мерой. В дальнейшем для всех 
$1\leq p\leq+\infty$ через $L_p(G)$ мы будем обозначать лебегово пространство функций 
интегрируемых со степенью $p$ по отношению к мере Хаара.

Мы будем рассматривать $L_1(G)$ как банахову алгебру со сверткой в качестве умножения. 
Эта банахова алгебра обладает сжимающей двусторонней аппроксимативной единицей 
\cite[теорема~3.3.23]{DalBanAlgAutCont}. Очевидно, алгебра $L_1(G)$ унитальна 
тогда и только тогда, когда группа $G$ дискретна. В этом случае индикаторная 
функция $e_G$, обозначим ее $\delta_{e_G}$, является единицей в $L_1(G)$. Аналогично, 
пространство комплексных конечных регулярных борелевских мер $M(G)$ со сверткой в 
качестве умножения становится унитальной банаховой алгеброй. Роль единицы играет 
мера Дирака $\delta_{e_G}$ сосредоточенная на $e_G$. Более того, $M(G)$ --- это 
копроизведение, в смысле теории категорий, в $L_1(G)-\mathbf{mod}_1$ 
(но не в $M(G)-\mathbf{mod}_1$) двустороннего идеала $M_a(G)$ мер абсолютно непрерывных 
по отношению к $m_G$ и подалгебры $M_s(G)$ состоящей из мер сингулярных по отношению 
к $m_G$. Заметим, что $M_a(G)\isom L_1(G)$ в $M(G)-\mathbf{mod}_1$ и $M_s(G)$ --- 
аннуляторный $L_1(G)$-модуль. Наконец, $M(G)=M_a(G)$ тогда и только тогда, когда 
группа $G$ дискретна.

Теперь приступим к обсуждению стандартных левых и правых модулей над алгебрами 
$L_1(G)$ и $M(G)$. Отметим, что банахова алгебра $L_1(G)$ является двусторонним 
идеалом в $M(G)$ посредством изометрического $M(G)$-морфизма левых и правых модулей 
$i:L_1(G)\to M(G):f\mapsto f m_G$. Следовательно, достаточно определить все модульные 
структуры над алгеброй $M(G)$. Для любых $1\leq p<+\infty$, 
$f\in L_p(G)$ и $\mu\in M(G)$ положим по определению
\[
(\mu\convol_p f)(s)=\int_G f(t^{-1}s)d\mu(t),
\qquad\qquad
(f\convol_p \mu)(s)=\int_G f(st^{-1})\Delta_G(t^{-1})^{1/p}d\mu(t)
\]
Эти внешние умножения превращают все банаховы пространства $L_p(G)$ для 
$1\leq p<+\infty$ в левые и правые $M(G)$-модули. Отметим, что для $p=1$ и 
$\mu\in M_a(G)$ мы получаем обычное определение свертки. Для $1<p\leq +\infty$, 
$f\in L_p(G)$ и $\mu\in M(G)$ мы определим
\[
(\mu\cdot_p f)(s)=\int_G \Delta_G(t)^{1/p}f(st)d\mu(t),
\qquad\qquad
(f\cdot_p \mu)(s)=\int_G f(ts)d\mu(t)
\]
Эти внешние умножения задают на всех пространствах $L_p(G)$ для $1<p\leq+\infty$ 
структуру левых и правых $M(G)$-модулей. Этот специальный выбор внешних умножений 
хорошо согласуется с двойственностью. Действительно, имеет место изоморфизм 
$(L_p(G),\convol_p)^*\isom (L_{p^*}(G),\cdot_{p^*})$ в $\mathbf{mod}_1-M(G)$ для 
всех $1\leq p<+\infty$. Тут мы полагаем по определению, что $p^*=p/(p-1)$ если 
$1<p<+\infty$ и $p^*=+\infty$ если $p=1$. Наконец, банахово пространство $C_0(G)$ 
также становится левым и правым $M(G)$-модулей с $\cdot_\infty$ в качестве 
внешнего умножения. Более того, $C_0(G)$ является левым и правым $M(G)$-подмодулем 
$L_\infty(G)$, причем  $(C_0(G),\cdot_\infty)^*\isom (M(G),\convol)$ в 
$M(G)-\mathbf{mod}_1$.

Через $\widehat{G}$ мы будем обозначать дуальную группу группы $G$. Любой характер 
$\gamma\in\widehat{G}$ задает непрерывный характер  
\[
\varkappa_\gamma^L:L_1(G)\to\mathbb{C}:f\mapsto \int_G f(s)\overline{\gamma(s)}d m_G(s),
\quad
\varkappa_\gamma^M:M(G)\to\mathbb{C}:\mu\mapsto\int_{G} \overline{\gamma(s)}d\mu(s).
\]
на $L_1(G)$ и $M(G)$ соответственно. Символом $\mathbb{C}_\gamma$ мы будем обозначать 
левый и правый аугментационный $L_1(G)$- или $M(G)$-модуль. Его внешние умножения 
определяются равенствами
\[
f\cdot_{\gamma}z=z\cdot_{\gamma}f=\varkappa_\gamma^L(f)z,
\qquad
\mu\cdot_{\gamma}z=z\cdot_{\gamma}\mu=\varkappa_\gamma^M(\mu)z
\]
для $f\in L_1(G)$, $\mu\in M(G)$ и $z\in\mathbb{C}$. 

Одно из многих определений аменабельной группы говорит, что локально компактная 
группа $G$ является аменабельной если существует $L_1(G)$-морфизм правых модулей $M:L_\infty(G)\to\mathbb{C}_{e_{\widehat{G}}}$ такой что $M(\chi_G)=1$ 
\cite[раздел~VII.2.5]{HelBanLocConvAlg}. Мы даже можем предполагать, что 
функционал $M$ сжимающий \cite[замечание~VII.1.54]{HelBanLocConvAlg}.

Все результаты этого раздела, для которых не было указано ссылок, подробно 
описаны в \cite[раздел~3.3]{DalBanAlgAutCont}.

%----------------------------------------------------------------------------------------
%	L_1(G)-modules
%----------------------------------------------------------------------------------------

\section{\texorpdfstring{$L_1(G)$}{L1(G)}-модули}
\label{SubSectionL1GModules}

Метрические гомологические свойства стандартных $L_1(G)$-модулей гармонического 
анализа впервые были изучены в \cite{GravInjProjBanMod}. Мы обобщим эти идеи на 
случай топологической банаховой гомологии. Чтобы прояснить определения мы начнем 
с общего результата об инъективности. Будет поучительно доказать его по определению.

\begin{propos}\label{AlgDualWithApproxIdIsMetrInj} Пусть $A$ --- банахова алгебра 
со сжимающей правой аппроксимативной единицей, тогда правый $A$-модуль $A^*$ 
метрически инъективен. 
\end{propos}
\begin{proof} Пусть $\xi:Y\to X$ --- изометрический $A$-морфизм правых $A$-модулей 
$X$ и $Y$ и пусть задан сжимающий $A$-морфизм $\phi: Y\to A^*$. По предположению 
$A$ обладает сжимающей аппроксимативной единицей, назовем ее $(e_\nu)_{\nu\in N}$. 
Для каждого $\nu\in N$ определим ограниченный линейный функционал
$f_\nu:Y\to\mathbb{C}:y\to \phi(y)(e_\nu)$. По теореме Хана-Банаха существует 
ограниченный линейный функционал $g_\nu:X\to\mathbb{C}$ такой что $g_\nu\xi=f_\nu$ 
и $\Vert g_\nu\Vert=\Vert f_\nu\Vert$. Легко проверить, что 
$\psi_\nu: X\to A^*:x\mapsto(a\mapsto g_\nu(x\cdot a))$ есть $A$-морфизм правых 
модулей такой, что $\Vert\psi_\nu\Vert\leq\Vert\phi\Vert$ и 
$\psi_\nu(\xi(y))(a)=\phi(y)(a e_\nu)$ для всех $y\in Y$ и $a\in A$. Поскольку 
направленность $(\psi_\nu)_{\nu\in N}$ ограничена по норме, существует
поднаправленность $(\psi_\mu)_{\mu\in M}$ с таким же ограничением на нормы, которая 
сходится в слабо${}^*$-операторной топологии к некоторому оператору $\psi:X\to A^*$.
Легко видеть, что $\psi$ является морфизмом правых $A$-модулей причем $\psi\xi=\phi$ и $\Vert\psi\Vert\leq\Vert\phi\Vert$. Поскольку $\phi$ произвольно, отображение $\operatorname{Hom}_{\mathbf{mod}_1-A}(\xi, A^*)$ строго коизометрично. 
Следовательно, модуль $A^*$ метрически инъективен.
\end{proof}

\begin{propos}\label{LInfIsL1MetrInj} Пусть $G$ --- локально компактная группа.
Тогда $L_\infty(G)$ метрически и топологически инъективен как $L_1(G)$-модуль. 
Как следствие, $L_1(G)$-модуль $L_1(G)$ метрически и топологически плоский.
\end{propos}
\begin{proof} Так как $L_1(G)$ обладает сжимающей аппроксимативной единицей, то 
по предложению \ref{AlgDualWithApproxIdIsMetrInj} правый $L_1(G)$-модуль 
$L_1(G)^*$ метрически инъективен. Как следствие, он топологически инъективен 
\cite[предложение~2.14]{NemGeomProjInjFlatBanMod}. Осталось напомнить, что 
$L_\infty(G)\isom L_1(G)^*$ в $\mathbf{mod}_1-L_1(G)$. Результат о плоскости 
$L_1(G)$ следует из \cite[предложение~2.21]{NemGeomProjInjFlatBanMod}.
\end{proof}

\begin{propos}\label{OneDimL1ModMetTopProjCharac} Пусть $G$ --- локально 
компактная группа и $\gamma\in\widehat{G}$. Тогда следующие условия эквивалентны:
\begin{enumerate}
    \item $G$ компактна;
    \label{OneDimL1ModMetTopProjCharac:i}
    \item $\mathbb{C}_\gamma$ --- метрически проективный $L_1(G)$-модуль;
    \label{OneDimL1ModMetTopProjCharac:ii}
    \item $\mathbb{C}_\gamma$ --- топологически проективный $L_1(G)$-модуль.
    \label{OneDimL1ModMetTopProjCharac:iii}
\end{enumerate}
\end{propos}
\begin{proof} \ref{OneDimL1ModMetTopProjCharac:i}) $\Longrightarrow$ 
\ref{OneDimL1ModMetTopProjCharac:ii}), \ref{OneDimL1ModMetTopProjCharac:iii}) 
$\Longrightarrow$ \ref{OneDimL1ModMetTopProjCharac:i}) Доказательство 
аналогично \cite[теорема~4.2]{GravInjProjBanMod}. 

\ref{OneDimL1ModMetTopProjCharac:ii}) $\Longrightarrow$ 
\ref{OneDimL1ModMetTopProjCharac:iii}) Импликация следует из 
\cite[предложение~2.4]{NemGeomProjInjFlatBanMod}.
\end{proof}

\begin{propos}\label{OneDimL1ModMetTopInjFlatCharac} Пусть $G$ --- локально 
компактная группа и $\gamma\in\widehat{G}$. Тогда следующие условия эквивалентны:
\begin{enumerate}
    \item $G$ аменабельна;
    \label{OneDimL1ModMetTopInjFlatCharac:i}
    \item $\mathbb{C}_\gamma$ --- метрически инъективный $L_1(G)$-модуль;
    \label{OneDimL1ModMetTopInjFlatCharac:ii}
    \item $\mathbb{C}_\gamma$ --- топологически инъективный $L_1(G)$-модуль.
    \label{OneDimL1ModMetTopInjFlatCharac:iii}
    \item $\mathbb{C}_\gamma$ --- метрически плоский $L_1(G)$-модуль;
    \label{OneDimL1ModMetTopInjFlatCharac:iv}
    \item $\mathbb{C}_\gamma$ --- топологически плоский $L_1(G)$-модуль.
    \label{OneDimL1ModMetTopInjFlatCharac:v}
\end{enumerate}
\end{propos}
\begin{proof} \ref{OneDimL1ModMetTopInjFlatCharac:i}) $\Longrightarrow$
\ref{OneDimL1ModMetTopInjFlatCharac:ii}), \ref{OneDimL1ModMetTopInjFlatCharac:iii}) 
$\Longrightarrow$ \ref{OneDimL1ModMetTopInjFlatCharac:i}) Доказательство аналогично 
\cite[теорема~4.5]{GravInjProjBanMod}.

\ref{OneDimL1ModMetTopInjFlatCharac:ii}) $\Longrightarrow$ 
\ref{OneDimL1ModMetTopInjFlatCharac:iii}) Импликация следует из 
\cite[предложение~2.14]{NemGeomProjInjFlatBanMod}.

\ref{OneDimL1ModMetTopInjFlatCharac:ii}) $\Longrightarrow$
\ref{OneDimL1ModMetTopInjFlatCharac:iv}), \ref{OneDimL1ModMetTopInjFlatCharac:iii}) 
$\Longrightarrow$ \ref{OneDimL1ModMetTopInjFlatCharac:v}) Напомним, что 
$\mathbb{C}_\gamma^*\isom \mathbb{C}_\gamma$ в $\mathbf{mod}_1-L_1(G)$, поэтому 
все эквивалентности следуют из трех предыдущих пунктов и того факта что плоские 
модули это в точности модули чьи сопряженные модули инъективны 
\cite[предложение~2.21]{NemGeomProjInjFlatBanMod}.
\end{proof}

В следующем предложении мы займемся изучением специальных идеалов банаховой 
алгебры $L_1(G)$, а именно идеалов вида $L_1(G)\convol\mu$ для некоторой 
идемпотентной меры $\mu$. На самом деле, этот класс идеалов в случае 
коммутативной группы совпадает с классом левых идеалов $L_1(G)$ 
обладающих правой аппроксимативной единицей.

\begin{theorem}\label{CommIdealByIdemMeasL1MetTopProjCharac} Пусть $G$ --- 
локально компактная группа и $\mu\in M(G)$ --- идемпотентная мера, то есть 
$\mu\convol\mu=\mu$. Допустим, что левый идеал $I=L_1(G)\convol\mu$ 
банаховой алгебры $L_1(G)$ топологически проективен как $L_1(G)$-модуль. 
Тогда $\mu=p m_G$ для некоторого $p\in I$.
\end{theorem}
\begin{proof} Рассмотрим произвольный морфизм $L_1(G)$-модулей 
$\phi:I\to L_1(G)$. Определим $L_1(G)$-морфизм 
$\phi':L_1(G)\to L_1(G):x\mapsto\phi(x\convol\mu)$. 
По теореме Венделя \cite[теорема~1]{WendLeftCentrzrs}, существует 
мера $\nu\in M(G)$ такая что $\phi'(x)=x\convol\nu$ для всех 
$x\in L_1(G)$. В частности, $\phi(x)=\phi(x\convol\mu)=\phi'(x)=x\convol\nu$ 
для любого $x\in I$. Отсюда следует, что 
$\psi:I\to I:x\mapsto\nu\convol x$ --- морфизм правых $I$-модулей такой, что 
$\phi(x)y=x\psi(y)$ для всех $x,y\in I$. Из 
\cite[лемма~2, пункт~\textup{(ii)}]{NemMetTopProjIdBanAlg} следует, что 
идеал $I$ обладает правой единицей, скажем $e\in I$. Тогда 
$x\convol\mu=x\convol\mu\convol e$ для всех $x\in L_1(G)$. Две меры 
равны если их свертки со всеми функциями из $L_1(G)$ совпадают 
\cite[следствие~3.3.24]{DalBanAlgAutCont}, поэтому 
$\mu=\mu\convol e m_G$. Так как $e\in I\subset L_1(G)$, то 
$\mu=\mu\convol e m_G\in M_a(G)$. Положим $p=\mu\convol e\in I$, 
тогда $\mu=p m_G$.
\end{proof}

Для случая метрической проективности наша гипотеза состоит в том, что 
левый идеал вида $L_1(G)\convol \mu$ для идемпотентной меры $\mu$ 
метрически проективен как $L_1(G)$-модуль тогда и только тогда, когда 
$\mu=p m_G$ для $p\in I$ нормы 1. В \cite{GravInjProjBanMod} Гравен 
дал критерий метрической проективности $L_1(G)$-модуля $L_1(G)$. 
Теперь мы можем получить этот результат как простое следствие.

\begin{theorem}\label{L1ModL1MetTopProjCharac} Пусть $G$ --- локально 
компактная группа. Тогда следующие условия эквивалентны:
\begin{enumerate}
    \item $G$ дискретна;
    \label{L1ModL1MetTopProjCharac:i}
    \item $L_1(G)$ --- метрически проективный $L_1(G)$-модуль;
    \label{L1ModL1MetTopProjCharac:ii}
    \item $L_1(G)$ --- топологически проективный $L_1(G)$-модуль.
    \label{L1ModL1MetTopProjCharac:iii}
\end{enumerate}
\end{theorem}
\begin{proof} \ref{L1ModL1MetTopProjCharac:i}) $\Longrightarrow$ 
\ref{L1ModL1MetTopProjCharac:ii}) Если группа $G$ дискретна, 
то $L_1(G)$ --- унитальная алгебра с единицей нормы $1$. Из 
\cite[предложение~7]{NemMetTopProjIdBanAlg} мы получаем, что 
$L_1(G)$ метрически проективен как $L_1(G)$-модуль.

\ref{L1ModL1MetTopProjCharac:ii}) $\Longrightarrow$ 
\ref{L1ModL1MetTopProjCharac:iii}) Импликация следует из 
\cite[предложение~2.4]{NemGeomProjInjFlatBanMod}.

\ref{L1ModL1MetTopProjCharac:iii}) $\Longrightarrow$ 
\ref{L1ModL1MetTopProjCharac:i}) Очевидно, $\delta_{e_G}$ --- идемпотентная 
мера. Так как $L_1(G)=L_1(G)\convol \delta_{e_G}$ --- метрически проективный 
$L_1(G)$-модуль, то из предложения \ref{CommIdealByIdemMeasL1MetTopProjCharac} 
следует, что $\delta_{e_G}=f m_G$ для некоторого $f\in L_1(G)$. Это возможно 
только если группа $G$ дискретна.
\end{proof}

Стоит отметить, что $L_1(G)$-модуль $L_1(G)$ относительно проективен для любой 
локально компактной группы $G$ \cite[упражнение~7.1.17]{HelBanLocConvAlg}.

\begin{propos}\label{L1MetTopProjAndMetrFlatOfMeasAlg} Пусть $G$ --- локально 
компактная группа. Тогда следующие условия эквивалентны:
\begin{enumerate}
    \item $G$ дискретна;
    \label{L1MetTopProjAndMetrFlatOfMeasAlg:i}
    \item $M(G)$ --- метрически проективный $L_1(G)$-модуль;
    \label{L1MetTopProjAndMetrFlatOfMeasAlg:ii}
    \item $M(G)$ --- топологически проективный $L_1(G)$-модуль;
    \label{L1MetTopProjAndMetrFlatOfMeasAlg:iii}
    \item $M(G)$ --- метрически плоский $L_1(G)$-модуль.
    \label{L1MetTopProjAndMetrFlatOfMeasAlg:iv}
\end{enumerate}
\end{propos}
\begin{proof} 
\ref{L1MetTopProjAndMetrFlatOfMeasAlg:i}) $\Longrightarrow$ 
\ref{L1MetTopProjAndMetrFlatOfMeasAlg:ii}) Как известно, $M(G)\isom L_1(G)$ 
в $L_1(G)-\mathbf{mod}_1$ когда группа $G$ дискретна, поэтому результат 
следует из теоремы \ref{L1ModL1MetTopProjCharac}. 

\ref{L1MetTopProjAndMetrFlatOfMeasAlg:ii}) $\Longrightarrow$ 
\ref{L1MetTopProjAndMetrFlatOfMeasAlg:iii}) Импликация следует из 
\cite[предложение~2.4]{NemGeomProjInjFlatBanMod}.

\ref{L1MetTopProjAndMetrFlatOfMeasAlg:ii}) $\Longrightarrow$ 
\ref{L1MetTopProjAndMetrFlatOfMeasAlg:iv}) Импликация следует из 
\cite[предложение~2.26]{NemGeomProjInjFlatBanMod}.

\ref{L1MetTopProjAndMetrFlatOfMeasAlg:iii}) $\Longrightarrow$ 
\ref{L1MetTopProjAndMetrFlatOfMeasAlg:i}) Так как 
$M(G)\isom L_1(G)\bigoplus_1 M_s(G)$ в $L_1(G)-\mathbf{mod}_1$, то модуль $M_s(G)$ 
топологически проективен как ретракт топологически проективного модуля 
\cite[предложение~2.2]{NemGeomProjInjFlatBanMod}. Заметим, что 
$M_s(G)$ --- аннуляторный $L_1(G)$-модуль, следовательно алгебра $L_1(G)$ 
обладает правой единицей \cite[предложение~3.3]{NemGeomProjInjFlatBanMod}. 
Поскольку $L_1(G)$ также обладает двусторонней аппроксимативной единицей, 
то алгебра $L_1(G)$ унитальна. Отсюда следует, что группа $G$ дискретна.

\ref{L1MetTopProjAndMetrFlatOfMeasAlg:iv}) $\Longrightarrow$ 
\ref{L1MetTopProjAndMetrFlatOfMeasAlg:i}) Поскольку 
$M(G)\isom L_1(G)\bigoplus_1 M_s(G)$ в $L_1(G)-\mathbf{mod}_1$, то 
$M_s(G)$ --- метрически плоский модуль как ретракт метрически плоского 
модуля \cite[предложение~2.27]{NemGeomProjInjFlatBanMod}. Так как $M_s(G)$ 
является аннуляторным $L_1(G)$-модулем над ненулевой алгеброй $L_1(G)$, 
то $M_s(G)$ должен быть нулевым модулем 
\cite[предложение~3.6]{NemGeomProjInjFlatBanMod}. Это возможно только 
если $G$ --- дискретная группа.
\end{proof}

\begin{propos}\label{MeasAlgIsL1TopFlat} Пусть $G$ --- локально компактная 
группа. Тогда $L_1(G)$-\\модуль $M(G)$ топологически плоский.
\end{propos}
\begin{proof} Напомним, что банахово пространство $M(G)$ является 
$L_1$-пространством и тем более $\mathscr{L}_1^g$-пространством 
\cite[пункт~3.13, упражнение~4.7(b)]{DefFloTensNorOpId}. Так как 
пространство $M_s(G)$ дополняемо в $M(G)$, то $M_s(G)$ так же является 
$\mathscr{L}_1^g$-пространством \cite[следствие~23.2.1(2)]{DefFloTensNorOpId}. 
Более того $M_s(G)$ --- аннуляторный $L_1(G)$-модуль, значит 
он топологически плоский $L_1(G)$-модуль 
\cite[предложение~3.6]{NemGeomProjInjFlatBanMod}. 
По предложению \ref{LInfIsL1MetrInj} топологически плоским является и 
$L_1(G)$-модуль $L_1(G)$. Снова используя изоморфизм 
$M(G)\isom L_1(G)\bigoplus_1 M_s(G)$ в $L_1(G)-\mathbf{mod}_1$, мы заключаем, 
что $L_1(G)$-модуль $M(G)$ топологически плоский как сумма 
топологически плоских модулей \cite[предложение~2.27]{NemGeomProjInjFlatBanMod}.
\end{proof}

%----------------------------------------------------------------------------------------
%	M(G)-modules
%----------------------------------------------------------------------------------------

\section{\texorpdfstring{$M(G)$}{M(G)}-модули}
\label{SubSectionMGModules}

Мы приступаем к обсуждению стандартных $M(G)$-модулей гармонического анализа. 
Как мы увидим, большая часть результатов может быть получена из предыдущих 
теорем и утверждений для $L_1(G)$-модулей.

\begin{propos}\label{MGMetTopProjInjFlatRedToL1} Пусть $G$ --- локально компактная 
группа и $X$ --- $\langle$~существенный / верный / существенный~$\rangle$ 
$L_1(G)$-модуль. Тогда,
\begin{enumerate}
    \item $X$ является метрически 
    $\langle$~проективным / инъективным / плоским~$\rangle$ $M(G)$-модулем 
    тогда и только тогда когда он метрически 
    $\langle$~проективный / инъективный / плоский~$\rangle$ как $L_1(G)$-модуль;
    \label{MGMetTopProjInjFlatRedToL1:i}
    \item $X$ является топологически 
    $\langle$~проективным / инъективным / плоским~$\rangle$ $M(G)$-модулем 
    тогда и только тогда, когда он топологически 
    $\langle$~проективный / инъективный / плоский~$\rangle$ как $L_1(G)$-модуль.
    \label{MGMetTopProjInjFlatRedToL1:ii}
\end{enumerate}
\end{propos}
\begin{proof} Напомним, что $L_1(G)$ --- двусторонний 1-дополняемый идеал алгебры 
$M(G)$. Теперь утверждения пунктов \ref{MGMetTopProjInjFlatRedToL1:i}) и 
\ref{MGMetTopProjInjFlatRedToL1:ii}) следуют из 
$\langle$~\cite[предложение~2.6]{NemGeomProjInjFlatBanMod} / 
\cite[предложение~2.16]{NemGeomProjInjFlatBanMod} / 
\cite[предложение~2.24]{NemGeomProjInjFlatBanMod}~$\rangle$.
\end{proof} 

Здесь следует упомянуть, что $L_1(G)$-модули $C_0(G)$, $L_p(G)$ для $1\leq p<\infty$ 
и $\mathbb{C}_\gamma$ для $\gamma\in\widehat{G}$ суть существенные модули, а 
$C_0(G)$, $M(G)$, $L_p(G)$ для $1\leq p\leq \infty$ и $\mathbb{C}_\gamma$ 
для $\gamma\in\widehat{G}$ суть верные $L_1(G)$-модули. 

\begin{propos}\label{MGModMGMetTopProjFlatCharac} Пусть $G$ --- локально компактная 
группа. Тогда $M(G)$ метрически и топологически проективен как $M(G)$-модуль. 
Как следствие, он является метрически и топологически плоским $M(G)$-модулем.
\end{propos} 
\begin{proof} Так как $M(G)$ --- унитальная алгебра с единицей нормы 1, то 
$\langle$~метрическая / топологическая~$\rangle$ проективность $M(G)$ следует 
из \cite[предложение~7]{NemMetTopProjIdBanAlg}, поскольку $M(G)$ можно 
рассмотреть унитальный как идеал алгебры $M(G)$. Остается напомнить что всякий 
$\langle$~метрически / топологически~$\rangle$ проективный модуль также 
является $\langle$~метрически / топологически~$\rangle$ плоским 
\cite[предложение~2.26]{NemGeomProjInjFlatBanMod}.
\end{proof}

%----------------------------------------------------------------------------------------
%	Banach geometric restrictions
%----------------------------------------------------------------------------------------

\section{Ограничения банаховой геометрии}
\label{SubSectionBanachGeometricRestriction}

В этом разделе мы покажем, что многие модули гармонического анализа не могут 
быть метрически или топологически проективными, инъективными или плоским 
по причинам своей плохой банаховой геометрии. В метрической теории для 
бесконечномерных $L_1(G)$-модулей $L_p(G)$, $M(G)$ и $C_0(G)$ это было 
сделано в \cite[теоремы~4.12--4.14]{GravInjProjBanMod}.
 
\begin{propos}\label{StdModAreNotRetrOfL1LInf} Пусть $G$ --- бесконечная 
локально компактная группа. Тогда
\begin{enumerate}
    \item банаховы пространства $L_1(G)$, $C_0(G)$, $M(G)$ и 
    $L_\infty(G)^*$ не являются топологически инъективными;
    \label{StdModAreNotRetrOfL1LInf:i}
    \item банаховы пространства $C_0(G)$ и $L_\infty(G)$ не 
    дополняемы ни в одном \\$L_1$-пространстве.
    \label{StdModAreNotRetrOfL1LInf:ii}
\end{enumerate}
\end{propos}
\begin{proof}
Так как $G$ --- бесконечная группа, то все рассматриваемые банаховы 
пространства бесконечномерны.

\ref{StdModAreNotRetrOfL1LInf:i}) Если бесконечномерное банахово пространство 
топологически инъективно, то оно содержит копию $\ell_\infty(\mathbb{N})$ 
\cite[следствие~1.1.4]{RosOnRelDisjFamOfMeas}, и как следствие 
копию $c_0(\mathbb{N})$. Банахово пространство $L_1(G)$ слабо 
секвенциально полно \cite[следствие~III.C.14]{WojBanSpForAnalysts}, 
поэтому из \cite[следствие~5.2.11]{KalAlbTopicsBanSpTh} мы знаем, 
что оно не может содержать копию $c_0(\mathbb{N})$. Таким образом, 
банахово пространство $L_1(G)$ не топологически инъективно. Допустим, 
что пространство $M(G)$ топологически инъективно, тогда инъективно и 
его дополняемое подпространство $M_a(G)$, изоморфное $L_1(G)$. Это 
противоречит рассуждениям выше. Из 
\cite[следствие~3]{LauMingComplSubspInLInfOfG} известно, что банахово 
пространство $C_0(G)$ не дополняемо в $L_\infty(G)$, следовательно оно 
не может быть топологически инъективным. Напомним, что пространство 
$L_1(G)$ дополняемо в $L_\infty(G)^*$, которое в свою очередь 
изометрически изоморфно $L_1(G)^{**}$ \cite[предложение~B10]{DefFloTensNorOpId}. 
Следовательно, если банахово пространство $L_\infty(G)^*$ топологически 
инъективно, то $L_1(G)$ тоже будет инъективным. Это противоречит 
рассуждениям выше.

\ref{StdModAreNotRetrOfL1LInf:ii}) Допустим, $C_0(G)$ является ретрактом 
некоторого $L_1$-пространства, тогда пространство $M(G)$, которое, как 
известно, изометрически изоморфно $C_0(G)^*$, будет ретрактом некоторого 
$L_\infty$-пространства. Следовательно, $M(G)$ --- топологически 
инъективное банахово пространство. Это противоречит пункту
\ref{StdModAreNotRetrOfL1LInf:i}). Так как пространство 
$\ell_\infty(\mathbb{N})$ вкладывается в $L_\infty(G)$, то существует 
и вложение пространства $c_0(\mathbb{N})$. Если $L_\infty(G)$ ретракт 
некоторого $L_1$-пространства, то такое $L_1$-пространство будет содержать 
копию $c_0(\mathbb{N})$. Как было показано в пункте 
\ref{StdModAreNotRetrOfL1LInf:i}) это невозможно.
\end{proof}

С этого момента через $A$ мы будем обозначать одну из алгебр $L_1(G)$ 
или $M(G)$. Напомним, что $L_1(G)$ и $M(G)$ являются  $L_1$-пространствами.

\begin{propos}\label{StdModAreNotL1MGMetTopProjInjFlat} Пусть $G$ --- 
бесконечная локально компактная группа. Тогда
\begin{enumerate}
    \item $A$-модули $C_0(G)$ и $L_\infty(G)$ не являются ни 
    метрически ни топологически проективными;
    \label{StdModAreNotL1MGMetTopProjInjFlat:i}
    \item $A$-модули $L_1(G)$, $C_0(G)$, $M(G)$ и $L_\infty(G)^*$ 
    не являются ни метрически ни топологически инъективными;
    \label{StdModAreNotL1MGMetTopProjInjFlat:ii}
    \item $A$-модули $L_\infty(G)$ и $C_0(G)$ не являются 
    ни метрически ни топологически плоскими.
    \label{StdModAreNotL1MGMetTopProjInjFlat:iii}
    \item $A$-модули $L_p(G)$ для $1<p<\infty$ не являются 
    ни метрически ни топологически проективными, инъективными 
    или плоскими.
    \label{StdModAreNotL1MGMetTopProjInjFlat:iv}
\end{enumerate}
\end{propos}
\begin{proof} \ref{StdModAreNotL1MGMetTopProjInjFlat:i}) Каждый метрически 
или топологически проективный $A$-модуль дополняем в некотором 
$L_1$-пространстве \cite[предложение~3.8]{NemGeomProjInjFlatBanMod}. 
Остается применить пункт \ref{StdModAreNotRetrOfL1LInf:ii}) предложения
\ref{StdModAreNotRetrOfL1LInf}.

\ref{StdModAreNotL1MGMetTopProjInjFlat:ii}) Каждый метрически или 
топологически инъективный $A$-модуль является топологически 
инъективным банаховым пространством 
\cite[предложение~3.8]{NemGeomProjInjFlatBanMod}. Теперь результат 
следует из пункта \ref{StdModAreNotRetrOfL1LInf:i}) предложения 
\ref{StdModAreNotRetrOfL1LInf}.

\ref{StdModAreNotL1MGMetTopProjInjFlat:iii}) Напомним, что 
$C_0(G)^*\isom M(G)$ в $\mathbf{mod}_1-A$. Теперь достаточно 
скомбинировать результаты пункта 
\ref{StdModAreNotL1MGMetTopProjInjFlat:i}) и тот факт, что 
модуль сопряженный к плоскому инъективен 
\cite[предложение~2.21]{NemGeomProjInjFlatBanMod}.

\ref{StdModAreNotL1MGMetTopProjInjFlat:iv}) Так как пространства 
$L_p(G)$ рефлексивны для $1<p<\infty$ то достаточно применить 
\cite[следствие~3.14]{NemGeomProjInjFlatBanMod}.
\end{proof}

Осталось рассмотреть метрические и топологические гомологические 
свойства $A$-модулей для конечной группы $G$.

\begin{propos}\label{LpFinGrL1MGMetrInjProjCharac} Пусть $G$ --- 
нетривиальная конечная группа и $1\leq p\leq +\infty$. Тогда $A$-модуль 
$L_p(G)$ является метрически 
$\langle$~проективным / инъективным~$\rangle$ тогда и только тогда, 
когда $\langle$~$p=1$ / $p=+\infty$~$\rangle$.
\end{propos}
\begin{proof} 
Допустим, что $A$-модуль $L_p(G)$ метрически 
$\langle$~проективен / инъективен~$\rangle$. Поскольку пространство 
$L_p(G)$ конечномерно, то в силу 
$\langle$~проективности / инъективности~$\rangle$ должны существовать 
изометрические изоморфизмы 
$\langle$~$L_p(G)\isom \ell_1(\mathbb{N}_n)$ / 
$L_p(G)\isom \ell_\infty(\mathbb{N}_n)$ ~$\rangle$ 
\cite[предложение~3.8, пункты~\textup{(i)}, \textup{(ii)}]{NemGeomProjInjFlatBanMod}, 
где $n=\operatorname{Card}(G)>1$. Теперь воспользуемся результатом 
теоремы 1 из \cite{LyubIsomEmdbFinDimLp} для банаховых пространств над 
полем $\mathbb{C}$: если для $2\leq m\leq k$ и $1\leq r,s\leq \infty$, 
существует изометрическое вложение из $\ell_r(\mathbb{N}_m)$ в 
$\ell_s(\mathbb{N}_k)$, то либо $r=2$, $s\in 2\mathbb{N}$ либо $r=s$. 
Таким образом, $\langle$~$p=1$ / $p=+\infty$~$\rangle$. Обратная импликация 
легко следует из $\langle$~теоремы \ref{L1ModL1MetTopProjCharac} / 
предложения \ref{LInfIsL1MetrInj}~$\rangle$.
\end{proof}

\begin{propos}\label{StdModFinGrL1MGMetrInjProjFlatCharac} 
Пусть $G$ --- конечная группа. Тогда,
\begin{enumerate}
    \item $A$-модули $C_0(G)$ и $L_\infty(G)$ метрически инъективны;
    \label{StdModFinGrL1MGMetrInjProjFlatCharac:i}
    \item $A$-модули $C_0(G)$ и $L_p(G)$ для $1<p\leq +\infty$ 
    метрически проективны тогда и только тогда, когда группа $G$ тривиальна;
    \label{StdModFinGrL1MGMetrInjProjFlatCharac:ii}
    \item $A$-модули $M(G)$ и $L_p(G)$ для $1\leq p< +\infty$ 
    метрически инъективны тогда и только тогда, когда группа $G$ тривиальна;
    \label{StdModFinGrL1MGMetrInjProjFlatCharac:iii}
    \item $A$-модули $C_0(G)$ и $L_p(G)$ для $1<p\leq +\infty$ 
    метрически плоские тогда и только тогда, когда группа $G$ тривиальна.
    \label{StdModFinGrL1MGMetrInjProjFlatCharac:iv}
\end{enumerate}
\end{propos}
\begin{proof}
\ref{StdModFinGrL1MGMetrInjProjFlatCharac:i}) Так как группа $G$ конечна, то 
$C_0(G)=L_\infty(G)$. Теперь необходимый результат следует из 
предложения \ref{LInfIsL1MetrInj}.

\ref{StdModFinGrL1MGMetrInjProjFlatCharac:i}) Если группа $G$ тривиальна, 
то есть $G=\{e_G\}$, то $L_p(G)=C_0(G)=L_1(G)$. Осталось 
воспользоваться пунктом \ref{StdModFinGrL1MGMetrInjProjFlatCharac:i}). 
Если группа $G$ нетривиальна, то достаточно вспомнить, что 
$C_0(G)=L_\infty(G)$ и использовать предложение \ref{LpFinGrL1MGMetrInjProjCharac}.

\ref{StdModFinGrL1MGMetrInjProjFlatCharac:iii}) Если $G=\{e_G\}$, 
то $M(G)=L_p(G)=L_\infty(G)$ и можно снова использовать пункт
\ref{StdModFinGrL1MGMetrInjProjFlatCharac:i}). Если группа $G$ нетривиальна, 
то можно применить предложение \ref{LpFinGrL1MGMetrInjProjCharac} поскольку 
в этом случае $M(G)=L_1(G)$.

\ref{StdModFinGrL1MGMetrInjProjFlatCharac:iv}) Из пункта 
\ref{StdModFinGrL1MGMetrInjProjFlatCharac:iii}) следует, что для 
$1\leq p<+\infty$ модули $L_p(G)$ метрически инъективны тогда и только тогда, 
когда группа $G$ тривиальна. Напомним, что банахов модуль плоский тогда и 
только тогда, когда его сопряженный модуль инъективен 
\cite[предложение~2.21]{NemGeomProjInjFlatBanMod}. Теперь результат 
для модулей $L_p(G)$ следует из отождествления $L_p(G)^*\isom L_{p^*}(G)$ 
в $\mathbf{mod}_1-L_1(G)$ для $1\leq p^*< +\infty$. Аналогично 
используя характеризацию плоских модулей и изоморфизмы 
$C_0(G)^*\isom M(G)\isom L_1(G)$ в $\mathbf{mod}_1-L_1(G)$ мы получаем 
критерий инъективности для $M(G)$.
\end{proof}

Следует сказать, что если бы мы рассматривали все банаховы пространства 
над полем действительных чисел, то модули $L_\infty(G)$ и $L_1(G)$ 
были бы метрически проективны и инъективны соответственно, для группы $G$ 
состоящей из двух элементов. Причина в том, что $L_\infty(\mathbb{Z}_2)\isom \mathbb{R}_{\gamma_0}\bigoplus\nolimits_1\mathbb{R}_{\gamma_1}$ в 
$L_1(\mathbb{Z}_2)-\mathbf{mod}_1$ и $L_1(\mathbb{Z}_2)\isom \mathbb{R}_{\gamma_0}\bigoplus\nolimits_\infty\mathbb{R}_{\gamma_1}$ в 
$\mathbf{mod}_1-L_1(\mathbb{Z}_2)$. Здесь, $\mathbb{Z}_2$ обозначает 
единственную группу из двух элементов и 
$\gamma_0,\gamma_1\in\widehat{\mathbb{Z}_2}$ -- ее характеры задаваемые 
равенствами $\gamma_0(0)=\gamma_0(1)=\gamma_1(0)=-\gamma_1(1)=1$.

\begin{propos}\label{StdModFinGrL1MGTopInjProjFlatCharac} Пусть $G$ --- конечная группа. 
Тогда $C_0(G)$, $M(G)$ и $L_p(G)$ для $1\leq p\leq +\infty$ являются топологически 
проективными, инъективными и плоскими $A$-модулями.
\end{propos} 
\begin{proof}
Для конечной группы $G$ мы имеем $M(G)=L_1(G)$ и $C_0(G)=L_\infty(G)$, поэтому 
модули $C_0(G)$ и $M(G)$ не требуют специального рассмотрения. Поскольку 
$M(G)=L_1(G)$, мы можем ограничиться случаем $A=L_1(G)$. Тождественное 
отображение $i:L_1(G)\to L_p(G):f\mapsto f$ является топологическим 
изоморфизмом банаховых пространств, так как $L_1(G)$ и $L_p(G)$ для 
$1\leq p<+\infty$ имеют одинаковую размерность. Так как группа $G$ конечна, 
то она унимодулярна. Поэтому, совпадают внешние умножения в 
$(L_1(G),\convol)$ и $(L_p(G),\convol_p)$ для $1\leq p<+\infty$. 
Следовательно $i$ --- изоморфизм в $L_1(G)-\mathbf{mod}$ и 
$\mathbf{mod}-L_1(G)$. Аналогично можно показать, что модули 
$(L_\infty(G),\cdot_\infty)$ и $(L_p(G),\cdot_p)$ для 
$1<p\leq+\infty$ изоморфны в $L_1(G)-\mathbf{mod}$ и 
$\mathbf{mod}-L_1(G)$. Наконец, легко проверить, что модули 
$(L_1(G),\convol)$ и $(L_\infty(G),\cdot_\infty)$ изоморфны в 
$L_1(G)-\mathbf{mod}$ и $\mathbf{mod}-L_1(G)$ посредством изоморфизма 
$j:L_1(G)\to L_\infty(G):f\mapsto(s\mapsto f(s^{-1}))$. Таким образом, 
все рассматриваемые модули попарно изоморфны. Осталось вспомнить, что по 
теореме \ref{L1ModL1MetTopProjCharac} и предложению 
\ref{LInfIsL1MetrInj} модуль $L_1(G)$ является топологически проективным 
и плоским, и  по предложению \ref{LInfIsL1MetrInj} модуль  $L_\infty(G)$ 
является топологически инъективным.
\end{proof}

В таблице \ref{HomolTrivModMetTh}, приведенной ниже, собраны результаты о 
гомологических свойствах модулей гармонического анализа для метрической, 
топологической и относительной теории. Каждая ячейка таблицы содержит 
условие при котором модуль обладает соответствующим свойством и ссылки 
на доказательства. Стрелка $\Longrightarrow$ обозначает, что на данный 
момент известно только необходимое условие. Стоит сказать, что результаты 
для модулей $L_p(G)$, где $1<p<\infty$, верны для обоих типов внешнего 
умножения $\convol_p$ и $\cdot_p$. Формулировки и доказательства теорем о 
гомологической тривиальности модулей $\mathbb{C}_\gamma$ в случае 
относительной теории будут такими же как и в предложениях 
\ref{OneDimL1ModMetTopProjCharac} и \ref{OneDimL1ModMetTopInjFlatCharac}, 
но на самом деле эти результаты давно известны. Например, критерий о 
проективности $\mathbb{C}_\gamma$ дан в 
\cite[теорема~IV.5.13]{HelBanLocConvAlg}, а инъективности в 
\cite[теорема~2.5]{JohnCohomolBanAlg}. Для алгбер $L_1(G)$ и $M(G)$ 
понятия $\langle$~проективности / инъективности / плоскости~$\rangle$ 
совпадают во всех трех теориях для модулей $\langle$~$M(G)$ и 
$\mathbb{C}_\gamma$ / $L_\infty(G)$, $C_0(G)$ и $\mathbb{C}_\gamma$ / 
$L_1(G)$ и $\mathbb{C}_\gamma$~$\rangle$. Наконец, плоские $M(G)$-модули 
$M(G)$ так же имеют одно и то же описание в метрической, топологической 
и относительной теории.

\begin{table}[ht]
    \centering
    \caption{Гомологически тривиальные модули гармонического анализа}
    \label{ThreeTheoriesComparisonTable}
    \begin{tiny}
        \setlength{\tabcolsep}{0.5em}{
        \begin{tabular}{|c|c|c|c|c|c|c|}
            \hline
                 & 
                 \multicolumn{3}{c|}{$L_1(G)$-модули} & 
                 \multicolumn{3}{c|}{$M(G)$-модули} \\
            \hline
                 & 
                 Проективность & 
                 Инъективность & 
                 Плоскость & 
                 Проективность & 
                 Инъективность & 
                 Плоскость \\
            \hline
                 \multicolumn{7}{c}{\textbf{Метрическая теория}} \\
            \hline
                $L_1(G)$ & 
                \shortstack{
                    $G$ дискретна \\ 
                    \ref{L1ModL1MetTopProjCharac}
                } & 
                \shortstack{
                    $G=\{e_G\}$ \\ 
                    \ref{StdModAreNotL1MGMetTopProjInjFlat}, 
                    \ref{StdModFinGrL1MGMetrInjProjFlatCharac}
                } &
                \shortstack{
                    $G$ любая \\ 
                    \ref{LInfIsL1MetrInj}
                } &
                \shortstack{
                    $G$ дискретна \\ 
                    \ref{L1ModL1MetTopProjCharac},
                    \ref{MGMetTopProjInjFlatRedToL1}
                } &
                \shortstack{
                    $G=\{e_G\}$ \\
                    \ref{StdModAreNotL1MGMetTopProjInjFlat},
                    \ref{StdModFinGrL1MGMetrInjProjFlatCharac}
                } & 
                \shortstack{
                    $G$ любая \\
                    \ref{LInfIsL1MetrInj},
                    \ref{MGMetTopProjInjFlatRedToL1}
                } \\
            \hline
                 $L_p(G)$ & 
                 \shortstack{
                    $G=\{e_G\}$ \\ 
                    \ref{StdModAreNotL1MGMetTopProjInjFlat},
                    \ref{LpFinGrL1MGMetrInjProjCharac}
                } &
                \shortstack{
                    $G=\{e_G\}$ \\ 
                    \ref{StdModAreNotL1MGMetTopProjInjFlat},
                    \ref{LpFinGrL1MGMetrInjProjCharac}
                } & 
                \shortstack{
                    $G=\{e_G\}$ \\
                    \ref{StdModAreNotL1MGMetTopProjInjFlat},
                    \ref{StdModFinGrL1MGMetrInjProjFlatCharac}
                } & 
                \shortstack{
                    $G=\{e_G\}$ \\ 
                    \ref{StdModAreNotL1MGMetTopProjInjFlat},
                    \ref{LpFinGrL1MGMetrInjProjCharac}
                } & 
                \shortstack{
                    $G=\{e_G\}$ \\ 
                    \ref{StdModAreNotL1MGMetTopProjInjFlat},
                    \ref{LpFinGrL1MGMetrInjProjCharac}
                } & 
                \shortstack{
                    $G=\{e_G\}$ \\ 
                    \ref{StdModAreNotL1MGMetTopProjInjFlat},
                    \ref{StdModFinGrL1MGMetrInjProjFlatCharac}
                } \\
            \hline
                $L_\infty(G)$ & 
                \shortstack{$
                    G=\{e_G\}$ \\ 
                    \ref{StdModAreNotL1MGMetTopProjInjFlat},
                    \ref{LpFinGrL1MGMetrInjProjCharac}
                } & 
                \shortstack{
                    $G$ любая  \\ 
                    \ref{LInfIsL1MetrInj}
                } & 
                \shortstack{
                    $G=\{e_G\}$ \\ 
                    \ref{StdModAreNotL1MGMetTopProjInjFlat},
                    \ref{StdModFinGrL1MGMetrInjProjFlatCharac}
                } & 
                \shortstack{
                    $G=\{e_G\}$ \\ 
                    \ref{StdModAreNotL1MGMetTopProjInjFlat},
                    \ref{LpFinGrL1MGMetrInjProjCharac}
                } & 
                \shortstack{
                    $G$ любая  \\ 
                    \ref{LInfIsL1MetrInj},
                    \ref{MGMetTopProjInjFlatRedToL1}
                } & 
                \shortstack{
                    $G=\{e_G\}$ \\ 
                    \ref{StdModAreNotL1MGMetTopProjInjFlat},
                    \ref{StdModFinGrL1MGMetrInjProjFlatCharac}
                } \\ 
            \hline
                $M(G)$ & 
                \shortstack{
                    $G$ дискретна  \\ 
                    \ref{L1MetTopProjAndMetrFlatOfMeasAlg}
                } & 
                \shortstack{
                    $G=\{e_G\}$ \\ 
                    \ref{StdModAreNotL1MGMetTopProjInjFlat},
                    \ref{StdModFinGrL1MGMetrInjProjFlatCharac}
                } & 
                \shortstack{
                    $G$ дискретна  \\ 
                    \ref{MeasAlgIsL1TopFlat}
                } & 
                \shortstack{
                    $G$ любая  \\ 
                    \ref{MGModMGMetTopProjFlatCharac}
                } & 
                \shortstack{
                    $G=\{e_G\}$ \\ 
                    \ref{StdModAreNotL1MGMetTopProjInjFlat},
                    \ref{StdModFinGrL1MGMetrInjProjFlatCharac}
                } & 
                \shortstack{
                    $G$ любая  \\ 
                    \ref{MGModMGMetTopProjFlatCharac}
                } \\ 
            \hline
                $C_0(G)$ & 
                \shortstack{
                    $G=\{e_G\}$ \\         
                    \ref{StdModAreNotL1MGMetTopProjInjFlat},
                    \ref{StdModFinGrL1MGMetrInjProjFlatCharac}
                } & 
                \shortstack{
                    $G$ конечна  \\ 
                    \ref{StdModAreNotL1MGMetTopProjInjFlat},
                    \ref{StdModFinGrL1MGMetrInjProjFlatCharac}
                } & 
                \shortstack{
                    $G=\{e_G\}$ \\ 
                    \ref{StdModAreNotL1MGMetTopProjInjFlat},
                    \ref{StdModFinGrL1MGMetrInjProjFlatCharac}
                } & 
                \shortstack{
                    $G=\{e_G\}$ \\ 
                    \ref{StdModAreNotL1MGMetTopProjInjFlat},
                    \ref{StdModFinGrL1MGMetrInjProjFlatCharac}
                } & 
                \shortstack{
                    $G$ конечна  \\ 
                    \ref{StdModAreNotL1MGMetTopProjInjFlat},
                    \ref{StdModFinGrL1MGMetrInjProjFlatCharac}
                } & 
                \shortstack{
                    $G=\{e_G\}$ \\ 
                    \ref{StdModAreNotL1MGMetTopProjInjFlat},
                    \ref{StdModFinGrL1MGMetrInjProjFlatCharac}
                } \\ 
            \hline
                $\mathbb{C}_\gamma$ & 
    			\shortstack{
    				$G$ компактна  \\ \ref{OneDimL1ModMetTopProjCharac}
    			} & 
    			\shortstack{
    				$G$ аменабельна  \\ 
    				\ref{OneDimL1ModMetTopInjFlatCharac}
    			} & 
    			\shortstack{
    				$G$ аменабельна  \\ 
    				\ref{OneDimL1ModMetTopInjFlatCharac}
    			} & 
    			\shortstack{
    				$G$ компактна  \\ 
    				\ref{OneDimL1ModMetTopProjCharac},
    				\ref{MGMetTopProjInjFlatRedToL1}
    			} & 
    			\shortstack{
    				$G$ аменабельна  \\ 
    				\ref{OneDimL1ModMetTopInjFlatCharac},
    				\ref{MGMetTopProjInjFlatRedToL1}
    			} & 
    			\shortstack{
    				$G$ аменабельна  \\ 
    				\ref{OneDimL1ModMetTopInjFlatCharac},
    				\ref{MGMetTopProjInjFlatRedToL1}
    			} \\ 
            \hline
                \multicolumn{7}{c}{\textbf{Топологическая теория}} \\
            \hline
                $L_1(G)$ & 
    			\shortstack{
    				$G$ дискретна \\ 
    				\ref{L1ModL1MetTopProjCharac}
    			} & 
    			\shortstack{
    				$G$ конечна \\ 
    				\ref{StdModAreNotL1MGMetTopProjInjFlat}, 
    				\ref{StdModFinGrL1MGTopInjProjFlatCharac}
    			} & 
    			\shortstack{
    				$G$ любая \\ 
    				\ref{LInfIsL1MetrInj}
    			} & 
    			\shortstack{
    				$G$ дискретна \\ 
    				\ref{L1ModL1MetTopProjCharac},
    				\ref{MGMetTopProjInjFlatRedToL1}
    			} & 
    			\shortstack{
    				$G$ конечна \\ 
    				\ref{StdModAreNotL1MGMetTopProjInjFlat}, 
    				\ref{StdModFinGrL1MGTopInjProjFlatCharac}
    			} & 
    			\shortstack{
    				$G$ любая \\ 
    				\ref{LInfIsL1MetrInj},
    				\ref{MGMetTopProjInjFlatRedToL1}
    			} \\ 
            \hline
                $L_p(G)$ & 
    			\shortstack{
    				$G$ конечна \\ 
    				\ref{StdModAreNotL1MGMetTopProjInjFlat},
    				\ref{StdModFinGrL1MGTopInjProjFlatCharac}
    			} & 
    			\shortstack{
    				$G$ конечна \\ 
    				\ref{StdModAreNotL1MGMetTopProjInjFlat},
    				\ref{StdModFinGrL1MGTopInjProjFlatCharac}
    			} & 
    			\shortstack{
    				$G$ конечна \\ 
    				\ref{StdModAreNotL1MGMetTopProjInjFlat},
    				\ref{StdModFinGrL1MGTopInjProjFlatCharac}
    			} & 
    			\shortstack{
    				$G$ конечна \\ 
    				\ref{StdModAreNotL1MGMetTopProjInjFlat},
    				\ref{StdModFinGrL1MGTopInjProjFlatCharac}
    			} & 
    			\shortstack{
    				$G$ конечна \\ 
    				\ref{StdModAreNotL1MGMetTopProjInjFlat},
    				\ref{StdModFinGrL1MGTopInjProjFlatCharac}
    			} & 
    			\shortstack{
    				$G$ конечна \\ 
    				\ref{StdModAreNotL1MGMetTopProjInjFlat},
    				\ref{StdModFinGrL1MGTopInjProjFlatCharac}
    			} \\ 
            \hline
                $L_\infty(G)$ & 
    			\shortstack{
    				$G$ конечна \\ 
    				\ref{StdModAreNotL1MGMetTopProjInjFlat},
    				\ref{StdModFinGrL1MGTopInjProjFlatCharac}
    			} & 
    			\shortstack{
    				$G$ любая \\ 
    				\ref{LInfIsL1MetrInj}
    			} & 
    			\shortstack{
    				$G$ конечна \\ 
    				\ref{StdModAreNotL1MGMetTopProjInjFlat},
    				\ref{StdModFinGrL1MGTopInjProjFlatCharac}
    			} & 
    			\shortstack{
    				$G$ конечна \\ 
    				\ref{StdModAreNotL1MGMetTopProjInjFlat},
    				\ref{StdModFinGrL1MGTopInjProjFlatCharac}
    			} & 
    			\shortstack{
    				$G$ любая \\ 
    				\ref{LInfIsL1MetrInj},
    				\ref{MGMetTopProjInjFlatRedToL1}
    			} & 
    			\shortstack{
    				$G$ конечна \\ 
    				\ref{StdModAreNotL1MGMetTopProjInjFlat},
    				\ref{StdModFinGrL1MGTopInjProjFlatCharac}
    			} \\ 
            \hline
                $M(G)$ & 
    			\shortstack{
    				$G$ дискретна \\ 
    				\ref{L1MetTopProjAndMetrFlatOfMeasAlg}
    			} & 
    			\shortstack{
    				$G$ конечна \\ 
    				\ref{StdModAreNotL1MGMetTopProjInjFlat},
    				\ref{StdModFinGrL1MGTopInjProjFlatCharac}
    			} & 
    			\shortstack{
    				$G$ любая \\ 
    				\ref{MeasAlgIsL1TopFlat}
    			} & 
    			\shortstack{
    				$G$ любая \\ 
    				\ref{MGModMGMetTopProjFlatCharac}
    			} & 
    			\shortstack{
    				$G$ конечна \\ 
    				\ref{StdModAreNotL1MGMetTopProjInjFlat},
    				\ref{StdModFinGrL1MGTopInjProjFlatCharac}
    			} & 
    			\shortstack{
    				$G$ любая \\ 
    				\ref{MGModMGMetTopProjFlatCharac}
    			} \\ 
            \hline
                $C_0(G)$ & 
    			\shortstack{
    				$G$ конечна \\ 
    				\ref{StdModAreNotL1MGMetTopProjInjFlat},
    				\ref{StdModFinGrL1MGTopInjProjFlatCharac}
    			} & 
    			\shortstack{
    				$G$ конечна \\ 
    				\ref{StdModAreNotL1MGMetTopProjInjFlat},
    				\ref{StdModFinGrL1MGTopInjProjFlatCharac}
    			} & 
    			\shortstack{
    				$G$ конечна \\ 
    				\ref{StdModAreNotL1MGMetTopProjInjFlat},
    				\ref{StdModFinGrL1MGTopInjProjFlatCharac}
    			} & 
    			\shortstack{
    				$G$ конечна \\ 
    				\ref{StdModAreNotL1MGMetTopProjInjFlat},
    				\ref{StdModFinGrL1MGTopInjProjFlatCharac}
    			} & 
    			\shortstack{
    				$G$ конечна \\ 
    				\ref{StdModAreNotL1MGMetTopProjInjFlat},
    				\ref{StdModFinGrL1MGTopInjProjFlatCharac}
    			} & 
    			\shortstack{
    				$G$ конечна \\ 
    				\ref{StdModAreNotL1MGMetTopProjInjFlat},
    				\ref{StdModFinGrL1MGTopInjProjFlatCharac}
    			} \\ 
            \hline
                $\mathbb{C}_\gamma$ & 
    			\shortstack{
    				$G$ компактна \\ 
    				\ref{OneDimL1ModMetTopProjCharac}
    			} & 
    			\shortstack{
    				$G$ аменабельна \\ 
    				\ref{OneDimL1ModMetTopInjFlatCharac}
    			} & 
    			\shortstack{
    				$G$ аменабельна \\ 
    				\ref{OneDimL1ModMetTopInjFlatCharac}
    			} & 
    			\shortstack{
    				$G$ компактна \\ 
    				\ref{OneDimL1ModMetTopProjCharac},
    				\ref{MGMetTopProjInjFlatRedToL1}
    			} & 
    			\shortstack{
    				$G$ аменабельна \\ 
    				\ref{OneDimL1ModMetTopInjFlatCharac},
    				\ref{MGMetTopProjInjFlatRedToL1}
    			} & 
    			\shortstack{
    				$G$ аменабельна \\ 
    				\ref{OneDimL1ModMetTopInjFlatCharac},
    				\ref{MGMetTopProjInjFlatRedToL1}
    			} \\ 
            \hline
                \multicolumn{7}{c}{\textbf{Относительная теория}} \\
            \hline
                $L_1(G)$ & 
    			\shortstack{
    				$G$ любая \\ \cite{DalPolHomolPropGrAlg}, \S 6
    			} & 
    			\shortstack{
    				$G$ аменабельна \\  и дискретна \\ \cite{DalPolHomolPropGrAlg}, \S 6
    			} & 
    			\shortstack{
    				$G$ любая \\ \cite{DalPolHomolPropGrAlg}, \S 6
    			} & 
    			\shortstack{
    				$G$ любая \\ \cite{RamsHomPropSemgroupAlg}, \S 3.5
    			} & 
    			\shortstack{
    				$G$ аменабельна \\  и дискретна \\ \cite{RamsHomPropSemgroupAlg}, \S 3.5
    			} & 
    			\shortstack{
    				$G$ любая \\ \cite{RamsHomPropSemgroupAlg}, \S 3.5
    			} \\ 
            \hline
                $L_p(G)$ & 
    			\shortstack{
    				$G$ компактна \\ \cite{DalPolHomolPropGrAlg}, \S 6
    			} & 
    			\shortstack{
    				$G$ аменабельна \\ \cite{RachInjModAndAmenGr}
    			} & 
    			\shortstack{
    				$G$ аменабельна \\ \cite{RachInjModAndAmenGr}
    			} & 
    			\shortstack{
    				$G$ компактна \\ \cite{RamsHomPropSemgroupAlg}, \S 3.5
    			} & 
    			\shortstack{
    				$G$ аменабельна \\ \cite{RamsHomPropSemgroupAlg}, \S 3.5, \cite{RachInjModAndAmenGr}
    			} & 
    			\shortstack{
    				$G$ аменабельна \\ \cite{RamsHomPropSemgroupAlg}, \S 3.5
    			} \\
            \hline
                $L_\infty(G)$ & 
    			\shortstack{
    				$G$ конечна \\ \cite{DalPolHomolPropGrAlg}, \S 6
    			} & 
    			\shortstack{
    				$G$ любая \\ \cite{DalPolHomolPropGrAlg}, \S 6
    			} & 
    			\shortstack{
    				$G$ аменабельна \\ \cite{DalPolHomolPropGrAlg}, \S 6
    			} & 
    			\shortstack{
    				$G$ конечна \\ \cite{RamsHomPropSemgroupAlg}, \S 3.5
    			} & 
    			\shortstack{
    				$G$ любая \\ \cite{RamsHomPropSemgroupAlg}, \S 3.5
    			} & 
    			\shortstack{
    				$G$ аменабельна \\ ($\Longrightarrow$)\cite{RamsHomPropSemgroupAlg}, \S 3.5
    			} \\ 
            \hline
                $M(G)$ & 
    			\shortstack{
    				$G$ дискретна \\ \cite{DalPolHomolPropGrAlg}, \S 6
    			} & 
    			\shortstack{
    				$G$ аменабельна \\ \cite{DalPolHomolPropGrAlg}, \S 6
    			} & 
    			\shortstack{
    				$G$ любая \\ \cite{RamsHomPropSemgroupAlg}, \S 3.5
    			} & 
    			\shortstack{
    				$G$ любая \\ \cite{RamsHomPropSemgroupAlg}, \S 3.5
    			} & 
    			\shortstack{
    				$G$ аменабельна \\ \cite{RamsHomPropSemgroupAlg}, \S 3.5
    			} & 
    			\shortstack{
    				$G$ любая \\ \cite{RamsHomPropSemgroupAlg}, \S 3.5
    			} \\ 
            \hline
                $C_0(G)$ & 
    			\shortstack{
    				$G$ компактна \\ \cite{DalPolHomolPropGrAlg}, \S 6
    			} & 
    			\shortstack{
    				$G$ конечна \\ \cite{DalPolHomolPropGrAlg}, \S 6
    			} & 
    			\shortstack{
    				$G$ аменабельна \\ \cite{DalPolHomolPropGrAlg}, \S 6
    			} & 
    			\shortstack{
    				$G$ компактна \\ \cite{RamsHomPropSemgroupAlg}, \S 3.5
    			} & 
    			\shortstack{
    				$G$ конечна \\ \cite{RamsHomPropSemgroupAlg}, \S 3.5
    			} & 
    			\shortstack{
    				$G$ аменабельна \\ \cite{RamsHomPropSemgroupAlg}, \S 3.5
    			} \\ 
            \hline
                $\mathbb{C}_\gamma$ & 
    			\shortstack{
    				$G$ компактна \\ 
    				\ref{OneDimL1ModMetTopProjCharac}
    			} & 
    			\shortstack{
    				$G$ аменабельна \\ 
    				\ref{OneDimL1ModMetTopInjFlatCharac}
    			} & 
    			\shortstack{
    				$G$ аменабельна \\ 
    				\ref{OneDimL1ModMetTopInjFlatCharac}
    			} & 
    			\shortstack{
    				$G$ компактна \\ 
    				\ref{OneDimL1ModMetTopProjCharac},
    				\ref{MGMetTopProjInjFlatRedToL1}
    			} & 
    			\shortstack{
    				$G$ аменабельна \\ 
    				\ref{OneDimL1ModMetTopInjFlatCharac},
    				\ref{MGMetTopProjInjFlatRedToL1}
    			} & 
    			\shortstack{
    				$G$ аменабельна \\ 
    				\ref{OneDimL1ModMetTopInjFlatCharac},
    				\ref{MGMetTopProjInjFlatRedToL1}
    			} \\                   
            \hline
            \end{tabular}
        }
    \end{tiny}
    \label{HomolTrivModMetTh}
\end{table}


\end{fulltext}
\begin{thebibliography}{99}

%
\RBibitem{DalPolHomolPropGrAlg}
\by H.\,G.~Dales, M.\,E.~Polyakov 
\paper Homological properties of modules over group algebras
\jour Proc. Lond.  Math. Soc. 
\vol 89
\issue 2 
\yr 2004 
\pages 390--426

%
\RBibitem{RamsHomPropSemgroupAlg}
\by P.~Ramsden 
\thesis Homological properties of semigroup algebras
\publaddr The University of Leeds 
\yr 2009

\RBibitem{RachInjModAndAmenGr}
\by G.~Racher 
\paper Injective modules and amenable groups
\jour Comment. Math. Helv. 
\vol 88
\issue 4 
\yr 2013 
\pages 1023--1031

\RBibitem{GravInjProjBanMod}
\by A.\,W.\,M.~Graven 
\paper Injective and projective Banach modules
\jour Indag. Math.
\publ Elsevier 
\vol 82
\issue 1 
\yr 1979
\pages 253--272

\RBibitem{WhiteInjmoduAlg}
\by M.\,C.~White
\paper Injective modules for uniform algebras
\jour Proc. London Math. Soc. 
\vol 73
\issue 1
\yr 1996
\pages 155--184

\RBibitem{HelemHomolDimNorModBanAlg}
\by А.\,Я.~Хелемский 
\paper О гомологической размерности нормированных модулей над банаховыми алгебрами
\jour Матем. Сб. 
\vol 81
\issue 3
\yr 1970
\pages 430--444

\RBibitem{NemGeomProjInjFlatBanMod}
\by Н.\,Т.~Немеш 
\paper Геометрия проективных, инъективных и плоских банаховых модулей
\jour Фундамент. и прикл. матем.
\vol 21
\issue 3
\yr 2016
\pages 161--184

\RBibitem{HelBanLocConvAlg}
\by А.\,Я.~Хелемский 
\book Банаховы и полинормированные алгебры: общая теория, представления, гомологии. 
\publ Наука
\yr 1989

\RBibitem{DalBanAlgAutCont}
\by H.\,G.~Dales 
\book Banach algebras and automatic continuity
\publ Clarendon Press
\yr 2000

\RBibitem{NemMetTopProjIdBanAlg}
\by Н.\,Т.~Немеш 
\paper Метрически и топологически проективные идеалы банаховых алгебр
\jour Матем. заметки
\vol 99 
\issue 4
\pages 526--536

\RBibitem{WendLeftCentrzrs}
\by J.\,G.~Wendel 
\paper Left centralizers and isomorphisms of group algebras
\jour Pacific J. Math. 
\vol 2
\issue 3
\yr 1952
\pages 251--261

\RBibitem{DefFloTensNorOpId}
\by A.~Defant, K.~Floret
\book Tensor norms and operator ideals 
\vol 176
\publ Elsevier
\yr 1992

\RBibitem{RosOnRelDisjFamOfMeas}
\by H.~Rosenthal
\paper On relatively disjoint families of measures, with some applications to Banach space theory
\jour Stud. Math. 
\vol 37
\issue 1
\yr 1970
\pages 13--36

\RBibitem{WojBanSpForAnalysts}
\by P.~Wojtaszczyk 
\book Banach spaces for analysts 
\publ Cambridge University Press
\vol 25
\yr 1996

\RBibitem{KalAlbTopicsBanSpTh}
\by F.~Albiac, N.\,J.~Kalton 
\book Topics in Banach space theory
\vol 233
\publ Springer
\yr 2006

\RBibitem{LauMingComplSubspInLInfOfG}
\by A.\,T.-M.~Lau,  V.~Losert
\paper Complementation of certain subspaces of $\it{L_\infty(G)}$ of a locally compact group
\jour Pacific J. Math
\vol 141
\issue 2
\yr 1990
\pages 295--310

\RBibitem{LyubIsomEmdbFinDimLp}
\by Yu.\,I.~Lyubich, O.\,A.~Shatalova
\paper Isometric embeddings of finite-dimensional $\ell_p$-spaces over the quaternions
\jour St. Petersburg Math. J.
\vol 16
\issue 1
\yr 2005
\pages 9--24

\RBibitem{JohnCohomolBanAlg}
\by B.~Johnson
\book Cohomology in Banach Algebras 
\publ Memoirs Series
\yr 1972

\end{thebibliography}
\end{document}